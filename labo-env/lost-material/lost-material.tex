\documentclass[a4paper,11pt]{style-esi/td}

\usepackage{style-esi/licence}
\usepackage{style-esi/exercice}
\usepackage{style-esi/exemple}
\usepackage{style-esi/question}
\usepackage{style-esi/tutoriel}
\usepackage{style-esi/listing}
\usepackage{style-esi/images}
\usepackage{style-dev1/dev1}

\begin{document}

\seance{1}{Introduction}
\entete
\titre
\ccbysa{esi-dev1-list@he2b.be}
\lastedit

\bigskip
\tableofcontents

	%====================
	\subsection{L'éditeur}
	%=====================

		Un \textbf{éditeur de texte} (ou, plus court, un \textbf{éditeur}) 
		est un programme qui vous permet d'entrer, modifier
		le contenu d'un fichier texte. 
		Vous connaissez probablement \name{Notepad} sous \name{Windows}. 
		Sur \name{Linux}, il en existe beaucoup. 
		Celui que nous vous proposons dans un premier temps au laboratoire 
		s'appelle \name{nano}. 
		Il est assez facile pour débuter et faire des choses simples. 
		Par la suite, vous pourrez choisir de rester sous \name{nano} 
		ou d'utiliser \name{vim}.
		
		\begin{Tutoriel}{Premiers pas avec nano}
			\vspace{-1em}
			\begin{steps}
			\item 
				En préambule, tapez la commande \kbd{cd} pour revenir dans \textit{votre home}.
			\item 
				Tapez \kbd{nano test} pour commencer à éditer le fichier \samp{test} 
				(comme il n'existe pas encore, il est créé).
			
				Une fenêtre s'ouvre. 
				Vous voyez qu'elle est scindée en 2 parties : 
				la partie supérieure où vous écrivez votre texte 
				et la partie inférieure où sont indiquées 
				les différentes commandes (le \char`\^{} représente la touche \verb|Ctrl|)
			\item 
				Entrez quelques mots.
			\item 
				Appuyez sur la combinaison de touches \verb|Ctrl X|, 
				confirmez que vous voulez sauver vos modifications et sortez.
			
				Vous êtes maintenant revenu à l'invite de commande.
			\item 
				Tapez à présent la commande \kbd{ls}.
				Vous pouvez constater que le fichier \verb_test_ est apparu dans la liste ;)
			\end{steps}
		\end{Tutoriel}

		Pour aller plus loin dans l'apprentissage de cet éditeur,
		consultez ce document :\\
		{\tiny\url{http://fr.openclassrooms.com/informatique/cours/reprenez-le-controle-a-l-aide-de-linux/nano-l-editeur-de-texte-du-debutant}}.

        	%==================================
	\subsection{La racine du système de fichiers}
	%==================================

		Rappelez-vous que les fichiers sont organisés en hiérarchie :
		un dossier contient d'autres dossiers qui lui-même\dots

		On a vu que chaque utilisateur dispose d'un dossier personnel. 
		Où se trouve-il ? Dans un dossier appelé \samp{home}. 
		Et où se trouve ce dossier \samp{home} ? 
		Dans un dossier appelé \og{}\samp{/}\fg{} (la barre oblique). 
		Et où se trouve ce dossier \og{}\samp{/}\fg{} ? Nulle part ;) 
		C'est le dossier principal (la \textbf{racine}).

		\begin{theorie}{La racine du système de fichiers}
			La \textbf{racine} est le dossier principal, 
			tout en haut de la hiérarchie des dossiers.
		\end{theorie}

	%==================================
	\subsection{Chemin absolu et relatif}
	%==================================

		\exergue{Tout est relatif, et cela seul est absolu.}{Auguste Comte}
		
		\bigskip
		Régulièrement, il faut indiquer un endroit du système de fichiers 
		(par exemple pour y aller). 
		Désigner simplement le nom du dossier ne suffit pas.  

		\begin{theorie}{Chemin absolu et relatif}
			\begin{description}
			\item[Chemin]
				Un \textbf{chemin} est une suite de dossiers à traverser 
				pour arriver au dossier ou au fichier qui nous intéresse.
				On les sépare par \og{}\samp{/}\fg{}.

				Par exemple : \samp{home/g12345/dev1/td1/test}
				indique que dans le dossier \samp{home}
				il y a un dossier \samp{g12345}
				qui contient un dossier \samp{dev1} 
				qui contient un dossier \samp{td1} 
				qui contient un fichier (ou un dossier) \samp{test}.
			\item[Chemin absolu]
				Un chemin est \textbf{absolu} 
				s'il commence à partir de la \textbf{racine}.
				Il commence donc par \og{}\samp{/}\fg{}.
				Par exemple : \samp{/home/g12345/td1/test}
			\item[Chemin absolu]
				Dans le cas contraire il est \textbf{relatif}.
				Il s'agit d'un chemin à suivre à partir du \textbf{dossier courant}.
			\end{description}
		\end{theorie}

		\begin{alertbox}
			Un chemin \textbf{absolu} désigne toujours le même endroit.
			Un chemin \textbf{relatif} non ! \c ca dépend d'où on est,
			du dossier courant.
		\end{alertbox}

		\begin{Exercice}{pwd}
			Entrez la commande \kbd{pwd} (que fait-elle encore ?). 
			Comprenez-vous la notation qu'elle utilise pour la réponse ?
			Est-ce une chemin absolu ou relatif ?
		\end{Exercice}

		\begin{infobox}
			Rappel : dans une commande, un nom de fichier (ou de dossier)
			désigne toujours un fichier (ou un dossier) se trouvant
			dans le dossier courant
	
			Pour indiquer un \textbf{fichier} (ou un dossier) 
			se trouvant \textbf{ailleurs},
			il faut donner son \textbf{chemin} (absolu ou relatif).
		\end{infobox}

		Par exemple : \kbd{ls /home/g12345}
		permet de visualiser le contenu de la home de \samp{g12345}
		(quel que soit l'endroit où on se trouve puisque c'est un chemin absolu).
		
		\begin{Exercice}{Utilisation d'un chemin}
			Supposons que vous êtes dans votre home 
			et que vous ne pouvez pas vous déplacer 
			(la commande \samp{cd} est interdite !).
			\begin{enumerate}
			\item 
				Affichez le contenu du fichier \samp{test}
				(qui se trouve dans le dossier \samp{td1})
				en utilisant un chemin \textbf{relatif}.
			\item 
				Faites la même chose 
				en utilisant un chemin \textbf{absolu}.
			\item
				Créez une copie du fichier \samp{welcome} 
				dans le dossier \samp{td1}.
			\item 
				Auriez-vous pu faire cette copie 
				sans utiliser de chemin,
				en vous déplaçant (\samp{cd}) ?
			\end{enumerate}
		\end{Exercice}

	%==================================
	\subsection{Raccourcis pour des chemins}
	%==================================

		\begin{theorie}{Raccourcis}
			Dans un chemin :
			\begin{itemize}
			\item \samp{\textasciitilde{}g12345} : désigne la home de l'utilisateur \samp{g12345} ;
			\item \samp{\textasciitilde{}} : désigne la home de l'utilisateur qui entre la commande ;
			\item \samp{..} : désigne le dossier parent du dossier courant ;
			\item \samp{.} : désigne le dossier courant.
			\end{itemize}
		\end{theorie}

		\begin{Exercice}{Se familiariser avec les raccourcis}
			Placez-vous dans votre dossier personnel.
			En utilisant les raccourcis 
			afin de trouver la solution la plus courte possible :
			\begin{enumerate}
			\item Affichez le contenu de votre dossier personnel
				en utilisant un chemin absolu.
			\item Affichez le contenu de la home de votre professeur
				en utilisant un chemin absolu.
			\item Refaites la même chose mais en utilisant un chemin relatif.
			\end{enumerate}
		\end{Exercice}

		\begin{Exercice}{Comprendre un chemin}
			Que désigne le chemin suivant :
			\samp{\textasciitilde{}mcd/../../home} ?
		\end{Exercice}

		\begin{Exercice}{Chemins absolus et relatifs}
			Parmi tous les chemins suivants, quels sont ceux qui sont 
			\textbf{relatifs} ?			
			\begin{selectmany} 
			\item \verb@~/../g12345/td2@
			\item \verb@/home/g12345/../g54321/Hello.java@
			\item \verb@./tds/td2@
			\item \verb@tds/td2/Hello.java@
			\item \verb@~g12345/tds/td2@
			\end{selectmany} 
        \end{Exercice}
        
\end{document}
