\documentclass[a4paper,11pt]{style-esi/td}

\usepackage{style-esi/licence}
\usepackage{style-esi/exercice}
\usepackage{style-esi/exemple}
\usepackage{style-esi/question}
\usepackage{style-esi/tutoriel}
\usepackage{style-esi/listing}
\usepackage{style-esi/images}
\usepackage{style-dev1/dev1}

\begin{document}

\seance{0}{Lost material}
\entete
\titre
\ccbysa{esi-dev1-list@he2b.be}
\lastedit

\bigskip
\tableofcontents

%===================
\section{Rechercher (dans) des fichiers}
%====================

	Lorsqu'on commence à avoir beaucoup de fichiers,
	il arrive qu'on ne sache plus trop bien où se trouve un fichier
	ou quel fichier contient un texte particulier.
	Les commandes \samp{find} et \samp{grep} sont là pour nous aider.

	%===================
	\subsection{Rechercher un fichier}
	%====================

		La commande \samp{find} permet de chercher des fichiers
		dans le système de fichiers en fonction de critères : 
		son nom, sa taille, sa date de création\dots

		\begin{theorie}{find}
			\kbd{find dossierOùChercher critèreDeRecherche...}
			cherche dans le dossier indiqué (et tous les sous dossiers)
			les fichiers qui respectent tous les critères indiqués.
			Par défaut, la commande affiche le nom des fichiers trouvés.

			Parmi les critères citons :
			\begin{itemize}
				\item \kbd{-name nomfichier}
			\end{itemize}
        \end{theorie}
        
%=================================
\section{Les variables d'environnement}  
%=================================

	\subsection{Introduction}
	%========================

		\begin{theorie}{Variable d'environnement}
			Une \textbf{variable d'environnement} 
			est une variable associée à votre shell 
			contenant un texte qui est accessible 
			par toutes les applications que vous lancez. 
			Généralement, elle permet de configurer certaines applications, 
			d'en modifier le comportement.
		\end{theorie}	

		\begin{theorie}{Manipuler une variable d'environnement}
			Si \samp{VARENV} est une variable d'environnement :
			\begin{itemize}
			\item \kbd{export VARENV=valeur} crée une nouvelle variable à la valeur donnée.
			\item \kbd{VARENV=valeur} modifie la valeur d'une variable existante.
			\item \kbd{\$VARENV} est remplacé par la valeur de la variable.
			\end{itemize}
		\end{theorie}

		\begin{Experience}{Manipuler une variable}
			\begin{steps}
			\item Entrez \kbd{VAR=12} pour créer la variable.
			\item Entrez \kbd{echo "Bonjour !"}.
				\\Cette commande affiche ce qu'on lui donne en paramètre.
			\item Entrez \kbd{echo \$VAR} pour afficher le contenu de la variable.
			\end{steps}
		\end{Experience}

		\begin{Exercice}{Comprendre la signification du \$}
			Supposons que la variable d'environnement \samp{VAR} vaut 12.
			Que va faire chacune des commandes suivantes ?
			Une fois que vous pensez le savoir, 
			vérifiez-le le tapant sur la machine.
			\begin{itemize}
			\item \kbd{echo VAR}
			\item \kbd{\$VAR=12}
			\item \kbd{VAR=\$VAR} 
			\item \kbd{echo \$VAR + 30} 
			\item \kbd{VAR=\$VAR+30} 
			\item \kbd{VAR=\$VAR + 30} 
			\item \kbd{VAR="\$VAR + 30"} 
			\item \kbd{VAR=VAR} 
			\end{itemize}
		\end{Exercice}

	\subsection{Le prompt}
	%=====================

		\begin{theorie}{Prompt}
			Le \textbf{prompt} (ou \textit{invite} en français) 
			est le texte qui apparait à gauche
			de ce que vous tapez dans votre shell. 
			Il est déterminé par la variable d'environnement \samp{PS1}.
		\end{theorie}
			
		\begin{Experience}{Le prompt} 
			Affichez la valeur de votre prompt. 
			Vous remarquerez qu'il contient des codes qui seront 
			remplacés par certaines valeurs. 
			Par exemple, \verb_\w_ indique le dossier courant.
		\end{Experience}
			
		\begin{Exercice}{Modifier le prompt} 
			Modifiez la valeur de votre prompt.
			Par exemple, modifiez l'invite en "Bonjour ! ".
		\end{Exercice}	

		\begin{Tutoriel}{Durée de vie de la variable} 
			\vspace{-1em}
			\begin{steps}
			\item Déconnectez-vous de linux1, puis reconnectez-vous.	
			\end{steps}
			Surprise, votre prompt a repris sa valeur d'origine ! 
			Pour rendre une modification permanente, 
			il faut ajouter la commande à votre fichier \verb@.bashrc@.
			C'est un fichier caché qui est exécuté 
			par votre shell lors de votre connexion.
        \end{Tutoriel}

        \subsection{Pour aller plus loin : le prompt}
        %=============================================
    
            \begin{Exercice}{Explorer les codes du prompt}
                Les possibilités de configuration du prompt sont nombreuses : 
                afficher l'heure, le nom de la machine, votre login,
                utiliser des couleurs\dots{}
                Examinez la documentation pour configurer le prompt comme il 
                vous plait (\verb@man bash@, section \verb@PROMPTING@).
                Rendez la modification permanente.
            \end{Exercice}	
            
\end{document}
