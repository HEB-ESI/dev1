\documentclass[a4paper,11pt]{style-esi/td}

\usepackage{style-esi/licence}
\usepackage{style-esi/exercice}
\usepackage{style-esi/exemple}
\usepackage{style-esi/question}
\usepackage{style-esi/tutoriel}
\usepackage{style-esi/listing}
\usepackage{style-esi/images}
\usepackage{style-dev1/dev1}

\begin{document}

\seance{5}{GNU/Linux (partie III)}
\entete
\titre
\ccbysa{esi-dev1-list@he2b.be}
\lastedit

\bigskip
\tableofcontents

\vfill
\begin{infobox}
    Dans votre répertoire \samp{\textasciitilde/dev1}, 
	créez un répertoire \samp{td5}. 
    Il contiendra tous les fichiers que vous allez créer aujourd'hui. 
\end{infobox}
\vfill

\newpage

%====================
\section{Les jokers}
%====================

	Le dossier \samp{/eCours/dev1/envl} contient plusieurs programmes \name{Java}.
	Imaginons qu'on veuille les copier tous chez nous. Comment faire ?
	\begin{enumerate}
	\item 
		On peut entrer une commande \samp{cp} par fichier à copier.
		C'est pénible.
	\item 
		La commande \samp{cp} accepte plusieurs noms de fichiers à copier
		pour autant que toutes les copies se fassent au même endroit.
		Ainsi on peut écrire : 
		\\\kbd{cp fichier1 fichier2 fichier3 dossierOùLesMettre}.
		\\C'est mieux mais encore pénible.
	\item
		S'il y a une certaine logique dans les fichiers à copier,
		on peut utiliser les \textbf{jokers}.
	\end{enumerate}

	\begin{theorie}{Les jokers}
		Dans un nom de fichier/dossier
		\begin{itemize}
			\item \og{}\samp{?}\fg{} remplace \textbf{un et un seul} caractère quelconque.
			\item \og{}\samp{*}\fg{} remplace \textbf{0, 1 ou plusieurs} caractères quelconques.
		\end{itemize}
	\end{theorie}

	\begin{Experience}{Copie de plusieurs fichiers}
		Utilisons les jokers pour facilement tout copier d'un coup.
		\begin{steps}
		\item 
			Si ce n'est pas encore fait, créez un dossier \samp{td4}
			et placez-vous dedans.
		\item 
			Entrez \kbd{cp /eCours/dev1/envl/*.java .}
			\\Que signifie le \og{}\samp{.}\fg{} (point) dans la commande ?
		\item 
			Affichez le contenu de \samp{td4}.
			Vous constatez que plusieurs fichiers y ont été copiés.
		\end{steps}
		Concrètement, 
		lorsqu'il voit un nom avec un joker,
		le shell le remplace par tous les fichiers/dossiers qui correspondent au motif.
		Ensuite il appelle la commande comme si tous les noms de fichiers/dossiers
		avaient été entrés explicitement.
	\end{Experience}
	
	\begin{Exercice}{Comprendre le \samp{*}}
		Quels sont les noms qui correspondent au motif \samp{*1*.java} ?
		\begin{selectmany}
		\item \samp{Ex10.java}
		\item \samp{Ex1.java}
		\item \samp{Ex10.class}
		\item \samp{1.java}		
		\end{selectmany}
	\end{Exercice}

	\begin{Exercice}{Comprendre le \samp{?}}
		Quels sont les noms qui correspondent au motif \samp{Ex?.java} ?
		\begin{selectmany}
		\item \samp{Ex10.java}
		\item \samp{Ex1.java}
		\item \samp{10.java}
		\item \samp{1.java}		
		\end{selectmany}
	\end{Exercice}

	\begin{Exercice}{Compiler en bloc}
		Sachant que la commande \samp{javac} 
		accepte plusieurs noms de fichiers à compiler,
		comment compiler facilement tous les fichiers \name{Java} 
		du dossier \samp{td4}.
	\end{Exercice}

%====================
\section{Recherche de fichiers}
%====================

	Vous commencez à avoir beaucoup de fichiers.
	Dans pareil cas, 
	il arrive souvent qu'on ne sache plus exactement où se trouve un fichier.
	Est-ce que le système peut nous aider à le trouver ?

	Les jokers peuvent parfois nous aider.
	Par exemple, si on sait que le fichier s'appelle \samp{Hello.java}
	et qu'il se trouve directement dans un des dossiers \samp{td\emph{i}},
	on peut le trouver avec 
	la commande \kbd{ls \textasciitilde/dev1/td*/Hello.java}.
	
	Dans des cas plus complexes, la commande \samp{find} vient à notre secours.

	\medskip
	\begin{coltbox}{\samp{find}}
		\kbd{find <oùChercher> <quoiChercher>}
		
		\medskip
		Permet de chercher des fichiers
		sur le disque dur en fonction de critères :
		le nom, la date de création, le propriétaire, la taille,
		les permissions\dots{}  {\color{colalert}mais pas le contenu !}
		Elle affiche la liste des fichiers trouvés.
		\begin{itemize}
		\item \og{}oùChercher\fg{} : le dossier dans lequel chercher ;
		\item \og{}quoiChercher\fg{} : le ou les critère(s) de recherche.
		\end{itemize}
	\end{coltbox}

	\medskip
	Exemple : \kbd{find \textasciitilde{} -name "*.java"}
	\\permet de trouver tous les fichiers \name{Java}%
	\footnote{
		Les guillements demande au shell de ne \textbf{pas}
		traiter le joker. Le motif est passé tel quel à la commande
		et c'est le \samp{find} qui va en tenir compte.
	} \emph{quelque part}
	dans son répertoire personnel.

	On vous laisse consulter la page de manuel pour trouver comment
	exprimer les autres critères.

	\begin{Exercice}{Recherche 1}
		Cherchez dans votre répertoire personnel,
		tous les fichiers ou dossiers qui sont vides
		(il n'y en a peut-être pas !).
	\end{Exercice}

	\begin{Exercice}{Recherche 2}
		Cherchez dans votre répertoire personnel,
		tous les fichiers qui ont une taille de plus de 512 octets.
	\end{Exercice}

	\begin{Exercice}{Recherche 3}
		Listez tous les répertoires personnels 
		qui sont composés d'exactement 3 caractères.
	\end{Exercice}

	\begin{Exercice}{Recherche 4}
		Listez tous vos fichiers qui ont été modifiés depuis moins de 24h.
	\end{Exercice}

\newpage

%====================
\section{Redirection}
%====================

	Imaginons qu'un programme \name{Java} produise un gros output.
	Est-il possible de le sauver dans un fichier pour le consulter tranquillement ?
	Oui !
	Ceci ,et bien d'autres choses encore, est possible grâce aux redirections.

	%===============================
	\subsection{Rediriger la sortie}
	%===============================

		\begin{Experience}{Rediriger la sortie d'un programme Java}
			Si ce n'est pas encore le cas,
			créez un dossier \samp{td4} et placez-vous dedans.
			\begin{steps}
			\item 
				\kbd{cp /eCours/dev1/envl/PasCroissant.java}
				pour copier un programme \name{Java} qui produit un output très long.
			\item 
				Examinez le code et essayez de comprendre ce qu'il fait.
			\item 
				\kbd{javac PasCroissant.java} pour le compiler.
			\item 
				\kbd{java PasCroissant} pour l'exécuter.
				Vous constatez que le résultat est très long.
			\item 
				\kbd{java PasCroissant > res} Plus rien n'est affiché !
			\item 
				\kbd{ls} pour constater qu'un fichier \samp{res} est apparu.
			\item 
				\kbd{cat res} 
				pour constater qu'il contient bien tout l'output du programme.
			\end{steps}
		\end{Experience}

		Ce qu'on vient de voir avec un programme \name{Java} est en fait
		vrai pour n'importe quel exécutable (les commandes aussi donc).

		\begin{theorie}{Rediriger la sortie}
			La \textbf{sortie standard} d'un exécutable est par défaut l'écran.
			\c Ca peut être changé. 
			\begin{itemize}
				\item \samp{> nomFichier} :
					à la suite d'une commande indique que les affichages
					se feront \textbf{dans le fichier} indiqué (qui est créé ou vidé avant)
				\item \samp{>{}> nomFichier} : 
					pour \textbf{ajouter} les affichages à la \textbf{fin} du fichier.
			\end{itemize}
		\end{theorie}

		\begin{alerttbox}{Transparent pour les exécutables}
			Comprenez-bien que ceci est \textbf{transparent}
			pour les commandes.
			Le programme \name{Java} n'a pas été modifié !
			Pour lui, \samp{System.out} représente la \emph{sortie standard}.
			Par défaut c'est l'écran mais la redirection 
			permet de rediriger vers un fichier.
		\end{alerttbox}

		\begin{Exercice}{Rediriger la sortie d'une commande}
			Sauvez dans un fichier toute l'arborescence (commande \samp{tree})
			de votre répertoire personnel.
		\end{Exercice}

	%===============================
	\subsection{Rediriger l'erreur}
	%===============================
	
		En \name{Java}, le code \samp{System.out.println()}
		est utilisé pour afficher quelque chose à l'écran.
		Il existe également le code \samp{System.err.println()}
		pour afficher un message d'erreur.
		Pourquoi fournir 2 techniques différentes si, au final,
		le résultat apparait au même endroit ?

		\begin{theorie}{Rediriger l'erreur}
			L'\textbf{erreur standard} d'un exécutable est par défaut l'écran.
			\c Ca peut être changé. 
			\begin{itemize}
				\item \samp{2> nomFichier} : 
				pour envoyer les \textbf{messages d'erreur} dans le fichier.
			\end{itemize}
		\end{theorie}

		Lorsqu'un programme affiche des messages d'erreurs,
		ils sont mélangés aux affichages normaux.
		Une redirection permet de les séparer pour mieux les voir.

		Ceci explique pourquoi c'est une bonne pratique en \name{Java}
		d'utiliser \samp{System.err} lorsque vous affichez un message d'erreur.

		\begin{Exercice}{Recherche}
			Si on utilise la commande \samp{find} pour chercher un fichier,
			il est possible que la commande tente d'accéder à des sous-dossiers
			pour lesquels vous n'avez pas les droits%
			\footnote{%
				Un prochain TD sera consacré aux permissions sous \name{GNU/Linux}.
			}.
			Dans ce cas, elle affiche un message d'erreur.

			Affichez à l'écran la liste des fichiers \samp{Hello.java}
			qui existent quelque part dans le système de fichier
			en enlevant les messages d'erreurs.
		\end{Exercice}

	%===============================
	\subsection{Rediriger l'entrée}
	%===============================

		Tout comme on peut rediriger la sortie,
		on peut rediriger l'entrée.

		\begin{theorie}{Rediriger l'entrée}
			L'\textbf{entrée standard} d'un exécutable est par défaut le clavier.
			\c Ca peut être changé. 
			\begin{itemize}
			\item \samp{< nomFichier} : 
				pour indiquer que les lectures ne se font pas au clavier
				mais \textbf{dans le fichier} (qui doit exister).
			\end{itemize}
		\end{theorie}

		\begin{Experience}{Rediriger l'entrée d'un programme Java}
			\begin{steps}
			\item 
				\kbd{cp /eCours/dev1/envl/Multiple4.java .}
				pour copier un programme \name{Java} qui lit des nombres
				au clavier (plus précisément, sur l'entrée standard).
			\item 
				Examiner le code et essayez de comprendre ce qu'il fait.
			\item 
				\kbd{javac Multiple4.java} pour le compiler.
			\item 
				\kbd{java Multiple4} pour l'exécuter.
				Il vous demande d'entrer des entiers et affiche
				uniquement ceux qui sont multiples de 4.
			\item 
				Pour indiquer qu'il n'y a plus de données,
				il faut appuyez sur \samp{CTRL-D}
				(et pas \samp{CTRL-C} qui tue brutalement le processus).
			\end{steps}
			Si on veut exécuter à nouveau le programme pour les mêmes valeurs,
			il faudra retaper les nombres ? Non !
			\begin{steps}
			\item 
				Créez un fichier \samp{data} avec quelques nombres.
			\item 
				\kbd{java Multiple4 <data} pour exécuter
				le programme en demandant que les lectures se fassent
				dans le fichier indiqué et pas au clavier.
			\end{steps}
		\end{Experience}	

		\medskip
		\begin{alerttbox}{Transparent pour les exécutables}
			À nouveau, ceci est \textbf{transparent} pour les commandes.
			C'est toujours \samp{System.in} (l'entrée standard) 
			qui est donné à \samp{Scanner}.
		\end{alerttbox}

		\begin{Exercice}{Pas croissant multiple de 4 }
			Produisez un fichier \samp{croissant4}
			qui contient parmi les 1000 premiers nombres de la suite
			des pas croissants (produite par le programme \samp{PasCroissant})
			uniquement ceux qui sont des multiples de 4.
			\\Aide: la sortie de l'un devient l'entrée de l'autre.
		\end{Exercice}

		\begin{Exercice}{cat et ses variantes}
			La commande \samp{cat} est plus riche que ce que vous avez vu.
			\begin{enumerate}
			\item 
				Faites l'expérience de taper juste \samp{cat}.
				Entrez quelques phrases puis appuyez sur \samp{CTRL-D}.
				En vous aidant du manuel, comprenez ce qui se passe.
			\item 
				Entrez maintenant \samp{cat <fichier} 
				où \samp{fichier} est le nom d'un fichier qui existe.
				Que se passe-t'il et pourquoi ?
			\item 
				Comment pouvez-vous utiliser la commande \samp{cat}
				pour créer un fichier contentant juste le mot \samp{Bonjour}
				(sans utiliser un éditeur donc) ?
			\item 
				Comment pouvez-vous utiliser la commande \samp{cat}
				pour copier un fichier ?
			\item 
				Comment pouvez-vous utiliser la commande \samp{cat}
				pour fusionner le contenu de deux fichiers ?
			\end{enumerate}
		\end{Exercice}

	%%%%%%%%%%%%%%%%%%%%%%%%%%%%%%%%%%%%%%%%
    \subsection{Les tubes (pipes en anglais)}
 	%%%%%%%%%%%%%%%%%%%%%%%%%%%%%%%%%%%%%%%%

		Pour résoudre l'exercice 
		qui produit les pas croissants multiples de 4 de la section précédente, 
		vous avez dû créer un fichier temporaire qui n'a servi qu'à \c ca. 
		Vous avez redirigé la sortie de la première commande dans un fichier 
		et redirigé ce fichier comme entrée de la seconde commande.
		Les tubes permettent de simplifier le processus.
						
		\begin{theorie}{Tubes}
			\samp{commande1 | commande2}

			\medskip
			Un \emph{tube} (ou \emph{pipe}) lie 2 commandes :
			la sortie de la première devient l'entrée de la seconde.
		\end{theorie}
			
		\begin{Exemple}{Enchaîner les commandes} 
			La commande \verb_less_ permet d'afficher un message page par page.
			Si on ne lui donne pas de nom de fichier, il pagine les données reçues sur l'entrée standard. 
			On peut donc remplacer
					
			\begin{Console}
				ls /home >temp
				less temp
			\end{Console}
			par					
			\begin{Console}
				ls /home | less
			\end{Console}	
		\end{Exemple}
			
		\begin{Exercice}{Utilisation des tubes}
			Utilisez un tube pour refaire l'exercice précédent
			(afficher parmi les 1000 premiers nombres de la suite des pas croissants,
			tous ceux qui sont des multiples de 4).
		\end{Exercice}

%===================
\section{Conclusion}
%====================

	\begin{theorie}{Notions importantes de ce TD}
		Voici les notions importantes que vous devez avoir assimilées à la fin de ce TD.
		\begin{itemize}
		\item Comprendre et savoir utiliser les jokers (\samp{*} et \samp{?}).
		\item Savoir utiliser la commande \samp{find} pour rechercher des fichiers.
			Pouvoir trouver dans la page de manuel la syntaxe pour un critère de recherche
			qui n'a pas été testé au laboratoire.
		\item Comprendre et savoir utiliser à bon escient les redirections et les tubes.
		\end{itemize}
	\end{theorie}

	\subsubsection*{Pour aller plus loin\dots}
	Ceux qui ont du temps peuvent aborder les exercices suivants.

	\begin{Exercice}{Traiter les fichiers trouvés}
		Comment afficher le \textbf{contenu} de tous vos fichiers \name{Java} ?
		Aide : consultez la section \samp{ACTIONS} de la page de manuel
		de \samp{find}.
	\end{Exercice}

\end{document}

	
