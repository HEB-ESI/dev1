\documentclass[a4paper,11pt]{style-esi/td}

\usepackage{style-esi/licence}
\usepackage{style-esi/exercice}
\usepackage{style-esi/exemple}
\usepackage{style-esi/question}
\usepackage{style-esi/tutoriel}
\usepackage{style-esi/listing}
\usepackage{style-esi/images}
\usepackage{style-dev1/dev1}
\usepackage{longtable} % Pour des tables sur plusieurs pages

\begin{document}

\seance{3}{Java sur linux}
\entete
\titre
\ccbysa{esi-dev1-list@he2b.be}
\lastedit

\bigskip
\begin{abstract}
    Lorsqu'on programme en \name{Java} sur \name{Linux},
    on peut, comme sur \name{Windows},
    utiliser un environnement de développement comme \name{Netbeans}.
    Mais on peut aussi tout faire en mode console, 
    c'est ce que nous allons voir ici.
    Nous en profiterons pour apprendre de nouvelles notions liées à Linux.
\end{abstract}

\bigskip
\tableofcontents

\vfill
\begin{infobox}
    Dans votre répertoire \samp{\textasciitilde/dev1}, 
    créez un répertoire \samp{td3}. 
    Il contiendra tous les fichiers que vous allez créer aujourd'hui. 
\end{infobox}

\newpage

%============================================================
\section{Quelques notions supplémentaires de Linux}
%============================================================

	Avant d'attaquer le c\oe{}ur du TD, 
	à savoir le compilation et l'exécution de programmes Java,
	voyons quelques éléments de Linux qui vous seront utiles.

	%============================================================
	\subsection{Les fichiers cachés}
	%============================================================

		\begin{theorie}{Les fichiers cachés}
			Un \textbf{fichier caché}
			est un fichier
			dont le nom commence par \og{}\samp{.}\fg{} (un point).
			\\
			Idem pour un dossier.
		\end{theorie}

		Par défaut, 
		les fichiers cachés ne sont pas montrés par la commande \kbd{ls}.
		Pour les voir, utiliser l'option \texttt{a} : \kbd{ls -a}.
		C'est surtout utilisé pour des fichiers de configuration.

		\begin{Exercice}{Les fichiers cachés de la home}
			Regardez s'il existe des fichiers cachés
			dans votre répertoire personnel.
		\end{Exercice}

%=====================================
\section{L'éditeur nano}
%======================================

	Un \textbf{éditeur de texte} (ou, plus court, un \textbf{éditeur}) 
	est un programme qui vous permet d'entrer, modifier
	le contenu d'un fichier texte. 
	Vous connaissez probablement \name{Notepad} sous \name{Windows}. 
	Sur \name{Linux}, il en existe beaucoup. 
	Celui que nous vous proposons dans un premier temps au laboratoire 
	s'appelle \name{nano}. 
	Il est assez facile pour débuter et faire des choses simples. 
	Par la suite, vous pourrez choisir de rester sous \name{nano} 
	ou d'utiliser \name{vim}.

	\subsection{Premiers pas}
	%=================================

	\begin{Tutoriel}{Premiers pas avec nano}
	\vspace{-1em}
	\begin{steps}
	\item 
		En préambule, tapez la commande \kbd{cd} pour revenir dans \textit{votre home}.
	\item 
		Tapez \kbd{nano test} pour commencer à éditer le fichier \samp{test} 
		(comme il n'existe pas encore, il est créé).
	
		Une fenêtre s'ouvre. 
		Vous voyez qu'elle est scindée en 2 parties : 
		la partie supérieure où vous écrivez votre texte 
		et la partie inférieure où sont indiquées 
		les différentes commandes (le \char`\^{} représente la touche \verb|Ctrl|)
	\item 
		Entrez quelques mots.
	\item 
		Appuyez sur la combinaison de touches \verb|Ctrl X|, 
		confirmez que vous voulez sauver vos modifications et sortez.
	
		Vous êtes maintenant revenu à l'invite de commande.
	\item 
		Tapez à présent la commande \kbd{ls}.
		Vous pouvez constater que le fichier \verb_test_ est apparu dans la liste ;)
	\end{steps}
	\end{Tutoriel}

	Pour aller plus loin dans l'apprentissage de cet éditeur,
	consultez ce document :\\
	{\tiny\url{http://fr.openclassrooms.com/informatique/cours/reprenez-le-controle-a-l-aide-de-linux/nano-l-editeur-de-texte-du-debutant}}.

	\subsection{Coloration syntaxique}
	%=================================

		La coloration syntaxique signifie 
		utiliser des couleurs pour mettre en évidence certaines parties d'un code :
		mots clés, constantes\dots{}
	
		Pour utiliser cette facilité, 
		il faut configurer \name{nano}. 
		Cette configuration se fait dans le fichier \samp{\textasciitilde{}/.nanorc}.

		\begin{Tutoriel}{Introduire la coloration sytaxique} 
			\vspace{-1em}
			\begin{steps}
			\item 
				Tapez \kbd{nano \textasciitilde/.nanorc} 
				pour éditer le fichier de configuration de \emph{nano}.
			\item 
				Ajoutez-y la ligne : \samp{include "/usr/share/nano/java.nanorc"}
			\item 
				Quittez l'éditeur.
			\item 
				Ouvrez le fichier \samp{Ex.java};
				il devrait être coloré.
			\end{steps}
		\end{Tutoriel}
				
	\subsection{Numérotation des lignes}
	%=================================
				
		Nano peut également indiquer le numéro de la ligne sur laquelle se trouve le curseur, 
		ce qui sera pratique pour corriger vos erreurs. 
		Pour cela, il existe deux méthodes.

		\begin{itemize} 
		\item \textbf{Méthode 1} : 
			Lancer nano avec l'option \samp{-c} : \kbd{nano -c monFichier}.
		\item \textbf{Méthode 2} : 
			Dans l'éditeur, appuyez sur \samp{CTRL-c}.
		\end{itemize}
			
	\subsection{Indentation}
	%=================================
				 
		Pour qu'un programme soit lisible, 
		il doit être \textit{indenté}. 
		Ce qui serait pratique lorsqu'on code ce serait qu'un retour à la ligne 
		positionne automatiquement le curseur de façon à être aligné avec la ligne précédente. 
		Pour que nano fasse ça pour nous, 
		il suffit d'ajouter	ceci à son fichier de configuration : \samp{set autoindent}.
			
	\subsection{Autres configurations} 
	%=================================
		
		Si vous désirez connaitre d'autres possibilités de configuration de nano, 
		vous pouvez lire le manuel : \kbd{man nanorc}.

		Pour une \textit{quick ref} en ligne, consultez (par exemple) : \\
		\url{www.codexpedia.com/text-editor/nano-text-editor-command-cheatsheet/}
		.

%=============================
\section{Java en mode console}
%==============================

	Avec Netbeans, lorsque vous appuyez sur des boutons
	(comme celui demandant d'exécuter le projet),
	cela lancent des commandes qui effectuent la tâche.
	Il est possible d'exécuter directement ces commandes.
	C'est ce que nous allons faire ici.

	%=============================
	\subsection{Compiler et exécuter}
	%==============================

		\begin{theorie}{Java sans package}
			Si le programme Java n'utilise pas de package :
			\begin{itemize}
				\item \kbd{javac MaClasse.java} compile le programme Java du fichier donné
				\item \kbd{java MaClasse} exécute le programme Java se trouvant dans la classe donnée.
			\end{itemize}
		\end{theorie}

		\begin{Tutoriel}{Compiler / exécuter un programme} 
			Commençons par un programme correct et tentons de l'exécuter.  
		
			Le fichier \samp{/eCours/dev1/envl/Hello.java} 
			contient un petit programme Java tout simple qui affiche un message de bienvenue. 
			
			\begin{steps}	
			\item 
				Si ce n'est pas le cas, placez-vous dans le dossier \samp{td3}.
			\item 
				Copiez le fichier indiqué dans cd dossier :
				\kbd{cp /eCours/dev1/envl/Hello.java .}
			\item 
				Lisez-le et voyez si vous devinez ce qu'il fait : \kbd{cat Hello.java}
			\item 
				Compilez-le : \kbd{java Hello.java}
			\item 
				Que fait cette phase, à quoi sert-elle ?
				Affichez le contenu du dossier pour le vérifier.
			\item Exécutez le programme : \kbd{java Hello}
			\end{steps}			
		\end{Tutoriel}

		% Lorsqu'on compile, il faut mettre le \samp{.java} 
		% mais lorsqu'on exécute, 
		% il ne faut pas mettre le \samp{.class}.

		\begin{alerttbox}{Retenez !} 
			On \textbf{compile un fichier} mais on \textbf{exécute une classe}.
		\end{alerttbox}

		Vous allez à présent écrire votre premier programme de bout en bout sur linux1. 
		
		\begin{Exercice}{Un programme Java}
			Écrivez un programme qui :
			\begin{itemize}
			\item Affiche votre nom ;
			\item Demande un nombre à l'utilisateur ;
			\item Affiche l'inverse du nombre (par ex: $0.25$ si le nombre entré était $4$).
			\end{itemize}
		\end{Exercice}

		% \begin{alerttbox}{Conseils}
		% 	Procédez par étape et vérifiez correctement votre programme.
		% 	Ce n'est pas parce qu'il compile qu'il est correct.
		% 	Et ce n'est pas parce qu'il affiche quelque chose que c'est la bonne réponse.
		% \end{alerttbox}

		\begin{Exercice}{Comprendre les erreurs (I)}
			Supposons que vous faites une erreur dans votre programme.
			Par exemple en écrivant \code{java}{Public} 
			au lieu de \code{java}{public}.

			À quelle étape le problème va-t'il apparaitre ?
			Sous quelle forme ? Testez !
		\end{Exercice}

		\begin{Exercice}{Comprendre les erreurs (II)}
			Supposons que vous demandez un calcul impossible dans votre programme.
			Par exemple \code{java}{1/0}.

			À quelle étape le problème va-t'il apparaitre ?
			Sous quelle forme ? Testez !
		\end{Exercice}

	%=============================
	\subsection{Produire la Javadoc}
	%==============================

		\begin{theorie}{Javadoc}
			\kbd{javadoc -doc dossierDoc MaClasse.java}
			génère la javadoc présente dans le fichier \texttt{MaClasse.java}
			dans le dossier \texttt{dossierDoc} (qui doit exister).
		\end{theorie}

		\begin{Exercice}{Javadoc de Hello}
			Produisez la javadoc du programme \samp{Hello} précédent.
		\end{Exercice}

		\subsubsection*{Visualiser le résultat}
		%--------------------------------------

			La javadoc est au format \name{HTML} et peut être consulté
			avec un simple navigateur web mais il n'y en a pas sur \samp{linux1}%
			\footnote{%
				Un Linux avec environnement graphique 
				dispose bien sûr d'un navigateur web.
			}.

			\begin{todo}
				Aborder le \samp{public\_html}.
			\end{todo}

			
%=================================
\section{Transfert de fichiers}  
%=================================

	Vous n'avez peut-être pas fini. Pour pouvoir continuer à la maison sans tout recommencer, 
    il serait bon de pouvoir récupérer ce que vous avez déjà fait sur linux1.
	Voici une façon de le faire.			
				
	\begin{Tutoriel}{Transférer des fichiers}
	\begin{steps}		
		\item 
			Ouvrez l'explorateur de fichier Windows (par exemple en cliquant sur l'icône "My Computer").
		\item 
			Dans le champ d'adresse, tapez l'adresse \samp{ftp://linux1}.
		\item 
			Une boite de dialogue vous demande votre login et mot de passe (sur \textit{linux1}).
		\item 
			Vous voyez apparaitre votre dossier personnel sur \textit{linux1}. 
		\item 
			Vous pouvez y prendre/déposer des fichiers comme vous le feriez pour un dossier Windows. 
			Vous pouvez par exemple les mettre sur une \textbf{clé USB},
			sur le cloud ou vous les envoyer par mail.
	\end{steps}
	\end{Tutoriel}			

%===================
\section{Conclusion}
%====================

	\subsection{Ce qu'il faut connaitre}
	%===================================

		\begin{theorie}{Notions importantes de ce TD}
			Voici les notions importantes que vous devez avoir assimilées à la fin de ce TD.
			\begin{itemize}
			\item 
				La notion de fichier caché et comment les voir.
			\item 
				Savoir utiliser \samp{nano} pour éditer un petit programme Java.
			\item 
				Savoir compiler et exécuter un programme Java 
				qui n'utilise pas de package.
			\item 
				Savoir créer/consulter/modifier une variable d'environnement.
			\end{itemize}
		\end{theorie}


		
\end{document}
			