\documentclass[a4paper,11pt]{style-esi/td}

\usepackage{style-esi/licence}
\usepackage{style-esi/exercice}
\usepackage{style-esi/exemple}
\usepackage{style-esi/question}
\usepackage{style-esi/tutoriel}
\usepackage{style-esi/listing}
\usepackage{style-esi/images}
\usepackage{style-dev1/dev1}

\begin{document}

\seance{5}{Linux (partie III)}
\entete
\titre
\ccbysa{esi-dev1-list@he2b.be}
\lastedit

\bigskip
\tableofcontents

\vfill
\begin{infobox}
    Dans votre répertoire \samp{\textasciitilde/dev1}, 
	créez un répertoire \samp{td6}. 
    Il contiendra tous les fichiers que vous allez créer aujourd'hui. 
\end{infobox}
\vfill

\newpage

%===================
\section{Grep - Recherche dans un fichier}
%====================

	On a déjà vu la commande \samp{find}
	qui permet de rechercher un fichier en fonction de critères.
	La commande \samp{grep} permet de faire une recherche 
	\textbf{à l'intérieur} d'un fichier.

	\begin{theorie}{Grep}
		\kbd{grep motif fichiers...}

		\medskip
		Dans son utilisation la plus simple, 
		permet d’extraire de fichiers 
		toutes les lignes qui contiennent un certain texte 
		(appelé motif).
	\end{theorie}

	\begin{infotbox}{Penser aux jokers}
		Comme à chaque fois qu'une commande peut reçevoir plusieurs noms de fichiers,
		on peut les indiquer explicitement et/ou utiliser les \emph{jokers}
		pour en désigner plusieurs d'un coup.
	\end{infotbox}

	\begin{Exemple}{Où ai-je déjà utiliser Scanner ?}
		La commande \kbd{grep Scanner *.java} montre toutes les lignes
		contenant le mot \samp{Scanner}
		dans tous les programmes Java du dossier courant.
	\end{Exemple}

	Consultez la page de manuel pour voir comment paramétrer la commande.

%===================
\section{Les filtres}
%====================

	\subsection{Présentation}
	%========================

		La commande \samp{grep}
		peut rechercher dans un fichier donné en paramètre
		mais aussi rechercher dans du texte donné en entrée.

		\begin{Exemple}{grep comme un filtre}
			\kbd{cat Hello.java | grep Scanner}
			
			\medskip
			Avec cette écriture, le contenu du fichier \samp{Hello.java}
			est envoyé à la commande \samp{grep} qui n'affichera
			que les lignes contenant le mot \samp{Scanner}.
		\end{Exemple}

		On peut voir la commande comme un \emph{filtre}.

		\begin{theorie}{Filtres - Définition}
			Les filtres sont des commandes qui lisent sur l'entrée standard
			et affichent sur la sortie standard 
			une version réduite/modifiée de ce qui a été lu.
		\end{theorie}
	
		Il y a beaucoup d'autres commandes qui agissent comme des filtres.
		C'est à la base même de Linux : fournir des commandes qui font peu 
		mais qui le font bien et les combiner (avec des tubes)
		pour obtenir un résultat conséquent.

\newpage

		\begin{Exemple}{Combiner les filtres}
			\kbd{cat Hello.java | grep Scanner | grep System}
			
			\medskip
			Produit sur la sortie standard toutes les lignes du fichier
			\samp{Hello.java} qui contiennent \textbf{à la fois}
			les mots \samp{Scanner} et \samp{System}.
		\end{Exemple}

	\subsection{Nourrir les filtres}
	%==================
	
		Les filtres traitent les données reçues.
		Quelles peuvent être ces données ? N'importe quoi.
		Nous avons vu que ça peut être le contenu d'un fichier
		mais ça peut aussi être le résultat d'une commande quelconque.
		Afin d'enrichir nos exemples, voyons 2 commandes qui produisent
		des données.

		\begin{theorie}{Commandes pour nourrir les filtres}
			\begin{itemize}
			\item 
				\kbd{du nomDossier} : 
				donne l'espace disque occupé par chaque sous-dossier
				du dossier indiqué.
			\item 
				\kbd{who} : 
				donne la liste des connexions à la machine.
			\end{itemize}
		\end{theorie}

	\subsection{Trier les lignes}
	%==================

		La commande \samp{sort} trie les lignes reçues.

		\begin{Experience}{Trier les lignes}
			Expérimentons la commande \samp{sort}.
			\begin{steps}
			\item 
				Placez-vous dans votre dossier \samp{dev1}.
			\item 
				\kbd{du .} 
				affiche les dossiers et leur taille
			\item 
				\kbd{du . | sort -n}
				affiche la même liste mais triée (numériquement sur la taille).
			\end{steps}
			Consultez la page de manuel pour toutes les options.
		\end{Experience}

		\begin{Exercice}{Trier le résultat de du}
			Reprenez l'exemple donné ci-dessus et voyez comment trier
			la liste des sous-dossiers et leur taille :
			\begin{enumerate}
			\item En ordre inverse de la taille (le plus gros d'abord).
			\item Sur le nom du dossier.
			\end{enumerate}
		\end{Exercice}

	\subsection{Enlever les doublons}
	%==================

		La commande \samp{uniq} ne laisse passer qu'une seule fois
		les lignes identiques \textbf{qui se suivent}.

		\begin{Experience}{Enlever les doublons}
			TODO
		\end{Experience}

		\begin{Exercice}{Les utilisations de Scanner}
			Affichez toutes les lignes \textbf{différentes}
			contenant le mot \samp{Scanner} dans tous vos fichiers Java
			du dossier \samp{td4}.
			\\Aide: 
			Si vous devez enchainer plusieurs filtres,
			vous pouvez tester, étape par étape.
		\end{Exercice}

	\subsection{Les premières/dernières lignes}
	%==================
	
		La commande \samp{head} (respectivement \samp{tail}
		ne laisse passer que les premières (respectivement dernières)
		lignes reçues.

		\begin{Experience}{Les premières/dernières lignes}
			TODO
		\end{Experience}

		\begin{Exercice}{Les gros dossiers}
			Affichez le nom et la taille des 3 dossiers 
			qui prennent le plus d'espace disque 
			parmi tous vos sous-dossiers de \samp{dev1}. 
		\end{Exercice}

	\subsection{Transformer des caractères}
	%======================================
	
		La commande \samp{tr} permet de faire des transformation du texte :
		remplacez certains caractères par d'autres ou simplifier 
		plusieurs occurrences consécutives d'un caractère 

		\begin{Experience}{Transformer des caractères}
			TODO
		\end{Experience}

		Elle est particulièrement utile pour la commande qui suit.

	\subsection{Ne garder que certaines colonnes}
	%==================
	
		La commande \samp{cut} sert à ne garder que certaines colonnes 
		parmi les lignes reçues.

		\begin{Experience}{Les premières/dernières lignes}
			TODO
		\end{Experience}

		\begin{Exercice}{Les gros dossiers}
			Affichez le nom et la taille des 3 dossiers 
			qui prennent le plus d'espace disque 
			parmi tous vos sous-dossiers de \samp{dev1}. 
		\end{Exercice}

	\subsection{Compter}
	%==================

		La commande \samp{wc} est un peu particulière.
		Elle compte le nombre de lignes/mots/caractères reçus.

		TODO

	\subsection{Résumé}
	%==================

		\begin{theorie}{Filtres}
			Voici les quelques filtres que nous avons appris à utiliser·
			\begin{itemize}
				\item \samp{grep} : ne laisse passer que certaines lignes.
				\item \samp{sort} : trie les lignes
				\item \samp{uniq} : enlève les lignes en double
				\item \samp{head} : ne laisse passer que les premières lignes
				\item \samp{tail} : ne laisse passer que les dernières lignes
				\item \samp{cut}  : ne laisse passer que certaines colonnes
				\item \samp{tr} : modifie certains (groupes de) caractères
				\item \samp{wc} : compte les lignes, les mots et les caractères
			\end{itemize}
		\end{theorie}
		
	
%===================
\section{Conclusion}
%====================

	\begin{theorie}{Notions importantes de ce TD}
		Voici les notions importantes que vous devez avoir assimilées à la fin de ce TD.
		\begin{itemize}
		\item Comprendre le concept de filtre.
		\item Savoir utiliser et enchainer les bons filtres pour répondre
			aux questions qu'on se pose.
		\end{itemize}
	\end{theorie}

	\subsubsection*{Pour aller plus loin\dots}
	Ceux qui ont du temps peuvent aborder les exercices suivants.
	(expressions régulières dans le grep)
	
\end{document}

	
