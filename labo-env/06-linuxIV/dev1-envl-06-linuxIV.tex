\documentclass[a4paper,11pt]{style-esi/td}
%\documentclass[a4paper,11pt,draft]{style-esi/td}

\usepackage{style-esi/licence}
\usepackage{style-esi/exercice}
\usepackage{style-esi/exemple}
\usepackage{style-esi/question}
\usepackage{style-esi/tutoriel}
\usepackage{style-esi/listing}
\usepackage{style-esi/images}
\usepackage{style-dev1/dev1}

\begin{document}

\seance{6}{GNU/Linux (partie IV)}
\entete
\titre
\ccbysa{esi-dev1-list@he2b.be}
\lastedit

\bigskip
\tableofcontents

\vfill
\begin{infobox}
    Dans votre répertoire \samp{\textasciitilde/dev1}, 
	créez un répertoire \samp{td6}. 
    Il contiendra tous les fichiers que vous allez créer aujourd'hui. 
\end{infobox}
\vfill

\newpage

%===================
\section{grep : recherche dans un fichier}
%====================

	On a déjà vu la commande \samp{find}
	qui permet de rechercher un fichier en fonction de critères.
	La commande \samp{grep} permet de faire une recherche 
	\emph{à l'intérieur} d'un fichier.

	\begin{theorie}{Grep}
		\kbd{grep motif fichiers...}

		\medskip
		Dans son utilisation la plus simple, 
		permet d’extraire de fichiers 
		toutes les lignes qui contiennent un certain texte 
		(appelé motif).
	\end{theorie}

	\begin{infotbox}{Penser aux jokers}
		Comme à chaque fois qu'une commande peut recevoir plusieurs noms de fichiers,
		on peut les indiquer explicitement et/ou utiliser les \emph{jokers}
		pour en désigner plusieurs d'un coup.
	\end{infotbox}

	\begin{Exemple}{Où ai-je déjà utiliser Scanner ?}
		La commande \kbd{grep Scanner *.java} montre toutes les lignes
		contenant le mot \samp{Scanner}
		dans tous les programmes \name{Java} du dossier courant.
	\end{Exemple}

	% Consultez la page de manuel pour voir comment paramétrer la commande.

%===================
\section{Filtres : présentation}
%====================

	%\subsection{Présentation}
	%========================

		La commande \samp{grep}
		peut rechercher dans un fichier donné en paramètre
		mais aussi rechercher dans du texte donné en entrée.

		\begin{Exemple}{grep comme un filtre}
			\kbd{cat -n Hello.java | grep Scanner}
			
			\medskip
			Avec cette écriture, le contenu du fichier \samp{Hello.java}
			est envoyé à la commande \samp{grep} qui n'affichera
			que les lignes (avec leur numéro, grâce à l'option \kbd{-n}) contenant le mot \samp{Scanner}.
		\end{Exemple}

		On peut voir la commande comme un \emph{filtre}. Le symbole \kbd{|} est appelé \textit{pipe} (ou \textit{tube}, en français).

		\begin{theorie}{Filtres - Définition}
			Les filtres sont des commandes qui lisent sur l'entrée standard
			et affichent sur la sortie standard 
			une version réduite/modifiée de ce qui a été lu.
		\end{theorie}
	
		Il y a beaucoup d'autres commandes qui agissent comme des filtres.
		C'est à la base même de \name{GNU/Linux} : fournir des commandes qui font peu 
		mais qui le font bien et les combiner (avec des tubes)
		pour obtenir un résultat conséquent%
		\footnote{%
			C'est ce qu'on appelle le principe \name{KISS}, 
			\emph{Keep it simple, stupid !}
		}.

\newpage

		\begin{Exemple}{Combiner les filtres}
			\kbd{cat -n Hello.java | grep Scanner | grep System}
			
			\medskip
			Produit sur la sortie standard toutes les lignes (avec leur numéro) du fichier
			\samp{Hello.java} qui contiennent \emph{à la fois}
			les mots \samp{Scanner} et \samp{System}.
		\end{Exemple}

	\section{Nourrir les filtres}
	%==================
	
		Les filtres traitent les données reçues.
		Quelles peuvent être ces données ? N'importe quoi.
		Nous avons vu que ça peut être le contenu d'un fichier (par exemple un fichier log)
		mais ça peut aussi être le résultat d'une commande quelconque.
		Afin d'enrichir nos exemples et exercices, 
		voyons 2 commandes qui produisent des données.

		\begin{theorie}{Commandes pour nourrir les filtres}
			\begin{itemize}
			\item 
				\kbd{du nomDossier} : 
				donne l'espace disque occupé par chaque sous-dossier
				du dossier indiqué.
			\item 
				\kbd{who} : 
				donne la liste des connexions à la machine.
			\end{itemize}
		\end{theorie}

	\section{sort : trier les lignes}
	%==================

		La commande \samp{sort} trie les lignes reçues.

		\begin{Experience}{Trier les lignes}
			Expérimentons la commande \samp{sort}.
			\begin{steps}
			\item 
				Placez-vous dans votre dossier \samp{dev1}.
			\item 
				\kbd{du .} 
				affiche les dossiers et leur taille
			\item 
				\kbd{du . | sort -n}
				affiche la même liste mais triée (numériquement sur la taille).
			\end{steps}
			Consultez la page de manuel pour toutes les options.
		\end{Experience}

		\begin{Exercice}{Trier le résultat de l'occupation disque}
			Reprenez l'exemple donné ci-dessus et voyez comment trier
			la liste des sous-dossiers et leur taille :
			\begin{enumerate}
			\item En ordre inverse de la taille (le plus gros d'abord).
			\item Sur le nom du dossier.
			\end{enumerate}
		\end{Exercice}

	\section{uniq : enlever les doublons}
	%==================

		La commande \samp{uniq} ne laisse passer qu'une seule fois
		les lignes identiques \emph{qui se suivent}.

		\begin{Experience}{Enlever les doublons}
			\begin{steps}
			\item 
				Exécutez \kbd{uniq}.
				La commande est en attente de texte et le reproduit tel quel.
			\item 
				Faites-en l'expérience en entrant \kbd{salut} puis, à la ligne,
				\kbd{hello}.
			\item 
				Entrez à nouveau \kbd{hello}.
				Il n'est \emph{pas reproduit sur la sortie} !
				La ligne précédente étant identique.
			\item 
				Entrez à nouveau  \kbd{salut} puis, à la ligne, \kbd{hello}.
				Ils sont reproduits car la ligne précédente est à chaque fois
				différente.
			\item 
				Rappel : \samp{CTRL-D} pour terminer l'expérience.
			\end{steps}
		\end{Experience}

		\begin{Exercice}{Les utilisations de Scanner}
			Affichez toutes les lignes \emph{différentes}
			contenant le mot \samp{Scanner} dans tous vos fichiers \name{Java}
			du dossier \samp{td4}.
			\\Aide: 
			Si vous devez enchainer plusieurs filtres,
			vous pouvez les tester étape par étape.
		\end{Exercice}

	\section{head / tail : les premières / dernières lignes}
	%==================
	
		La commande \samp{head} (respectivement \samp{tail})
		ne laisse passer que les premières (res\-pec\-ti\-ve\-ment 
		dernières)
		lignes reçues.

		\begin{Experience}{Les premières/dernières lignes}
			\kbd{ls *.java | head -5}
			pour afficher le nom de 5 fichiers \name{Java} de votre dossier courant.\footnote{Nous utilisons ici et dans la suite la commande \samp{ls} comme exemple simple. Cependant, filtrer l'affichage de \samp{ls} n'est généralement pas une bonne idée, par exemple parce qu'un nom de fichier pourrait contenir un retour à la ligne. Voir à ce sujet: \url{https://mywiki.wooledge.org/ParsingLs}.}
                        %Dans le cas présent, on pourrait proposer \kbd{files=(*.java); printf \textquotesingle\%s\textbackslash n\textquotesingle{} "\$\{files[@]:0:5\}"}
                        Cela fonctionne-t-il ?
                        Essayez maintenant \kbd{ls *.java} sans filtre. Combien de lignes s'affichent ?

                        À retenir : certaines commandes changent de comportement lorsqu'elles sont insérées dans un pipeline.
		\end{Experience}

		\begin{Exercice}{Les gros dossiers}
			Affichez le nom et la taille des 3 dossiers 
			qui prennent le plus d'espace disque 
			parmi tous vos sous-dossiers de \samp{dev1}. 
		\end{Exercice}

	\section{tr : transformer des caractères}
	%======================================
	
		La commande \samp{tr} permet de faire des transformation du texte :
		remplacez certains caractères par d'autres, en supprimer ou simplifier 
		plusieurs occurrences consécutives d'un caractère 

		\begin{Exemple}{Transformer des caractères}
			Voici quelques utilisations possibles :
			\begin{itemize}
			\item 
				\kbd{ls | tr "jv" "gf"} 
				transforme tous les \samp{j} en \samp{g}
				et les \samp{v} en \samp{f}.
			\item
				\kbd{ls | tr -d "a"} 
				supprime tous les \samp{a}.
			\item 
				\kbd{who | tr -s " "}
				remplace toute suite de plusieurs espaces par un seul.
				Comparez le résultat à celui produit par \samp{who} tout seul.
			\end{itemize}
			Consultez le manuel pour bien comprendre les exemples
			et pouvoir aller plus loin.
		\end{Exemple}

		Elle est particulièrement utile pour la commande qui suit.

	\section{cut  : ne garder que certaines colonnes}
	%==================
	
		La commande \samp{cut} sert à ne garder que certaines colonnes 
		(\emph{champs})
		parmi les lignes reçues.

		\begin{Exemple}{Utilisation de la commande \samp{cut}}
			\begin{itemize}
			\item 
				\kbd{du . | cut -f 2}
				pour ne garder que les noms des dossiers 
				(qui se trouvent dans le \emph{champ}/\emph{colonne} 2).
			\item 
				\kbd{ls | cut -d "." -f 1}
				pour ne garder que les noms des fichiers sans leur extension.
				L'option \samp{d} modifie le séparateur de champ
				qui est une tabulation par défaut.
                                \footnote{À votre avis, que fait cette commande si le fichier \samp{ld.so.conf} est présent dans le répertoire ? Comment essayer ?}
			\end{itemize}
		\end{Exemple}

		\begin{Experience}{Les heures de connexions}
			Imaginons qu'on veuille les heures 
			de connexions des utilisateurs actuels de \samp{linux1}
			(et rien que cela).
			\begin{steps}
			\item 
				\kbd{who}. 
				L'information demandée est en colonne 4 mais il y a beaucoup plus.
			\item 
				On aurait envie d'écrire \kbd{who | cut -f 4}.
				Ca ne fonctionne pas (essayez !)
				car la commande affiche des espaces
				et pas des tabulations entre les éléments.
			\item 
				Alors ce sera \kbd{who | cut -d " " -f 4}.
				Toujours pas !
				Lorsque 2 espaces se suivent,
				la commande considère qu'il y a une colonne (vide) entre eux.
			\item 
				La bonne commande est \kbd{who | tr -s " " | cut -d " " -f 4}.
			\end{steps}
		\end{Experience}

		\begin{Exercice}{Les machines de connexion}
			Affichez la liste des machines à partir desquelles les utilisateurs
			de \samp{linux1} sont connectés.
			\begin{enumerate}[label=\alph*)]
				\item Plus facile : le nom complet;
				\item Plus difficile : le nom court 
					(ex: \verb_L301P01_ sans le domaine).
			\end{enumerate} 
		\end{Exercice}

	\section{wc : compter}
	%==================

		La commande \samp{wc} est un peu particulière.
		Elle compte le nombre de lignes/mots/ca\-rac\-tè\-res reçus.

		\begin{Exemple}{Le nombre de connexions}
			\kbd{who | wc} donne le nombre de connexions actuellement.
		\end{Exemple}

		\begin{Exercice}{Clarifier le résultat}
			L'exemple précédent affiche 3 nombres.
			Regardez comment afficher une et une seule fois le nombre
			de connexions.
		\end{Exercice}

	\section{Récapitulatif}
	%==================

		\begin{theorie}{Filtres}
			Voici les quelques filtres que nous avons appris à utiliser :
			\begin{itemize}
				\item \samp{grep} : ne laisse passer que certaines lignes.
				\item \samp{sort} : trie les lignes
				\item \samp{uniq} : enlève les lignes en double
				\item \samp{head} : ne laisse passer que les premières lignes
				\item \samp{tail} : ne laisse passer que les dernières lignes
				\item \samp{cut}  : ne laisse passer que certaines colonnes
				\item \samp{tr} : modifie certains (groupes de) caractères
				\item \samp{wc} : compte les lignes, les mots et les caractères
			\end{itemize}
		\end{theorie}
		
		Voici quelques exercices récapitulatifs qui vont vous demander 
		d'utiliser la plupart des concepts vus.

		\begin{Exemple}{Le nombre de personnes connectées différentes}
			Vous avez déjà écrit une solution pour afficher le nombre de connexions.
			Modifiez-le pour affichez le nombre de personnes connectées.
			La différence ? Si une personne est connectée plusieurs fois,
			elle ne doit être comptée qu'une seule fois !
		\end{Exemple}
	
		\begin{Exemple}{Le plus gros dossier}
			Affichez le nom (et rien que cela)
			de votre sous-dossier de \samp{dev1}
			qui occupe le plus d'espace sur le disque.
		\end{Exemple}

%===================
\section{Conclusion}
%====================

	\begin{theorie}{Notions importantes de ce TD}
		Voici les notions importantes que vous devez avoir assimilées à la fin de ce TD.
		\begin{itemize}
		\item Comprendre le concept de filtre.
		\item Savoir utiliser et enchainer les bons filtres pour répondre
			aux questions qu'on se pose.
		\end{itemize}
	\end{theorie}

	\subsubsection*{Pour aller plus loin\dots}
	Ceux qui ont du temps peuvent aborder les exercices suivants.
	
	\begin{Exercice}{Un mot ou un autre}
		Vous avez vu comment utiliser la commande \samp{grep}
		pour sélectionner les lignes qui con\-tien\-nent \verb_Scanner_
		\emph{et} \verb_System_.
		Comment faire pour sélectionner celles qui con\-tien\-nent
		\verb_Scanner_ \emph{ou} \verb_System_.
	\end{Exercice}
	
	\begin{Exercice}{Un chiffre}
		Comment sélectionner avec \samp{grep}
		toutes les lignes qui con\-tien\-nent un chiffre, quel qu'il soit.
	\end{Exercice}

	\begin{Exercice}{La casse}
		Affichez le contenu de votre dossier en remplaçant
		toutes les minuscules par des ma\-jus\-cu\-les. 
	\end{Exercice}

\end{document}

	
