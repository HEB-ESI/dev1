\documentclass[a4paper,11pt]{article}

%=========================
% Les styles
%=========================
\usepackage{style-esi/french}	% Francise LaTeX
\usepackage{style-esi/td}
\usepackage{style-esi/licence}	% Affiche une licence dans le document
\usepackage{style-esi/exercice}
\usepackage{style-esi/listing}
\usepackage{style-esi/tutoriel}
%\marginsectiontrue
\usepackage{booktabs}
\usepackage{pifont}  



\definecolor{verylightgray}{rgb}{0.98,0.98,0.98}

\date{2018 -- 2019}
\siglecours{DEV1}
\libellecours{Laboratoires d'environnement}
\libelledocument{TD 01 -- Prise en main de l’environnement }
\sigleprof{}



\begin{document}

\entete
\titre
\ccbysa{esi-dev1-list@he2b.be}
\lastedit


	%===================
	%  Contenu
	%====================
	\begin{tcolorbox}[blanker,
	before skip=10mm,after skip=10mm,
	borderline west={1mm}{-4mm}{lightgray},
	title=Objectifs, coltitle=black, fonttitle=\sffamily\bfseries\large]
	Ce premier TD (\textit{Travail Dirigé}) a pour but de vous permettre de prendre en main les outils informatiques 
	avec lesquels vous allez travailler aux laboratoires d'environnement. Il vous accompagne dans vos premiers pas sur \verb_Linux_.
	\end{tcolorbox}
	
	\tableofcontents

	\newpage

%===================
\section{Introduction}
%====================	
Au cours des laboratoires java, vous serez amener à concevoir et écrire des programmes en java à l'aide de l'environnement de développement Netbeans.
Les laboratoires d'environnement ont, de leur côté, pour objectifs de vous familiariser avec l'environnement de développement au sens large. Vous y découvrirez des outils souvent incontournables lorsqu'on développe du logiciel. Entre autres, nous utiliseront linux, bash, git et ssh.
%====================
\subsection{Consignes}
%====================
Quelques conseils pour bien travailler et progresser.
\par
\begin{itemize}
\item Faites bien tous les exercices proposés.
\item Vous pouvez \textbf{coopérer} avec vos condisciples mais nous vous demandons de ne \textbf{pas copier} les réponses. Si vous voulez progresser, \textbf{chercher} la réponse est plus important que de la trouver. 
\item N'hésitez pas à \textbf{montrer votre travail} à votre professeur.
\item N'hésitez pas à \textbf{poser des questions} si vous n'avez pas bien compris ce qu'on vous demande.
\item \textbf{Prenez des notes} ! Ce que vous allez apprendre aujourd'hui vous servira les semaines prochaines mais vous en aurez oublié une grande partie si vous ne notez rien. Le plus pratique est probablement d'annoter la version \textbf{papier}. Nous vous expliquons plus loin comment l'imprimer si ce n'est pas déjà fait.
\end{itemize}
%====================
\subsection{Ressources}
%=====================
Nous avons rassemblé sur la plateforme d'e-learning \verb_poési_ (\url{https://poesi.esi-bru.be}), dans la rubrique \textbf{Aide et documents associés} du Laboratoire d'environnement de développement, une série de documents qui peuvent être utiles. Voyez notamment :
\par
\begin{itemize}
\item un \textbf{guide visuel Linux} : document écrit par nos soins qui explique de façon simple et visuelle les bases de Linux. Vous pouvez le consulter quand vous n'avez pas \textbf{compris} un point de matière. Certains points sont à lire \textbf{avant} de venir au laboratoire ;
\item un \textbf{aide-mémoire} : document écrit par nos soins sur l'utilisation de Windows et Linux. Vous pouvez le consulter quand vous avez \textbf{oublié} quelque chose (le nom d'une commande, une procédure...) ;
\item	un \textbf{quick reference Linux} : reprend, en condensé, toutes les commandes Linux les plus utiles. 
\end{itemize}


%===================
\section{Windows}
%====================

Comme vous avez pu le constater, les PC des laboratoires sont équipés du système \verb_Windows_.
\par			
Au laboratoire d'environnement, vous vous connecterez sur un serveur \verb_Linux_. Windows vous servira essentiellement à : vous connecter au Linux, 
effectuer des recherches sur Internet, imprimer et transférer des fichiers.
\par
Si vous avez une question concernant l'utilisation de Windows vous trouverez peut-être la réponse dans l'aide-mémoire que nous avons déjà cité dans la partie "ressources". Il est disponible sur poÉSI. 
\par		

%====================
\subsection{Imprimer}
%====================
Si vous voulez imprimer ce TD (ce qui est une bonne idée), vous devez \textit{installer} une imprimante. Vous trouverez comment faire en consultant l'aide-mémoire que vous venons de mentionner.
\par
%====================
\subsection{Changer le mot de passe sous Windows}
%=====================


\subparagraph{Réflexion}
\textcolor{white}{.} 
\par
				
\`A votre avis, pourquoi vous demande-t-on de modifier votre mot de passe ?
				
\par
			
%\subparagraph{Exemples de mots de passe} 
\begin{Exercice}{Exemples de mots de passe} 		
%\textcolor{white}{.} 
%\par
Quelles sont les propositions qui vous paraissent correctes comme mot de passe ?
\begin{itemize} 
  \item[ \ding{"6F} ] nadia
  \item[ \ding{"6F} ] M0nAm1eN@di@
  \item[ \ding{"6F} ] m@C0p1ne
  \item[ \ding{"6F} ] GH5).jg
 \end{itemize} 
\end{Exercice} 
 
\subparagraph{Changer le mot de passe} 
		
\textcolor{white}{.} 
\par
				
\par
        
Il est  temps de \textbf{changer votre mot de passe}. Consultez l'aide-m\'emoire si vous ne savez pas comment faire. 
				
\par
        
			
\subparagraph{FAQ Windows} 
		
\textcolor{white}{.} \par
				
 \par
 \textbf{Je ne suis pas content du mot de passe que j'ai choisi. Est-ce que je peux le changer ?}
 \par
Oui mais pas tout de suite. L'administrateur des machines Windows de l'\'ecole impose un temps minimum (1 jour) entre 2 modifications de mot de passe.
 \par
 \textbf{Est-ce que je vais pouvoir garder ce mot de passe toute l'ann\'ee ?}
\par
Oui. La politique en matière de mots de passe sous Windows a été redéfinie pour cette année académique. Vous pourrez garder votre mot de passe toute l'année.
\par
 \textbf{J'ai oubli\'e mon mot de passe. Qu'est-ce que je peux faire ?}
 \par
 Les professeurs ne peuvent ni retrouver votre nouveau mot de passe, ni remettre le mot de passe de d\'epart. Par contre les techniciens (bureau au 5\textsuperscript{\`eme}) peuvent remettre le mot de passe de d\'epart. Allez les trouver (et prenez garde \`a ce que \c ca n'arrive plus !)
 \par

\clearpage
%===================
\section{Linux}
%====================
\begin{quotation}
\guillemotleft  \textit{Linux ? Il y a moins bien mais c'est plus cher} \guillemotright . Auteur inconnu
\end{quotation}
%====================
\subsection{Présentation}
%====================
Vous ne travaillerez pas directement sur votre PC durant les laboratoires d'environnement. Celui-ci vous servira pour vous connecter au serveur Linux (son nom est \verb_linux1_)
 \par
 \textbf{Tiens, c'est quoi Linux et pourquoi l'utiliser ? C'est quoi une machine partag\'ee?}
 \par
  Si vous vous posez ce genre de questions (et c'est bien !), je vous invite vivement \`a (re)lire le point 1 du guide visuel (cf. documents d'aide).
 \par
%====================
\subsection{Se connecter}
%=====================
Lorsque vous allez vous connecter, \verb_linux1_ va vous demander de vous identifier.
				
 \par
        
\begin{itemize}
				
\item Votre \textbf{\textit{username}} est le m\^eme que sous Windows (avec un \textbf{'g' minuscule} obligatoirement ; ex:	\verb_g32010_).
\par
\textbf{Note} : pour Linux, les minuscules et les majuscules sont toujours des caract\`eres diff\'erents.
					
\item Votre \textit{\textbf{mot de passe}} est le m\^eme que votre \textbf{mot de passe initial sous Windows}.
					
\end{itemize}
				
Le mot de passe sous Windows et sous Linux sont 2 mots de passe diff\'erents (initialis\'es \`a la m\^eme valeur).
        
\par
        
Vous avez modifi\'e votre mot de passe sous Windows mais pas encore sous Linux (vous le ferez plus tard...)
				
%\subparagraph{Connectez-vous \`a linux1} 
\begin{Tutoriel}{Connectez-vous \`a linux1} 
Il y a 4 \'etapes :
\par
        
\begin{steps}
				
\item lancez l'application \verb_putty_ (vous la trouverez dans le menu ou comme raccourci sur le bureau) ;
					
\item indiquez \`a \verb_putty_ le nom de la machine (\textit{Host Name}) \`a laquelle vous voulez vous connecter (ici \verb_linux1_) ;
% TODO capture putty

\item cliquez sur "\textbf{Open}" ; la connexion se fait ! S'il vous pr\'esente une boite de message avec un "\textbf{Security Alert}", cliquez sur "\textbf{Yes}" en toute confiance ;
					
\item identifiez-vous !
						
\begin{steps}
\item Tapez votre nom d'utilisateur (\,\verb|gxxxxx|\,) puis sur la touche \,\verb|ENTREE|\,.\par 
\textbf{Note} : Soyez attentif au \verb|g|\, minuscule.

\item Tapez votre mot de passe puis sur la touche \,\verb|ENTREE|\,.\par
\textbf{Note} : Rien ne s'affiche quand vous tapez votre mot de passe ; c'est normal.
\end{steps}
\end{steps}
\end{Tutoriel}		       

%====================
\subsection{Le mode console}
%=====================
Si vous ne voyez pas du tout ce qu'est le mode console ou comment entrer une commande, allez d'abord faire un petit tour aux points 2 et 3 du guide visuel.
				
\par
        
%\subparagraph{Ma premi\`ere commande} 
\begin{Tutoriel}{Ma premi\`ere commande}
\begin{steps}
\item Entrez la commande \,\verb|ls|\, (n'oubliez pas la touche \,\verb|ENTREE|\,).
\end{steps}				
\par
        
Vous constatez que le bash a affich\'e quelque chose (d'incompr\'ehensible pour le moment ; ne vous inqui\'etez pas nous y reviendrons) et qu'il vous propose \`a nouveau l'invite de commande.
				
\par
        
			
\subparagraph{Il faut \^etre pr\'ecis !} 
\textcolor{white}{.} \par
\begin{steps}
\item Entrez \`a pr\'esent  la commande \,\verb|LS|\,.
\end{steps}
				
\par
        
Vous voyez que le r\'esultat est diff\'erent : il ne comprend pas ce que vous lui voulez.
        
\par
        
En Linux, \textbf{les majuscules et les minuscules n'ont pas le m\^eme sens, vous devez respecter la casse.}
\par
        
Faites une autre exp\'erience. 
        
\par
        
Tapez les 3 commandes suivantes qui ne se diff\'erencient que par la pr\'esence ou non d'espaces.
				
\par
        
\begin{steps}
				
\item \,\verb|ls /home|\,
\item \,\verb|ls/home|\,
\item \,\verb|ls / home|\,

\end{steps}
				
\`A nouveau le r\'esultat est diff\'erent dans les 3 cas. \textbf{Les espaces ont de l'importance}.
\end{Tutoriel}				
\par

%====================
\subsection{Changer le mot de passe sous Linux}
%=====================
La commande pour changer le mot de passe est \,\verb|passwd|\,.
				
 \par
        
\begin{itemize}
				
\item Les r\`egles \`a respecter sont quasiment les m\^emes que sur Windows. Attention toutefois \`a ne pas choisir un mot du dictionnaire.
\item Vous pouvez d'ailleurs reprendre le m\^eme mot de passe que celui que vous avez choisi pour Windows. 
%Mais contrairement \`a celui sous Windows, le mot de passe Linux pourra \^etre conserv\'e toute l'ann\'ee.
					
\end{itemize}
				
			
\subparagraph{\`A vous !} 
		
\textcolor{white}{.} \par
				
 \par
        
Tapez la commande ad\'equate pour changer votre mot de passe.
				
\par
        
\begin{itemize}
				
\item Le syst\`eme vous demande de taper le mot de passe actuel (vous ne le voyez pas quand vous le tapez, c'est normal !)
\item Ensuite, vous entrez le nouveau mot de passe que vous venez de choisir.
\item Vous retapez une deuxi\`eme fois ce mot de passe pour le confirmer.

\end{itemize}
\textbf{Si \c ca va mal...}
\par
        
\begin{itemize}
				
\item \textit{Quand je tape la commande rien ne se passe !}

\begin{itemize}				
\item Avez-vous bien appuy\'e sur la touche \,\verb|ENTREE|\, ?
\item Une seule personne \`a la fois peut changer son mot de passe et vous \^etes tous connect\'es \`a la m\^eme machine. Soyez patient.
\end{itemize}
				
\item \textit{Apr\`es avoir tout entr\'e, il me met un message d'erreur !}
\begin{itemize}
\item \textbf{Lisez le message} ! Il est en g\'en\'eral assez explicite.
\item Peut-\^etre que le mot de passe est trop simple.
\item Peut-\^etre n'avez-vous pas respect\'e les minuscules/majuscules.
\end{itemize}
\end{itemize}
				
			
\subparagraph{V\'erification} 
		
\textcolor{white}{.} \par
				
\par
        
Pour v\'erifier que tout s'est bien pass\'e, vous pouvez vous d\'econnecter et vous reconnecter.
        
 \par
        
Pour quitter proprement \verb_linux1_, la commande est \,\verb|exit|\,.
				
\par

%====================
\subsection{Le dossier personnel et le dossier courant}
%=====================
Un petit tour pr\'ealable aux points 4 \`a 7 du guide visuel est vivement conseill\'e.
				
 \par
        
			
\subparagraph{Examiner son dossier} 
		
\textcolor{white}{.} \par
				
\par
        
Comment voir le contenu de votre dossier ? Simplement avec la commande\,\verb|ls|\, que vous avez d\'ej\`a rencontr\'ee.
				
\par

%\subparagraph{Exp\'erimentation} 
\begin{Tutoriel}{Exp\'erimentation}  
\begin{steps}
\item Tapez la commande \,\verb|ls|\,.
\end{steps}				
\par
        
\begin{itemize}
\item Il vous montre le contenu de votre dossier.
\item Vous constatez qu'il contient d\'ej\`a des \'el\'ements.
\item La couleur permet de distinguer un dossier (en bleu) d'un fichier (en blanc).
\item Comme sur Windows, la notion de dossier est hi\'erarchique : un dossier peut contenir des fichiers mais aussi d'autres dossiers qui \`a leur tour...
\end{itemize}
				
\begin{steps}
\item Tapez la commande \,\verb|ls bin|\,.
\end{steps}
				
\par
\begin{itemize}
\item Cette fois, il vous montre le contenu du dossier \textit{bin} (ne vous inqui\'etez pas, la commande n'affiche rien parce que le dossier est vide).
\end{itemize}
\begin{steps}
\item	 \`A pr\'esent, tapez la commande \,\verb|cd bin|\,.
\end{steps}				
\par
        
\begin{itemize}
\item Cette commande demande de se \textit{\textbf{d\'eplacer}} dans le dossier \textit{bin}.
\end{itemize}
\begin{steps}
\item	Retapez la commande	\,\verb|ls|\, du d\'ebut.
\end{steps}
 \par
 \begin{itemize}
\item Le r\'esultat est diff\'erent. Est-ce que vous comprenez pourquoi ?
\end{itemize}
\end{Tutoriel}
\subparagraph{Le dossier courant} 
		
\textcolor{white}{.} \par
				
\par
        
\`A tout moment, vous \^etes \textit{dans} un dossier, appel\'e le \textbf{dossier courant} (\textit{working directory} en anglais).
				
 \par
        
Il est repr\'esent\'e par "\,\verb|.|\,".
				
\par
        
\begin{itemize}
\item La commande \,\verb|cd|\, (\textit{change directory}) permet de changer de dossier courant.
\item La commande \,\verb|cd|\, sans rien derri\`ere vous ram\`ene toujours dans votre dossier personnel.
\item La commande \,\verb|cd .|\, vous laisse l\`a o\`u vous \^etes, \,\verb|.|\, repr\'esentant votre dossier courant.
\item La commande \,\verb|cd ..|\, vous am\`ene dans le dossier juste au-dessus de celui o\`u vous \^etes, on parle de \textit{r\'epertoire parent}. \,\verb|..|\, repr\'esente le r\'epertoire parent du r\'epertoire courant.
\item La commande \,\verb|pwd|\, (\textit{print working directory}) permet d'afficher le chemin du dossier courant (o\`u vous \^etes pour le moment).
\end{itemize}
\textbf{C'est quoi le chemin ?}
\par
C'est la suite des dossiers qu'il faut traverser. Nous verrons \c ca plus en d\'etail dans le prochain TD.
				
\par
        
%\subparagraph{Exp\'erimentation} 
\begin{Tutoriel}{Exp\'erimentation}      
\begin{steps}
				
\item En pr\'eambule, tapez la commande \,\verb|cd|\, pour revenir dans \textit{votre home} (dossier personnel).
\item Tapez \`a pr\'esent la commande \,\verb|ls bin|\,.
\item Comparez le r\'esultat avec celui produit par les 2 commandes suivantes : \,\verb|cd bin|\, et \,\verb|ls|\,
\end{steps}
\end{Tutoriel}				
			
\begin{Exercice}{Question} 

Est-ce qu'on peut dire que \,\verb|ls bin|\, est strictement \'equivalent \`a \,\verb|cd bin|\, suivi de \,\verb|ls|\, ?
				
\par
 \fcolorbox{gray}{verylightgray}{\parbox{\textwidth}{\textcolor{verylightgray}{\LARGE Non ! Dans le 1 er cas, on ne modifie pas le dossier courant. Dans le 2 ème cas oui. }}}
  Comment le mettre en \'evidence ?
 \par
 \fcolorbox{gray}{verylightgray}{\parbox{\textwidth}{\textcolor{verylightgray}{\LARGE On peut le mettre en évidence en tapant la commande pwd après chaque cas.}}} 
										
			
 \end{Exercice}       
			


%====================
\subsection{L'éditeur}
%=====================

Un petit tour pr\'ealable au point 8 du guide visuel est vivement conseill\'e.
				
\par
        
%\subparagraph{Exp\'erimentation} 
 \begin{Tutoriel}{Exp\'erimentation}     
\begin{steps}
\item En pr\'eambule, tapez la commande \,\verb|cd|\, pour revenir dans \textit{votre home} (dossier personnel).
\item Tapez \,\verb|nano test|\, pour commencer \`a \'editer le fichier \verb_test_ (comme il n'existe pas encore, il est cr\'e\'e).
\item Une fen\^etre s'ouvre. Vous voyez qu'elle est scind\'ee en 2 parties : la partie sup\'erieure o\`u vous \'ecrivez votre texte et la partie inf\'erieure o\`u sont indiqu\'ees les diff\'erentes commandes (le \char`\^ repr\'esente la touche Ctrl)
\item Entrez quelques mots.
\item Appuyez sur la combinaison de touches \,\verb|Ctrl X|\,, confirmez que vous voulez sauver vos modifications et sortez.
\item Vous \^etes maintenant revenu \`a l'invite de commande.
\item Tapez \`a pr\'esent la commande \,\verb|ls|\,.Vous pouvez constater que le fichier \verb_test_ est apparu dans la liste ;)
\end{steps}
\end{Tutoriel}
%====================
\subsection{Quelques commandes courantes}
%=====================
%\subparagraph{Faisons le point} 
\begin{Exercice}{Faisons le point}		
 \textcolor{white}{.} \par
            
Vous avez d\'ej\`a eu l'occasion d'utiliser 6 commandes : \,\verb|passwd|\,,\,\verb|ls|\,,\,\verb|cd|\,,\,\verb|pwd|\,,\,\verb|exit|\, et \,\verb|nano|\,.
							
Voyons voir si vous avez retenu leur signification.
						
\begin{itemize}
\item La commande pour voir le contenu d'un dossier (la liste de ce qu'il contient) est \textcolor{gray}{\underline{\hspace*{2em}}} 
\item La commande pour \'editer le contenu d'un fichier est \textcolor{gray}{\underline{\hspace*{3em}}} 
\item La commande pour changer son mot de passe est \textcolor{gray}{\underline{\hspace*{5em}}} 
\item La commande pour se d\'econnecter de linux1 est \textcolor{gray}{\underline{\hspace*{3em}}} 
\item La commande pour changer de dossier courant est \textcolor{gray}{\underline{\hspace*{2em}}} 
\item La commande pour voir le chemin du dossier courant est \textcolor{gray}{\underline{\hspace*{2em}}} 
\end{itemize}
\end{Exercice}
				
\subparagraph{Quelques commandes en plus...} 
\textcolor{white}{.} \par
 \par
 Il est temps de voir quelques commandes suppl\'ementaires.
 \par
        
\begin{itemize}
\item \,\verb|cat nomDuFichier|\,
affiche \`a l'\'ecran le contenu du fichier dont le nom est donn\'e (ce n'est pas un \'editeur, on voit le contenu et c'est tout) ;
\item \,\verb|mkdir nomDuDossier|\, cr\'ee un dossier (vide) nomm\'e "\textit{nomDuDossier}" ;
\item \,\verb|mv nomDuFichier nouveauNomDeFichier|\, renomme le fichier donn\'e "\textit{nomDuFichier}" sous le nom "\textit{nouveauNomDeFichier}" ;
\item \,\verb|mv nomDuFichier nomDuDossier|\, d\'eplace le fichier donn\'e dans le dossier indiqu\'e ;
\item \,\verb|cp nomDuFichier nouveauNomDeFichier|\, cr\'ee une copie du fichier sous le nom "\textit{nouveauNomDeFichier}" ;
\item \,\verb|cp nomDuFichier nomDuDossier|\, copie le fichier donn\'e dans le dossier indiqu\'e ;
\item \,\verb|rm nomDuFichier|\, d\'etruit le fichier dont on donne le nom ;
\item \,\verb|rmdir nomDuDossier|\, d\'etruit le dossier dont on donne le nom (Attention, le dossier doit \^etre vide !).
\end{itemize}



\begin{Exercice}{}
Cr\'eez un dossier \verb_td1_ et d\'eplacez-y le fichier \verb_test_ que vous avez d\'ej\`a cr\'e\'e.\\
\textbf{Rappel} : Notez bien votre r\'eponse. Il est difficile de tout retenir la premi\`ere fois; vous serez bien content en relisant vos notes de pouvoir retrouver comment vous avez fait !
\end{Exercice}	
		
\begin{Exercice}{}
\begin{enumerate}
\item Prenez une copie de votre fichier \verb_test_ (appelez-la \verb_test2_).
\item \'Editez ce fichier et ajoutez-y quelques mots.
\item Affichez le contenu des 2 fichiers pour v\'erifier qu'ils sont bien diff\'erents.
\end{enumerate}
\end{Exercice}
				
\begin{Exercice}{}
\begin{enumerate}
\item Cr\'eez, dans votre dossier \verb_td1_, un dossier \verb_monDossier_.
\item D\'eplacez-y votre fichier \verb_test2_.
\end{enumerate}
\end{Exercice}	

\begin{Exercice}{}
D\'etruisez le dossier \verb_monDossier_ (ainsi que son contenu).
\end{Exercice}	

%====================
\subsection{La ligne de commande}
%=====================
Nous avons d\'ej\`a parl\'e de la ligne de commande mais il reste un \'el\'ement dont nous n'avons pas parl\'e, les \textbf{options} :  
				
 \par
\begin{itemize}
\item une option modifie le sens d'une commande ;
\item elle commence par le signe "\verb|-|" suivi d'une seule lettre ;
\item ou encore par le double tiret "\verb|--|" suivi d'un nom d'option.
\end{itemize}
				
Les options sont plac\'ees n'importe o\`u apr\`es le nom de la commande.   
				
\par
        
			
%\subparagraph{Exp\'erience} 
\begin{Tutoriel}{Exp\'erimentation}		
\begin{steps}
\item Tapez la commande \,\verb|ls -l|\,. Vous constatez que le r\'esultat obtenu est beaucoup plus verbeux que celui obtenu sans l'option.
\item Tapez la commande \,\verb|cat -n td1/test|\,. L'option demande de num\'eroter les lignes (note: nous supposons que vous avez bien cr\'e\'e le r\'epertoire \textit{td1} qui contient le fichier \textit{test} et que vous vous trouvez dans \textit{votre home}).
\item Essayez \,\verb|cat --number td1/test|\,. C'est la version \textit{longue} tout-\`a-fait \'equivalente \`a la pr\'ec\'edente.
\end{steps}
\end{Tutoriel}
%====================
\subsection{FAQ Linux}
%=====================
\textbf{Vous me dites que la commande pour changer le mot de passe est}\,\verb|passwd|\,\textbf{et que celle pour quitter est}\,\verb|exit|\,. \textbf{Je vais devoir retenir tout \c ca ?}
 \par
Oui ! En tout cas pour les plus fr\'equentes mais l'apprentissage se fera naturellement \`a force de les utiliser.
						
\par
 \textbf{Et si j'ai oubli\'e le nom d'une commande ou une option?}
 \par
         
Vous verrez la semaine prochaine les moyens mis \`a votre disposition pour retrouver le nom d'une commande ou pour apprendre \`a l'utiliser correctement.
						
 \par
\textbf{J'ai quitt\'e en fermant la fen\^etre, ce n'est pas plus simple ?}
\par
        
Oui ! Mais c'est impoli de quitter quelqu'un sans lui dire au revoir ! ;) 
						
\par
Plus s\'erieusement, vous coupez brutalement la conversation avec \verb_linux1_ ce qui peut laisser trainer des programmes actifs et vous emp\^echer de vous connecter la prochaine fois.
						
\par
 \textbf{J'ai oubli\'e mon mot de passe. Je dois aussi aller voir les techniciens ?}
 \par
        
Non ! Votre professeur de labo peut r\'einitialiser le mot de passe Linux \`a sa valeur initiale.
						
 \par

%===================
\section{Les outils virtuels}
%====================
Cette partie d\'epasse un peu le cadre strict des laboratoires d'environnement. Nous voudrions profiter de votre pr\'esence \`a un laboratoire pour vous pr\'esenter les diff\'erents outils virtuels mis \`a votre disposition par l'\'ecole.
\par

%====================
\subsection{Les outils virtuels}
%=====================
\subparagraph{A. Le site de l'\'ecole} 
		
\textcolor{white}{.} \par
				
Vous avez probablement d\'ej\`a visit\'e le site web (\url{esi-bru.be}) de l'\'ecole. C'est l\`a que vous trouverez tous les documents officiels.
			
 \par
        
\begin{Exercice}{}
Allez sur le site de l'\'ecole et retrouvez-y deux documents qui vous seront utiles dans votre parcours scolaire :
\begin{itemize}
\item Le calendrier acad\'emique : les dates de cong\'es, d'examens...
\item Le r\'eglement des \'etudes qui reprend l'ensemble de vos droits et devoirs en tant qu'\'etudiant.
\end{itemize}
\end{Exercice}				

        
\subparagraph{B. po\'ESI} 
\textcolor{white}{.} \par
Vous avez d\'ej\`a fait connaissance avec po\'ESI. Cette \textbf{plateforme d'apprentissage en ligne} est l'outil principal utilis\'e dans tous les cours pour mettre toutes les notes \`a disposition des \'etudiants
			
 \par
 \textbf{Attention !} M\^eme si le syst\`eme de messagerie de po\'ESI envoie une copie par mail, pr\'ef\'erez l'envoi direct d'un mail pour nous contacter. 
			
 \par
        
			
\subparagraph{C. Le mail} 
		
\textcolor{white}{.} \par On vous a d\'ej\`a montr\'e comment utiliser votre messagerie pour connaitre votre mot de passe 
po\'ESI. 
			
 \par N'utilisez que cette adresse pour toute communication avec l'\'ecole : professeurs, service administratif...
 \par
Si vous voulez (sans en abuser), poser une question \`a tous les profs de l'unit\'e d'enseignement DEV1 en un seul mail, vous pouvez (sans en abuser, j'insite) utiliser l'adresse \textit{esi-dev1-list@he2b.be}.
			
 \par
        
\subparagraph{D. Le drive} 
		
\textcolor{white}{.} \par
				
Vous disposez d'un espace de 30Go dans le cloud, g\'er\'e par Google. Vous pouvez y d\'eposer vos fichiers li\'es \`a l'\'ecole et utiliser tous les services li\'es : partage de document, travail collaboratif... Apprenez \`a l'utiliser !
			
 \par 
 Pour y acc\'eder, il y a plusieurs possibilit\'es. Si vous \^etes d\'ej\`a connect\'e \`a votre mail, vous pouvez simplement cliquer sur l'ic\^one en haut \`a droite de l'\'ecran. Sinon, vous pouvez vous rendre \`a l'adresse https://drive.google.com (\url{https://drive.google.com}).
			
 \par
Pour le moment, cet espace est peu utilis\'e dans les \'echanges \'etudiants-professeurs car nous privil\'egions l'utilisation de \textit{po\'ESI}mais n'h\'esitez pas \`a l'utiliser pour vous et entre vous !
			
\par
\begin{Exercice}{}
Allez sur votre drive, cr\'eez un dossier \verb_dev1_ et, dedans, un fichier texte \verb_Notes_ que vous pourrez utiliser pour prendre quelques notes.
\end{Exercice}

 \par
        
\subparagraph{E. L'agenda} 

\textcolor{white}{.} \par
Google met \'egalement \`a disposition un agenda. Nous ne l'\textbf{utilisons pas} pour l'instant pour communiquer des dates importantes (cours, interrogations,...) mais vous pouvez l'utiliser pour vous ; il est facilement int\'egrable \`a tout autre syst\`eme d'agenda que vous pourriez d\'ej\`a utiliser.
			
\par
\subparagraph{F. git-esi}
\textcolor{white}{.} \par
Nous utiliserons également un serveur gitlab avec lequel nous travaillerons au TD5 disponible à l'adresse \url{https://git.esi-bru.be}
\textcolor{white}{.} \par


\par
%===================
\section{Conclusion}
%====================

Félicitations ! Vous \^etes arriv\'es au bout de ce premier TD.
 \par
Avant de quitter le laboratoire, n'oubliez pas de quitter proprement la connexion avec \verb_linux1_ (\,\verb|exit|\,). et d'\'eteindre l'ordinateur ou de vous d\'eloguer.
\par
Attention, afin d'arriver au laboratoire dans les meilleures conditions, il est bien de revoir la mati\`ere qui sera mise en pratique. 
% C'est pourquoi nous vous fournissons quelques \textbf{exercices pr\'eparatoires} \`a faire \`a la maison pour vous permettre d'\'evaluer si vous \^etes pr\^et.  
 \par
\`A la semaine prochaine et soyez \`a l'heure !
\par
	
	
\end{document}