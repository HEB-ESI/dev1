\documentclass[a4paper,11pt]{article}

%=========================
% Les styles
%=========================
\usepackage{style-esi/french}	% Francise LaTeX
\usepackage{style-esi/td}
\usepackage{style-esi/licence}	% Affiche une licence dans le document
\usepackage{style-esi/exercice}
\usepackage{style-esi/exemple}
\usepackage{style-esi/listing}
\usepackage{style-esi/tutoriel}
%\marginsectiontrue
\usepackage{booktabs}
\usepackage{pifont} 
%\usepackage{tabularx} 
%\usepackage{multicol}
%\usepackage{multirow}
\usepackage{longtable}
%\usepackage{array}

\definecolor{verylightgray}{rgb}{0.98,0.98,0.98}

\date{2018 -- 2019}
\siglecours{DEV1}
\libellecours{Laboratoires d'environnement}
\libelledocument{TD 4 -- Commandes grep, wc, find et variables d'environnement}
\sigleprof{}



\begin{document}

\entete
\titre
\ccbysa{esi-dev1-list@he2b.be}
\lastedit


	%===================
	%  Contenu
	%====================
	\begin{tcolorbox}[blanker,
	before skip=10mm,after skip=10mm,
	borderline west={1mm}{-4mm}{lightgray},
	title=Objectifs, coltitle=black, fonttitle=\sffamily\bfseries\large]
	Ce TD a pour objectif de vous familiariser avec :
	\begin{itemize}
		\item les commandes Unix \verb_grep_, \verb_wc_ et \verb_find_ ;
		\item les variables d'environnement.  
	\end{itemize}
	\end{tcolorbox}
	
	\tableofcontents

	\newpage


%%%%%%%%%%%%%%%%%%%%%%%%%%%%%%%%%%%%%%%%%%%%%%%%%%%%%%%%%%%%%%%%%%%%%%%%%%%%%%%%%%%%%	   
\section{TD4 - Les commandes Unix grep, wc, find et les variables d'environnement}
%%%%%%%%%%%%%%%%%%%%%%%%%%%%%%%%%%%%%%%%%%%%%%%%%%%%%%%%%%%%%%%%%%%%%%%%%%%%%%%%%%%%%
	 \begin{consigne}
	 	\begin{itemize}
			%\item Ce TD est accompagn\'e d'exercices \`a faire \textbf{avant} de venir au laboratoire.
			\item Prenez bien note des r\'eponses aux exercices ainsi que de la fa\c con dont vous avez trouv\'e ces r\'eponses.
		\end{itemize}				
	\end{consigne}
	
%%%%%%%%%%%%%%%%%%%%%%%	
	\subsection{Recherche dans des fichiers (grep)}
%%%%%%%%%%%%%%%%%%%%%%%%
	
		Vous avez d\'ej\`a \'ecrit pas mal de programmes et il devient parfois difficile de vous y retrouver. Vous allez \^etre amen\'es \`a vous poser des questions du genre :
				
            	\par
        
		\begin{itemize}
				\item Dans quel fichier ai-je \'ecrit un programme qui calcule une surface ?
				\item Dans quel fichier ai-je d\'ej\`a utilis\'e un double ?
		\end{itemize}
        
            	\par
        
			
		La commande \verb_grep_ peut venir \`a votre secours. Dans son utilisation la plus simple, elle permet d'extraire de fichiers toutes les lignes qui contiennent un certain texte (appel\'e 
		\textit{pattern}).
					
		\par
		Syntaxe : \,\verb|grep pattern fichier...|\,
                 \par
        
			
				
           	\begin{Exemple}{}
			Pour trouver dans quel fichier vous avez utilis\'e une variable nomm\'ee "surface", vous pouvez \'ecrire :
				\begin{Console}
					grep surface *.java
				\end{Console}
		\end{Exemple}
            \par
        

        
			\textbf{Rappel} Lorsque vous appuyez sur la touche \,\verb|TAB|\,,
			le shell tente de compl\'eter le d\'ebut de commande que vous avez
			d\'ej\`a tap\'e.  Si plusieurs possibilit\'es existent, elles sont
			affich\'ees si vous appuyez 2x sur  \,\verb|TAB|\,.  
        
            	\par
        
			
		\begin{Exercice}{Recherche d'un pattern dans un fichier} 
			\begin{enumerate}
				\item Comment trouver les programmes Java du TD4 o\`u vous avez d\'ej\`a utilis\'e un "double" ?
		 
		 		\item Comment trouver, parmi \textbf{tous} les programmes Java que vous avez d\'ej\`a \'ecrits, ceux qui utilisent des entiers ?
			\end{enumerate}
			\fcolorbox{gray}{verylightgray}{
			\begin{minipage}[c][1.2cm][c]{\textwidth}\textcolor{verylightgray}{X}\end{minipage}
		}\par\medskip
            \par

		\end{Exercice}		
%%%%%%%%%%%%%%%%%%%%%%%%%%%%%%%%%%%%%%%%%%%%%%%%%%%%%%%%%%%%%%%%%%%%%%%%%%%%%%%%%%			
 	\subsection{Compter (wc)}
%%%%%%%%%%%%%%%%%%%%%%%%%%%%%%%%%%%%%%%%%%%%%%%%%%%%%%%%%%%%%%%%%%%%%%%%%%%%%%%%%%		
		Vous vous demandez peut-\^etre combien de lignes fait un programme que vous avez \'ecrit
		ou encore combien de lignes vous avez \'ecrites aujourd'hui.
		La commande \verb_wc_ peut vous r\'epondre ; elle indique le nombre de lignes, de mots et de caract\`eres
		contenus dans les fichiers donn\'es.
		\par
				
		Syntaxe : \,\verb|wc fichier...|\,
            \par
           
        
			
		\begin{Exemple}{}
			La commande : \,\verb|wc Ex2.java|\, affiche le nombre de lignes, mots et caract\`eres contenus dans le fichier
			\verb_Ex2.java_.

		\end{Exemple}	
		
		\begin{Exercice}{Compter le nombre de lignes} 
			\begin{enumerate}
				
				\item Comment afficher le nombres de lignes de tous vos fichiers Java de ce TD.
						
				\item Examinez les options de la commande et trouvez comment n'afficher \textbf{que}
					le nombre de lignes et pas le nombre de mots et de caract\`eres.
						
				\item Une convention d'\'ecriture Java indique de ne pas d\'epasser la colonne 80 dans les programmes.
					Trouvez l'option qui permet de v\'erifier que tous vos programmes actuels v\'erifient cette convention.
			\end{enumerate}
				
			\fcolorbox{gray}{verylightgray}{
			\begin{minipage}[c][3cm][c]{\textwidth}\textcolor{verylightgray}{X}\end{minipage}
		}\par\medskip
		
		\end{Exercice}
            \par
		
%%%%%%%%%%%%%%%%%%%%%%%%%%%%%%%		
	\subsection{Recherche de fichier (find)}
%%%%%%%%%%%%%%%%%%%%%%%%%%%%%%%			
		\verb_find_ est une commande linux tr\`es puissante qui vous fera gagner beaucoup de temps.
		Elle permet de rechercher dans une arborescence de fichiers ceux qui correspondent \`a un crit\`ere donn\'e (taille, droits, nom, dates...).
		Elle permet \'egalement d'appliquer une commande \`a chacun des fichiers ainsi trouv\'es.
			
           	\par
        
		Attention \`a ne pas la confondre avec la commande \verb_grep_ qui va examiner le contenu des fichiers.
			
            	\par
	
		\begin{Tutoriel}{Rechercher un fichier} 
		
		        \begin{Console}
				find ~ -name Ex.java
			\end{Console}
			\begin{steps}
				\item Recherche, chez vous (\verb_~_), un fichier nomm\'e \verb_Ex.java_
			\end{steps}
		\end{Tutoriel}
            	\par
        
			
		\begin{Exercice}{}
		
			Trouvez avec la commande \verb_find_ tous les fichiers Java que vous avez d\'ej\`a \'ecrits.
			
            		\par
        
			\fcolorbox{gray}{verylightgray}{
				\begin{minipage}[c][1cm][c]{\textwidth}\textcolor{verylightgray}{X}\end{minipage}
			}\par\medskip
				
			Nous avons \'ecrit pour vous une classe \verb_Color_ mais nous ne savons plus tr\`es bien o\`u nous l'avons stock\'ee.
			Nous nous rappelons juste l'avoir mise quelque part dans \verb_/eCours_.
			Pouvez-vous la retrouver pour nous ?
			
            		\par
        
				\fcolorbox{gray}{verylightgray}{
					\begin{minipage}[c][1cm][c]{\textwidth}\textcolor{verylightgray}{X}\end{minipage}
				}\par\medskip
%%%%%%%%%%%%%%%%%%%%%%%%%%%%%%%%%%%%%%%%%%%%%%%%%%%%%%%%%%%%%%%%%%%%%%%%%%%%%%				
	\subsection{Passer \`a la couleur}
%%%%%%%%%%%%%%%%%%%%%%%%%%%%%%%%%%%%%%%%%%%%%%%%%%%%%%%%%%%%%%%%%%%%%%%%%%%%%%				
			La classe \verb_Color_ que vous avez trouv\'ee \`a l'exercice pr\'ec\'edent offre quelques m\'ethodes pour colorier des textes :
			
            		\par
        
			\begin{itemize}
				
				\item \verb_String toRed(String text)_ qui colore le texte donn\'e en rouge.
					C'est-\`a-dire que si on l'affiche \`a l'\'ecran, il apparaitra en rouge.
				
				\item Idem pour \verb_toGreen_, \verb_toBlue_, \verb_toYellow_, \verb_toPurple_
					et \verb_toCyan_.
				
				\item Exemple d'utilisation dans la classe \verb_Color_ :
					\begin{Code}{Java}{
						System.out.println( toRed("Hello ") + toGreen("World !"));
					\end{Code}
			\end{itemize}
		
		\end{Exercice}
				
			
		\begin{Exercice}{}
		
			\begin{enumerate}
				
				\item Copiez notre fichier \verb_Color_ chez vous.
				
				\item Compilez-le.
				
				\item Pour apprendre \`a l'utiliser, cr\'eez une m\'ethode principale
					qui lit deux nombres au clavier et affiche la valeur maximale.
					\par
				
					Afin de rendre le programme plus agr\'eable \`a utiliser, chaque lecture sera pr\'ec\'ed\'ee
					d'un message en couleur indiquant ce qui est demand\'e (ex: "Entrez un nombre entier").
					La r\'eponse sera, elle aussi, affich\'ee en couleur.
				
			\end{enumerate}
		\end{Exercice}
        
			

%%%%%%%%%%%%%%%%%%%%%%%%%%%%%%%%%%		
	\subsection{Les variables d'environnement}
%%%%%%%%%%%%%%%%%%%%%%%%%%%%%%%%%%
		Une \textit{variable d'environnement} est une variable associ\'ee \`a votre shell contenant un texte qui est accessible par toutes
		les applications que vous lancez. G\'en\'eralement, elle permet de configurer certaines applications, d'en modifier le comportement.
			
            \par
        
			
		\subparagraph{Le prompt} 
		
	         	\par
        
			Le \textit{prompt} (ou \textit{invite} en fran\c cais) est le texte qui apparait \`a gauche
			de ce que vous tapez dans votre shell. Il est d\'etermin\'e par la variable d'environnement \verb_PS1_.
			
            		\par
        
			
		\subparagraph{FAQ} 
           		\par
        			\textbf{Comment voir le contenu d'une variable d'environnement ?}
           		 \par
        
			Gr\^ace \`a la commande : \,\verb|echo $NOM_VARIABLE|\,
            		\par
       			 \textbf{Puis-je changer son contenu?}
            		\par
        
			Oui, via la commande : \verb@NOM_VARIABLE=nouvelle_valeur@
            		\par
        
			
		\begin{Exercice}{} 
			Affichez la valeur de votre prompt. Vous remarquerez qu'il contient des codes qui seront remplac\'es par certaines valeurs. 
			Par exemple, \verb_\w_ indique le dossier courant.
		\end{Exercice}
			
            \par
        
			
		\begin{Exercice}{}        
			Modifiez la valeur de votre variable \verb_PS1_.
			Par exemple, modifiez l'invite en "Bonjour ! ".
		\end{Exercice}	
           	\par
        
			
		\begin{Tutoriel}{Expérience} 
			\begin{steps}
				\item D\'econnectez-vous de linux1, puis reconnectez-vous.
				
			\end{steps}
			
            \par
        
			Surprise, votre prompt a repris sa valeur d'origine ! Pour rendre une modification permanente, il faut ajouter la commande \`a votre fichier \verb@.bashrc@.
			C'est un fichier cach\'e qui est ex\'ecut\'e par votre shell lors de votre connexion.
		\end{Tutoriel}
			
            \par
        
			
		\begin{Exercice}{}
			Les possibilit\'es de configuration du prompt sont nombreuses : afficher l'heure, le nom de la machine, votre login,
			utiliser des couleurs... 
			Examinez la documentation pour configurer le prompt comme il vous plait. (\verb@man bash@, section \verb@PROMPT@)
		\end{Exercice}	
            \par



				
				
\end{document}
			
