\documentclass[a4paper,11pt]{article}

%=========================
% Les styles
%=========================
\usepackage{style-esi/french}	% Francise LaTeX
\usepackage{style-esi/td}
\usepackage{style-esi/licence}	% Affiche une licence dans le document
\usepackage{style-esi/exercice}
\usepackage{style-esi/exemple}
\usepackage{style-esi/listing}
\usepackage{style-esi/tutoriel}
%\marginsectiontrue
\usepackage{booktabs}
\usepackage{pifont} 
%\usepackage{tabularx} 
%\usepackage{multicol}
%\usepackage{multirow}
\usepackage{longtable}
%\usepackage{array}

\definecolor{verylightgray}{rgb}{0.98,0.98,0.98}

\date{2018 -- 2019}
\siglecours{DEV1}
\libellecours{Laboratoires d'environnement}
\libelledocument{TD 07 -- Filtres, tubes et redirections}
\sigleprof{}



\begin{document}

\entete
\titre
\ccbysa{esi-dev1-list@he2b.be}
\lastedit


	%===================
	%  Contenu
	%====================
	\begin{tcolorbox}[blanker,
	before skip=10mm,after skip=10mm,
	borderline west={1mm}{-4mm}{lightgray},
	title=Objectifs, coltitle=black, fonttitle=\sffamily\bfseries\large]
	Dans ce TD, vous étudierez les mécanismes de redirections, de tubes et les filtres sous Linux.
	Vous apprendrez à :
	\begin{itemize}
		\item rediriger les entrées et les sorties de vos programmes ;
		\item combiner plusieurs commandes Linux pour réaliser des opérations complexes sur vos fichiers ;
	\end{itemize}
	
	\end{tcolorbox}
	
	\tableofcontents

	\newpage


%%%%%%%%%%%%%%%%%%%%%%%%%%%%%%%%%%%%%%%%%%%%%%%%%%%%%%%%%%%%%%%%%%%%%%%%%%%%%%%%%%%%%	   
\section{TD7 - Filtres, tubes et redirections}
%%%%%%%%%%%%%%%%%%%%%%%%%%%%%%%%%%%%%%%%%%%%%%%%%%%%%%%%%%%%%%%%%%%%%%%%%%%%%%%%%%%%%
		
%%%%%%%%%%%%%%%%%%%%%%%	
	\subsection{Entr\'ees et sorties standards}
%%%%%%%%%%%%%%%%%%%%%%%%
	\begin{coltbox}{Les fichiers standards}
		Tout programme qui s'ex\'ecute dispose de trois fichiers ouverts d'office 
		par le syst\`eme pour lui : l'entr\'ee standard, la sortie standard et la sortie d'erreur,
		identifi\'es respectivement par les num\'eros 0, 1 et 2.
		\par
		En Java, on retrouve ces trois fichiers :
	   			
                  
			\begin{itemize}
				
				\item \verb_System.in_ pour 0 (entr\'ee standard) qu'on retrouve dans la d\'eclaration
					\verb_Scanner clavier = new Scanner(System.in);_
				\item \verb_System.out_ pour 1 (sortie standard) qu'on retrouve dans l'instruction
					\verb_System.out.println();_
				\item \verb_System.err_ pour 2 (erreur standard) qu'on retrouve dans l'instruction
						\verb_System.err.println();_
				\end{itemize}
	\end{coltbox}
	
	\medskip	
	
	Cela peut vous paraitre bizarre de dire que le clavier et l'\'ecran sont des
	fichiers mais c'est bien ainsi que le programme les voit. Et c'est pratique,
	car nous allons pouvoir \textit{rediriger} ces entr\'ees et ces sorties vers
	de vrais fichiers de fa\c con tout-\`a-fait transparente pour le programme
	; il ne sera pas n\'ecessaire de le modifier.
				


%%%%%%%%%%%%%%%%%%%%%%%%%%%%%%%		
	\subsection{Rediriger la sortie}
%%%%%%%%%%%%%%%%%%%%%%%%%%%%%%%			
		Il est possible, au moment o\`u on lance un programme, de rediriger sa sortie.
		Tout ce que le programme enverra sur sa sortie standard
		(par exemple avec un \verb_System.out.println()_ en Java) ne sera pas visible \`a l'\'ecran
		mais sera envoy\'e dans, par exemple, un fichier.
		\medskip	

		\begin{coltbox}{Notation}
		
			Une redirection de sortie standard se note \og > \fg\,  ou \og 1>
			\fg\, lors du lancement du programme.  Ces redirections sont
			r\'ealis\'ees par le shell avant l'ex\'ecution de la commande et
			sont transparentes pour cette commande.
		
		\end{coltbox}
					
		\begin{Exemple}{Redirection de sortie}
		        \begin{Console}
				ls -l > liste
		 	\end{Console}
			ou
			 \begin{Console}
				ls -l 1> liste
		 	\end{Console}
		\end{Exemple}
		
		Ces deux commandes sont \'equivalentes; elles n'affichent pas le r\'esultat \`a l'\'ecran,
		mais l'\'ecrivent dans le fichier \verb_liste_ cr\'e\'e ici ou \'ecras\'e s'il pr\'eexistait.
		Par d\'efaut \verb_ls -l_  affiche le r\'esultat \`a l'\'ecran. Nous avons \textit{redirig\'e} 
		cette sortie standard vers un fichier, en l'occurrence \verb_liste_.
				
            \par
        		\begin{steps}
			\item Faites l'essai et v\'erifiez le contenu du fichier cr\'e\'e.
		\end{steps}
				
            \newpage
\begin{Exercice}{Rediriger la sortie}
	\begin{itemize}
			\item Écrivez un programme qui affiche la suite des pas croissants :\\
				 1, 2, 4, 7, 11, 16, 22, 29, ... .\\
				  Ex\'ecutez-le pour afficher les 1000 premiers nombres de cette suite.
					
			\item Sauvez le r\'esultat dans un fichier pour pouvoir l'examiner \`a votre aise.\\
				\textbf{Rappel} : pour examiner le contenu d'un fichier, inutile de passer par un \'editeur,
				la commande \verb_more_ suffit.
					
			\item Est-ce que le nombre 15007 en fait partie ? (aide : vous vous rappelez de la commande \verb_grep_ ?)
					
	\end{itemize}
				

\end{Exercice}

\begin{coltbox}{Note} 
	Avec la simple redirection en sortie, si le fichier existe d\'ej\`a, il est \'ecras\'e. 
	Nous pouvons choisir de ne pas l'\'ecraser, mais de le compl\'eter 
	(ajouter du contenu \`a la fin du fichier) via la double redirection (en sortie) not\'ee 
					\verb_>>_.			
	
\end{coltbox}


%%%%%%%%%%%%%%%%%%%%%%%%%%%%%%%%%%		
	\subsection{Rediriger l'entrée}
%%%%%%%%%%%%%%%%%%%%%%%%%%%%%%%%%%

		L'\og entr\'ee standard\fg\,  peut \^etre associ\'ee \`a un fichier au
		lieu du clavier.  Cela permet d'utiliser les donn\'ees \`a partir du
		fichier au lieu de les entrer au clavier.  C'est ce qu'on appelle une
		redirection de l'entr\'ee.
			
		\medskip	
		\begin{coltbox}{Notation}
			La redirection d'entr\'ee se note \guillemotleft <\guillemotright .
		\end{coltbox}
        \par
		\begin{Exemple}{Redirection d'entrée}
		        \begin{Console}
		        		commande < data
			\end{Console}
		\end{Exemple}
		Les lectures de la commande se feront dans le fichier \verb_data_
		et pas au clavier.
		
		\begin{Tutoriel}{Exp\'erience} 
				\begin{steps}
					\item Écrivez une classe \verb_Multiples4_ qui lit une série de nombres et n'affiche que ceux qui sont des multiples de 4. 
					À la fin, elle affiche le nombre de multiples de 4.
			
					\item Exécutez votre programme et entrez des nombres au
						clavier.  La combinaison de touches \verb_Ctrl-d_ est
						l'\'equivalent de la marque \og fin de
						fichier\fg\,  pour le clavier, c'est ainsi que
						vous terminerez l'acquisition de la s\'erie de nombres
						au clavier.
					
					\item Il ne faut pas confondre \verb_Ctrl-d_ et \verb_Ctrl-c_
						qui tue le processus. Comment mettre en \'evidence la diff\'erence ?
					
					\item Ex\'ecutez la classe en associant le clavier 
						\`a un petit fichier texte o\`u vous aurez \'ecrit les nombres au pr\'ealable,
						s\'epar\'es par des blancs (caractères d'espacement).						
					
				\end{steps}
		\end{Tutoriel}
				
			
		\begin{Exercice}{Rediriger l'entrée} 
			Pour r\'esoudre cet exercice, vous devez combiner votre programme des pas croissants
			et le programme qui s\'electionne les multiples de 4. On vous demande d'afficher,
			parmi les 1000 premiers nombres de la suite des pas croissants, tous ceux qui sont des multiples de 4.
			Combien y en a-t-il ?
		\end{Exercice}
            		
 %%%%%%%%%%%%%%%%%%%%%%%%%%%%%%%%%%%%%%%%
        \subsection{Les tubes (pipes en anglais)}
  %%%%%%%%%%%%%%%%%%%%%%%%%%%%%%%%%%%%%%%%
Pour r\'esoudre l'exercice de la section pr\'ec\'edente, vous avez d\^u cr\'eer un fichier
temporaire qui n'a servi qu'\`a \c ca. Vous avez redirig\'e la sortie de la premi\`ere commande dans un fichier et
redirig\'e ce fichier comme entr\'ee de la seconde commande.
					
			\begin{coltbox}{Les tubes}
		
				Le symbole \guillemotleft |\guillemotright\,  permet de chainer des commandes ; la sortie de l'une sert d'entr\'ee \`a la suivante. 
				On parle de "pipe" en anglais et de "tube" en fran\c cais.
					
			\end{coltbox}

			C'est une situation qui se pr\'esente souvent, surtout sous Linux qui
			propose de nombreuses commandes qui font une seule chose (plut\^ot
			bien) et qu'on combine pour obtenir un r\'esultat plus cons\'equent.
			Des commandes comme \verb_more_, \verb_grep_, \verb_wc_... peuvent
			prendre leurs donn\'ees sur l'entr\'ee standard.
					
          
		\begin{Exemple}{Enchaîner les commandes} 
			La commande \verb_more_ permet d'afficher un message page par page.
			Si on ne lui donne pas de nom de fichier, il pagine les donn\'ees re\c cues sur l'entr\'ee standard. 
			On peut donc remplacer
				
                  	 \begin{Console}
				ls /home >temp
				more temp
			\end{Console}
					par
				
                		 \begin{Console}
				ls /home | more
			\end{Console}
			
		\end{Exemple}
			
		\begin{Exercice}{Utilisation des tubes}
				\begin{itemize}
				
						\item Utilisez un tube pour refaire l'exercice pr\'ec\'edent
						(afficher parmi les 1000 premiers nombres de la suite des pas croissants,
						tous ceux qui sont des multiples de 4).
					
						\item Supprimez du programme \verb_Multiples4_
							la ligne finale qui affiche le nombre de multiples trouv\'es. 
					
						\item Relancez votre commande de l'exercice pr\'ec\'edent.
							Vous ne voyez plus, \`a la fin, le nombre de multiples, ce qui est normal.
							Quel enchainement de commandes permet d'afficher ce nombre
							(et uniquement ce nombre) ? Rappelez-vous, il existe une commande Linux qui "compte". 
					
						\item Affichez, parmi les 1000 premiers nombres 
							de la suite des pas croissants, tous ceux qui contiennent un 0.
					
					\end{itemize}
	\end{Exercice}
%%%%%%%%%%%%%%%%%%%%%%%%%%%%%%%%%%%%%%%%%%%%%
	\subsection{Rediriger les erreurs}  
%%%%%%%%%%%%%%%%%%%%%%%%%%%%%%%%%%%%%%%%%%%%%
          	On a vu qu'il est possible de rediriger la sortie d'un programme.
		Il est aussi possible de rediriger ses erreurs.
				
          	Attention ! C'est au programme \`a d\'ecider ce qui est un message d'erreur (en utilisant
		\verb_err_ au lieu de \verb_out_).
			
		\begin{coltbox}{Notation}
		
					Une redirection de la sortie d'erreur se note \guillemotleft 2>\guillemotright .
				
		\end{coltbox}
\newpage
			
		\begin{Exemple}{Redirection d'erreurs}
			Supposons que vous ayez \'ecrit une classe \verb_Malfaite_ qui provoque beaucoup d'erreurs \`a la compilation.
			Pour les regarder \`a votre aise, vous pouvez les rediriger dans un fichier 
					\verb_erreur_ via :
					\begin{Console}
						javac MalFaite.java 2>erreur
						
					\end{Console}
					Vous pouvez alors examiner les erreurs en ouvrant le fichier \verb_erreur_,
					par exemple via : 
					\begin{Console} 
						more erreur
					\end{Console}
					
		\end{Exemple}		
          	
		\begin{Exercice}{Redirection}        
			Les professeurs se demandent combien d'\'etudiants ont chez eux
			un fichier \verb_Multiples4.java_. Pouvez-vous indiquer la (suite de) commande(s)
			qui permet de r\'epondre \`a la question ?
		\end{Exercice}
				
            \par
%%%%%%%%%%%%%%%%%%%%%%%%%%%%%%%%%%%%%%%%
        \subsection{Les filtres Linux}  
 %%%%%%%%%%%%%%%%%%%%%%%%%%%%%%%%%%%%%%%%%
      		De nombreuses commandes Linux sont bas\'ees sur le principe \verb_KISS_ (Keep It Simple, Stupid).
		Elles font peu, mais le font bien et surtout, peuvent facilement coop\'erer pour, au final,
		obtenir un r\'esultat bluffant. Parmi toutes les commandes, les filtres sont \`a mettre en \'evidence.
	
		\medskip
		\begin{coltbox}{Les filtres}
				Un \verb_filtre_ est une commande Linux qui acquiert des donn\'ees sur l'entr\'ee standard 
				et les envoie vers la sortie standard apr\`es les avoir \'eventuellement transform\'ees.
		\end{coltbox}

		Nous allons nous concentrer sur les filtres suivants : \verb_tr_,
		\verb_cut_, \verb_cat_, \verb_sort_, \verb_head_, \verb_tail_,
		\verb_split_, \verb_uniq_, \verb_grep_, \verb_more_, \verb_less_,
		\verb_wc_...  Vous en connaissez d\'ej\`a ; voyez-en les pages de
		manuel respectives et l'aide (\verb_--help_) pour en connaitre le
		d\'etail.
				
           
			
		\begin{Tutoriel}{Utilisation des filtres}
					On voudrait connaitre le nombre d'utilisateurs qui sont connect\'es \`a linux1 pour le moment.
					Voyons, \'etape par \'etape, comment on peut obtenir le r\'esultat.
				
           		\begin{steps}
				
				\item La premi\`ere \'etape est de penser \`a la commande \verb_who_
					qui affiche toutes les connexions actives.
					
				\item C'est gagn\'e, il suffit de compter le nombre de lignes, direz-vous !
						Allons ! Ne nous arr\^etons pas l\`a ; l'ordinateur peut compter pour nous,
						ce qui donne :
						\begin{Console}
							who | wc -l
						\end{Console}
						
				\item Cette fois, \c ca y est, on a la r\'eponse ! Ben, non !
					Il y a un pi\`ege car la commande \verb_who_ ne donne pas les utilisateurs
					mais les connexions ce qui n'est pas tout \`a fait la m\^eme chose ;
					un utilisateur peut avoir ouvert plusieurs fen\^etres "putty". 
						
            \par
        
					La commande \verb_uniq_ peut venir \`a notre secours en supprimant
					les doublons mais il faut que les lignes soient parfaitement identiques
					et contig\"ues.
						
            \par
        
					Parfait ! La commande \verb_cut_ peut ne garder que certaines colonnes
					et la commande \verb_sort_ peut trier les lignes. On obtient alors :							
						
            \par
        					\begin{Console}
						who | cut -f 1 -d ' ' | sort | uniq | wc -l
					\end{Console}
					
					Je ne me r\'ejouis pas trop vite ; je suis s\^ur que vous allez encore
					trouver une faille, n'est-ce pas ? Non, pas cette-fois ; on y est !
			\end{steps}
		\end{Tutoriel}
\newpage				
			
		\begin{Exercice}{Nombre de connexions d'un utilisateur} 
				Trouvez un enchainement de commandes qui permet de donner le nombre de connexions
				d'un utilisateur donn\'e.
				
            \par
        
				Il existe de nombreuses fa\c cons de r\'esoudre cet exercice. 
				Celle \`a laquelle nous pensons fait intervenir : \verb_grep_, \verb_wc_ et \verb_who_.
				
            \end{Exercice}
        
			
	\begin{Exercice}{Nombre de PC connect\'es} 
			Trouvez un enchainement de commandes qui permet de donner le nombre de PC
			connect\'es \`a linux1. Ce n'est pas exactement le nombre d'utilisateurs car un utilisateur pourrait \^etre connect\'e
			sur plusieurs machines.
				
            \par
        
			\`A nouveau, il existe de nombreuses fa\c cons de r\'esoudre cet exercice. 
			Celle \`a laquelle nous pensons fait intervenir la commande \verb_tr -s ' '_
			qui supprime plusieurs occurences cons\'ecutives d'un m\^eme caract\`ere facilitant ainsi la s\'election par colonne
			de la commande \verb_cut_.
				
            \end{Exercice}
        
			
	\begin{Exercice}{Droits sur les dossiers personnels} 
			Trouvez un enchainement de commandes qui permet de donner le nombre de professeurs
			qui ont donn\'e le droit \`a ceux qui ne font pas partie de leur groupe d'entrer dans leur dossier personnel.
				
            \par
			\`A votre imagination...
				
            \end{Exercice}
   		
				
				
\end{document}
			
