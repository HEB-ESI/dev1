\documentclass[a4paper,11pt]{article}

%=========================
% Les styles
%=========================
\usepackage{style-esi/french}	% Francise LaTeX
\usepackage{style-esi/td}
\usepackage{style-esi/licence}	% Affiche une licence dans le document
\usepackage{style-esi/exercice}
\usepackage{style-esi/listing}
\usepackage{style-esi/tutoriel}
%\marginsectiontrue
\usepackage{booktabs}
\usepackage{pifont}  



\definecolor{verylightgray}{rgb}{0.98,0.98,0.98}

\date{2018 -- 2019}
\siglecours{DEV1}
\libellecours{Laboratoires d'environnement}
\libelledocument{TD 03 -- Exercices pr\'eparatoires }
\sigleprof{}



\begin{document}

\entete
\titre
\ccbysa{esi-dev1-list@he2b.be}
\lastedit


	%===================
	%  Contenu
	%====================
	\begin{consigne}
	Afin d'arriver au laboratoire dans les meilleures conditions, il est bien de revoir la mati\`ere qui sera mise en pratique.
         C'est pourquoi nous vous fournissons quelques exercices pr\'eparatoires \`a faire \`a la maison pour vous permettre d'\'evaluer si vous \^etes pr\^et.  
	\end{consigne}
	
	\tableofcontents

	\newpage


				
%%%%%%%%%%%%%%%%%%%%%%%%%%%%  
 \section{Exercices pr\'eparatoires au TD3} 
 %%%%%%%%%%%%%%%%%%%%%%%%%%%%
  	Ces exercices pr\'eparatoires sont compos\'es d'une r\'evision des tds pr\'ec\'edents et de la th\'eorie \`a revoir.  
	
				
%%%%%%%%%%%%%%%%%%%%%%%%%%%%%          
        \subsection{Rappel des tds pr\'ec\'edents}
%%%%%%%%%%%%%%%%%%%%%%%%%%%%%			
		\begin{Exercice}{Les commandes linux} 
            		Quelle commande linux permet de faire l'action suivante ?
			\begin{itemize}
				\item se d\'econnecter  \textcolor{gray}{\underline{\hspace*{3em}}} 
				\item changer son mot de passe  \textcolor{gray}{\underline{\hspace*{5em}}} 
				\item nettoyer l'\'ecran  \textcolor{gray}{\underline{\hspace*{3em}}} 
				\item voir le contenu d'un dossier au format long \textcolor{gray}{\underline{\hspace*{2em}}}  \textcolor{gray}{\underline{\hspace*{2em}}} 
				\item cr\'eer un r\'epertoire  \textcolor{gray}{\underline{\hspace*{3em}}} 
				\item d\'eplacer un fichier  \textcolor{gray}{\underline{\hspace*{2em}}} 
				\item connaitre les groupes d'un utilisateur  \textcolor{gray}{\underline{\hspace*{5em}}} 
				\item modifier le groupe d'un fichier  \textcolor{gray}{\underline{\hspace*{3em}}} 
				\item modifier le propri\'etaire d'un fichier (par le super-utilisateur!) \textcolor{gray}{\underline{\hspace*{3em}}} 
				\item retirer la permission de traverser le r\'epertoire \verb@tds@ \`a tous \textcolor{gray}{\underline{\hspace*{3em}}}  \textcolor{gray}
					{\underline{\hspace*{2em}}}  \textcolor{gray}{\underline{\hspace*{2em}}} 
				\item placer le fichier \verb@Hello.java@ dans le groupe \verb@etudiants1@ \textcolor{gray}{\underline{\hspace*{3em}}}  \textcolor{gray}	
					{\underline{\hspace*{10em}}}  \textcolor{gray}{\underline{\hspace*{10em}}} 
			\end{itemize}
				
		\end{Exercice}	
		
		\begin{Exercice}{Les permissions} 
            		Remplissez les blancs avec la permission minimale correcte (r, w, x ou -),
			\begin{enumerate}
				\item pour que le r\'epertoire \verb@/home/gxxxxx/td3@ permette \`a un autre \'etudiant d'y cr\'eer le fichier  \verb@/home/gxxxxx/td3/fichier@\par
				 \textcolor{gray}{\underline{\hspace*{1em}}}  \textcolor{gray}{\underline{\hspace*{1em}}}  \textcolor{gray}{\underline{\hspace*{1em}}}  \textcolor{gray}	
				 {\underline{\hspace*{1em}}}  \textcolor{gray}{\underline{\hspace*{1em}}}  \textcolor{gray}{\underline{\hspace*{1em}}}  \textcolor{gray}
				 {\underline{\hspace*{1em}}}  \textcolor{gray}{\underline{\hspace*{1em}}}  \textcolor{gray}{\underline{\hspace*{1em}}} 
				 
				\item pour que le r\'epertoire \verb@/home/gxxxxx/td3@ permette \`a un autre \'etudiant d'acc\'eder au fichier \verb@/home/gxxxxx/td3/fichier@
					dont il connait le chemin
					\par
					 \textcolor{gray}{\underline{\hspace*{1em}}}  \textcolor{gray}{\underline{\hspace*{1em}}}  \textcolor{gray}{\underline{\hspace*{1em}}}  \textcolor{gray}
				 	{\underline{\hspace*{1em}}}  \textcolor{gray}{\underline{\hspace*{1em}}}  \textcolor{gray}{\underline{\hspace*{1em}}}  \textcolor{gray}
				 	{\underline{\hspace*{1em}}}  \textcolor{gray}{\underline{\hspace*{1em}}}  \textcolor{gray}{\underline{\hspace*{1em}}} 
			\end{enumerate}
				
		\end{Exercice}
		
			
		\begin{Exercice}{ Modifiez les permissions} 
		        \begin{itemize}
				\item pour que le fichier \verb@/home/gxxxxx/td3/fichier@ puisse \^etre lu et modifi\'e par votre professeur et vous m\^eme mais seulement lu par les autres 						\'etudiants 
					\par
					 \textcolor{gray}{\underline{\hspace*{3em}}}  \textcolor{gray}{\underline{\hspace*{1em}}}  \textcolor{gray}{\underline{\hspace*{1em}}}  \textcolor{gray}	
					 {\underline{\hspace*{1em}}}  \textcolor{gray}{\underline{\hspace*{16em}}} 
				\item \`A quel groupe ce fichier doit-il appartenir ?
					\par
				 	\textcolor{gray}{\underline{\hspace*{10em}}} 
					
				\item Quelle commande permet de modifier le groupe du fichier afin de l'adapter \`a ce qui est demand\'e ci-dessus ?
					\par \textcolor{gray}{\underline{\hspace*{3em}}}  \textcolor{gray}{\underline{\hspace*{10em}}} 
			\end{itemize}
				
		\end{Exercice}	
		\clearpage
		
		\begin{Exercice}{Chemins absolus et relatifs} 
            		Parmi les propositions suivantes, lesquelles repr\'esentent des chemins absolus ?
           		 \begin{itemize} 
        
           			 \item[ \ding{"6F} ] \verb@/usr/local/java/@
        
           			 \item[ \ding{"6F} ] \verb@/home/g31000/td3@
        
           			 \item[ \ding{"6F} ] \verb@g31000/td3@
        
           			 \item[ \ding{"6F} ] \verb@~/td3@
        
           			 \item[ \ding{"6F} ] \verb@td3@
        
          			  \item[ \ding{"6F} ] \verb@~g31000/td3@
        
           		 \end{itemize} 
		 \end{Exercice}
%        \subsection{Survol du langage}
%			
%		\subparagraph{Structure d'un programme} 
%		
%                \textcolor{white}{.} \par
%            Quelles structures g\'en\'erales d'un programme sont correctes 
%							(c-\`a-d qu'il doit compiler et respecter les conventions) parmi les suivantes ?
%						
%            \begin{itemize} 
%        
%            \item[ \ding{"6F} ]  
%							code 1
%							\begin{verbatim}
%    public class exercice {
%        // put methods here  
%    }\end{verbatim}
%        
%            \item[ \ding{"6F} ]  
%							code 2
%							\begin{verbatim}
%    public CLASS Exercice {
%        // put methods here  
%    }\end{verbatim}
%        
%            \item[ \ding{"6F} ] 
%							code 3 
%							\begin{verbatim}
%    public class Exercice {  
%        // put methods here  
%    }\end{verbatim}
%        
%            \item[ \ding{"6F} ]  
%							code 4 
%							\begin{verbatim}
%    public class MonExercice {  
%        // put methods here  
%    }\end{verbatim}
%        
%            \item[ \ding{"6F} ]  
%							code 5 
%							\begin{verbatim}
%    PUBLIC CLASS EXERCICE { 
%        // put methods here  
%    }\end{verbatim}
%        
%            \end{itemize} 
%        
%			
%		\subparagraph{Nom d'un programme} 
%		
%                \textcolor{white}{.} \par
%            Quels noms doivent avoir les fichiers dans lesquels sont plac\'es les programmes suivants :\begin{Java}
%    public class Exercice {  
%        // Methods    
%    }						\end{Java} \textcolor{gray}{\underline{\hspace*{10em}}} \begin{Java}
%    public class SommeChiffres {
%        // Methods    
%    }						\end{Java} \textcolor{gray}{\underline{\hspace*{16em}}} \begin{Java}
%    public class sommechiffres {    
%        // Methods    
%    }						\end{Java} \textcolor{gray}{\underline{\hspace*{16em}}} 
%			
%		\subparagraph{Compiler/ex\'ecuter} 
%		
%                \textcolor{white}{.} \par
%             
%							Quelle commande permet de compiler le fichier nomm\'e \verb@SommeChiffres.java@ ?  
%							\par
%				 \textcolor{gray}{\underline{\hspace*{3em}}}  \textcolor{gray}{\underline{\hspace*{16em}}} \par
%				
%							Quelle commande permet d'ex\'ecuter ce programme ?  
%							\par
%				 \textcolor{gray}{\underline{\hspace*{3em}}}  \textcolor{gray}{\underline{\hspace*{10em}}} 
%			
%		\subparagraph{M\'ethode principale} 
%		
%                \textcolor{white}{.} \par
%            
%								Comment s'\'ecrit l'ent\^ete de la m\'ethode principale (1 mot par case) ?
%							
%            \par
%         \textcolor{gray}{\underline{\hspace*{5em}}}  \textcolor{gray}{\underline{\hspace*{5em}}}  \textcolor{gray}{\underline{\hspace*{3em}}}  \textcolor{gray}{\underline{\hspace*{3em}}}  
%							(            
%							 \textcolor{gray}{\underline{\hspace*{5em}}}  \textcolor{gray}{\underline{\hspace*{2em}}}  \textcolor{gray}{\underline{\hspace*{3em}}}  
%							)   

%%%%%%%%%%%%%%%%%%%%%%%%%%%
	\subsection{Th\'eorie} 
%%%%%%%%%%%%%%%%%%%%%%%%%%%%
		Avant de venir au prochain labo, lisez attentivement le point 20 du guide visuel.  
				
            	\par
        
			
		\begin{Exercice}{Joker} 
            		\begin{enumerate}
				\item La commande \verb@rm td*.java@ supprime le(s) fichier(s) :
						
            				\begin{itemize} 
        
           					 \item[ \ding{"6F} ]  td.java
						
        
           					 \item[ \ding{"6F} ]  td2
						
        
           					 \item[ \ding{"6F} ]  td2.java
						
        
           					 \item[ \ding{"6F} ]  td3Prepa.java
						
        
           					 \item[ \ding{"6F} ]  td3.java
						
        
            					\item[ \ding{"6F} ]  td10.java
						
        
           				 \end{itemize} 
        
		
				\item La commande \verb@rm td?.java@ supprime le(s) fichier(s) :
						
           				 \begin{itemize} 
        
           					 \item[ \ding{"6F} ]  td.java
						
        
           					 \item[ \ding{"6F} ]  td2
						
        
           					 \item[ \ding{"6F} ]  td2.java
						
        
            					\item[ \ding{"6F} ]  td3Prepa.java
						
        
           					 \item[ \ding{"6F} ]  td3.java
						
        
           					 \item[ \ding{"6F} ]  td10.java
						
        
           				 \end{itemize} 
		\end{enumerate}
            \end{Exercice}
%        \subsection{Algorithmes}  
%					Assurez-vous de venir au laboratoire 
%					avec vos solutions des exercices 
%					de votre cours d'algorithmique.  
				
           
        
\end{document}
			