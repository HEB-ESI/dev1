\documentclass[a4paper,11pt]{article}

%=========================
% Les styles
%=========================
\usepackage{style-esi/french}	% Francise LaTeX
\usepackage{style-esi/td}
\usepackage{style-esi/licence}	% Affiche une licence dans le document
\usepackage{style-esi/exercice}
\usepackage{style-esi/listing}
\usepackage{style-esi/tutoriel}

\date{2018 -- 2019}
\siglecours{DEV1}
\libellecours{Laboratoires Java I}
\libelledocument{TD 5 -- Méthodes}
\sigleprof{}



\begin{document}

\entete
\titre
\ccbysa{esi-dev1-list@he2b.be}
\lastedit


	%===================
	%  Contenu
	%====================	
	Ce TD introduit les notions de méthode, entête et déclaration, paramètre, et valeur de retour, l'instruction \texttt{return}
	 ainsi que le type \texttt{boolean}.
	 
	\tableofcontents

	\newpage

%===================
\section{Introduction}
%====================	

	Une méthode est une construction qui permet de décomposer une solution sous forme de petit module.
	Dans l'exemple ci-dessous la méthode \texttt{périmètre} de la classe \texttt{Cercle} calcule le périmètre d'un cercle 
	étant donné son rayon.
	
	
 	\listing{java}{code/Cercle.java}

	\emph{L'entête}\footnote{aussi appelée \emph{déclaration} de la méthode. } de cette méthode, \code{java}{static double périmètre(double rayon)},  nous signale que:
	\begin{itemize}
		\item cette méthode s'appelle \texttt{périmètre}~;
		\item elle reçoit en paramètre un \texttt{double} qui sera manipulé par son nom \texttt{rayon}~;
		\item elle retourne une valeur de type \texttt{double}.
	\end{itemize}
	Nous verrons ultérieurement la signification du mot clef \texttt{static}, qui doit toujours se trouver au début de l'entête des méthodes
	que vous écrivez. Une méthode non-static a un tout autre sens que nous verrons en dev2.

	Le \emph{corps} de la méthode est ici très simple, il \emph{retourne} (à l'aide de l'instruction  \texttt{return})
	 la valeur reçue en paramètre (le rayon) multiplié par 2*3.141596.
	Le type de la valeur retournée doit être compatible avec le type déclaré dans l'entête, 
	ici cela doit être un \texttt{double}.	
	
	La méthode principale \emph{appelle}  cette méthode avec la valeur 10 comme rayon. 


	 \begin{Exercice}{Périmètre et aire d'un cercle}
		
		Créer un package \texttt{g12345.dev1.td5} et, dans celui-ci, une classe \texttt{MathUtil}.
		Dans la classe \texttt{MathUtil} écrivez les 2 méthodes suivantes: 
		\begin{itemize}
			\item	\code{java}{double périmètreCercle(double rayon)} 
				qui reçoit le rayon (un double) en paramètre et retourne le périmètre du cercle.
			\item \code{java}{double aireCercle(double rayon)} 
				qui reçoit un rayon en paramètre et retourne l'aire du cercle.
		\end{itemize}
		
		Rappel: l'aire d'un cercle se calcule par la formule: $\pi r^2$ ou  $r$ est le rayon du cercle.
		
		Attention: n'oubliez pas le mot-clef \texttt{static} au début de l'entête de vos méthodes.
		
		Dans la méthode principale testez ces 2 méthodes comme cela est fait dans l'exemple ci-dessus avec la méthode 
		\texttt{périmètre}.
	\end{Exercice} 


%===================
\section{Appel d'une méthode}
%====================	

	Dans l'exemple ci-dessus la méthode principale fait appel à la méthode périmètre en utilisant le nom 
	et en fournissant une valeur pour le paramètre~: 
	\code{java}{périmètre(10)}.
	
	Il est possible de faire appel à une méthode depuis une \emph{autre classe}.
	
 	\listing{java}{code/CercleApp.java}

	La méthode principale de la classe \texttt{CercleApp} fait appel à la méthode \texttt{périmètre}
	de la classe \texttt{Cercle} en utilisant le nom de la classe suivi d'un point et du nom de la méthode: \texttt{Cercle.périmètre(10)}.
	
	 \begin{Exercice}{Périmètre et aire d'un cercle}
		
		Dans une classe \texttt{CercleApp} écrivez un programme (et donc une méthode \texttt{main})
		qui demande à l'utilisateur le rayon d'un cercle et affiche son périmètre et son aire.
		
		Votre programme fera appel aux méthodes \texttt{périmètreCercle} et \texttt{aireCercle} de la classe
		\texttt{MathUtil} écrites précédemment.
	\end{Exercice} 
	

%===================
\section{Méthodes à plusieurs paramètres}
%====================	

 	\listing{java}{code/Maximum.java}
	
	Dans la classe \texttt{Maximum} ci-dessus sont définies 2 méthodes: \texttt{max2} et \texttt{max3}.
	Dans l'entête de la méthode \texttt{max2} on constate que cette méthode 
	\begin{itemize}
		\item s'appelle \texttt{max2}~;
		\item reçoit \emph{en paramètres} deux entiers, ces entiers seront manipulés grâce à leur nom \texttt{a} et \texttt{b}~;
		\item \emph{retourne} un entier.
	\end{itemize}
		

	La méthode principale fait un appel à la méthode \texttt{max2} avec les paramètres effectifs 10 et 6 
	et affiche la valeur retournée par la méthode max2: 10.
	
	 Dans l'entête de la méthode \texttt{max3} on constate que cette méthode 
	\begin{itemize}
		\item s'appelle \texttt{max3}~;
		\item reçoit \emph{en paramètres} trois entiers, ces entiers seront manipulés grâce à leur nom \texttt{a}, \texttt{b} et \texttt{c}~;
		\item \emph{retourne} un entier.
	\end{itemize}

	Cette méthode fait appel par deux fois à la méthode \texttt{max2}.
	
	La méthode principale fait un appel à la méthode \texttt{max3} avec les paramètres effectifs 10, 6 et 19 
	et affiche la valeur retournée: 19.

 
 	\begin{Exercice}{Minimum}
		Dans la classe \texttt{MathUtil} écrivez les 2 méthodes suivantes~:		
		\begin{itemize}
			\item \code{java}{double min2(double x, double y)} qui reçoit deux doubles en paramètres et retourne le minimum.
			\item \code{java}{double min3(double x, double y, double z)} qui reçoit trois doubles en paramètres et retourne le minimum.
		\end{itemize}
		
		Dans la méthode principale testez ces 2 méthodes comme cela est fait dans l'exemple ci-dessus avec les fonctions 
		\texttt{max2} et \texttt{max3}.
	\end{Exercice} 
		
		

	 \begin{Exercice}{Moyenne}
		Dans la classe \texttt{MathUtil} ajoutez la méthode 
		\texttt{double moyenne(double x, double y)} qui reçoit deux doubles en paramètres et retourne leur moyenne.
		
		Par exemple si la méthode reçoit 10.5 et 15.5 elle retourne 13.
		
		Testez-la dans la méthode principale.
	\end{Exercice} 

\section{Méthodes et chaînes de caractères}

	Les méthodes peuvent traiter n'importe quel type de données.
	Les méthodes de cette section traitent des chaînes de caractères. 

 	\listing{java}{code/Mot.java}

	La méthode \texttt{premièreLettre} reçoit en paramètre une chaine de caractères et retourne la première lettre de ce mot.

	\begin{Exercice}{Première et dernière lettre}
		Créez une classe \texttt{ChaineUtil} et ajoutez-y les méthodes suivantes~: 
		\begin{itemize}
			\item \code{java}{char premièreLettre(String mot)} 
				qui reçoit une chaîne de caractères en paramètre et retourne sa première lettre.
				
				Par exemple si la méthode reçoit "Java" elle retournera 'J'.
			\item \code{java}{char dernièreLettre(String mot)} 
				qui reçoit une chaîne de caractères en paramètre et retourne sa dernière lettre.
				
				Par exemple si la méthode reçoit "Java" elle retournera 'a'.
		\end{itemize}
		
		Testez-les dans la méthode principale.
	\end{Exercice} 



\section{Méthodes et booléens}

	Nous avons vu que les \emph{booléens} sont les deux valeurs \texttt{true} (vrai) et \texttt{false} (faux). 
	Le type associé aux booléens est \texttt{boolean}.
	
	Une méthode retournant vrai ou faux sera de type \texttt{boolean} comme illustré dans l'exemple suivant.
	 	\listing{java}{code/Pair.java}
	La méthode \texttt{estPair} retourne la valeur de l'expression \code{java}{(nb\%2)==0}, 
	cette expression est vraie si \texttt{nb} est pair et faux sinon.
	
	La condition de l'instruction \texttt{if}, \code{java}{estPair(10)}, sera vraie si l'appel à la méthode retourne vrai (et donc si \texttt{10} est pair)
	 et faux sinon.

	 \begin{Exercice}{Divisible}
		Dans la classe \texttt{MathUtil} ajoutez la méthode 
		\code{java}{boolean estDivisible(int a, int b)} qui reçoit deux entiers en paramètres et 
		retourne vrai si le premier est divisible par le second.
		
		Par exemple si la méthode reçoit 10 et 5 elle retourne \texttt{true}, car 10 est divisible par 5.
		
		Rappel: a est divisible par b si l'expression \texttt{a\%b==0} est vraie.
		
		Testez-la dans la méthode principale.
	\end{Exercice} 

%===================
\section{Méthodes sans retour}
%====================	

 	\listing{java}{code/Ligne.java}
	
	Dans la classe \texttt{Ligne} ci-dessus la méthode \texttt{afficherLigne(int longueur)}

	TODO

	
\section{Exercices récapitulatifs}
				
	 \begin{Exercice}{Valeur absolue}
		
		Dans la classe \texttt{MathUtil} ajoutez la méthode 
		\texttt{abs} qui reçoit un entier en paramètre et retourne sa valeur absolue.
		
		Par exemple si la méthode reçoit -4 elle retourne 4, et si elle reçoit 10 elle retourne 10.
		
		Testez-la dans la méthode principale avec une valeur négative et ensuite une valeur positive.
		
%		Astuce: utilisez l'instruction \texttt{if} dans votre méthode. 
	\end{Exercice} 
					
	\begin{Exercice}{Décomposition, recomposition}
		Dans la classe \texttt{MathUtil} ajoutez les méthodes suivantes~: 
		\begin{itemize}
			\item \code{java}{int unité(int nb)} qui reçoit un entier en paramètre et retourne la valeur des unités de cet entier.
				Par exemple si la méthode reçoit 123 elle retournera 3.
			\item \code{java}{int dizaine(int nb)} qui reçoit un entier en paramètre et retourne la valeur des dizaines de cet entier.
				Par exemple si la méthode reçoit 123 elle retournera 2.
			\item \code{java}{int centaine(int nb)} qui reçoit un entier en paramètre et retourne la valeur des centaines de cet entier.
				Par exemple si la méthode reçoit 123 elle retournera 1.
			\item \code{java}{int miroir(int nb)} qui reçoit un nombre compris entre 100 et 999 et retourne son miroir.
				Par exemple si la méthode reçoit 123 elle retournera 321.
				
				Astuce: utilisez judicieusement les méthodes \texttt{unité}, \texttt{dizaine} et \texttt{centaine}.
		\end{itemize}
		Testez-les dans la méthode principale.
	\end{Exercice} 
	
	\begin{Exercice}{Voyelles et consonnes}
		Dans la classe \texttt{ChaineUtil} ajoutez les méthodes suivantes~: 
		\begin{itemize}
			\item \code{java}{int nombreVoyelles(String mot)} 
				qui reçoit une chaîne de caractères en paramètre et retourne le nombre de voyelles de ce mot.
				
				Par exemple si la méthode reçoit "Programmation" elle retournera 5.
			\item \code{java}{int nombreConsonnes(String mot)} 
				qui reçoit une chaîne de caractères en paramètre et retourne le nombre de consonnes de ce mot.
				
				Par exemple si la méthode reçoit "Programmation" elle retournera 8.
		\end{itemize}
		Testez-les dans la méthode principale.
	\end{Exercice} 

	 \begin{Exercice}{Palindrome}
		Dans la classe \texttt{ChaineUtil} ajoutez la méthode 
		\code{java}{boolean estPalindrome(String mot)} qui reçoit une chaîne de caractères en paramètre et 
		retourne vrai si cette chaîne est un palindrome.
		
		Par exemple si la méthode reçoit "été" elle retourne \texttt{true}.
		
		Testez-la dans la méthode principale avec un palindrome et ensuite avec un mot qui n'en est pas un.
	\end{Exercice} 





\end{document}