\documentclass[a4paper,11pt]{article}

%=========================
% Les styles
%=========================
\usepackage{style-esi/french}	% Francise LaTeX
\usepackage{style-esi/td}
\usepackage{style-esi/licence}	% Affiche une licence dans le document
\usepackage{style-esi/exercice}
\usepackage{style-esi/listing}
\usepackage{style-esi/tutoriel}
\usepackage{amsmath}
\usepackage{amssymb}

\date{2018 -- 2019}
\siglecours{DEV1}
\libellecours{Laboratoires Java I}
\libelledocument{TD 13 -- Mise en pratique \\ La gestion des clients (ébauche)}
\sigleprof{}



\begin{document}

\entete
\titre
\ccbysa{esi-dev1-list@he2b.be}
\lastedit

\vspace{0.5cm}

	%===================
	%  Contenu
	%====================	
	
\section{Introduction}

	Dans ce TD, on vous propose de mettre en pratique les notions vues auparavant pour créer un programme gérant l'enregistrement de nouveaux clients d'une entreprise (imaginaire). De nombreuses données différentes peuvent être associées à un client d'une entreprise donnée: son identifiant (numéro d'identification au sein de l'entreprise), son nom, son numéro de compte courant, son adresse postale, son adresse e-mail, sa date de naissance, sa nationalité, .... Pour simplifier les choses dans ce TD, nous considérerons uniquement les identifiants, noms, prénoms et l'adresse e-mail.
Nous considérerons qu'un identifiant client est simplement une suite de $10$ chiffres (dans l'intervalle \path{0-9}).

Cet encodage au clavier n'aurait pas beaucoup d'int\'er\^et si les donn\'ees ne sont pas enregistr\'ees \`a la fin de l'ex\'ecution du programme. Dans un syst\`eme professionnel de gestion de clients, celles-ci sont le plus souvent enregistr\'ees dans une base de donn\'ee. Ici, nous simplifierons les choses et enregistrerons simplement les donn\'ees dans un fichier plat (un fichier texte).

\section{D\'eveloppement}
	
	
	%Dans ce TD vous utiliserez les notions vues précédemment afin de réaliser 
	%le jeu 'le mot le plus long'. 
	
	%Le mot le plus long est un jeu de lettres à deux joueurs qui se déroule en
	%plusieurs manches. Le nombre de manches est choisi au départ, par exemple 5 manches.
	%Une manche se déroule en 2 étapes. 
	
	%Tout d'abord les joueurs, chacun à leur tour, 
	%tirent au hasard soit une voyelle, soit une consonne (le joueur choisit), 
	%jusqu'à obtenir 9 lettres. A chaque fois qu'une lettre est tirée 
	%(voyelle ou consonne), la lettre est dévoilée aux joueurs afin qu'ils puissent
	%continuer en connaissance de cause.

	%Ensuite les joueurs cherchent le mot le plus long avec les lettres disponibles.
	%Chaque joueur annonce la longueur du mot trouvé. Celui qui a trouvé le mot le 
	%plus long le forme avec les lettres disponibles. On vérifie alors que c'est 
	%bien un mot du dictionnaire. Le joueur gagne autant de points que le nombre
	%de lettres de son mot. 
			

\vspace{0.3cm}


%Nous allons vous guider tout au long de ce TD afin de développer les différentes méthodes nécessaires pour une version simplifiée à un seul joueur de ce jeu.



		%Créer un package \texttt{g12345.dev1.lemotlepluslong} et une classe
		 %\texttt{LeMotLePlusLong}. Créer la méthode principale qui, dans un premier 
		%temps, affiche un message de bienvenue au 'mot le plus long'.


 	\begin{Exercice}{Créer la structure Client}
		D\'eclarer une nouvelle structure \path{Client} comportant les champs repris dans l'introduction: identifiant, nom, prénom, adresse-e-mail.
		
		Vous prendrez soin de d\'eclarer un constructeur appropri\'e ainsi qu'une m\'ethode \path{toString()} renvoyant une description du client (il suffit pour ce faire de concat\'ener les diff\'erents champs les uns \`a la suite des autres).
		
	\end{Exercice} 

 
 	\begin{Exercice}{Initialiser un tableau de clients}
		 \`Ecrire une classe \path{GestionClients} contenant une m\'ethode \path{main} dans laquelle on demande \`a l'utilisateur le nombre de nouveaux clients qu'il souhaite encoder. Si ce nombre est $n$, le programme initialisera alors un tableau de $n$ clients. Pour ce faire, une référence vers le tableau contenant les clients peut-être déclarée dans la classe:\newpage
		 
		
		\begin{Code}{Java} 
			public class GestionClients
			{
				// Déclare une référence destinée à contenir un tableau de structures Client.
				static Client [] clients;
				
				public static void main (String args[])
				{
					//Assigner un nouveau tableau de taille n à la variable clients ici.
					// n est choisi par l'utilisateur.
				}
			}
		\end{Code}
			
	\end{Exercice}

 	\begin{Exercice}{Encoder les différents clients dans le tableau}
		Apr\`es avoir créer un nouveau tableau de $n$ clients, if faut initialiser chaque case du tableau avec un nouveau client dont les données sont entrées au clavier par l'utilisateur.
		Pour ce faire, le programme demandera $n$ fois à l'utilisateur d'entrer un nom, un prénom, un identifiant et une adresse e-mail.		
		
  \end{Exercice}
  
  	\begin{Exercice}{Afficher les clients encod\'es}
	\`Ecrire une m\'ethode
	
		\begin{Code}{Java}
		    static void afficher()
		\end{Code}
	qui affiche \`a l'\'ecran, pour chaque client du tableau \path{clients[]} les donn\'es du clients. Cette m\'ethode fera appel a la m\'ethode \path{toString()} de la classe \path{Client}.
	\end{Exercice}

	\begin{Exercice}{V\'erifier le format de l'adresse e-mail}
		L'utilisateur peut se tromper en entrant les donn\'ees du client. En particulier, une adresse e-mail doit \^etre au format 
		
		\path{s1@s2.s3}
		
		ou s1,s2,s3 sont des cha\^ines de caract\`ere. \`Ecrire une m\'ethode 
		
		\begin{Code}{Java}
		boolean verifierEmail(String email)
		\end{Code}
		
		qui renvoie \path{true} si email a le format d'une adresse e-mail et \path{false} sinon.
		Mettre la fonction \path{main} \`a jour: si l'utilisateur entre une mauvaise adresse pour un certain client, le programme refuse de cr\'eer le nouveau client et lui demande de corriger      l'adresse entr\'ee.
	\end{Exercice}

\begin{Exercice}{V\'erifier l'absence de doublons}
		Bien que ce soit rare, il est possible que deux clients aient le m\^eme nom et le m\^eme pr\'enom. Par contre, ils ne peuvent avoir le m\^eme identifiant au sein de la soci\'et\'e. Ecrire une m\'ethode
		
		\begin{Code}{Java}
		boolean verifierID(String identifiant)
		\end{Code}
		
qui renvoie \path{true} si l'identifiant que l'utilisateur vient de rentrer n'a pas d\'eja \'et\'e utilis\'e et \path{false} sinon. Adapter la m\'ethode \path{main} de sorte que l'utilisateur ne puisse jamais encoder deux clients avec le m\^eme identifiant. \newpage

\end{Exercice}

\begin{Exercice}{\'Ecrire les donn\'ees dans un fichier}

Il existe plusieurs fa\c cons d'\'ecrire des donn\'ees dans un fichier en Java. Celle que nous vous proposons utilise la classe \path{FileWriter}:\\

\begin{Code}{Java}

import java.io.FileWriter;

public class Td13
{    
    static void Ecrire () throws Exception
    {
		// On va écrire dans le fichier Truc.txt
		// Si le fichier n'existe pas il est créé, sinon il est écrasé
		FileWriter w = new FileWriter ("Truc.txt");
		w.write("Turlututu chapeau pointu !");
        
        // Une fois qu'on a fini d'écrire, on ferme le fichier.
        w.close();
        
    }
}

\end{Code}

Vous aurez remarqué la ligne \begin{Code}{Java} 
throws Exception 
\end{Code}
Celle-ci signifie simplement que notre méthode Ecrire peut générer des exceptions. En effet, si, pour une raison ou pour une autre, le fichier "Truc.txt" ne peut \^etre ouvert ou qu'on ne peut \'ecrire dedans (nous n'avons pas les droits d'acc\`es) alors \path{FileWriter} renvoie un exception à la m\'ethode Ecrire qui doit se charger de la gérer. Le m\'ecanisme de gestion des exceptions en Java sera vu dans un cours ult\'erieur. Ici, on se contente de dire \` a Java qu'on ne g\`ere pas l'exception explicitement. \\

Ecrire une m\'ethode:

\begin{Code}{Java} 
static void WriteData (String filename) throws Exception
\end{Code}

qui \'ecrira toute les données du tableau \path{clients[]} dans le fichier dont le nom est sp\'ecifi\'e dans \path{filename}. Mettre \`a jour la m\'ethode \path{main} pour demander un nom de fichier a l'utilisateur et valider l'\'ecriture du fichier de donn\'ees.





\end{Exercice}
	

	
\end{document}


