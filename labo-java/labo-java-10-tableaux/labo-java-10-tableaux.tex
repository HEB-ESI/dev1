\documentclass[a4paper,11pt]{article}

%=========================
% Les styles
%=========================
\usepackage{style-esi/french}	% Francise LaTeX
\usepackage{style-esi/td}
\usepackage{style-esi/licence}	% Affiche une licence dans le document
\usepackage{style-esi/exercice}
\usepackage{style-esi/listing}
\usepackage{style-esi/tutoriel}


\newcommand{\publicbasepath}{https://git.esi-bru.be/dev1/labo-java/tree/master/td10-tableaux}
\renewcommand{\listingpublicpath}{\publicbasepath/code/}
\renewcommand{\listingsrcpath}{code/}

\date{2018 -- 2019}
\siglecours{DEV1}
\libellecours{Laboratoires Java I}
\libelledocument{TD 10 -- Tableaux}
\sigleprof{}

\marginnumberfalse
\marginsectiontrue


\begin{document}

\entete
\titre
\ccbysa{esi-dev1-list@he2b.be}
\lastedit


	%===================
	%  Contenu
	%====================	
	Dans ce TD vous trouverez une introduction aux tableaux.
	 
	Les codes sources et les solutions de ce TD se trouvent à l'adresse~: 
	
	\url{\publicbasepath}	


	\tableofcontents

	\newpage

%===================
\section{Créer et manipuler des tableaux}
%====================	

 	\listing{java}{Tableau.java}

 
	\begin{Exercice}{\texttt{TableauUtil}}	
		Dans une classe \texttt{TableauUtil} écrivez les méthodes suivantes et leur 
		javadoc~:
		\begin{itemize}
			\item \code{java}{static double min(double[] tab)} qui retourne le minimum
				du tableau passé en paramètre ~;
			\item \code{java}{static double max(double[] tab)} qui retourne le maximum
				du tableau~;
			\item \code{java}{static double somme(double[] tab)} qui retourne la somme
				des éléments du tableau~;
			\item \code{java}{static double moyenne(double[] tab)} qui retourne la
				 moyenne des éléments du tableau~;
		\end{itemize}
	\end{Exercice}

	\begin{Exercice}{Initialisation par défaut}	
		Dans la méthode principale d'une classe \texttt{TestInit}~:
		\begin{itemize}
			\item créer un tableau d'entier de taille 10 et afficher-le 
				(comme à  la ligne 18 de la classe \texttt{Tableau} ci-dessus). 
				
				Quelle valeur par défaut initialise chacune des cases du tableau ?
			\item créer un tableau de double de taille 10 et afficher-le~;
			\item créer un tableau de 10 booléens et afficher-le~;
			\item créer un tableau de 10 \texttt{String} et afficher-le~;
		\end{itemize}
	\end{Exercice}

\section{Tableaux de différents types}
	Les tableaux peuvent contenir des éléments de tous les types, 
	mais un tableau contient toujours des éléments du même type.
	Par exemple un tableau d'entiers ne contient que des entiers,
	un tableau de \texttt{String} ne contient que des \texttt{String}.

	\begin{Exercice}{\texttt{Tableau de \texttt{String}}}	
		Créer un programme qui demande initialise un tableau de \texttt{String}
		avec les chaines suivantes: "		
	\end{Exercice}
	
	

\section{Tableaux et tests}

 	\listing{java}{TableauTest.java}

	\begin{Exercice}{Tests}	
		Créer des tests JUnit pour les méthodes de l'exercice 1. Pour chacune 
		de ces méthodes testez un cas général ainsi que les cas limites (tableau vide,
		maximum en début/fin de tableau, valeurs négatives, etc).
	\end{Exercice}


	


\section{Retourner un tableau}

\section{Tableaux et \texttt{String}}

\section{Exercices récapitulatifs}

	\begin{Exercice}{\texttt{TableauUtil} (suite)}	

		Dans la classe \texttt{TableauUtil} ajoputez les méthodes suivantes, meur javadoc ainsi que les tests JUnit correspondant~:
	\begin{itemize}
		\item \code{java}{int indiceMax(double[] tab)}: 
				retourne l'indice du maximum du talbeau passé en paramètre~;
		\item \code{java}{boolean estTrié(double[] tab)}: 
				retourne vrai si le tableau est trié par ordre croissant et 	
				faux sinon~;
		\item \code{java}{int indice(int[] tab, int valeur)}: 
				retourne l'indice de cette valeur dans le tableau. 
				Que retournez-vous si la valeur n'apparait pas?
		\item  \code{java}{boolean contient(String[] tab, String mot)}: 
			retourne vrai si un tableau de \texttt{String} contient un mot passé en
			paramètre~;
%		\item \texttt{inverser}: retourne ~;
%		\item \texttt{indiceMax}: retourne l'indice du maximum d'un tableau de double~;
%		\item \texttt{indiceMax}: retourne l'indice du maximum d'un tableau de double~;
%		\item \texttt{indiceMax}: retourne l'indice du maximum d'un tableau de double~;
%	
%		\item créer
%		\item inverser
%		\item positionsMin
%		\item position(int valeur)
%		\item premièreOccurence
%		\item dernièreOccurence
%		\item échange(int position1, position2)
%		\item glissementCyclique
%		\item doublons ?
%		\item nb éléments négatifs
%		\item inverserSigne
%		\item contientChaine
%		\item tri
%		\item recherche dichotomique
%		\item recherche d'une chaine
	\end{itemize}
	\end{Exercice}
	
	A faire:	
\begin{itemize}
	\item tableaux de string
	\item remplissage de tableau par system.in
	\item retourner un tableau
	\item tableau et String (split, chars)
	\item tableau de booléens ?
\end{itemize}	

\end{document}