\documentclass[a4paper,11pt]{style-esi/td}

\usepackage{style-esi/licence}
\usepackage{style-esi/exercice}
\usepackage{style-esi/listing}
\usepackage{style-esi/tutoriel}
\usepackage{style/dev1}

\begin{document}

\seance{11}{Les tableaux}{td11-tableaux}{
	Dans ce TD vous trouverez une introduction aux tableaux.
}
	 
%===================
\section{Créer et manipuler des tableaux}
%====================	

	Un tableau est une collection indexée d'éléments de même type.
	Un tableau est créé avec une taille donnée et ne peut plus en changer 
	par la suite. Dans l'exemple ci-dessous nous créons et manipulons
	un tableau de taille 5 contenant des entiers.
	Comme cela est illustré dans l'exemple, on accède aux éléments 
	d'un tableau par son indice.

	\bigskip 

 	\listing{java}{Tableau.java}

 	\begin{Exercice}{Multiplier}	
		Reprenez le code ci-dessus et ajoutez-y une méthode 
		\code{java}{void multiplier(int[] tab, int valeur)} qui
		 multiplie chaque élément du tableau par la valeur donnée.
	\end{Exercice}
	
	\bigskip
	
	\begin{Exercice}{Afficher}
		Modifiez la méthode \texttt{afficherTab} afin que l'affichage soit par exemple pour un tableau contenant les valeurs 1, 2, 3, 4, 5:
		
		\texttt{[1, 2, 3, 4, 5]}
	\end{Exercice}


\section{Tableaux de différents types}
	Les tableaux peuvent contenir des éléments de tous les types.	
	L'exemple suivant illustre l'utilisation d'un tableau de \texttt{String}.

 	\listing{java}{TableauChaines.java}



	\begin{Exercice}{Les tailles}	
		Complètez l'exemple ci-dessus en y ajoutant une méthode 
		\code{java}{void afficherTailles(String[] tab)}
			qui affiche la taille de chaque \texttt{String} d'un
			 tableau. Par exemple si le tableau est celui de l'exemple la méthode 		
			 affiche 3 5 5 3 5 4 3 4 3.
	\end{Exercice}
	
	\begin{Exercice}{Le plus long mot}
		Toujours dans l'exemple, ajoutez
		 une méthode  \code{java}{String plusLongMot(String[] tab)}
			qui reçoit un tableau de \texttt{String} et retourne le 
			plus long mot. S'il y a plusieurs mots les plus longs, la méthode
			retournera n'importe lequel parmi ceux-ci. 
			Dans l'exemple ci-dessus la méthode retournera
			 \texttt{quick}, \texttt{brown} ou \texttt{jumps}.
				
	\end{Exercice}
	
	


\section{Retourner un tableau}

	Il est bien sûr possible qu'une méthode retourne un tableau.
	Dans l'exemple suivant la méthode \texttt{copie} 
	effectue une copie d'un tableau reçu en paramètre. 
	La méthode \texttt{pasUneCopie} illustre une erreur fréquente
	lorsque l'on débute dans l'apprentissage des tableaux en Java.
	
 	\listing{java}{TableauCopie.java}

	\begin{Exercice}{Miroir}	
		Complétez l'exemple ci-dessus en y ajoutant une méthode 
		\code{java}{int[] miroir(int[] tab)}
			qui retourne un nouveau tableau dont les éléments sont les 
			éléments du tableau reçu en paramètre mais en ordre inverse. 
			Par exemple si le tableau est celui de l'exemple la méthode 		
			 retourne un tableau contenant les entiers 5, 4, 3, 2, 1.
	\end{Exercice}


%\section{Tableaux et \texttt{String}}
%TODO v2 ? split, getCharArray

\section{Tableaux et tests}

	La manipulation des tableaux n'est pas aisée, les tests JUnit 
	sont un outil essentiel afin de s'assurer de la justesse de nos méthodes.
	
	La méthode \code{java}{assertArrayEquals} de la classe \texttt{org.junit.Assert}
	permet de s'assurer que deux tableaux sont strictement égaux: de même taille, 
	de même type et que tous leurs éléments sont égaux et 
	apparaissent dans le même ordre.

 	\listing{java}{TableauTest.java}

	\begin{Exercice}{Tests}	
		Créer des tests JUnit pour les méthodes des exercices 1, 4 et 5. Pour chacune 
		de ces méthodes testez un cas général ainsi que les cas limites (tableau vide,
		maximum en début/fin de tableau, valeurs négatives, etc).
	\end{Exercice}



\section{Exercices récapitulatifs}

	\begin{Exercice}{\texttt{TableauUtil}}	
		Dans une classe \texttt{TableauUtil} écrivez les méthodes suivantes, leur 
		javadoc ainsi que les tests JUnit correspondant~:
		\begin{itemize}
			\item \code{java}{double min(double[] tab)} qui retourne le minimum
				du tableau passé en paramètre ~;
			\item \code{java}{double max(double[] tab)} qui retourne le maximum
				du tableau~;
			\item \code{java}{double somme(double[] tab)} qui retourne la somme
				des éléments du tableau~;
			\item \code{java}{double moyenne(double[] tab)} qui retourne la
				 moyenne des éléments du tableau~;
			\item \code{java}{double[] copie(double[] tab)} qui retourne une
			 copie du tableau passé en paramètre~;
		\end{itemize}
	\end{Exercice}



	\begin{Exercice}{\texttt{TableauUtil} (suite)}	

		Dans la classe \texttt{TableauUtil} ajoutez les méthodes suivantes, leur javadoc ainsi que les tests JUnit correspondant~:
	\begin{itemize}
		\item \code{java}{int indiceMax(double[] tab)}: 
				retourne l'indice du maximum du tableau passé en paramètre~;
		\item \code{java}{boolean estTrié(double[] tab)}: 
				retourne vrai si le tableau est trié par ordre croissant et 	
				faux sinon~;
		\item \code{java}{int indice(int[] tab, int valeur)}: 
				retourne l'indice de cette valeur dans le tableau. 
				
				Que retournez-vous si la valeur n'apparait pas?
				Indiquez-le dans la javadoc.
		\item  \code{java}{boolean contient(String[] tab, String mot)}: 
			retourne vrai si un tableau de \texttt{String} contient un mot passé en
			paramètre~;
			
			Astuce: utiliser la méthode \texttt{equals} afin de tester 
			l'égalité entre deux \texttt{String} et non pas l'opérateur '=='.
		\item  \code{java}{boolean contient(double[] tab, double valeur, 
						double erreur)}: 
			retourne vrai si un élément du tableau est proche de la \texttt{valeur}
			à l'\texttt{erreur} près, c'est-à-dire si la distance entre l'élément 
			et la valeur est inférieure à \texttt{erreur}.
%	
%		\item créer
%		\item position(int valeur)
%		\item premièreOccurence
%		\item dernièreOccurence
%		\item glissementCyclique
%		\item inverserSigne
%		\item tri
%		\item recherche dichotomique
%		\item recherche d'une chaine
	\end{itemize}
	\end{Exercice}
	
	\begin{Exercice}{\texttt{TableauUtil} (suite de la suite)}	
		Dans la classe \texttt{TableauUtil} ajoutez les méthodes suivantes, leur
		 javadoc ainsi que les tests JUnit correspondant~:
		\begin{itemize}
			\item \code{java}{boolean doublons(int[] tab)} 
				qui retourne vrai si le tableau possède des doublons 
				(2 fois la même valeur) et faux sinon~;
			\item \code{java}{int nbNégatifs(double[] tab)} 
				qui retourne le nombre d'éléments négatifs du tableau~;
			\item \code{java}{void échange(double[] tab, int indice1,int indice2)} 
				qui échange la valeur se trouvant à l'indice \texttt{indice1} 
				avec la valeur se trouvant à l'indice \texttt{indice2}~;
			\item \code{java}{void inverser(double[] tab)} 
				qui inverse l'ordre des éléments  du tableau passé en paramètre ~;				
			\item \code{java}{int[] indicesMin(double[] tab)} 
				qui retourne un tableau contenant les indices des minimums.
		\end{itemize}
	\end{Exercice}
	
		
	\begin{Exercice}{Initialisation par défaut}	
		Dans la méthode principale d'une classe \texttt{TestInit}~:
		\begin{itemize}
			\item créez un tableau d'entier de taille 10 et afficher-le 
				(comme à  la ligne 18 de la classe \texttt{Tableau} ci-dessus). 
				
				Quelle valeur par défaut initialise chacune des cases du tableau ?
			\item créez un tableau de double de taille 10 et afficher-le~;
			\item créez un tableau de 10 booléens et afficher-le~;
			\item créez un tableau de 10 \texttt{String} et afficher-le.
		\end{itemize}
	\end{Exercice}

%	\begin{Exercice}{\texttt{Tri}}	
%		Dans la classe \texttt{Tri} ajoutez les méthodes suivantes, leur
%		 javadoc ainsi que les tests JUnit correspondant~:
%		\begin{itemize}
%			\item \code{java}{void triBulle(int[] tab)} 
%				qui trie le tableau en utilisant le tri bulle vu au cours~;
%			\item \code{java}{void triSélection(int[] tab)} 
%				qui trie le tableau en utilisant le tri par sélection vu au cours~;
%			\item \code{java}{void triInsertion(int[] tab)} 
%				qui trie le tableau en utilisant le tri par insertion vu au cours~;
%			\item \code{java}{boolean rechercheDichotomique(int[] tab, 
%			int valeur )} 
%				qui retourne vrai si l'élément se trouve dans le tableau trié et 
%				faux sinon. On suppose que le tableau reçu est trié. Vous utiliserez
%				l'algorithme de recherche dichotomique vu au cours.
%
%		\end{itemize}
%	\end{Exercice}



%	A faire:	
%\begin{itemize}
%	\item remplissage de tableau par system.in
%	\item tableau et String (split, chars)
% 	\item différentes façons de créer un tableaux
% 	\item tri bulle, insertin, sélection
%   \item recherche dichotomique
%\end{itemize}	

\end{document}
