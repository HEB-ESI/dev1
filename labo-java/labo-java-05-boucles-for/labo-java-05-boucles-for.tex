\documentclass[a4paper,11pt]{style-esi/td}

\usepackage{style-esi/licence}
\usepackage{style-esi/exercice}
\usepackage{style-esi/listing}
\usepackage{style-esi/tutoriel}
\usepackage{style/dev1}

\begin{document}

\seance{5}{Boucles \texttt{for}}{td05-boucles-for}{
	Ce TD introduit l'instruction \texttt{for} 
	ainsi que la notion de chaîne de caractères 
	et le type associé \texttt{String}.
}
	
%===================
\section{Boucle - \texttt{for}}
%====================	

 	\listing{java}{BoucleFor.java}
	Le code ci-dessus affiche les nombres de 1 à 10.
	Ce programme s'exécute comme suit~:
	\begin{itemize}
		\item  le programme commence à l'instruction \texttt{for} de la ligne 6~:
			\begin{itemize}
				\item l'initialisation \code{java}{int i = 1} est d'abord exécutée: 
					la variable $i$ de type entier est déclarée et initialisée à 1~;
				\item  ensuite, la condition du \texttt{for} \code{java}{i<=10}
					est évaluée, sa valeur est vrai car $i$ vaut 1 et est donc inférieur à 10~;
				\item puisque la condition est vraie le \emph{corps} de la boucle s'exécute, 
					la ligne 7 affiche la valeur de l'entier $i$: 1~;
				\item le programme revient à la ligne 6 et l'incrément \code{java}{i = i +1} est exécuté, 
					$i$ vaut 2 maintenant~;
				\item la condition est à nouveau évaluée avec la nouvelle valeur de i qui vaut maintenant 2~;
				\item le programme continue ainsi jusqu'à ce que $i$ atteigne la valeur 11. 
					\`A ce moment la condition est évaluée à faux et l'instruction 
					\texttt{for} prend fin.
			\end{itemize}
		\item	Comme aucune instruction ne suit ce \texttt{for}, le programme se termine.

	\end{itemize}

 
	\begin{Exercice}{Suites d'entiers}	
		Dans un package \texttt{g12345.dev1.td5} créez un classe \texttt{Exercice01}.
		Dans cette classe écrivez un programme qui demande à l'utilisateur un nombre entier $n$ et affiche
		à l'aide d'une boucle \texttt{for}:
		\begin{itemize}
			\item  les nombres de 1 à n~;
			\item  les nombres pairs qui sont compris entre 1 et n~;
			\item les nombres de -n à n~;
			\item les multiples de 5 qui sont compris entre 1 et n ;
			\item les multiples de n compris entre 1 et 100.
		\end{itemize}
	\end{Exercice} 
 
%===================
\section{Les chaînes de caractères}
%===================
 
 	Un caractère se représente en Java avec des guillemets simples.
	Par exemple, \texttt{'a'} représente le caractère a et \texttt{'1'} représente le caractère 1 (et non pas l'entier 1).
	Une variable de type \texttt{char} permet de manipuler les caractères.
	On écrira par exemple \code{java}{char lettre = 'a';} 
	
	En Java, le texte est représenté par une chaîne de caractères (une suite de caractères). 
	Le type associé est \texttt{String}.
	Une variable de type \texttt{String} permet de stocker et manipuler une chaine de caractères~:
	\code{java}{String mot = "Bonjour";}

	\bigskip
	\listing[basicstyle=\footnotesize\vtt]{java}{Texte.java}


	\begin{Exercice}{Voyelle}	
		\'Ecrire un programme qui demande à l'utilisateur 
		un mot et affiche si la première lettre est une voyelle ou non.
	\end{Exercice}

	\begin{Exercice}{Consonne}	
		\'Ecrire un programme qui demande à l'utilisateur 
		un mot et affiche si la première lettre est une consonne ou non.
		
		Astuce: une lettre est une consonne si ce n'est pas une voyelle.
	\end{Exercice}

	
	\begin{Exercice}{Première == dernière ?}	
		\'Ecrire un programme qui demande à l'utilisateur 
		un mot et affiche si la première lettre est la même que la dernière ou non.
		Par exemple, si l'utilisateur entre "java", le programme affiche que la première et 
		la dernière lettre ne sont pas égales.
	\end{Exercice}

	\begin{Exercice}{Miroir}	
		\'Ecrire un programme qui demande à l'utilisateur 
		un mot et affiche son miroir.
		
		Par exemple, si l'utilisateur entre "java", le programme affiche
		
		\begin{verbatim}
		avaj
		\end{verbatim}
	\end{Exercice}


	\begin{Exercice}{Voyelles}	
		\'Ecrire un programme qui demande à l'utilisateur 
		un mot et affiche les voyelles.
		
		Par exemple, si l'utilisateur entre "programmation", le programme affiche
		
		\begin{verbatim}
		oaaio
		\end{verbatim}
	\end{Exercice}

\section{Exercices Récapitulatifs}

	\begin{Exercice}{Une phrase}	
		\'Ecrire un programme qui demande à l'utilisateur 
		une phrase et affiche si la première lettre est une majuscule et la dernière un point.
		
		Par exemple si l'utilisateur entre "Bonjour Marco." votre programme affichera que c'est une phrase.
		Consulter Internet pour savoir comment vérifier qu'un caractère est une majuscule. 
		
	\end{Exercice}

	\begin{Exercice}{Nombre de voyelles et de consonnes}	
		\'Ecrire un programme qui demande à l'utilisateur 
		un mot et affiche le nombre de voyelles et de consonnes dans ce mot.
		
		Par exemple, si l'utilisateur entre "programmation", le programme affiche "5 voyelles et 8 consonnes".	
	\end{Exercice}


	\begin{Exercice}{Palindrome}
		Un palindrome est un mot dont la succession de lettres est la même 
		de gauche à droite et de droite à gauche. 
		Les mots été, ressasser ou kayak sont des palindromes.
		  	
		\'Ecrire un programme qui demande à l'utilisateur 
		un mot et affiche si ce mot est un palindrome ou non.
	\end{Exercice}

	\begin{Exercice}{Transformer un \texttt{for} en \texttt{while}}
		Modifier vos solutions aux deux exercices précédents afin d'utiliser des \texttt{while} à la place des \texttt{for}. 
	\end{Exercice}




\end{document}