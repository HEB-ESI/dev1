\documentclass[a4paper,11pt]{article}

%=========================
% Les styles
%=========================
\usepackage{style-esi/french}	% Francise LaTeX
\usepackage{style-esi/td}
\usepackage{style-esi/licence}	% Affiche une licence dans le document
\usepackage{style-esi/exercice}
\usepackage{style-esi/listing}
\usepackage{style-esi/tutoriel}
\usepackage{booktabs}

\renewcommand{\listingpublicpath}{https://github.com/HEB-ESI/dev1/tree/master/labo-java/labo-java-02-alternatives/code/}
\renewcommand{\listingsrcpath}{code/}

\date{2018 -- 2019}
\siglecours{DEV1}
\libellecours{Laboratoires Java I}
\libelledocument{TD 2 -- Alternatives}
\sigleprof{}



\begin{document}

\entete
\titre
\ccbysa{esi-dev1-list@he2b.be}
\lastedit


	%===================
	%  Contenu
	%====================	
	Dans ce TD vous trouverez une introduction à la manipulation des nombres à virgule,
	aux expressions booléennes et à l'instruction \texttt{if/else}.
	\tableofcontents

	\newpage

%===================
\section{Les nombres à virgule}
%====================	


	Au td1 nous avons travaillé avec des entiers, nous allons maintenant 
	manipuler des nombres à virgule.

	En java les \emph{nombres à virgule} se représentent avec un point. 
	Par exemple 12,3 s'écrira 12.3.

	Les variables pour représenter un entier sont de type \texttt{int}.
	Pour les nombres à virgule les variables sont de type \texttt{double}:
	
	\code{java}{double taille = 1.92;}

	Pour lire un entier on utilise la méthode \texttt{nextInt()} du scanner.
	Pour lire un nombre à virgule on utilise la méthode \texttt{nextDouble()}.
	
	\code{java}{double y = clavier.nextDouble();}


	Le programme suivant illustre l'utilisation de ces nombres.

	\bigskip
 	\listing{java}{\detokenize{Calculs.java}}

	\begin{itemize}
		\item \`A la ligne 3, on importe le Scanner afin de pouvoir lire les entrées de l'utilisateur.
		\item \`A la ligne 8, on initialise le Scanner, il scanne \emph{l'entrée standard}: 
			\texttt{System.in}.

		\item \`A la ligne 10, on déclare une variable nommée x, de type \texttt{int}. 
			On lui assigne la valeur entrée par l'utilisateur~: \code{java}{clavier.nextInt()}.

		\item \`A la ligne 11, on déclare une variable nommée y, de type \texttt{double}.
			On lui assigne la valeur entrée par l'utilisateur~: \code{java}{clavier.nextDouble()}.

		\item Aux lignes 13 à 19 le programme affiche le résultat des différents calculs.

	\end{itemize}

%	Le tableau suivant reprend les différences entre entiers et réels~:
%	\begin{center}
%		\begin{tabular}{| c | c | c | c |}
%			\hline
%			& type & opérateurs & lecture \\
%			\hline
%			Entier & \texttt{int} & +, - *, /, \% & \texttt{nextInt()} \\
%			Réel & \texttt{double} & +, - *, / & \texttt{nextDouble()} \\
%			\hline			
%		\end{tabular}
%	\end{center}

\begin{Exercice}{Calculs}
	Créez un package \texttt{g12345.dev1.td2} où vous remplacez \texttt{g12345} par votre identifiant.
	Dans ce package créez une classe \texttt{Exercice1}.
	Dans cette classe créez un programme qui affiche la valeur des expressions suivantes:
	
	\begin{itemize}		
		\item \texttt{12.3+13.5}
		\item \texttt{12.3-13.5}
		\item \texttt{12.3*13.5}
		\item \texttt{2.0/3.0}
	\end{itemize}
		
\end{Exercice}

\begin{Exercice}{Périmètre et aire d'un cercle}
	Dans une classe \texttt{Exercice2} écrivez un programme
	qui demande à l'utilisateur le rayon d'un cercle (un nombre à virgule)
	et affiche son périmètre et son aire.
	
	Rappel: le périmètre se calcule par la formule $2\pi r$ et 
	l'aire par la formule $\pi r^2$ où vous utiliserez 3.141593 comme valeur approchée de $\pi$.
	
	Attention: si votre système est configuré en français (cela s'appelle la \emph{locale}) 
	vous devrez entrer ce nombre avec une virgule, s'il est en anglais ce sera avec un point.

\end{Exercice}



\section{Les conditions}

	En informatique, on appelle \emph{booléens} les deux valeurs 'vrai' et 'faux'. 
	En java les booléens se représentent par \texttt{true} (vrai) et \texttt{false} (faux).  

	Une condition est une expression dont la valeur s'évalue à \texttt{true} ou \texttt{false}.
	Voici quelques exemples de conditions:
	\begin{itemize}
		\item \code{java}{age < 18}~:  vaut \texttt{true} si \texttt{age} a une valeur inférieure à 18 et \texttt{false} sinon~;
		\item \code{java}{nb >= 0}~:  vaut \texttt{true} si \texttt{nb} est supérieur à 0~;
		\item \code{java}{b*b - 4*a*c < 0}~: vaut \texttt{true} si $b^2-4ac$ est négatif~;
		\item \code{java}{(nb >= 0) && (nb <=100)}~: vaut \texttt{true} si le nombre \texttt{nb} est compris entre 0 et 100~;
		\item \code{java}{a < b}~:  vaut \texttt{true} si la valeur de la variable \texttt{a} est inférieure à celle de \texttt{b}.
	\end{itemize}


	Pour construire une condition on utilisera les opérateurs de comparaison de nombres:
	
        \begin{center}
	\begin{tabular}{lll}\toprule
          signification            & symbole  & exemple\\ \midrule
		plus petit         & <        & \code{java}{age < 18}\\
		plus petit ou égal & <=       & \code{java}{age <= 10} \\
		plus grand         & >        & \code{java}{age > 18} \\
		plus grand ou égal & >=       & \code{java}{age >= 21} \\
		 égal              & ==       & \code{java}{i == 4}\\
		 différent         & !=       & \code{java}{nb != 42}\\ \bottomrule
	\end{tabular} 
      \end{center}
	Et on combinera des conditions avec les \emph{opérateurs booléens}~:
	\begin{itemize}
		\item Le \emph{ET} logique s'écrit \texttt{\&\&}.
			
			Exemple~: \code{java}{(nb >= 0) && (nb <= 100)} vérifie si \texttt{nb} est compris entre 0 et 100.
			Cette condition sera vraie si \code{java}{nb >= 0} ET si \code{java}{nb <= 0}. 
			
		\item Le \emph{OU} logique s'écrit \texttt{||}. 
			
			Exemple~: \code{java}{a < b || a < c} sera vraie si \code{java}{a < b} ou bien si \code{java}{a < c} (ou les deux).
			
		\item La négation s'écrit \texttt{!}. 
		
			Exemple~: \texttt{ !(a < b)} sera vraie si \code{java}{a < b} est faux, c'est-à-dire si \code{java}{a >= b}.
	\end{itemize}
	

	\begin{Exercice}{Conditions}
		\'Ecrivez un programme qui affiche la valeur des expressions suivantes:
		\begin{itemize}
			\item \texttt{10 < 20} (écrivez simplement: \code{java}{System.out.println(10 < 20);})
			\item \texttt{10 > 20}
			\item \texttt{1 == 2}
			\item \texttt{20.0/2 != 10.0}
		\end{itemize}
		
		Qu'affiche votre programme pour chacune des expressions ci-dessus ?
	\end{Exercice}

	\begin{Exercice}{Conditions}
		\'Ecrivez un programme qui demande à l'utilisateur 
		3 nombres entiers a, b et c et affiche la valeurs des expressions suivantes:
		\begin{itemize}
			\item \texttt{a\%2 == 0}  (a est divisible par 2 c'est-à-dire a est pair)
			\item \texttt{a\%2 == 1}  (a est impair)
			\item \texttt{a\%b == 0}  (a est divisible par b)
			\item \texttt{a < b}
			\item \texttt{(a <= b) \&\& (a <= c)} (a est le minimum)
			\item \texttt{(a < b \&\& b < c) || (a > b \&\& b > c)} (b est strictement compris entre a et c)
		\end{itemize}
	\end{Exercice}


\section{Les alternatives: \texttt{if/else}}

	L'instruction \texttt{if} permet d'exécuter des instructions si une certaine condition est vérifiée.
	
	Le programme suivant affichera "ce nombre est positif" 
	si \texttt{nb} est plus grand ou égal à 0 et n'affichera rien dans le cas contraire.
	 
	 \listing{java}{Positif.java}

	L'instruction \texttt{if/else} permet d'exécuter des instructions si une certaine condition est vérifiée
	et d'autres instructions si la condition n'est pas vérifiée. On peut traduire 'else' par 'sinon'.
	
	On peut remplacer le \texttt{if} du programme précédent par le \texttt{if/else} suivant~:
	
	\begin{Code}{java}
		if(nb >= 0) {
			System.out.println("ce nombre est positif.");
		} else {
			System.out.println("ce nombre est négatif.");
		}
	\end{Code}

	Le programme affichera "ce nombre est positif" 
	si \texttt{nb} est plus grand ou égal à 0 et "ce nombre est négatif" sinon.


	Il est aussi possible d'utiliser une succession de \texttt{if/else}~:
		
	\begin{Code}{java}
		if(nb > 0) {
			System.out.println("ce nombre est positif.");
		} else if(nb < 0) {
			System.out.println("ce nombre est négatif.");
		} else {
			System.out.println("ce nombre est nul.");
		}
	\end{Code}

	
	\begin{Exercice}{Majeur - \texttt{if}}
		\'Ecrivez un programme qui demande à l'utilisateur 
		son age et affiche s'il est majeur (s'il a plus de 18 ans). 
		S'il n'est pas majeur le programme n'affiche rien.
		
		Exemple: si l'utilisateur entre 19 le programme affiche "vous êtes majeur". 
	\end{Exercice}

	\begin{Exercice}{Pair ou impair - \texttt{if/else}}
		\'Ecrivez un programme qui demande à l'utilisateur 
		un nombre entier et affiche "ce nombre est pair" ou "ce nombre est impair" selon le cas.
				
		Exemple: si l'utilisateur entre -23 le programme affiche "ce nombre est impair". 

		Astuce: un nombre est pair si le reste de la division par 2 vaut 0.
	\end{Exercice}

	\begin{Exercice}{Maximum de 2 nombres}
		\'Ecrivez un programme qui demande à l'utilisateur deux nombres
		et affiche le plus grand des deux.
		
		Exemple: si l'utilisateur entre 7,5 et 2,3 le programme affiche 7,5. 
	\end{Exercice}
	
	
	
\section{Exercices supplémentaires}




	\begin{Exercice}{Les intérêts}
		\'Ecrivez un programme qui demande un montant (un nombre à virgule)
		à l'utilisateur et affiche les intérêts reçus après 1 an pour un montant placé en banque à du 2\% d'intérêt.
	\end{Exercice}

	\begin{Exercice}{Prix TTC}
		\'Ecrivez un programme qui demande à l'utilisateur:
		\begin{itemize}
			\item le prix unitaire d’un produit hors TVA, 
			\item le taux de TVA en \% (un entier) 
			\item la quantité de produit vendue à un client (un entier)
		\end{itemize}
		et affiche le prix total à payer par ce client.
		
		Exemple: si le client achète 5 drones au prix de 1000 euros hors TVA et que la TVA est de 21\%,
				le programme affichera 6050 euros.
	\end{Exercice}

	\begin{Exercice}{Maximum de 3 nombres}
		\'Ecrivez un programme qui demande à l'utilisateur 
		trois nombres  et affiche le maximum des trois.
		
		Exemple: si l'utilisateur entre 7,5, 17,9 et 2,3 le programme affiche 17,9. 
	\end{Exercice}

	\begin{Exercice}{Le type de triangle}
 		\'Ecrivez un programme qui demande à l'utilisateur 
		 la longueur des 3 côtés d'un triangle et affiche s'il est~: 
		 équilatéral (tous égaux), isocèle (2 égaux) ou quelconque.
		 
		 Exemple: si l'utilisateur entre 2,5, 5 et 5 le programme affiche "le triangle est isocèle". 
	\end{Exercice}
	
	\begin{Exercice}{Divisions entière et à virgule}
		Créez un programme qui affiche la valeur des expressions suivantes:
	
		\begin{itemize}		
			\item \texttt{2.0/3.0}
			\item \texttt{2/3.0}
			\item \texttt{2.0/3}
			\item \texttt{2/3}
			\item \texttt{2.0/0.0}
			\item \texttt{2/0}
		\end{itemize}
	
		Notez la différence de résultat entre les expressions \texttt{2.0/3.0} qui est une division entre nombres à virgule 
		et \texttt{2/3} qui est une division entre entiers (et donc une division entière).
	
		Notez également la différence de résultat entre les 2 dernières expressions qui sont des divisions par zéro.
		La première est une division entre nombres à virgule,
		la seconde est une division entre entiers.
	\end{Exercice}
	
	\begin{Exercice}{Conditions}
		\'Ecrivez un programme qui affiche la valeur des expressions suivantes:
		\begin{itemize}
			\item \texttt{9.99 == 10.0}
			\item \texttt{9.999999999999999999999999 == 10.0}
			\item \texttt{9.999999999999999999999999 == 10.00000000000001}
			\item \texttt{true}  (écrivez simplement: \code{java}{System.out.println(true);})
			\item \texttt{false} 
			\item \texttt{!true} 
			\item \texttt{!false} 
			\item \texttt{true \&\& true} 
			\item \texttt{true \&\& false} 
			\item \texttt{false \&\& true} 
			\item \texttt{false \&\& false} 
			\item \texttt{true || true} 
			\item \texttt{true || false} 
			\item \texttt{false || true} 
			\item \texttt{false || false} 
		\end{itemize}
		
		Qu'affiche votre programme pour chacune des expressions ci-dessus ?
	\end{Exercice}

\end{document}


%%===================
%\section{Variables entières, réelles et booléennes}
%%====================	
%
%
%	Au td1 nous avons vu que les variables permettant de manipuler des entiers 
%	se \emph{déclarent} avec le type \texttt{int}. 
%	
%	Par exemple l'instruction~:
%	
%	\code{java}{int longueur = 12;}
%	
%	déclare une variable de type entier nommée \texttt{longueur} à laquelle on \emph{assigne}
%	la valeur 12.
%	
%	\begin{itemize}
%		\item Les variables de type \emph{réel}  se déclarent avec le type \texttt{double}. 
%	
%			Par exemple:
%	
%			\code{java}{double taille = 1.92;}
%	
%			déclare une variable de type réel nommée \texttt{taille} à laquelle on \emph{assigne}
%			la valeur $1,92$.
%
%%		\item Les variables de type \emph{booléen}  se déclarent avec le type \texttt{boolean}. 
%%	
%%			Par exemple~:
%%	
%%			\code{java}{boolean jeSuisContent = true;}
%%
%%			déclare une variable de type booléen nommée \texttt{jeSuisContent} à laquelle on \emph{assigne}
%%			la valeur \texttt{true}. %Ou encore~: 
%%%	
%%%			\code{java}{boolean majeur = age > 18;}
%%%	
%%%			déclare une variable de type booléen nommée \texttt{majeur} à laquelle on \emph{assigne}
%%%			la valeur de l'expression \texttt{age > 18}, \texttt{majeur} vaudra \texttt{true} si la variable \texttt{age} 
%%%			a une valeur supérieure à 18 (au moment ou s'exécute cette instruction). 
%%
%%		\item On peut également stocker du texte dans des variables.
%%			Les variables contenant du texte sont de type \texttt{String},
%%			on dit également \emph{chaîne de caractères}.
%%	 
%%			Par exemple~:
%%
%%			\code{java}{String prénom = "John";}
%%
%%			déclare une variable de type chaîne de caractères nommée \texttt{prénom} 
%%			à laquelle on \emph{assigne} la valeur \texttt{"John"}.
%	\end{itemize}
%
%	L'exemple suivant illustre ces notions. 
%
%	
% 	\listing{java}{code/Personne1.java}
%
%	Notez l'utilisation de \texttt{print} et \texttt{println} cette dernière instruction affiche 
%	le texte entre guillements
%	\emph{et} passe à la ligne alors que le \texttt{print} simple ne passe pas à la ligne. 
%
%
%	\begin{Exercice}{Une personne} %TODO
%		Dans votre package \texttt{g12345.dev1.td2} créez une classe \texttt{Personne}.
%		Dans cette classe déclarez les variables :
%	
%		\begin{itemize}		
%			\item \texttt{prénom} de type \texttt{String}
%			\item \texttt{nom} de type \texttt{String}
%			\item \texttt{année} de type \texttt{int}
%			\item \texttt{poids} de type \texttt{double}
%			\item \texttt{taille} de type \texttt{double}
%		\end{itemize}
%		
%		Initialisez les avec des valeurs de votre choix, par exemple le prénom et 
%		le nom seront "John" et "Smith".
%		
%		Le programme affiche "Bonjour", suivi du prénom et du nom, 
%		de l'âge de la personne (2018-année), son poids et sa taille. 
%	\end{Exercice}

%\section{Lecture de réels et de textes}
%
%	Pour lire un entier au clavier il faut 
%	\begin{itemize} 
%		\item importer le \texttt{Scanner}: \code{java}{import java.util.Scanner;}
%		\item initialiser le \texttt{Scanner}: \code{java}{Scanner clavier = new Scanner(System.in);}
%		\item lire un entier: \code{java}{int nb = clavier.nextInt();}
%	\end{itemize}
%	
%	Pour lire un réel on utilise la méthode \texttt{nextDouble()}~: 
%	\code{java}{double taille = clavier.nextDouble();}
%	
%	Pour lire une chaine de caractères on utilise la méthode \texttt{next()} ou \texttt{nextLine()}~: 
%	
%	\code{java}{String prénom = clavier.next();}
%	
%		
% 	\listing{java}{code/Personne2.java}
%
%	\begin{Exercice}{Une personne (suite)}
%		Modifiez votre programme de l'exercice précédent afin de demander à l'utilisateur
%		des valeurs pour le prénom, le nom, l'année, la taille et le poids.
%	\end{Exercice}
%	
%	Le programme suivant demande à l'utilisateur d'entrée le rayon d'un cercle et affiche son aire.
% 	\listing{java}{code/AireCercle.java}
%
%	\begin{itemize}
%		\item \`A la ligne 9 on lit un réel au clavier. 
%		\item Attention lorsqu'on entre un réel au clavier il faut suivre les conventions du système.
%			Si votre système est en français, il faudra écrire le nombre avec une virgule (par exemple 50,5), 
%			s'il est en anglais ce sera avec un point (par exemple 50.5).
%		 
%		\item \`A la ligne 10 on déclare une variable de type réel qui se nomme \texttt{PI}
%		et on lui assigne la valeur 3.141596. 
%	\end{itemize}
%	


