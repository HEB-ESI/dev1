\documentclass[a4paper,11pt]{article}

%=========================
% Les styles
%=========================
\usepackage{style-esi/french}	% Francise LaTeX
\usepackage{style-esi/td}
\usepackage{style-esi/licence}	% Affiche une licence dans le document
\usepackage{style-esi/exercice}
\usepackage{style-esi/listing}
\usepackage{style-esi/tutoriel}

\date{2018 -- 2019}
\siglecours{DEV1}
\libellecours{Laboratoires Java I}
\libelledocument{TD 3 -- Boucles}
\sigleprof{}



\begin{document}

\entete
\titre
\ccbysa{esi-dev1-list@he2b.be}
\lastedit


	%===================
	%  Contenu
	%====================	
	Dans ce TD vous trouverez une introduction aux boucles et à l'instruction \texttt{while}.
	Ensuite nous verrons comment lire une série de données.
	 
	\tableofcontents

	\newpage

%===================
\section{Boucle - \texttt{while}}
%====================	

 	\listing{java}{code/Boucle.java}
	Le code ci-dessus affiche les nombres de 1 à 10.
	Ce programme s'exécute comme suit~:
	\begin{itemize}
		\item  le programme commence à la ligne 6 où la variable $i$ de type entier est déclarée et initialisée à 1~;
		\item  ensuite, la condition du \texttt{while} de la ligne 7 est évaluée, sa valeur est vrai car $i$ vaut 1 et est donc inférieur à 10~;
		\item puisque la condition est vraie le \emph{corps} de la boucle s'exécute:
			\begin{itemize}
				\item la ligne 8 affiche la valeur de l'entier $i$: 1
				\item la ligne 9 assigne la valeur 2 à i (c'est la valeur de l'expression $i+1$)
			\end{itemize}
		\item l'exécution du corps de la boucle étant terminé le programme retourne à l'instruction 7, 
			et évalue à nouveau la condition avec la nouvelle valeur de i qui vaut maintenant 2.
		\item  la condition de la ligne 7 étant vraie (car $i$ vaut 2 et est donc inférieur à 10) on exécute à nouveau le corps de la boucle~;
		\item le programme continue ainsi jusqu'à ce que $i$ atteigne la valeur 11. 
			\`A ce moment la condition du \texttt{while} est évaluée à faux et l'instruction 
			\texttt{while} prend fin.
			Comme aucune instruction ne suit ce \texttt{while}, le programme se termine.
	\end{itemize}

 
	\begin{Exercice}{Suites d'entiers}	
		Dans votre package créez un classe \texttt{Exercice2}.
		Dans cette classe écrivez un programme qui demande à l'utilisateur un nombre entier $n$ et affiche:
		\begin{itemize}
			\item  les nombres de 1 à n~;
			\item  les nombres pairs qui sont compris entre 1 et n~;
			\item les nombres de -n à n~;
			\item les multiples de 5 qui sont compris entre 1 et n ;
			\item les multiples de n compris entre 1 et 100.
		\end{itemize}
	\end{Exercice}

	\begin{Exercice}{Ligne}	
		\'Ecrire un programme qui demande à l'utilisateur 
		la longueur d'une ligne et affiche une ligne (une suite de tirets \texttt{'-'}) de cette longueur.
		
		Par exemple, si l'utilisateur entre 10, le programme affiche
		
		\begin{verbatim}
		----------
		\end{verbatim}
		
		Astuce: utiliser la méthode \texttt{System.out.print('-')} et non pas \texttt{System.out.println('-')}.
	\end{Exercice}
 
 \section{Lecture de données multiples}
 
 	Dans cette section nous allons voir comment demander une série de données à l'utilisateur.
	Par exemple demander d'entrer la température de chaque jour du mois écoulé afin d'en calculer la moyenne.
	Il y a plusieurs manières de procéder, nous allons en voir 3.

	\subsection{Variante 1 : le nombre de valeurs est fixe}

		Dans cette variante le nombre de valeurs à lire est connu et fixé au moment de l'écriture du 
		programme. Ce nombre sera le même chaque fois que l'on
		lancera le programme.
		Par exemple on va traiter 10 entiers. 	
			
 		\listing{java}{code/LectureMultiple1.java}

		\begin{Exercice}{Moyenne}\label{ex:moyenne}
			\'Ecrivez un programme qui demande à l'utilisateur $5$ valeurs
			 et affiche la somme de ces valeurs et la moyenne de ces valeurs.
			 
			 Par exemple si l'utilisateur entre 15,  20, 10, 12 et 18, 
			 le programme affichera 75 (somme) et 15 (moyenne).
		\end{Exercice}

		\begin{Exercice}{Pair ou impair}\label{ex:pair}
			\'Ecrivez un programme qui demande 10 nombres entiers à l'utilisateur et 
			pour chacun de ces nombres affiche, au fur et à mesure, s'il est pair ou impair. 
		\end{Exercice}
	
	\subsection{Variante 2 : le nombre de valeurs est connu}
	
		Ici le nombre de valeurs à lire est demandé à l'utilisateur.
		Ce nombre peut être différent à chaque exécution du programme mais il sera connu
		avant de commencer la lecture des valeurs.
		 
 		\listing{java}{code/LectureMultiple2.java}

		\begin{Exercice}{Moyenne}
			Modifiez l'Exercice \ref{ex:moyenne} afin que le programme demande
			 à l'utilisateur le nombre $n$ de valeurs qu'il veut introduire,
			lise ces $n$ valeurs au clavier et affiche la somme et la moyenne. 
		\end{Exercice}

		\begin{Exercice}{Positifs, négatifs et nuls}
			\'Ecrire un programme qui demande à l'utilisateur 
			le nombre $n$ de valeurs qu'il veut introduire,
			lit ces $n$ valeurs au clavier et affiche le nombres de valeurs positives, 
			le nombre de valeurs négatives et le nombre de valeurs nulles. 
			
			Exemple, si l'utilisateur entre les 8 valeurs:
			
			\texttt{ 5 9 -1 0 12 -7 -4 3}
			
			le programme affiche
			
			\begin{verbatim}
				positifs : 4
				négatifs : 3
				nuls : 1
			\end{verbatim}
		\end{Exercice}





	\subsection{Variante 3 : valeur sentinelle}
	
		Dans cette variante, l'utilisateur entre une série de nombres et une valeur spéciale permet d'indiquer
		la fin des données. Par exemple si l'utilisateur doit entrer des nombres positifs on pourra utiliser -1
		comme valeur de fin. Cette valeur spéciale s'appelle \emph{valeur sentinelle}.
 		\listing{java}{code/LectureMultiple3.java}

		\begin{Exercice}{Premier et dernier}
			\'Ecrivez un programme qui demande à l'utilisateur une série de nombres positifs 
			qui se termine par -1 (valeur sentinelle). 
			Le programme affiche le premier et le dernier nombre de la série.
		
			Exemple, si l'utilisateur entre
			
			\texttt{ 5 9 1 0 12 7 4 3 -1}

			le programme affiche
			
			\begin{verbatim}
				premier : 5
				dernier : 3
			\end{verbatim}

		\end{Exercice}

		\begin{Exercice}{Maximum}
			\'Ecrivez un programme qui demande à l'utilisateur une série de nombres positifs 
			qui se termine par -1 (valeur sentinelle). 
			Le programme affiche le maximum de la série.
			
			Par exemple, si l'utilisateur entre
			
			\texttt{ 5 9 1 0 12 7 4 3 -1}

			le programme affiche
			
			\begin{verbatim}
				maximum : 12
			\end{verbatim}
		\end{Exercice}


	\section{Exercices Récapitulatifs}

		\begin{Exercice}{Maximum et minimum}
			\'Ecrivez un programme qui demande à l'utilisateur le nombre $n$ de valeurs qu'il veut introduire,
			lit ces $n$ valeurs au clavier et affiche le maximum et le minimum. 
			
			Attention les valeurs peuvent être négatives.
		\end{Exercice}


		\begin{Exercice}{Comparer le premier et le dernier}
			\'Ecrivez un programme qui demande à l'utilisateur une série de nombres positifs 
			qui se termine par -1 (valeur sentinelle). 
			Le programme affiche l'un des messages suivants~:
			\begin{itemize}
				\item Le premier est inférieur au dernier
				\item Le premier est égal au dernier
				\item Le premier est supérieur au dernier
				\item Le série est vide (si le premier nombre entré est -1)
			\end{itemize}			
		\end{Exercice}

		\begin{Exercice}{Série croissante}
			\'Ecrivez un programme qui demande à l'utilisateur une série de nombres positifs 
			qui se termine par -1 (valeur sentinelle) et affiche si cette série est strictement croissante ou non.
			On suppose que la série contient au moins 2 nombres.
		\end{Exercice}



\end{document}