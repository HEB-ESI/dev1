\documentclass[a4paper,11pt]{article}

%=========================
% Les styles
%=========================
\usepackage{style-esi/french}	% Francise LaTeX
\usepackage{style-esi/td}
\usepackage{style-esi/licence}	% Affiche une licence dans le document
\usepackage{style-esi/exercice}
\usepackage{style-esi/listing}
\usepackage{style-esi/tutoriel}


\newcommand{\publicbasepath}
{https://git.esi-bru.be/dev1/labo-java/tree/master/td08-exceptions}
\renewcommand{\listingpublicpath}{\publicbasepath/code/}
\renewcommand{\listingsrcpath}{code/}

\marginnumberfalse
\marginsectiontrue


\date{2018 -- 2019}
\siglecours{DEV1}
\libellecours{Laboratoires Java I}
\libelledocument{TD 8 -- Exceptions, Javadoc}
\sigleprof{}



\begin{document}

\entete
\titre
\ccbysa{esi-dev1-list@he2b.be}
\lastedit


	%===================
	%  Contenu
	%====================	
	Ce TD introduit la notion \emph{d'exception} qui permet de lancer une alerte lorsqu'une erreur est détectée.
	Ensuite nous verrons comment \emph{documenter} le code Java à l'aide de l'outil Javadoc.
	
		Les codes sources et les solutions de ce TD se trouvent à l'adresse~: 
	
	\url{\publicbasepath}	

		 
	\tableofcontents

	\newpage

%====================
\section{Les exceptions}
%====================	

	
	
	
 	\listing{java}{Cercle.java}

	La méthode \texttt{périmètre} de la classe \texttt{Cercle} ci-dessus
	reçoit en paramètre le rayon du cercle.
	
	Le périmètre d'un cercle n'a de sens que si le rayon est strictement positif.  
	La méthode va \emph{lancer} une \texttt{IllegalArgumentException} dans la cas contraire. 
	Lorsque cette erreur est lancée le programme s'arrête et affiche un message d'erreur.
	Lorsqu'on exécute la classe \texttt{Cercle} on obtient le message suivant~:
	\begin{Console}
> java esi.dev1.td8.Cercle
62.83185307179586
Exception in thread "main" java.lang.IllegalArgumentException: Le rayon doit être positif: -3.0
	at esi.dev1.td8.Cercle.périmètre(Cercle.java:13)
	at esi.dev1.td8.Cercle.main(Cercle.java:20)
\end{Console}

	\begin{Exercice}{Calendrier~: méthodes robustes}
		Dans votre projet calendrier, vérifiez que les paramètres des méthodes sont corrects et lancer une 
		\texttt{IllegalArgumentException} avec un message adéquat sinon~:
		\begin{itemize}
			\item \code{java}{String nomMois(int mois)}: le mois est compris entre 1 et 12~;
			\item \code{java}{void afficherTitre(int mois, int année)}: le mois est compris entre 1 et 12~;
			\item \code{java}{void afficherMois(int décalage, int nombreJours)}: le décalage est compris entre 0 et 6 et 
				le nombre de jours entre 1 et 31.
			\item  \code{java}{int nombreJours(int mois, int année)}: le mois est compris entre 1 et 12~;
			\item \code{java}{int numéroJour(int jour, int mois, int année)}: la date est correcte, c'est-à-dire 
			que \texttt{mois} est compris entre 1 et 12 et \texttt{jour} est compris entre 1 et le nombre de 
			jours dans le mois. 
		\end{itemize}
		
		Vérifiez qu'une exception est lancée lorsque vous entrez un mois incorrect dans votre application.
	\end{Exercice}

%====================
\section{Gérer les entrées de l'utilisateur}
%====================	

	Dans la classe \texttt{Saisie} ci-dessous, la méthode \texttt{lireEntier} demande à l'utilisateur d'entrer un entier.
	Tant que l'utilisateur entre autre chose qu'un entier, la méthode lui demande à nouveau d'en entrer un. 
	
 	\listing{java}{Saisie.java}

	
	
	\begin{Exercice}{Lectures robustes}
		Créez une classe \texttt{Lecture} et écrivez-y les méthodes suivantes~:
		\begin{itemize}
			\item \code{java}{int lireEntier(String message)}: comme dans l'exemple ci-dessus, lit et retourne un entier.
				Tant que l'utilisateur n'entre pas un entier, la méthode lui demande à nouveau.
			\item \code{java}{double lireDouble(String message)}: lit et retourne un double.
				Tant que l'utilisateur n'entre pas un double, la méthode lui demande à nouveau.
			\item \code{java}{int lireEntier(String message, int min, int max)}: lit et retourne un entier 
				compris entre \texttt{min} et \texttt{max}.
				Tant que l'utilisateur n'entre pas un entier compris entre \texttt{min} et \texttt{max}, 
				la méthode lui demande à nouveau.
				
				Astuce: faites appel à la méthode \texttt{lireEntier}.
		\end{itemize}
	\end{Exercice}
	
	\begin{Exercice}{Calendrier avec lectures robustes}
		Utilisez des méthodes robustes pour gérer les éventuelles erreurs de l'utilisateur dans votre application calendrier.
	\end{Exercice}


%====================
\section{Documentation des méthodes}
%====================	

	Dans l'exemple du cercle ci-dessus la documentation de la méthode \texttt{périmètre} 
	se trouve juste avant l'entête de la méthode entre les balises \texttt{/**} et \texttt{*/}.
	Cette documentation suit le format \texttt{Javadoc} vu au cours.
	
	Pour générer le documentation faites un clic-droit sur l'icône du projet et choisissez 
	\textbf{Generate Javadoc} dans le menu déroulant qui apparait. 
	Un navigateur s'ouvre avec votre Javadoc. Dans la console (output) les éventuels 
	erreurs et avertissements sont affichés.
	Les fichiers générés se trouvent dans le répertoire \texttt{dist/javadoc}.
	

	\begin{Exercice}{Documentation du calendrier}
		Documentez chaque méthode du projet calendrier et générez sa documentation.
	\end{Exercice}

	
				






\end{document}