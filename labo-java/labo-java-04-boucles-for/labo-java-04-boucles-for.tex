\documentclass[a4paper,11pt]{article}

%=========================
% Les styles
%=========================
\usepackage{style-esi/french}	% Francise LaTeX
\usepackage{style-esi/td}
\usepackage{style-esi/licence}	% Affiche une licence dans le document
\usepackage{style-esi/exercice}
\usepackage{style-esi/listing}
\usepackage{style-esi/tutoriel}

\date{2018 -- 2019}
\siglecours{DEV1}
\libellecours{Laboratoires Java I}
\libelledocument{TD 4 -- Boucles \texttt{for}}
\sigleprof{}



\begin{document}

\entete
\titre
\ccbysa{esi-dev1-list@he2b.be}
\lastedit


	%===================
	%  Contenu
	%====================	
	Ce TD introduit l'instruction \texttt{for} ainsi que la notion de chaîne de caractères et le type
	associé \texttt{String}.
	 
	\tableofcontents

	\newpage

%===================
\section{Boucle - \texttt{for}}
%====================	

 	\listing{java}{code/BoucleFor.java}
	Le code ci-dessus affiche les nombres de 1 à 10.
	Ce programme s'exécute comme suit~:
	\begin{itemize}
		\item  le programme commence à l'instruction \texttt{for} de la ligne 5~:
			\begin{itemize}
				\item l'initialisation \code{java}{int i = 1} est d'abord exécutée: 
					la variable $i$ de type entier est déclarée et initialisée à 1~;
				\item  ensuite, la condition du \texttt{for} \code{java}{i<=10}
					est évaluée, sa valeur est vrai car $i$ vaut 1 et est donc inférieur à 10~;
				\item puisque la condition est vraie le \emph{corps} de la boucle s'exécute, 
					la ligne 6 affiche la valeur de l'entier $i$: 1. 
					Et l'incrément \code{java}{i = i +1} est exécuté, $i$ vaut 2 maintenant~;
				\item la condition est à nouveau évaluée avec la nouvelle valeur de i qui vaut maintenant 2~;
				\item etc...
				\item le programme continue ainsi jusqu'à ce que $i$ atteigne la valeur 11. 
					\`A ce moment la condition est évaluée à faux et l'instruction 
					\texttt{for} prend fin.
			\end{itemize}
		\item	Comme aucune instruction ne suit ce \texttt{for}, le programme se termine.

	\end{itemize}

 
	\begin{Exercice}{Suites d'entiers}	
		Dans votre package créez un classe \texttt{Exercice1}.
		Dans cette classe écrivez un programme qui demande à l'utilisateur un nombre entier $n$ et affiche
		à l'aide d'une boucle \texttt{for}:
		\begin{itemize}
			\item  les nombres de 1 à n~;
			\item  les nombres pairs qui sont compris entre 1 et n~;
			\item les nombres de -n à n~;
			\item les multiples de 5 qui sont compris entre 1 et n ;
			\item les multiples de n compris entre 1 et 100.
		\end{itemize}
	\end{Exercice}

	\begin{Exercice}{Ligne}	
		\'Ecrire un programme qui demande à l'utilisateur 
		la longueur d'une ligne et affiche une ligne (une suite de tirets \texttt{'-'}) de cette longueur.
		
		Par exemple, si l'utilisateur entre 10, le programme affiche
		
		\begin{verbatim}
		----------
		\end{verbatim}
		
		Astuce: utiliser la méthode \texttt{System.out.print('-')} et non pas \texttt{System.out.println('-')}.
	\end{Exercice}
 
 
%===================
\section{Les chaînes de caractères}
%===================
 
 	Un caractère se représente en Java avec des simples guillemets.
	Par exemple, \texttt{'a'} représente le caractère a et \texttt{'1'} représente le caractère 1 (et non pas l'entier 1).
	Une variable de type \texttt{char} permet de manipuler les caractères.
	On écrira par exemple \code{java}{char lettre = 'a';} 
	
	En Java, le texte est représenté par une chaîne de caractères (une suite de caractères). 
	Le type associé est \texttt{String}.
	Une variable de type \texttt{String} permet de stocker et manipuler une chaine de caractères~:
	\code{java}{String mot = "Bonjour";}

	\bigskip
	  \listing{java}{code/Texte.java}


	\begin{Exercice}{Voyelle}	
		\'Ecrire un programme qui demande à l'utilisateur 
		un mot et affiche si la première lettre est une voyelle ou non.
	\end{Exercice}

	\begin{Exercice}{Consonne}	
		\'Ecrire un programme qui demande à l'utilisateur 
		un mot et affiche si la première lettre est une consonne ou non.
		
		Astuce: une lettre est une consonne si ce n'est pas une voyelle.
	\end{Exercice}

	
	\begin{Exercice}{Première == dernière ?}	
		\'Ecrire un programme qui demande à l'utilisateur 
		un mot et affiche si oui la première lettre est la même que la dernière ou non.
		
		Par exemple, si l'utilisateur entre "java", le programme affiche
		
		\begin{verbatim}
		avaj
		\end{verbatim}
	\end{Exercice}

	\begin{Exercice}{Miroir}	
		\'Ecrire un programme qui demande à l'utilisateur 
		un mot et affiche son miroir.
		
		Par exemple, si l'utilisateur entre "java", le programme affiche
		
		\begin{verbatim}
		avaj
		\end{verbatim}
	\end{Exercice}


	\begin{Exercice}{Voyelles}	
		\'Ecrire un programme qui demande à l'utilisateur 
		un mot et affiche les voyelles.
		
		Par exemple, si l'utilisateur entre "programmation", le programme affiche
		
		\begin{verbatim}
		oaaio
		\end{verbatim}
	\end{Exercice}

\section{Exercices Récapitulatifs}


	\begin{Exercice}{Palindrome}
		Un palindrome est un mot dont la succession de lettres est la même 
		de gauche à droite et de droite à gauche. 
		\'Eté, ressasser ou kayak sont des palindromes.
		  	
		\'Ecrire un programme qui demande à l'utilisateur 
		un mot et affiche si ce mot est un palindrome ou non.
	\end{Exercice}

	


\end{document}