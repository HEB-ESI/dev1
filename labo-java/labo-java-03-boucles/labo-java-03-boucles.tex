\documentclass[a4paper,11pt]{article}

%=========================
% Les styles
%=========================
\usepackage{style-esi/french}	% Francise LaTeX
\usepackage{style-esi/td}
\usepackage{style-esi/licence}	% Affiche une licence dans le document
\usepackage{style-esi/exercice}
\usepackage{style-esi/listing}
\usepackage{style-esi/tutoriel}

\date{2018 -- 2019}
\siglecours{DEV1}
\libellecours{Laboratoires Java I}
\libelledocument{TD 3 -- Boucles}
\sigleprof{}



\begin{document}

\entete
\titre
\ccbysa{esi-dev1-list@he2b.be}
\lastedit


	%===================
	%  Contenu
	%====================	
	Dans ce TD vous trouverez xxx
	\tableofcontents

	\newpage

%===================
\section{Boucles}
%====================	
 
	\begin{Exercice}{}	
		Dans un package \texttt{g12345.dev1.td3} créez une classe \texttt{Exercice1}.
		Dans cette classe écrivez un programme qui affiche les entiers de 1 à n.
	\end{Exercice}

	\begin{Exercice}{}	
		Dans votre package \texttt{g12345.dev1.td3} créez une classe \texttt{Exercice2}.
		Dans cette classe écrivez un programme qui affiche les nombres pair compris entre 1 à n.
	\end{Exercice}


	\begin{Exercice}{}	
		Dans votre package \texttt{g12345.dev1.td3} créez une classe \texttt{Exercice2}.
		Dans cette classe écrivez un programme qui affiche les entiers de n à 1.
	\end{Exercice}

	\begin{Exercice}{}	
		Dans votre package \texttt{g12345.dev1.td3} créez une classe \texttt{Exercice2}.
		Dans cette classe écrivez un programme qui affiche les entiers de -n à n.
	\end{Exercice}


b) les nombres de 1 à n en ordre décroissant ;
c) les nombres de -n à n ;
d) les multiples de 5 de 1 à n ;
e) les multiples de n de 1 à 100

 

% 	\listing{java}{code/Expression.java}



\end{document}