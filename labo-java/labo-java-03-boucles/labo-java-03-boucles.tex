\documentclass[a4paper,11pt]{article}

%=========================
% Les styles
%=========================
\usepackage{style-esi/french}	% Francise LaTeX
\usepackage{style-esi/td}
\usepackage{style-esi/licence}	% Affiche une licence dans le document
\usepackage{style-esi/exercice}
\usepackage{style-esi/listing}
\usepackage{style-esi/tutoriel}

\date{2018 -- 2019}
\siglecours{DEV1}
\libellecours{Laboratoires Java I}
\libelledocument{TD 3 -- Boucles}
\sigleprof{}



\begin{document}

\entete
\titre
\ccbysa{esi-dev1-list@he2b.be}
\lastedit


	%===================
	%  Contenu
	%====================	
	Dans ce TD vous trouverez xxx
	\tableofcontents

	\newpage

%===================
\section{Boucles - \texttt{while}}
%====================	

 	\listing{java}{code/Boucle.java}
	Le code ci-dessus affiche les nombres de 1 à 10.
	Ce programme s'exécute comme suit~:
	\begin{itemize}
		\item  le programme commence à la ligne 6 où la variable $i$ de type entier est déclarée et initialisée à 1~;
		\item  ensuite, la condition du \texttt{while} de la ligne 7 est évaluée, sa valeur est vrai car $i$ vaut 1 et est donc inférieur à 10~;
		\item puisque la condition est vraie le \emph{corps} de la boucle s'exécute:
			\begin{itemize}
				\item la ligne 8 affiche la valeur de l'entier $i$: 1
				\item la ligne 9 assigne la valeur 2 à i (c'est la valeur de l'expression $i+1$)
			\end{itemize}
		\item l'exécution du corps de la boucle étant terminé le programme retourne à l'instruction 7, 
			et évalue à nouveau la condition avec la nouvelle valeur de i qui vaut maintenant 2.
		\item  la condition de la ligne 7 étant vraie (car $i$ vaut 2 et est donc inférieur à 10) on exécute à nouveau le corps de la boucle~;
		\item le programme continue ainsi jusqu'à ce que $i$ atteigne la valeur 11. 
			\`A ce moment la condition du \texttt{while} est évaluée à faux et l'instruction 
			\texttt{while} prend fin.
			Comme aucune instruction ne suit ce \texttt{while}, le programme se termine.
	\end{itemize}


	\begin{Exercice}{Suites d'entiers}	\label{ex1}
		Créez un package \texttt{g12345.dev1.td3}.
		Dans ce package créez un classe \texttt{Exercice1} dans laquelle vous 
		écrivez un programme qui affiche~:
		\begin{itemize}
			\item les nombres de 1 à 10~;
			\item les nombres de -10 à 10~;
			\item les nombres de 10 à 1~;
			\item les nombres pairs de 1 à 20~;
			\item les multiples de 5 de 1 à 100 ;
		\end{itemize}
	\end{Exercice}
 
	\begin{Exercice}{Suites d'entiers paramétrées}	
		Dans votre package créez un classe \texttt{Exercice2}.
		Dans cette classe écrivez un programme qui demande à l'utilisateur un nombre entier $n$ et affiche:
		\begin{itemize}
			\item  les nombres de 1 à n~;
			\item  les nombres pairs de 1 à n~;
			\item les nombres de -n à n~;
			\item les multiples de 5 de 1 à n ;
			\item les multiples de n de 1 à 100.
		\end{itemize}
	\end{Exercice}

 
 \section{Lecture de données multiples}
 
 	Dans cette section nous allons voir comment demander une série de données à l'utilisateur.
	Il y a plusieurs manières de procéder, nous allons en voir 3.

	\subsection{Variante 1 : le nombre de valeurs est fixe}

 		\listing{java}{code/LectureMultiple1.java}

	
	\subsection{Variante 2 : le nombre de valeurs est connu}
	
 		\listing{java}{code/LectureMultiple2.java}

		\begin{Exercice}{Moyenne}
			\'Ecrivez un programme qui demande à l'utilisateur le nombre $n$ de valeurs qu'il veut introduire,
			lit ces $n$ valeurs au clavier et affiche le nombre de valeurs, la somme de ces valeurs, la moyenne. 
		\end{Exercice}

		\begin{Exercice}{Pair ou impair}
			\'Ecrivez un programme qui demande à l'utilisateur le nombre $n$ de valeurs qu'il veut introduire,
			lit ces $n$ valeurs au clavier et pour chacun des nombres affiche au fur et à mesure s'il est pair ou impair. 
		\end{Exercice}


		\begin{Exercice}{Maximum et minimum}
			\'Ecrivez un programme qui demande à l'utilisateur le nombre $n$ de nombres qu'il veut introduire,
			lit ces $n$ nombres au clavier et affiche le plus grand. 
		
			Astuce: utilisez une variable \texttt{maximum} et un \texttt{if} à l'intérieur de la boucle.
		\end{Exercice}


		\begin{Exercice}{}
		
		\end{Exercice}


	\subsection{Variante 3 : valeur sentinelle}
	
 		\listing{java}{code/LectureMultiple3.java}

		\begin{Exercice}{}
		
		\end{Exercice}

		\begin{Exercice}{}
		
		\end{Exercice}


\section{Exercices Récapitulatifs}

\end{document}