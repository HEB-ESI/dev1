\documentclass[a4paper,11pt]{article}

%=========================
% Les styles
%=========================

\usepackage{style-esi/french}
\usepackage{style-esi/td}
\usepackage{style-esi/licence}
\usepackage{style-esi/exercice}
\usepackage{style-esi/listing}
\usepackage{style-esi/tutoriel}

\newcommand{\publicbasepath}
{https://git.esi-bru.be/dev1/labo-java/tree/master/td02-variables}

\renewcommand{\listingpublicpath}{\publicbasepath/code/}
\renewcommand{\listingsrcpath}{code/}
\marginnumberfalse
\marginsectiontrue



\date{2018 -- 2019}
\siglecours{DEV1}
\libellecours{Laboratoires Java I}
\libelledocument{TD 2 -- Variables}
\sigleprof{}



\begin{document}

\entete
\titre

\ccbysa{esi-dev1-list@he2b.be}
\lastedit


	%===================
	%  Contenu
	%====================	
	Dans ce TD vous vous familiariserez avec la notion de variable et vous 
	réaliserez vos premiers programmes interactifs.

	Les codes sources et les solutions de ce TD se trouvent à l'adresse~: 
	
	\url{\publicbasepath}	
	
	\tableofcontents

	\newpage
	
%===================
\section{Variables}
%====================	

	Le programme suivant affiche l'aire d'un rectangle de longueur 12 et de largeur 4.
	
	\listing{Java}{Variables.java}

	\begin{itemize}
		\item \`A la ligne 6 la variable \texttt{longueur} est \emph{déclarée} 
			avec le type \texttt{int}, elle peut donc 'contenir' des entiers. 
			Sur cette même ligne on lui \emph{assigne} la valeur 12.  

		\item \`A la ligne 7 la variable \texttt{largeur} est \emph{déclarée} avec le type \texttt{int}. 
			Sur cette même ligne on lui \emph{assigne} la valeur 4.
	
		\item \`A la ligne 9 on affiche le résultat de la multiplication de la valeur de la variable \texttt{longueur}, 
			qui vaut 12, et de la valeur de la variable \texttt{largeur}, qui elle vaut 4.  
  	\end{itemize}

	Pour manipuler des nombres décimaux on utilise une variable de type 
	\texttt{double}~:
	
	\code{java}{double taille = 1.92;}




	\begin{Exercice}{Petits calculs avec 2 variables} 		
		Créez un package \texttt{g12345.dev1.td2} où vous remplacez \texttt{g12345} par 
		votre matricule.

		Dans un classe \texttt{Exercice01} déclarez 2 variables de type \texttt{int}~: 
		\texttt{a}, \texttt{b} et initialisez-les avec les valeurs 51 et 17.
		
		Ensuite affichez sur la sortie standard la valeur de :
		\begin{itemize}
		 	\item a+b
			\item a-b
			\item a*b
			\item a/b
			\item a\%b
			\item a*a+b*b
		\end{itemize} 
	\end{Exercice}

	\begin{Exercice}{Petits calculs avec 3 variables} 
		Dans un classe \texttt{Exercice02} déclarez 3 variables de type \texttt{double}~: 
		\texttt{a}, \texttt{b} et \texttt{c} et initialisez avec les valeurs 
		2.5, 3.3 et 4.9 respectivement.
		
		Ensuite affichez sur la sortie standard la valeur de :
		\begin{itemize}
		 	\item 4 * a * c
			\item b*b - 4*a*c
		\end{itemize} 
	\end{Exercice}


\newpage
%===================
\section{Lecture au clavier}
%====================	


	En java la lecture au clavier se fait en 3 étapes.

	\begin{enumerate}
		\item \emph{Importer} le \emph{lecteur} (\texttt{Scanner}) ligne 3 du code ci-dessous.
		\item \emph{Déclarer} et \emph{initialiser} le lecteur:  \texttt{Scanner clavier = new Scanner(System.in);}
		\item La lecture proprement dite: \texttt{int longueur = clavier.nextInt();}
	\end{enumerate}

	\listing{java}{AireRectangle.java}

	Pour lire un nombre décimal on utilise la méthode \texttt{nextDouble()}~:
	
	\code{java}{double longueur = clavier.nextDouble();}


	\begin{Exercice}{Aire d'un carré}
		Dans un classe \texttt{Exercice03} écrivez un programme qui demande 
		le côté d'un carré (un nombre entier) à l'utilisateur et 
		affiche l'aire de ce carré.
	\end{Exercice}
	
	\begin{Exercice}{Périmètre et aire d'un cercle}
		\'Ecrivez un programme
		qui demande à l'utilisateur le rayon d'un cercle (un nombre décimal)
		et affiche son périmètre et son aire.
	
		Rappel: le périmètre se calcule par la formule $2\pi r$ et 
		l'aire par la formule $\pi r^2$ où vous utiliserez 3.141593 comme valeur
		approchée de $\pi$.
	
		Attention: si votre système est configuré en français (cela s'appelle la 
		\emph{locale}) 
		vous devrez entrer ce nombre avec une virgule, s'il est en anglais ce sera avec
		un point.
	\end{Exercice}

	\begin{Exercice}{Petits calculs avec 2 nombres lus au clavier} 
		\'Ecrivez un programme qui demande 
		deux nombres entiers, \texttt{a} et \texttt{b}, à l'utilisateur et affiche la 
		valeur de :
		\begin{itemize}
		 	\item a+b
			\item a-b
			\item a*b
			\item a/b
			\item a\%b
			\item a*a+b*b
		\end{itemize} 
	\end{Exercice}


\section{Exercices Récapitulatifs}

		
	\begin{Exercice}{Centaines, dizaines, unités} 
		\'Ecrivez un programme qui demande à l'utilisateur 
		un nombre entier \texttt{nb} et affiche la valeur des 
		expressions suivantes:
		\begin{itemize}
			\item \texttt{nb\%10} - les unités
			\item \texttt{(nb/10)\%10} - les dizaines
			\item \texttt{(nb/100)\%10} - les centaines
		\end{itemize}
		Par exemple, si l'utilisateur entre 362 votre programme affiche
		\begin{verbatim}
		2
		6
		3
		\end{verbatim}
	\end{Exercice}	

	\begin{Exercice}{Miroir} 
		\'Ecrivez un programme qui demande à l'utilisateur 
		un nombre \texttt{nb} compris entre 100 et 999 et affiche la valeur miroir:

		Par exemple, si l'utilisateur entre 736 votre programme affiche 637.
		
		Astuce: utilisez la technique de l'exercice précédent afin d'extraire les
		unités, les dizaines et les centaines.
	\end{Exercice}	

	
	\begin{Exercice}{Secondes en minutes} 
		\'Ecrivez un programme qui demande un nombre de secondes à l'utilisateur
		et qui affiche le nombre de minutes que cela représente.

		Par exemple: 
		si l'utilisateur entre 217 secondes, le programme affiche 3, 
		car 217 secondes correspond à 3 minutes et 37 secondes.
	\end{Exercice}

	\begin{Exercice}{Temps en secondes} 
		\'Ecrivez un programme qui demande 
		un nombre d'heures, un nombre de minutes et un nombre de secondes
		et qui affiche le nombre de secondes totales.
		
		Par exemple: si l'utilisateur entre 2 heures, 10 minutes et 27 secondes, le 
		programme affiche
		7827. En effet 2 heures donnent 7200 secondes, 10 minutes sont 600 secondes 
		auxquelles il faut ajouter les 27 secondes: 7200 + 600 + 27 = 7827. 
	\end{Exercice}


	\begin{Exercice}{Les intérêts}
		\'Ecrivez un programme qui demande un montant (un nombre décimal)
		à l'utilisateur et affiche les intérêts reçus après 1 an pour un montant placé 
		en banque à du 2\% d'intérêt.
	\end{Exercice}

	\begin{Exercice}{Prix TTC}
		\'Ecrivez un programme qui demande à l'utilisateur:
		\begin{itemize}
			\item le prix unitaire d’un produit hors TVA, 
			\item le taux de TVA en \% (un entier) 
			\item la quantité de produit vendue à un client (un entier)
		\end{itemize}
		et affiche le prix total à payer par ce client.
		
		Exemple: si le client achète 5 drones au prix de 1000 euros hors TVA et que la
		TVA est de 21\%, le programme affichera 6050 euros.
	\end{Exercice}



\end{document}
