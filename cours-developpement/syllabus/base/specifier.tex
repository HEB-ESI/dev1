%==============================
\chapter{Spécifier le problème}
%==============================

	Comme nous l’avons dit, 
	un problème ne sera véritablement bien spécifié 
	que s’il s’inscrit dans le schéma suivant~:
		
	\medskip
	\begin{center}
	\begin{Ovalbox}
		{\textbf{étant donné} [les données] 
		\textbf{on demande} [résultat]}
	\end{Ovalbox}
	\end{center}
	\medskip
	
	La première étape dans la résolution d’un problème est de
	préciser ce problème à partir de l’énoncé,
	c-à-d de déterminer et préciser les données et le résultat.
	
	\minitoc

	%----------------------------------------------
	\section{Déterminer les données et le résultat}
	%----------------------------------------------
	
		La toute première étape est de parvenir à extraire
		d’un énoncé de problème, quelles sont les données
		et quel est le résultat attendu%
		\footnote{%
			Plaçons-nous pour le moment dans le cadre
			de problèmes où il y a exactement un résultat.%
		}. Dans la suite, nous utiliserons un exemple très simple afin 
		d'illustrer les concepts qui nous intéressent. 
	
		\clearpage
		\vspace{1cm}
	
		\begin{Emphase}
			\paragraph{Exemple.}
			Soit l’énoncé suivant~:
			\og
				Calculer la surface d’un rectangle 
				à partir de sa longueur et sa largeur
			\fg.
			
			Quelles sont les données~? Il y en a deux~:	
			\begin{itemize}
				\item la longueur du rectangle~;
				\item sa largeur.
			\end{itemize}
		
			Quel est le résultat attendu~? la surface du rectangle.
		\end{Emphase}
		
	%----------------------------------------------
	\section{Les noms}\index{noms}
	%----------------------------------------------
	
		Pour identifier clairement chaque \textbf{donnée}
		et pouvoir y faire référence dans le futur algorithme
		et dans le programme
		nous devons lui attribuer un \textbf{nom}%		
		\footnote{%
				Dans ces notes, nous nous efforcerons de choisir des noms en 
				anglais,
				mais vous pouvez très bien choisir
				des noms français si vous ne vous sentez pas encore suffisamment 
				à l'aise avec l'anglais.
		}.
		Il est important de bien choisir les noms. 
		Le but est de trouver un nom qui soit suffisamment court,
		tout en restant explicite et ne prêtant pas à confusion.
		
	
		\begin{Emphase}
			\paragraph{Exemple.}	
			Quel nom choisir pour la longueur d’un rectangle~?
	
			On peut envisager les noms suivants~:
			\begin{itemize}
			\item
				\pc{length} est probablement le plus approprié.
			\item
				\pc{rectangleLength} peut se justifier
				pour éviter toute ambigüité avec une autre longueur.
			\item
				\pc{len} peut être admis
				si le contexte permet de comprendre immédiatement
				l’abréviation.
			\item
				\pc{l} est à proscrire car pas assez explicite.
			\item
				\pc{theLengthOfMyRectangle} est inutilement long. 
			\item
				\pc{foo} (truc en anglais) ou \pc{tmp} ne sont pas de bons choix
				car ils n’ont aucun lien avec la donnée.
			\end{itemize}
		\end{Emphase}

		Les noms donnés à chaque donnée seront directement associés à une 
		\textbf{variable} lorsque l'algorithme sera traduit en un programme dans 
		un langage de programmation. 
		
		\marginicon{definition}\index{variable} \textbf{Variable}~:~ Emplacement
		mémoire nommé pouvant contenir une valeur. Cette valeur peut être
		remplacée par une autre — elle est variable — au fil de l'exécution du
		programme. 
		
		Nous allons également donner un \textbf{nom à l’algorithme}
		de résolution du problème.
		Cela permettra d’y faire référence dans les explications
		mais également de l’utiliser dans d’autres algorithmes.
		Généralement, un nom d’algorithme est~:
		\begin{itemize}
		\item soit un verbe indiquant ce que fait l’algorithme~;
		\item soit un nom indiquant le résultat fourni.	
		\end{itemize}
	
		\clearpage
		\begin{Emphase}
			\paragraph{Exemple.}	
			Quel nom choisir pour l’algorithme 
			qui calcule la surface d’un rectangle~?
	
			On peut envisager 
			le verbe \pc{computeRectangleArea}
			ou le nom \pc{rectangleArea} (notre préféré).
			On pourrait aussi simplifier en \pc{area}
			s’il est évident que le problème traite des rectangles.
		\end{Emphase}
	
		Les noms donnés aux algorithmes deviendront généralement les noms des
		programmes lorsque ces algorithmes seront implémentés. 

		Notons que les langages de programmation imposent certaines limitations
		(parfois différentes d’un langage à l’autre) ce qui peut nécessiter une
		modification du nom lors de la traduction de l’algorithme en un
		programme. À chaque langage de programmation sont également associées
		des \textbf{conventions d'écriture} que les développeurs respectent.
		Bien que nous y reviendrons plus en détail dans la section
		\ref{lisibilite}, notons déjà que la simple convention de nom de méthode
		change d'un langage à l'autre. 
		
		Par exemple pour \textit{rectangle area}, nous utiliserons
		\pc{rectangleArea} en langage Java mais \pc{rectangle\_area} en langage
		C, C++ et Python. 

	%----------------------------------------------
	\section{Les types}\index{types}
	%----------------------------------------------
		
		Nous allons également attribuer un \textbf{type} à chaque donnée ainsi
		qu’au résultat.  Le \textbf{type} décrit la nature de son contenu,
		quelles valeurs elle peut prendre.
		
		Certains langages imposent et vérifient le type de chaque variable tandis
		que d'autres non. En Java, chaque variable et chaque donnée ont un type. 
		
		Dans un premier temps, nous utiliserons ces \textbf{types}
		\footnote{Écrire ces différents types en français n'est pas une 
		faute. Dans ces notes nous utiliserons l'anglais}~:
		
		\begin{center}
			\begin{tabular}[t]{|p{1.4cm}|p{8cm}|}
				\hline
				\rowcolor{black!40}
				\color{white}\bf\large Type & \\
				\hline
				\pc{int} & pour les nombres entiers\\
				\pc{double} & pour les nombres réels\\
				\pc{String}
						\footnote{Notez la majuscule, indispensable en Java} 
						& \makecell[tl]{
							pour les chaines de caractères, les textes\\
							par exemple~:\\
							\hspace{1cm}\pc{"Bonjour"},\\
							\hspace{1cm}\pc{"Bonjour le monde !"},\\
							\hspace{1cm}\pc{"a"}, \pc{""}\dots
						}
					\\
				\pc{boolean} & quand la valeur 
			ne peut être que \pc{true} (vrai) ou \pc{false} (faux)\\
			\hline
			\end{tabular}
		\end{center}


		\clearpage
		\begin{Emphase}
			\paragraph{Exemples.}	
			\begin{itemize}
			\item Pour la longueur, la largeur et la surface d’un rectangle, 
				on prendra un réel (le type \pc{double}).
			\item Pour le nom d’une personne, on choisira une chaine de caractère
				(le type \pc{String}).
			\item Pour l’âge d’une personne, un entier est indiqué (\pc{int}).
			\item Pour décrire si un étudiant est doubleur ou pas, un booléen 
				(\pc{boolean}) est adapté.
			\item Pour représenter un mois, on préférera souvent un entier
				donnant le numéro du mois (par ex: 3 pour le mois de mars)
				plutôt qu’une chaine (par ex: "mars")
				car les manipulations, les calculs seront plus simples.
			\end{itemize}
		\end{Emphase}
	
		\subsection{Il n’y a pas d’unité}
		%-----------------------------------------
	
			Un type numérique indique que les valeurs possibles seront
			des nombres. Il n’y a là aucune notion d’unité.
			Ainsi, la longueur d’un rectangle, un réel, 
			peut valoir $2.5$ mais certainement pas $2.5$\,$cm$.
			Si cette unité a de l’importance,
			il faut la spécifier dans le nom de la donnée ou en commentaire.
			
			\begin{Emphase}
				\paragraph{Exemple.}
				Faut-il préciser les unités 
				pour les dimensions d’un rectangle~?
				
				Si la longueur d’un rectangle vaut $6$, 
				on ne peut pas dire s’il s’agit de centimètres, 
				de mètres ou encore de kilomètres.
				Pour notre problème de calcul de la surface,
				ce n’est pas important~;
				la surface n’aura pas d’unité non plus.

				Notre seule contrainte est que la longueur et la largeur soient
				exprimées dans la même unité.
				
				Si, par contre, 
				il est important de préciser que la longueur
				est donnée en centimètres,
				on pourrait l’expliciter en la nommant
				\pc{lengthCm}.	
			\end{Emphase}
	
		\subsection{Préciser les valeurs possibles}
		%-----------------------------------------
		
			Nous aurions pu introduire un seul type numérique
			mais nous avons choisi de distinguer les entiers et les réels.
			Pourquoi~?
			Préciser qu’une donnée ne peut prendre que des valeurs entières
			(par exemple dans le cas d’un numéro de mois)
			aide le lecteur à mieux la comprendre.
			Nous allons aussi pouvoir définir des opérations propres aux entiers
			(le reste d’une division par exemple).
			Enfin, pour des raisons techniques,
			beaucoup de langages font cette distinction.
			
			Même ainsi, le type choisi n’est pas toujours assez précis.
			Souvent, la donnée ne pourra prendre que certaines valeurs.
			
			\clearpage
			\begin{Emphase}
				\paragraph{Exemples.}	
				\begin{itemize} 
				\item Un âge est un entier qui ne peut pas être négatif.
				\item Un mois est un entier compris entre 1 et 12.
				\end{itemize}
			\end{Emphase}
			
			Ces précisions pourront être données en commentaire
			pour aider à mieux comprendre le problème et sa solution.
	
		\subsection{Le type des données complexes}
		%-----------------------------------------
		
			Parfois, aucun des types disponibles ne permet de représenter 
			la donnée.
			Il faut alors la décomposer.
			
			\begin{Emphase}
				\paragraph{Exemple.}	
				Quel type choisir 
				pour la date de naissance d’une personne~?
				
				On pourrait la représenter dans une chaine 
				(par ex: "17/3/1985")
				mais cela rendrait difficile le traitement, les calculs
				(par exemple, déterminer le numéro du mois).
				Le mieux est sans doute de la décomposer en trois parties~: 
				le jour, le mois et l’année, tous des entiers.
			\end{Emphase}
			
			Plus tard, vous verrez qu’il est possible de définir de nouveaux
			types de données grâce aux \emph{structures} pour les algorithmes et
			aux \emph{classes} pour les langages orientés objets tels que Java. 
			
			Il sera alors possible de définir et utiliser un type \pc{Date}
			et il ne sera plus nécessaire de décomposer une date en trois
			morceaux bien que le type sera bien entendu composé de trois valeurs.

	
		\subsection{Exercice}
		%----------------------------------------------
		
			Quel(s) type(s) de données utiliseriez-vous pour représenter~:
			\begin{itemize}
				\item le prix d’un produit en grande surface~;
				\item la taille de l’écran de votre ordinateur~;
				\item votre nom~;
				\item votre adresse~;
				\item le pourcentage de remise proposé pour un produit~;
				\item une date du calendrier~;
				\item un moment dans la journée~?
			\end{itemize}
			
	%----------------------------------------------
	\section{Résumés}
	%----------------------------------------------
	
		Toutes les informations déjà collectées sur le problème
		peuvent être résumées. 
		
		\subsection{Résumé graphique}
		
		Représentation graphique du problème.
	
		\begin{Emphase}
			\paragraph{Exemple.}
			Pour le problème, de la surface du rectangle, 
			on fera le schéma suivant~:
			
			\center\flowalgodd{longueur (réel positif)}
			{largeur (réel positif)}{surfaceRectangle}{réel}	
		\end{Emphase}

		\subsection{Résumé textuel}

		Résumé textuel du problème, indiquant clairement les données et les 
		résultats recherchés. 

		\begin{Emphase}
			\paragraph{Exemple.}
			\begin{tabular}[t]{|>{\columncolor{black!40}}r|l|}
				\hline
				\textbf{Données} & \makecell[tl]{
					longueur (un réel positif)\\
					largeur (un réel positif)
				}\\
				\hline
				\textbf{Résultat} & la surface du rectangle\\
				\hline
			\end{tabular}
		\end{Emphase}
		
	%----------------------------------------------
	\section{Exemples numériques}
	%----------------------------------------------
	
		Une dernière étape pour vérifier que le problème
		est bien compris est de donner quelques exemples numériques.
		Il est possible de les spécifier en français, 
		via un graphique ou via une notation compacte.

		\begin{Emphase}
			\paragraph{Exemples.}
			Voici différentes façons de présenter des exemples numériques
			pour le problème de calcul de la surface d’un rectangle~:
			\begin{itemize}
			\item En français~: 
				si la longueur du rectangle vaut 3 et sa largeur vaut 2, 
				alors sa surface vaut 6.			
			\item Via un schéma~:	
				\begin{center}
					\flowalgodd{length (3)}{width (2)}{rectangleArea}{6}
				\end{center}
			\item En notation compacte~:
				\pc{rectangleArea(3, 2)} donne/vaut $6$.
			\end{itemize}
		\end{Emphase}
	
