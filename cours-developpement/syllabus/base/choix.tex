%===========================================
\chapter{Une question de choix}
%===========================================

	Les \textbf{alternatives}\index{alternatives} permettent de n’exécuter des
	instructions que si une certaine \emph{condition} est vérifiée.  
	Par exemple, si le filtre à café est vide, le remplir. Dans la vie, nous 
	testons notre environnement, dans nos algorithmes
	et dans nos programmes, nous allons tester les données.
	
	Les algorithmes et les programmes vus jusqu’à présent ne proposent qu’un
	seul \og{}chemin\fg{}, une seule \og{}histoire\fg{}.  À chaque exécution de
	l’algorithme, les mêmes instructions s’exécutent dans le même ordres.  Les
	alternatives permettent de créer des histoires différentes, d’adapter les
	instructions aux valeurs concrètes des données.  

	\minitoc 

%===========================================
\section{Le si (\textit{if-then})}
\index{si}\index{if}
%===========================================
		
	Il existe des situations où des instructions ne doivent pas toujours être
	exécutées et un test va nous permettre de le savoir.
	
	\begin{langagenaturel}
		si une certaine condition est vraie alors\\
		\tab exécuter une ou plusieurs actions\\
		fin si\\
		continuer l'algorithme
	\end{langagenaturel}

	Dans la grammaire du langage Java, cette instruction s'écrit~:
	
	\begin{grammaire}
		\grammarrule{IfThenStatement:}
		    if ( \grammarrule{Expression} )
 		        \grammarrule{Statement}
	\end{grammaire}

	\begin{enumerate}
		\item \grammarrule{Expression} représente une expression booléenne. 
			c'est-à-dire ayant comme valeur \pc{true} ou \pc{false}.
		\item \grammarrule{Statement} représente une instruction ou un 
			\textbf{bloc} d'instructions. En langage Java, un bloc 
			\index{bloc} d'instructions est toujours délimités par une paire 
			d'accolades. 
	\end{enumerate}

	\marginicon{attention}
	\textbf{Remarque}

	Attention, un \og{}si\fg{} n’est pas une règle que l’ordinateur doit
	apprendre et exécuter à chaque fois que l’occasion se présente.  La
	condition n’est testée que lorsqu’on arrive à cet endroit de l’algorithme.
	
	\textbf{Exemple}

	Supposons que la variable \pc{nb} contienne un nombre positif ou négatif.
	Et supposons que l’on veuille le rendre positif. Il faudra tester son signe
	et, s’il est négatif, l'inverster.  Par contre, s’il est positif, il n’y
	a rien à faire.  
	
	Voici comment écrire un algorithme sous différentes formes~:
		
	\begin{langagenaturel}
		si nb < 0 alors\\
			\tab nb = -nb\\
		fin si
	\end{langagenaturel}
	
	ou
	
	\begin{langagenaturel}
		if (nb < 0)\\
			\tab nb = -nb\\
	\end{langagenaturel}
	
	Un organigramme aurait cette forme~:

	\begin{center}
		\begin{tikzpicture}[node distance = 2cm, auto]
			\node (start) [startstop] {Start};
			\node (dec1) [decision, below of=start, yshift=-1cm] {nb < 0};
			\draw [arrow] (start) -- (dec1);
			\node (proc1) [process, below of=dec1, yshift=-2cm] {nb = -nb};
			\node (stop) [startstop, below of=proc1] {End};
			\draw [arrow] (dec1) -- node[anchor=south, xshift=-5mm] {true} (proc1);
			\draw [arrow] (dec1) .. controls (4.5,-6) 
				.. node[anchor=east] {false} (stop);
			\draw [arrow] (proc1) -- (stop);
		\end{tikzpicture}	
	\end{center}
	
	En langage Java, le \emph{si} s'écrit comme suit~:

	\begin{java}
if (nb < 0)
	nb = -nb;
	\end{java}

	Pour s'éviter des erreurs, nous utiliserons toujours le bloc d'instructions.
	Nous conseillons de faire de même et nous écrirons~:

	\begin{java}
if (nb < 0){
	nb = -nb;
}
	\end{java}


	Traçons l'algorithme dans deux cas différents pour bien illustrer 
	son déroulement.
	
	\begin{center}
	\begin{tabular}{|>{\centering\arraybackslash}m{1cm}
					|*{2}{>{\centering\arraybackslash}m{1cm}}|}
		\hline
		\rowcolor{black!40}
			\verb_#_  & nb & test \\			
		\hline
			  & -3                   & {}   \\
			1 & {\color{gray}$\mid$} & vrai \\
			2 & 3                    & {}   \\
		\hline
	\end{tabular}
	\qquad
	\begin{tabular}{|>{\centering\arraybackslash}m{1cm}
					|*{2}{>{\centering\arraybackslash}m{1cm}}|}
		\hline
		\rowcolor{black!40}
			\verb_#_  & nb & test \\			
		\hline
			  & 3                    & {}   \\
			1 & {\color{gray}$\mid$} & faux \\
		\hline
	\end{tabular}
	\end{center}
	
	\begin{Emphase}
		\paragraph{Exercice de compréhension}
		Tracez  l'algorithme écrit en langage Java avec les valeurs fournies 
		et donnez la valeur de retour.
		
		\begin{java}
public static int exercice(int a, int b){
	int c;
	c = 2 * a;
	if (c > b){
		c = c-b;
	}
	return c;
}
		\end{java}
		
		\begin{multicols}{2}
		\begin{itemize}
		\item \pc{exercice(2, 5)} = \_\_\_
		\item \pc{exercice(4, 1)} = \_\_\_
		\end{itemize}
		\end{multicols}	
	\end{Emphase}


	\pagebreak[4]
%===============================
\section{Le si-sinon (\textit{if-then-else})}
\index{if-else}\index{si-sinon}
%===============================
	
	La construction \pc{si-sinon} permet d’exécuter certaines instructions ou
	d’autres en fonction d’un test.  
	
	\begin{langagenaturel}
		si une certaine condition est vraie alors\\
			\tab exécuter une ou plusieurs actions\\
		sinon (la condition est alors fausse)\\
			\tab exécuter d'autres actions\\
		fin si\\
		continuer l'algorithme 
	\end{langagenaturel}

	Dans la grammaire du langage Java, cette instruction s'écrit~:

	\begin{grammaire}
		\grammarrule{IfThenElseStatement:}
		    if ( \grammarrule{Expression} )
		        \grammarrule{Statement}
		    else
		        \grammarrule{Statement}
	\end{grammaire}

	\textbf{Exemple}

	Pour illustrer cette instruction, nous allons écrire un algorithme qui 
	recherche le maximum de deux nombres. 
	
	Pour déterminer le maximum de deux nombres, c’est-à-dire la plus grande
	des deux valeurs, il y aura deux chemins possibles.  Le maximum devra
	prendre la valeur du premier nombre ou du second selon que le premier est
	plus grand que le second ou pas.

	Voici comment écrire cet algorithme sous différentes formes~:

	\begin{langagenaturel}
		if (nb1 > nb2)\\
			\tab max = nb1\\
		else\\
			\tab max = nb2\\
	\end{langagenaturel}

	Un organigramme aurait cette forme~:
	
	\begin{center}
		\begin{tikzpicture}[node distance = 2cm, auto]
			\node (start) [startstop] {Start};
			\node (dec1) [decision, below of=start, yshift=-1cm] {nb1 > nb2};
			\draw [arrow] (start) -- (dec1);
			\node (proc1) [process, below of=dec1, yshift=-2cm, xshift=-2cm] 
				{max = nb1};
			\node (proc2) [process, below of=dec1, yshift=-2cm, xshift=2cm] 
				{max = nb2};
			\node (stop) [startstop, below of=dec1, yshift=-5cm] {End};
			\draw [arrow] (dec1) -- node[anchor=west, xshift=-10mm] {true} (proc1);
			\draw [arrow] (dec1) -- node[anchor=east, xshift=-1mm] {false} (proc2);
			\draw [arrow] (proc1) -- (stop);
			\draw [arrow] (proc2) -- (stop);
		\end{tikzpicture}	
	\end{center}
	
	En supposant les variables déclarées, le \textit{si-alors}, s'écrit comme
	suit en langage Java~:

	\begin{java}
if (nb1 > nb2){		
	max = nb1;
} else {
	max = nb2;
}
	\end{java}

	Traçons l'algorithme dans différentes situations.
	
	\begin{center}
	\begin{tabular}{|>{\centering\arraybackslash}m{6mm}
					|*{4}{>{\centering\arraybackslash}m{1cm}}|}
		\hline
		\rowcolor{gray!40}
			\verb_#_  & nb1 & nb2 & max & test \\			
		\hline
			  & 3 & 2 & indéfini & {} \\
			1 & {\color{gray}$\mid$} & {\color{gray}$\mid$} & indéfini & vrai \\
			2 & {\color{gray}$\mid$} & {\color{gray}$\mid$} & 3        & {} \\
		\hline
	\end{tabular}
	\end{center}

	\begin{center}
	\begin{tabular}{|>{\centering\arraybackslash}m{6mm}
					|*{4}{>{\centering\arraybackslash}m{1cm}}|}
		\hline
		\rowcolor{gray!40}
			\verb_#_  & nb1 & nb2 & max & test \\			
		\hline
			  & 4 & 42 & indéfini & {} \\
			1 & {\color{gray}$\mid$} & {\color{gray}$\mid$} & indéfini & faux \\
			4 & {\color{gray}$\mid$} & {\color{gray}$\mid$} & 42        & {} \\
		\hline
	\end{tabular}
	\end{center}
	
	Le cas où les deux nombres sont égaux est également géré.
	
	\begin{center}
	\begin{tabular}{|>{\centering\arraybackslash}m{1cm}
					|*{4}{>{\centering\arraybackslash}m{1cm}}|}
		\hline
		\rowcolor{gray!40}
			\verb_#_  & nb1 & nb2 & max & test \\			
		\hline
			  & 4 & 4 & indéfini & {} \\
			1 & {\color{gray}$\mid$} & {\color{gray}$\mid$} & indéfini & faux \\
			4 & {\color{gray}$\mid$} & {\color{gray}$\mid$} & 4        & {} \\
		\hline
	\end{tabular}
	\end{center}
	
	\begin{Emphase}
		\paragraph{Exercice de compréhension}

		Tracez ces algorithmes ou programmes avec les valeurs fournies et donnez
		la valeur de retour.
	

		\begin{java}
public static int exercice(int a, int b){
	int c;
	if (a > b){
		c = a/b;
	} else {
		c = b%a;
	}
}
		\end{java}
	
		\begin{multicols}{2}
		\begin{itemize}
		\item \pc{exercice(2, 3)} = \_\_\_
		\item \pc{exercice(4, 1)} = \_\_\_
		\end{itemize}
		\end{multicols}
	

		\begin{java}
public static int exercice(int x1, int x2){
	boolean ok;
	ok = x1 > x2;
	if (ok){
		ok = ok && x1 == 4;
	} else {
		ok = ok || x2 == 3;
	}
	return x1 + x2;
}
		\end{java}
		
		\medskip
		\begin{multicols}{2}
		\begin{itemize}
		\item \pc{exercice(2, 3)} = \_\_\_
		\item \pc{exercice(4, 1)} = \_\_\_	
		\end{itemize}
		\end{multicols}	
		
	\end{Emphase}	
	
	\clearpage

%============================
\section{Le si-sinon-si}
\index{si-sinon-si}
%============================

	Avec cette construction, il est possible d'indiquer à un endroit de
	l’algorithme plus de deux chemins possibles et dès lors que la première
	condition est fausse, tester une condition supplémentaire à chaque étape. 
	
	\begin{langagenaturel}
		si une certaine condition est vraie alors\\
			\tab exécuter une ou plusieurs actions\\
		sinon si une autre condition est vraie alors\\
			\tab exécuter d'autres actions\\
		sinon \\
			\tab exécuter encore d'autres actions\\
		fin si\\
		continuer l'algorithme
	\end{langagenaturel}

	Dans la grammaire du langage Java, il n'y a pas d'instruction
	supplémentaire. Un \textit{si-sinon-si} n'étant que des \textit{si-sinon}
	imbriqués un peu comme suit~:

	\begin{grammaire}
	    if ( \grammarrule{Expression} )
	        \grammarrule{Statement}
	    else
	        if ( \grammarrule{Expression} )
	            \grammarrule{Statement}
	        else
	            \grammarrule{Statement}
	\end{grammaire}
	
	…qui seront généralement indentés comme suit~:
	
	\begin{grammaire}
	    if ( \grammarrule{Expression} )
	        \grammarrule{Statement}
	    else if ( \grammarrule{Expression} )
	        \grammarrule{Statement}
	    else
	       	\grammarrule{Statement}
	\end{grammaire}
	
	… mais voyons cela sur un exmeple. 

	\textbf{Exemple}
	
	Supposons que l'on veuille mettre dans la chaine \pc{signe} la valeur
	\pc{"positif"}, \pc{"négatif"} ou \pc{"nul"} selon qu’un nombre donné est
	positif, négatif ou nul.

	Voici comment écrire cet algorithme sous différents formats~:

	\begin{langagenaturel}
		if (nb > 0)\\
			\tab signe = "positif"\\
		else if (nb == 0)\\
			\tab signe = "nul"\\
		else\\
			\tab signe = "négatif"
	\end{langagenaturel}

	Un organigramme aurait cette allure~:
	
	\begin{center}
		\begin{tikzpicture}[node distance = 2cm, auto]
			\node (start) [startstop] {Start};
			\node (dec1) [decision, below of=start, yshift=-.5cm, 
				text width=.5cm] {};
			\draw [arrow] (start) -- (dec1);
			\node (proc1) [process, below of=dec1, yshift=-2cm, xshift=-4cm] 
				{s = "positif"};
			\node (proc2) [process, below of=dec1, yshift=-2cm] 
				{s = "nul"};
			\node (proc3) [process, below of=dec1, yshift=-2cm, xshift=4cm] 
				{s = "négatif"};
			\node (stop) [startstop, below of=dec1, yshift=-5cm] {End};
			\draw [arrow] (dec1) -- 
				node[anchor=west, xshift=-17mm] {nb>0} (proc1);
			\draw [arrow] (dec1) -- 
				node[anchor=east, xshift=-1mm] {nb == 0} (proc2);
			\draw [arrow] (dec1) -- 
				node[anchor=east, xshift=-1mm] {nb<0} (proc3);
			\draw [arrow] (proc1) -- (stop);
			\draw [arrow] (proc2) -- (stop);
			\draw [arrow] (proc3) -- (stop);
		\end{tikzpicture}	
	\end{center}
	
	Comme dit plus haut, en langage Java, il n'y a pas de structure particulière
	pour ce test. Le \pc{if-then-else} fait bien l'affaire. Seule l'indentation
	change un peu pour plus de lisibilité. Ce test peut donc s'écrire — en
	supposant toujours que les variables sont déclarées — comme ceci~:

	\begin{java}
if (nb > 0){
	s = "positif";
} else if (nb == 0) {
	s = "nul";
} else {
	s = "négatif";
}
	\end{java}

	Traçons l'algorithme dans différentes situations. 
	
	\begin{center}
	\begin{tabular}{|*{2}{>{\centering\arraybackslash}m{4mm}}
					 *{2}{>{\centering\arraybackslash}m{9mm}}|}
		\hline
		\rowcolor{black!40}
			\verb_#_  & nb & signe & test \\			
		\hline
			  & 2                    & indéfini             & {}   \\
			1 & {\color{gray}$\mid$} & {\color{gray}$\mid$} & vrai \\
			2 & {\color{gray}$\mid$} & "positif"            & {}   \\
		\hline
	\end{tabular}
	\end{center}
	
	\begin{center}
	\begin{tabular}{|*{2}{>{\centering\arraybackslash}m{4mm}}
					 *{2}{>{\centering\arraybackslash}m{9mm}}|}
		\hline
		\rowcolor{black!40}
			\verb_#_  & nb & signe & test \\			
		\hline
			  & 0                    & indéfini             & {}   \\
			1 & {\color{gray}$\mid$} & {\color{gray}$\mid$} & faux \\
			3 & {\color{gray}$\mid$} & {\color{gray}$\mid$} & vrai \\
			4 & {\color{gray}$\mid$} & "nul"                & {}   \\
		\hline
	\end{tabular}
	\quad
	\begin{tabular}{|*{2}{>{\centering\arraybackslash}m{4mm}}
					 *{2}{>{\centering\arraybackslash}m{9mm}}|}
		\hline
		\rowcolor{black!40}
			\verb_#_  & nb & signe & test \\			
		\hline
			  & -5                   & indéfini             & {}   \\
			1 & {\color{gray}$\mid$} & {\color{gray}$\mid$} & faux \\
			3 & {\color{gray}$\mid$} & {\color{gray}$\mid$} & faux \\
			6 & {\color{gray}$\mid$} & "négatif"            & {}   \\
		\hline
	\end{tabular}
	\end{center}
	

	\paragraph{Remarques.}
	\begin{itemize}
	\item
		Pour le dernier cas, on se contente
		d’un \pc{\K{sinon}} sans indiquer la condition~;
		ce serait inutile, elle serait toujours vraie.
	\item
		Le \pc{\K{si}} et le \pc{\K{si-sinon}} 
		peuvent être vus comme des cas particuliers du 
		\pc{\K{si-sinon-si}}.
	\item
		On pourrait écrire la même chose 
		avec des \pc{\K{si-sinon}} imbriqués
		mais le \pc{\K{si-sinon-si}} est plus lisible.
	\item Lorsqu’une condition est testée, on sait que toutes celles au-dessus se
		sont avérées fausses.  Cela permet parfois de simplifier la condition.

		\textbf{Exemple.}
		Supposons que le prix unitaire d’un produit (\pc{prixUnitaire})
		dépende de la quantité achetée (\pc{quantité}). 
		En dessous de 10 unités, on le paie 10\texteuro{} l’unité.
		De 10 à 99 unités, on le paie 8\texteuro{} l’unité.
		À partir de 100 unités, on paie 6\texteuro{} l’unité.

		\begin{java}
if (quantité < 10){
	prixUnitaire = 10;
} else if (quantité < 100) {
	// On sait que la quantité est plus grande
	// ou égale à 10. Inutile de tester. 
	prixUnitaire = 8;
} else {
	prixUnitaire = 6;
}
		\end{java}
		
	\end{itemize}


%==========================================
\section{Expression booléenne}
\index{expression booléenne}
\label{expression booléenne}
%==========================================

Nous avons dit que dans un test, l'\textit{expression} était une expression
booléenne et nous avons vu quelques opérateurs intervenant dans ces
expressions.  Revenons plus en détail sur ce concept. 

\marginicon{definition}
\textbf{Definition.} Une expression booléenne est une expression — c'est-à-dire
le résultat d'un calcul — dont la valeur est booléenne~: \pc{true} ou
\pc{false}. 

Une telle expression se compose grâce~: 

\begin{enumerate}
	\index{opérateurs relationnels}
	\item aux opérateurs relationnels (\textit{relational operator} ou 
		\textit{comparators}\index{comparators});
		
		Un opérateur relationnel est un opérateur dont la valeur
		est booléenne et les opérandes numériques. 

		\begin{grammaire}
			\grammarrule{RelationalOperator:}
			    \grammarrule{(one of)}
			    < > <= >=
		\end{grammaire}
		
	\index{opérateurs d'égalité}
	\item aux opérateurs d'égalité (\textit{equality operators});

		Un opérateur d'égalité est un opérateur dont la valeur
		est booléenne et les opérandes de même type (à conversion près). 

		\begin{grammaire}
			\grammarrule{EqualityOperator:}
			    \grammarrule{(one of)}
			    == != 
		\end{grammaire}

	\index{opérateurs logiques}
	\item au complément logique (\textit{logical complement operator}) et 
		aux opérateurs conditionnels 
		(\textit{conditionals operators}\index{conditionals operators});

		Le complément logique et les opérateurs conditionnels sont des
		opérateurs dont la valeur est booléenne et le ou les opérandes également
		booléens. 

		Le \pc{\&\&} est prioritaire sur le \pc{||}.

		\begin{grammaire}
			\grammarrule{LogicalComplementOperator:}
			    !
			\grammarrule{ConditionalOperator:}
			    \grammarrule{(one of)}
			     || &&
		\end{grammaire}
\end{enumerate}



%==========================================
\section{Le selon-que (\textit{switch})}
\index{selon-que}\index{switch}
%==========================================

	Cette nouvelle instruction permet d’écrire plus lisiblement \emph{certains}
	\pc{\K{si-sinon-si}}, plus précisément quand le choix d’une branche dépend
	de la valeur précise d’une variable (ou d’une expression).

	\begin{langagenaturel}
		selon que la variable vale~:\\
			\tab — une valeur~:\\
				\tab\tab exécuter une ou plusieurs actions\\
			\tab — une autre valeur~:\\
				\tab\tab exécuter d'autres actions\\
			\tab — encore une autre valeur~:\\
				\tab\tab exécuter d'autres actions\\
			\tab — et une dernière valeur~:\\
				\tab\tab exécuter d'autres actions\\
		fin selon que 
	\end{langagenaturel}

	Dans la grammaire  du langage Java, cette instruction, \textit{switch}, 
	s'écrit (grammaire simplifiée)~:

	\begin{grammaire}
		\grammarrule{SwitchStatement:}
		    switch ( \grammarrule{Expression} ) \grammarrule{SwitchBlock}

		\grammarrule{SwitchBlock:}
		    \grammarrule{SwitchLabels Statement}

		\grammarrule{SwitchLabel:}
		    case \grammarrule{ConstantExpression}:
		    default:
	\end{grammaire}
	
	\begin{itemize}
		\item l'expression ne peut pas être de n'importe quel type. À ce stade, 
			elle peut être, un entier ou une chaine et nous rencontrerons 
			d'autres types plus tard~;
		\item \textit{Statement} peut être une instuction ou plusieurs~;
		\item \textit{SwitchLabels} (avec un \textit{s}) ce sont plusieurs 
			«~\texttt{case}~»~;
		\item le \textit{switch} en java peut être vu comme un «~\textit{saut}
			au bon label~». Dès lors que l'instruction a trouvé le bon
			\textit{case}, l'éxécution des instructions continue. Il faut
			explicitement demander de sortir du \textit{switch} en utilisant
			l'instruction \textbf{\pc{break}}\index{break}~.
	\end{itemize}

	… mais voyons ça sur un exemple. 	

	\textbf{Exemple.}
	
	Imaginons qu’une variable (\pc{dayNumber}) contienne un numéro de jour de la
	semaine et qu’on veuille mettre dans une variable (\pc{dayName}) le nom du
	jour correspondant ("lundi" pour 1, "mardi" pour 2\dots)
	
	Une solution avec un \pc{\K{si-sinon-si}} est possible 
	mais le \pc{\K{selon-que}} (\textit{switch}) est plus lisible dans ce cas.

	Voyons comment écrire un algorithme sous différentes formes~:

	\begin{langagenaturel}
		switch ( dayNumber )\\
			\tab — 1~:\\
				\tab\tab dayName = "lundi"\\
			\tab — 2~:\\
				\tab\tab dayName = "mardi"\\
			\tab — 3~:\\
				\tab\tab dayName = "mercredi"\\
			\tab — 4~:\\
				\tab\tab dayName = "jeudi"\\
			\tab — 5~:\\
				\tab\tab dayName = "vendredi"\\
			\tab — 6~:\\
				\tab\tab dayName = "samedi"\\
			\tab — 7~:\\
				\tab\tab dayName = "dimanche"
	\end{langagenaturel}

	En langage Java~:

	\begin{java}
switch (dayNumber) {
	case 1: 
		dayName = "lundi"; 
		break;
	case 2: 
		dayName = "mardi"; 
		break;
	case 3: 
		dayName = "mercredi"; 
		break;
	case 4: 
		dayName = "jeudi"; 
		break;
	case 5: 
		dayName = "vendredi"; 
		break;
	case 6: 
		dayName = "samedi"; 
		break;
	case 7: 
		dayName = "dimanche";
}
	\end{java}
		
	\paragraph{Remarques.}
	\begin{itemize}
	\item
		Il peut y avoir plusieurs valeurs pour un cas donné.
	\item
		Il peut y avoir un cas par défaut, \pc{default}
		qui sera exécuté si la valeur n’est pas reprise par ailleurs.
	\end{itemize}
	
	
	
