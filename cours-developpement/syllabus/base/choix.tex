%===========================================
\chapter{Une question de choix}
%===========================================

	% Dans la section \ref{codestudio} (p. \pageref{codestudio}) nous proposions
	% une initiation ludique aux algorithmes.  Vous avez eu l’occasion d’y aborder
	% les alternatives.  Par exemple, vous avez indiqué au zombie quelque chose
	% comme~: \og{}S’il existe un chemin à gauche alors tourner à gauche\fg{}.
	
	Les \textbf{alternatives}\index{alternatives} permettent de n’exécuter des
	instructions que si une certaine \emph{condition} est vérifiée.  Avec le
	zombie, par exemple, vous testiez son environnement~; dans nos algorithmes
	et dans nos programmes, nous allons tester les données.
	
	Les algorithmes et les programmes vus jusqu’à présent ne proposent qu’un
	seul \og{}chemin\fg{}, une seule \og{}histoire\fg{}.  À chaque exécution de
	l’algorithme, les mêmes instructions s’exécutent dans le même ordres.  Les
	alternatives permettent de créer des histoires différentes, d’adapter les
	instructions aux valeurs concrètes des données.  

	\minitoc 

%===========================================
\section{Le si (\textit{if-then})}
\index{si}\index{if}
%===========================================
		
	Il existe des situations où des instructions ne doivent pas toujours être
	exécutées et un test va nous permettre de le savoir.
	
	\paragraph{Exemple.}

	Supposons que la variable \pc{nb} contienne un nombre positif ou négatif.
	Et supposons que l’on veuille le rendre positif. Il faudra tester son signe
	et, s’il est négatif, l'inverster.  Par contre, s’il est positif, il n’y
	a rien à faire.  Voici comment écrire, graphiquement et en pseudocode, un 
	algorithme~:
	
	\qquad
	\begin{minipage}{6cm}
		\begin{tikzpicture}[node distance = 2cm, auto]
			\node (start) [startstop] {Start};
			\node (dec1) [decision, below of=start, yshift=-1cm] {nb < 0};
			\draw [arrow] (start) -- (dec1);
			\node (proc1) [process, below of=dec1, yshift=-2cm] {nb = -nb};
			\node (stop) [startstop, below of=proc1] {End};
			\draw [arrow] (dec1) -- node[anchor=south, xshift=-5mm] {true} (proc1);
			\draw [arrow] (dec1) .. controls (4.5,-6) 
				.. node[anchor=east] {false} (stop);
			\draw [arrow] (proc1) -- (stop);
		\end{tikzpicture}	
	\end{minipage}
	\quad
	\begin{minipage}{6cm}
		\begin{pseudocode}[1]
			\If{nb<0}
				\Let nb \Gets -nb
			\EndIf
		\end{pseudocode}
	\end{minipage}

	Traçons l'algorithme dans deux cas différents pour bien illustrer 
	son déroulement.
	
	\begin{center}
	\begin{tabular}{|>{\centering\arraybackslash}m{1cm}
					|*{2}{>{\centering\arraybackslash}m{1cm}}|}
		\hline
		\rowcolor{black!40}
			\verb_#_  & nb & test \\			
		\hline
			  & -3                   & {}   \\
			1 & {\color{gray}$\mid$} & vrai \\
			2 & 3                    & {}   \\
		\hline
	\end{tabular}
	\qquad
	\begin{tabular}{|>{\centering\arraybackslash}m{1cm}
					|*{2}{>{\centering\arraybackslash}m{1cm}}|}
		\hline
		\rowcolor{black!40}
			\verb_#_  & nb & test \\			
		\hline
			  & 3                    & {}   \\
			1 & {\color{gray}$\mid$} & faux \\
		\hline
	\end{tabular}
	\end{center}
	
	La condition peut être n’importe quelle expression (calcul)
	dont le résultat est un booléen (vrai ou faux).

	\marginicon{attention}
	\textbf{Remarque}
	Attention, la confusion est fréquente.

	Un \og{}si\fg{} n’est pas une règle que l’ordinateur doit apprendre et
	exécuter à chaque fois que l’occasion se présente.  La condition n’est
	testée que lorsqu’on arrive à cet endroit de l’algorithme.
	
	En langage Java, le \emph{si} s'écrit comme suit~:

	\begin{grammaire}
		if ( \grammarrule{Expression} ) \grammarrule{Statement}
	\end{grammaire}

	\begin{enumerate}
		\item \grammarrule{Expression} représente une expression booléenne. 
			c'est-à-dire ayant comme valeur \pc{true} ou \pc{false}.
		\item \grammarrule{Statement} représente une instruction ou un 
			\textbf{bloc} d'instructions. Un bloc \index{bloc} d'instructions 
			est toujours délimités par une paire d'accolades. 
	\end{enumerate}

	\clearpage
	Le test précédent pourrait s'écrire comme suit~:

	\begin{java}
if (nb < 0)
	nb = -nb;
	\end{java}

	Pour s'éviter des erreurs, nous utiliserons toujours le bloc d'instructions
	et nous écrirons~:

	\begin{java}
if (nb < 0){
	nb = -nb;
}
	\end{java}

	\begin{Emphase}
		\paragraph{Exercice de compréhension}
		Tracez  l'algorithme ou le programme avec les valeurs fournies et donnez
		la valeur de retour.
		
		\begin{pseudocode}
		\Algo{exercice}{a, b~: entiers}{entier}
			\Decl{c}{entier}
			\Let c \Gets 2 * a
			\If{c > b}
				\Let c \Gets c - b
			\EndIf
			\Return c
		\EndAlgo
		\end{pseudocode}		
		
		\begin{java}
public static int exercice(int a, int b){
	int c;
	c = 2 * a;
	if (c > b){
		c = c-b;
	}
	return c;
}
		\end{java}
		
		\begin{multicols}{2}
		\begin{itemize}
		\item \pc{exercice(2, 5)} = \_\_\_
		\item \pc{exercice(4, 1)} = \_\_\_
		\end{itemize}
		\end{multicols}	
	\end{Emphase}


	\pagebreak[4]
%===============================
\section{Le si-sinon (\textit{if-then-else})}
\index{if-else}\index{si-sinon}
%===============================
	
	La construction \pc{si-sinon} permet d’exécuter certaines instructions ou
	d’autres en fonction d’un test.  Pour illustrer cette instruction, nous
	allons nous pencher sur un grand classique, la recherche de maximum.
	
	\paragraph{Exemple.}
	
	Pour déterminer le le maximum de deux nombres, c’est-à-dire la plus grande
	des deux valeurs, il y aura deux chemins possibles.  Le maximum devra
	prendre la valeur du premier nombre ou du second selon que le premier est
	plus grand que le second ou pas.
	
	\begin{center}
		\begin{tikzpicture}[node distance = 2cm, auto]
			\node (start) [startstop] {Start};
			\node (dec1) [decision, below of=start, yshift=-1cm] {nb1 > nb2};
			\draw [arrow] (start) -- (dec1);
			\node (proc1) [process, below of=dec1, yshift=-2cm, xshift=-2cm] 
				{max = nb1};
			\node (proc2) [process, below of=dec1, yshift=-2cm, xshift=2cm] 
				{max = nb2};
			\node (stop) [startstop, below of=dec1, yshift=-5cm] {End};
			\draw [arrow] (dec1) -- node[anchor=west, xshift=-10mm] {true} (proc1);
			\draw [arrow] (dec1) -- node[anchor=east, xshift=-1mm] {false} (proc2);
			\draw [arrow] (proc1) -- (stop);
			\draw [arrow] (proc2) -- (stop);
		\end{tikzpicture}	
	\end{center}
	

	\begin{pseudocode}[1]
			\If{nb1>nb2}
				\Let max \Gets nb1
			\Else
				\Let max \Gets nb2
			\EndIf
	\end{pseudocode}

	Traçons-le dans différentes situations.
	
	\begin{tabular}{|>{\centering\arraybackslash}m{6mm}
					|*{4}{>{\centering\arraybackslash}m{1cm}}|}
		\hline
		\rowcolor{gray!40}
			\verb_#_  & nb1 & nb2 & max & test \\			
		\hline
			  & 3 & 2 & indéfini & {} \\
			1 & {\color{gray}$\mid$} & {\color{gray}$\mid$} & indéfini & vrai \\
			2 & {\color{gray}$\mid$} & {\color{gray}$\mid$} & 3        & {} \\
		\hline
	\end{tabular}
	\quad
	\begin{tabular}{|>{\centering\arraybackslash}m{6mm}
					|*{4}{>{\centering\arraybackslash}m{1cm}}|}
		\hline
		\rowcolor{gray!40}
			\verb_#_  & nb1 & nb2 & max & test \\			
		\hline
			  & 4 & 42 & indéfini & {} \\
			1 & {\color{gray}$\mid$} & {\color{gray}$\mid$} & indéfini & faux \\
			4 & {\color{gray}$\mid$} & {\color{gray}$\mid$} & 42        & {} \\
		\hline
	\end{tabular}
	
	Le cas où les deux nombres sont égaux est également géré.
	
	\begin{tabular}{|>{\centering\arraybackslash}m{1cm}
					|*{4}{>{\centering\arraybackslash}m{1cm}}|}
		\hline
		\rowcolor{gray!40}
			\verb_#_  & nb1 & nb2 & max & test \\			
		\hline
			  & 4 & 4 & indéfini & {} \\
			1 & {\color{gray}$\mid$} & {\color{gray}$\mid$} & indéfini & faux \\
			4 & {\color{gray}$\mid$} & {\color{gray}$\mid$} & 4        & {} \\
		\hline
	\end{tabular}

	En langage Java, le \emph{si-sinon} s'écrit~:

	\begin{grammaire}
		\grammarrule{IfThenElseStatement:}
		    if ( \grammarrule{Expression} )
		        \grammarrule{Statement}
		    else
		        \grammarrule{Statement}
	\end{grammaire}

	Le test précédent peut donc s'écrire en supposant les variables déclarées~:

	\begin{java}
if (nb1 > nb2){		
	max = nb1;
} else {
	max = nb2;
}
	\end{java}
	
	\begin{Emphase}
		\paragraph{Exercice de compréhension}

		Tracez ces algorithmes ou programmes avec les valeurs fournies et donnez
		la valeur de retour.
	
		\begin{pseudocode}
		\Algo{exercice}{a, b : integer}{integer}
			\Decl{c}{integer}
			\If{a > b}
				\Let c \Gets a DIV b
			\Else
				\Let c \Gets b MOD a	
			\EndIf
			\Return c
		\EndAlgo
		\end{pseudocode}

		\begin{java}
public static int exercice(int a, int b){
	int c;
	if (a > b){
		c = a/b;
	} else {
		c = b%a;
	}
}
		\end{java}
	
		\begin{multicols}{2}
		\begin{itemize}
		\item \pc{exercice(2, 3)} = \_\_\_
		\item \pc{exercice(4, 1)} = \_\_\_
		\end{itemize}
		\end{multicols}
	
		\begin{pseudocode}
		\Algo{exercice}{x1, x2~: integer}{integer}
			\Decl{ok}{boolean}
			\Let ok \Gets x1 > x2
			\If{ok}
				\Let ok \Gets ok ET x1 == 4
			\Else
				\Let ok \Gets ok OU x2 == 3
			\EndIf
			\If{ok}
				\Let x1 \Gets x1 * 1000
			\EndIf
			\Return x1 + x2
		\EndAlgo
		\end{pseudocode}

		\begin{java}
public static int exercice(int x1, int x2){
	boolean ok;
	ok = x1 > x2;
	if (ok){
		ok = ok && x1 == 4;
	} else {
		ok = ok || x2 == 3;
	}
	return x1 + x2;
}
		\end{java}
		
		\medskip
		\begin{multicols}{2}
		\begin{itemize}
		\item \pc{exercice(2, 3)} = \_\_\_
		\item \pc{exercice(4, 1)} = \_\_\_	
		\end{itemize}
		\end{multicols}	
		
	\end{Emphase}	
	
	\clearpage

%============================
\section{Le si-sinon-si}
\index{si-sinon-si}
%============================

	Avec cette construction, il est possible d'indiquer à un endroit de
	l’algorithme plus de deux chemins possibles.  Partons à nouveau d’un exemple
	pour illustrer cette instruction.
	
	\paragraph{Exemple.} On voudrait mettre dans la chaine \pc{signe} la valeur
	\pc{"positif"}, \pc{"négatif"} ou \pc{"nul"} selon qu’un nombre donné est
	positif, négatif ou nul.
	
	\begin{center}
		\begin{tikzpicture}[node distance = 2cm, auto]
			\node (start) [startstop] {Start};
			\node (dec1) [decision, below of=start, yshift=-.5cm, 
				text width=.5cm] {};
			\draw [arrow] (start) -- (dec1);
			\node (proc1) [process, below of=dec1, yshift=-2cm, xshift=-4cm] 
				{s = "positif"};
			\node (proc2) [process, below of=dec1, yshift=-2cm] 
				{s = "nul"};
			\node (proc3) [process, below of=dec1, yshift=-2cm, xshift=4cm] 
				{s = "négatif"};
			\node (stop) [startstop, below of=dec1, yshift=-5cm] {End};
			\draw [arrow] (dec1) -- 
				node[anchor=west, xshift=-17mm] {nb>0} (proc1);
			\draw [arrow] (dec1) -- 
				node[anchor=east, xshift=-1mm] {nb == 0} (proc2);
			\draw [arrow] (dec1) -- 
				node[anchor=east, xshift=-1mm] {nb<0} (proc3);
			\draw [arrow] (proc1) -- (stop);
			\draw [arrow] (proc2) -- (stop);
			\draw [arrow] (proc3) -- (stop);
		\end{tikzpicture}	
	\end{center}


		\begin{pseudocode}[1]
			\If{nb>0}
				\Let signe \Gets "positif"
			\ElsIf{nb==0}
				\Let signe \Gets "nul"
			\Else
				\Let signe \Gets "négatif"
			\EndIf
		\end{pseudocode}
	
	Traçons-le
	
	\begin{tabular}{|*{2}{>{\centering\arraybackslash}m{4mm}}
					 *{2}{>{\centering\arraybackslash}m{9mm}}|}
		\hline
		\rowcolor{black!40}
			\verb_#_  & nb & signe & test \\			
		\hline
			  & 2                    & indéfini             & {}   \\
			1 & {\color{gray}$\mid$} & {\color{gray}$\mid$} & vrai \\
			2 & {\color{gray}$\mid$} & "positif"            & {}   \\
		\hline
	\end{tabular}
	
	\begin{tabular}{|*{2}{>{\centering\arraybackslash}m{4mm}}
					 *{2}{>{\centering\arraybackslash}m{9mm}}|}
		\hline
		\rowcolor{black!40}
			\verb_#_  & nb & signe & test \\			
		\hline
			  & 0                    & indéfini             & {}   \\
			1 & {\color{gray}$\mid$} & {\color{gray}$\mid$} & faux \\
			3 & {\color{gray}$\mid$} & {\color{gray}$\mid$} & vrai \\
			4 & {\color{gray}$\mid$} & "nul"                & {}   \\
		\hline
	\end{tabular}
	\quad
	\begin{tabular}{|*{2}{>{\centering\arraybackslash}m{4mm}}
					 *{2}{>{\centering\arraybackslash}m{9mm}}|}
		\hline
		\rowcolor{black!40}
			\verb_#_  & nb & signe & test \\			
		\hline
			  & -5                   & indéfini             & {}   \\
			1 & {\color{gray}$\mid$} & {\color{gray}$\mid$} & faux \\
			3 & {\color{gray}$\mid$} & {\color{gray}$\mid$} & faux \\
			6 & {\color{gray}$\mid$} & "négatif"            & {}   \\
		\hline
	\end{tabular}

	En langage Java, il n'y a pas de structure particulière pour ce test. Le
	\pc{if-then-else} fait bien l'affaire. Seule l'indentation change un peu 
	pour plus de lisibilité. Ce test peut donc s'écrire — en supposant toujours
	que les variables sont déclarées — comme ceci~:

	\begin{java}
if (nb > 0){
	s = "positif";
} else if (nb == 0) {
	s = "nul";
} else {
	s = "négatif";
}
	\end{java}

	\paragraph{Remarques.}
	\begin{itemize}
	\item
		Pour le dernier cas, on se contente
		d’un \pc{\K{sinon}} sans indiquer la condition~;
		ce serait inutile, elle serait toujours vraie.
	\item
		Le \pc{\K{si}} et le \pc{\K{si-sinon}} 
		peuvent être vus comme des cas particuliers du 
		\pc{\K{si-sinon-si}}.
	\item
		On pourrait écrire la même chose 
		avec des \pc{\K{si-sinon}} imbriqués
		mais le \pc{\K{si-sinon-si}} est plus lisible.
		
		\begin{pseudocode}
			\If{nb>0}
				\Let signe \Gets "positif"
			\Else
				\If{nb=0}
					\Let signe \Gets "nul"
				\Else
					\Let signe \Gets "négatif"
				\EndIf
			\EndIf
		\end{pseudocode}
	\item Lorsqu’une condition est testée, on sait que toutes celles au-dessus se
		sont avérées fausses.  Cela permet parfois de simplifier la condition.

		\textbf{Exemple.}
		Supposons que le prix unitaire d’un produit (\pc{prixUnitaire})
		dépende de la quantité achetée (\pc{quantité}). 
		En dessous de 10 unités, on le paie 10\texteuro{} l’unité.
		De 10 à 99 unités, on le paie 8\texteuro{} l’unité.
		À partir de 100 unités, on paie 6\texteuro{} l’unité.
		
		\begin{pseudocode}
			\If{quantité<10}
				\Let prixUnitaire \Gets 10
				\ElsIf{quantité<100} \RComment{On sait que ce n’est pas <10}
								\Let\RComment {inutile de le tester}
				\Let prixUnitaire \Gets 8
			\Else
				\Let prixUnitaire \Gets 6
			\EndIf
		\end{pseudocode}
	\end{itemize}


%==========================================
\section{Expression booléenne}
\index{expression booléenne}
\label{expression booléenne}
%==========================================

Nous avons dit que dans un test, l'\textit{expression} était une expression
booléenne et nous avons vu quelques opérateurs intervenants dans ces
expressions.  Revenons plus en détail sur ce concept. 

\marginicon{definition}
\textbf{Denifition.} Une expression booléenne est une expression — c'est-à-dire
le résultat d'un calcul — dont la valeur est booléenne~: \pc{true} ou
\pc{false}. 

Une telle expression se compose grâce~: 

\begin{enumerate}
	\item aux opérateurs relationnels (\textit{relational operator} ou 
		\textit{comparators}\index{comparators});
		
		Un opérateur relationnel est un opérateur dont la valeur
		est booléenne et les opérandes numériques. 

		\begin{grammaire}
			\grammarrule{RelationalOperator:}
			    \grammarrule{(one of)}
			    < > <= >=
		\end{grammaire}
		
	\item aux opérateurs d'égalité (\textit{equality operators});

		Un d'égalité est un opérateur dont la valeur
		est booléenne et les opérandes de même type (à conversion près). 

		\begin{grammaire}
			\grammarrule{EqualityOperator:}
			    \grammarrule{(one of)}
			    == != 
		\end{grammaire}

	\item au complément logique (\textit{logical complement operator}) et 
		aux opérateurs conditionnels 
		(\textit{conditionals operators}\index{conditionals operators});

		Le complément logique et les opérateurs conditionnels sont des
		opérateurs dont la valeur est booléenne et le ou les opérandes également
		booléens. 

		Le \pc{\&\&} est prioritaire sur le \pc{||}.

		\begin{grammaire}
			\grammarrule{LogicalComplementOperator:}
			    !
			\grammarrule{ConditionalOperator:}
			    \grammarrule{(one of)}
			     || &&
		\end{grammaire}
\end{enumerate}



%==========================================
\section{Le selon-que (\textit{switch})}
\index{selon-que}\index{switch}
%==========================================

	Cette nouvelle instruction permet d’écrire plus lisiblement \emph{certains}
	\pc{\K{si-sinon-si}}, plus précisément quand le choix d’une branche dépend
	de la valeur précise d’une variable (ou d’une expression).

	\paragraph{Exemple.}
	
	Imaginons qu’une variable (\pc{dayNumber}) contienne un numéro de jour de la
	semaine et qu’on veuille mettre dans une variable (\pc{dayName}) le nom du
	jour correspondant ("lundi" pour 1, "mardi" pour 2\dots)
	
	Une solution avec un \pc{\K{si-sinon-si}} est possible 
	mais le \pc{\K{selon-que}} (\textit{switch}) est plus lisible.

		\begin{pseudocode}
			\Switch{dayNumber}
				\Case{1} dayName \Gets "lundi"
				\Case{2} dayName \Gets "mardi"
				\Case{3} dayName \Gets "mercredi"
				\Case{4} dayName \Gets "jeudi"
				\Case{5} dayName \Gets "vendredi"
				\Case{6} dayName \Gets "samedi"
				\Case{7} dayName \Gets "dimanche"
			\EndSwitch
		\end{pseudocode}

	remplace avantageusement

	\begin{wrong}
		\begin{pseudocode}
			\If{dayNumber=1}
				\Let dayName \Gets "lundi"
			\ElsIf{dayNumber=2}
				\Let dayName \Gets "mardi"
			\ElsIf{dayNumber=3}
				\Let dayName \Gets "mercredi"
			\ElsIf{dayNumber=4}
				\Let dayName \Gets "jeudi"
			\ElsIf{dayNumber=5}
				\Let dayName \Gets "vendredi"
			\ElsIf{dayNumber=6}
				\Let dayName \Gets "samedi"
			\Else
				\Let dayName \Gets "dimanche"
			\EndIf
		\end{pseudocode}
	\end{wrong}

	En langage Java, le \textit{switch} s'écrit (grammaire simplifiée)~:

	\begin{grammaire}
		\grammarrule{SwitchStatement:}
		    switch ( \grammarrule{Expression} ) \grammarrule{SwitchBlock}

		\grammarrule{SwitchBlock:}
		    \grammarrule{SwitchLabels Statement}

		\grammarrule{SwitchLabel:}
		    case \grammarrule{ConstantExpression}:
		    default:
	\end{grammaire}

	\begin{itemize}
		\item l'expression ne peut pas être de n'importe quel type. À ce stade, 
			elle peut être, un entier ou une chaine;
		\item \textit{Statement} peut être une instuction ou plusieurs;
		\item \textit{SwitchLabels} (avec un \textit{s}) se sont plusieurs 
			«~\texttt{case}~»
		
		\item le \textit{switch} en java peut être vu comme un «~\textit{saut}
			au bon label~». Dès lors que l'instruction a trouvé le bon
			\textit{case}, l'éxécution des instructions continue. Il faut
			explicitement demander de sortir du \textit{switch} en utilisant
			l'instruction \textbf{\pc{break}}

	\end{itemize}

	Le \textit{switch} précédent s'écrit~:

	\begin{java}
switch (dayNumber) {
	case 1: 
		dayName = "lundi"; 
		break;
	case 2: 
		dayName = "mardi"; 
		break;
	case 3: 
		dayName = "mercredi"; 
		break;
	case 4: 
		dayName = "jeudi"; 
		break;
	case 5: 
		dayName = "vendredi"; 
		break;
	case 6: 
		dayName = "samedi"; 
		break;
	case 7: 
		dayName = "dimanche";
}
	\end{java}
		
	\paragraph{Remarques.}
	\begin{itemize}
	\item
		Il peut y avoir plusieurs valeurs pour un cas donné.
	\item
		Il peut y avoir un cas par défaut, \pc{default}
		qui sera exécuté si la valeur n’est pas reprise par ailleurs.
	\end{itemize}
	
	La syntaxe générale est~:
	
	\begin{pseudocode}
		\Switch{expression}
			\Case{liste${}_1$ de valeurs séparées par des virgules }
				\Stmt Instructions
			\Case{liste${}_2$ de valeurs séparées par des virgules }
				\Stmt Instructions
			\Empty \dots
			\Case{liste${}_k$ de valeurs séparées par des virgules }
				\Stmt Instructions
			\Case{\K{default }}
				\Stmt Instructions
		\EndSwitch
	\end{pseudocode}
	
	
