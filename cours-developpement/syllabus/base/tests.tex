% ============================
%\renewcommand{\chaptername}{Errata}
\chapter{Les tests}
\label{chap:tests}
% ============================

\begin{Exergue}
	\og{}J'ai fait tourner le programme, il fonctionne bien.\fg{}
\end{Exergue}

Tous les programmes contiennent des erreurs, des \textit{bugs}, des
défauts\dots{} Ces «~\textit{bugs}~» entrainent, au mieux, un inconfort pour
l'utilisateur ou l'utilisatrice. Ils peuvent occasionner une perte de temps,
d'argent, de données, de matériel… ou, pire, un danger pour la vie humaine dans
un environnement industriel par exemple. 

Le travail de développeur ou de la développeuse est de réduire le nombre
d'erreurs dans son programme. Pour ce faire, il ou elle le \textbf{testera}
abondamment.  Ce chapitre s'intéresse aux tests. 

\minitoc

\section{Les types d'erreurs et de tests}

Les défauts d'un programme peuvent être de différentes sortes:

\begin{enumerate}
	
	\item les erreurs de compilation~;

		Ces erreurs sont détectées par le compilateur et rapidement corrigées.

	\item les erreurs d'exécution
		
		\dots{} et le programme s'arrête~;


		\dots{} et le programme ne fournit pas les bonnes valeurs~;

		Ces erreurs sont difficiles à corriger. Elles nécessitent
		que le programme soit \textbf{testé} avec un nombre suffisant de
		situations différentes. Elles peuvent être dues à des erreurs de
		programmation mais aussi à des erreurs dans nos algorithmes. 

	\item les erreurs de programmation qui entrainent une lenteur du programme 
		ou une consommation excessive de mémoire. 

		Pour éviter ces erreurs, respecter les bonnes pratiques de développement
		est une première étape qu'il ne faut pas négliger. Ensuite, il faudra
		détecter les endroits dans le code où les ralentissements ont lieu ou la
		consommation de mémoire augmente. Il existe des logiciels dédiés à ce
		genre de cas. 

\end{enumerate}

Pour montrer --- garantir, prouver --- qu'un programme fonctionne bien, il ne
suffit pas de montrer qu'il fournit les bons résultats dans certains cas, il
faut convaincre, qu'il fournira les bons résultats dans \textbf{tous} les cas :

\begin{itemize}
	\item les cas généraux;
	\item les cas particuliers;
	\item lors d'une défaillance de l'environnement;
		
		Par exemple, si un capteur de distance est défectueux, il faut que la 
		chaine de production s'arrête. 
	
	\item lors d'une utilisation non conforme du programme.
\end{itemize}

Pour répondre à ces attentes, le développeur ou la développeuse sera attentive
\footnote{Je m'autorise un accord de proximité.}
à utiliser une méthodologie éprouvée dans toutes les étapes de développement de
son projet; de la phase d'analyse au déploiement.  Il mettra en place différents
types de tests
\footnote{Voir \url{https://fr.wikipedia.org/wiki/Test_(informatique)}}
; des tests unitaires, d'intégration, fonctionnels, de non-régression… Dans ce
premier cours de développement, nous nous intéressons aux \textbf{tests
unitaires}.  Pour ces tests unitaires, le développeur ou la développeuse
fournira un ensemble de valeurs à tester représentatives et convaincantes. 

\index{test unitaire}
\marginicon{definition}
\textbf{Définition}

\textbf{Test unitaire} Procédure permettant de tester le bon fonctionnement
d'une unité de code, une méthode \bsc{Java} dans notre cas. En fonction des
paramètres qu'elle reçoit en entrée, la méthode fournit-elle les bons résultats~? 

L'idée sous-jacente des tests unitaires est que si chaque partie est correcte,
l'ensemble sera correct. 

Pour tester régulièrement --- très régulièrement --- notre programme nous avons
besoin d'outils qui vont automatiser ces tests mais commençons par les
planifier. 


\section{Planifier les tests}

Dans la méthode de résolution de problèmes que nous avons présenté dans le
chapitre~\ref{refpblm} p.\,\pageref{refpblm} et dans l'exercice résolu de la
section~\ref{exerciceresolu} p.\,\pageref{exerciceresolu} nous précisions que
l'écriture d'une solution complète dans ce premier cours de développement
consistait en: 

\begin{enumerate}
	\item spécifier le problème;
	\item fournir des exemples;
	\item écrire un algorithme;
	\item vérifier les exemples (en traçant l'algorithme);
	\item écrire un programme exécutable correspondant à l'algorithme;
	\item tester le programme et constater qu'il fournit bien les résultats attendus.
\end{enumerate}

Les points 2 et 4 sont les prémisses à l'écriture des tests puisque ces exemples
vont nous convaincre que l'algorithme --- et le programme --- fournit bien les
\og{}bons\fg{} résultats. Ces exemples constituent le \textbf{plan de tests} que nous allons utiliser. Dans ces différents cas, outre les cas généraux, les cas particuliers, les valeurs limites, apparaitront aussi des cas montrant les erreurs de programmation fréquentes. Par exemple~:

\begin{itemize}
	\item commencer ou arrêter trop tôt ou trop tard une boucle;
	\item ne pas initialiser ou mal initialiser une variable;
	\item confondre < et $\le$ ou > et $\ge$… voire < et >;
	\item confondre AND et OR;
	\item mal écrire la négation d'une proposition;
	\item \dots
\end{itemize}

C'est l'expérience qui permettra d'écrire un plan de tests efficace et
convaincant.

\index{plan de test}
\marginicon{definition}
Un \textbf{plan de test}\index{plan de test} est donc une liste ou un tableau
reprenant un certain nombre d'exemples qui mettent en œuvre la méthode testée
sur des cas tels que ceux décrits ci-dessus et permettent de faire apparaitre
des problèmes lors de l'exécution automatique. Chacun des tests sera défini par
une méthode qui porte un nom qui explique l'objectif du test sans la nécessité
de commentaires additionnels. Autant que possible, ce nom décrit aussi la
portée ou le contexte de ce qui est vérifié.

\paragraph{Remarque} Dans certaines méthodologies, la mise en exergue de l'importance des tests se fait en écrivant d'abord les tests avant le code.
Voir par exemple \href{https://fr.wikipedia.org/wiki/Test_driven_development}{Wikipedia} ou \href{http://www.extremeprogramming.org/rules/testfirst.html}{cette article d'\textit{extreme programming}}\cite{testfirst}. 

Dès lors que le plan de tests est écrit, nous pouvons nous intéresser aux outils
qui vont permettre d'automatiser l'exécution des tests et faciliter leur
écriture.  En effet pour que nos tests soient utiles il faudra tester souvent
— idéalement dès qu'une méthode est écrite — et «~tout~». Ce sont toutes les
méthodes qui sont testées parce que l'écriture d'une méthode pourrait amener une
erreur dans une autre méthode. Dans ces conditions, pour que le développeur ou
la développeuse teste son programme, ce doit être automatique.  

Cette automatisation est fournie par \text{JUnit}.



\section{JUnit}

\paragraph{Remarque préalable} JUnit n'est pas fourni avec le JDK (\textit{java
development kit}), c'est un \textit{framework} de tests indépendant que l'on
trouve sur \url{junit.org}. Il est donc nécessaire de l'installer. 

Par contre, il est fourni avec NetBeans. 

\index{junit}
\marginicon{definition}
\textbf{JUnit} est un \textit{framework} simple pour l'écriture de tests
répétables. 

Pour tester la méthode — \texttt{foo} — dans la classe \texttt{MyClass}, il faut
écrire une classe \texttt{MyClassTest} qui contiendra les méthodes de tests.  Le
nom de la classe de test porte le nom de la classe qu'elle teste, suffixé par
\texttt{Test}. Ensuite, il faut demander à JUnit — dès lors qu'il est installé
— d'exécuter cette classe. Un clic dans un IDE ou une commande du style fait
l'affaire~:

\begin{term}
	java org.junit.runner.JUnitCore my.package.MyClassTest
\end{term}

La classe \texttt{MyClassTest} contient plusieurs méthodes. Une méthode de test
par cas. Une méthode de test~: 

\begin{itemize}
	\item est autonome. Elle ne reçoit pas de paramètre et ne retourne rien;
	
	\item contient des affirmations qui seront évaluées;

		\textit{La méthode doit me retourner «~vrai~».}~: \pc{assertTrue()}\\
		\textit{La méthode doit me retourner «~faux~».}~: \pc{assertFalse()}\\
		\textit{La méthode doit me donner cette valeur si ses paramètres sont
		ceux-ci.}~:\\\pc{assertEquals(<valeur attendue>, <valeur>)}

              Ces trois méthodes \pc{assertTrue}, \pc{assertFalse}, \pc{assertEquals} sont définies dans la classe \pc{org.junit.Assert}.
              On peut les utiliser directement après les avoir préalablement importées :
\begin{java}
import static org.junit.Assert.*;
\end{java}
              

	\item est précédée de l'annotation \pc{@Test} permettant de la faire
		reconnaitre comme un test unitaire;

	\item n'a pas de mot clé \pc{static}.

\end{itemize}

Un test aura donc l'allure suivante~:

\begin{java}
@Test
public void max2_cas1(){
	assertEquals(2, max2(-1, 2));
}
\end{java}

Après lancement des tests, JUnit fournira un rapport l'allure suivante~:

\begin{term}
JUnit version 4.12
.

Time: 0,012

OK (1 test)
\end{term}

\section{Exemple}

Reprenons l'exemple du calcul du maximum de 2 nombres décrit dans la fiche
\ref{fiche:max2nb} p\,\pageref{fiche:max2nb}. Voici un plan de tests~:

\newcolumntype{C}[1]{>{\centering\arraybackslash}p{#1}}
\bigskip 
\begin{center}
	\begin{tabular}{|c|p{4cm}|c|c|C{1.5cm}|}
	\hline
	\rowcolor{black!40}
	test \no & description & nb1 & nb2 & réponse attendue  \\
	\hline 
	1 & maximum en position 2 et nombre négatif & -3    & 4   & 4	\\\hline
	2 & maximum en position 1 & 7     & 4   & 7 \\\hline
	3 & tous égaux & 4     & 4   & 4 \\\hline
	4 & avec un nul & 0     & -4  & 0 \\\hline
\end{tabular}
\end{center}		
\bigskip 

Et la classe de tests \texttt{MaximumTest} sans les commentaires.

\bigskip 
\begin{java}
package be.he2b.esi.cours.dev1;

import org.junit.Test;
import static org.junit.Assert.*;

public class MaximumTest {
    
    
	@Test
	public void max2_maximumEnPosition2EtUnNégatif(){
		assertEquals(4, max2(-3, 4));
	}

	@Test
	public void max2_maximumEnPosition1(){
		assertEquals(7, max2(7, 4));
	}

	@Test
	public void max2_tousÉgaux(){
		assertEquals(4, max2(4, 4));
	}

	@Test
	public void max2_avecUnNul(){
		assertEquals(0, max2(0, -4));
	}
}
\end{java}


\begin{term}
JUnit version 4.12
....

Time: 0,012

OK (4 tests)
\end{term}


