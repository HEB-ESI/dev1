\clearpage 
\section{Exercices~: structures itératives}

\subsection{while}
\bigskip 

	\begin{Exercice}{Compréhension d’algorithmes}
		Quels sont les affichages réalisés lors de l’exécution
		des algorithmes suivants~?

		\begin{java}
public static void while1(){
	int x;
	x = 0;
	while (x < 12){
		x = x + 2;
	}
	System.out.println(x);
}
		\end{java}

		\begin{java}
public static void while2(){
	boolean shouldContinue = true;
	int x = 5;
	while (shouldContinue) {
		x = x + 7;
		shouldContinue = x % 11 != 0
	}
	System.out.println(x);
}
		\end{java}

		\begin{java}
public static void while3(){
	boolean isDone = false;
	int count = 0, 
		x = 10;
	while (!isDone && count < 3){
		if (x % 2 == 0){
			x = x + 1;
			isDone = x < 20;
		} else {
			x = x + 3
			count++
		}
	}
	System.out.println(x);
}	
		\end{java}

	\end{Exercice}

		\begin{Exercice}{Afficher des nombres}

			En utilisant un \pc{\algorithmicwhile},
			écrire un algorithme qui reçoit un entier $n$ positif et affiche
			\begin{enumerate}[label=\alph*)]
				\item les nombres de 1 à $n$~;
				\item les nombres de $n$ à 1 en ordre décroissant~;
				\item les nombres impairs de 1 à $n$~;
				\item les nombres de -$n$ à $n$~;
				\item les multiples de 5 de 1 à $n$~;
				\item les multiples de $n$ de 1 à 100.
			\end{enumerate}
		\end{Exercice}




\subsection{for}
\bigskip 

	\begin{Exercice}{Compréhension d’algorithmes}
	
		Quels sont les affichages réalisés lors de l’exécution des algorithmes
		suivants~?

		\begin{java}
public static void for1(){
	int x;
	boolean b;
	x = 3;
	b = true;
	for (int i = 0; i < 5; i++){
		x = x + i + 1;
		b = b && x % 2 == 0;
	}
	if (b){
		System.out.println(x);
	} else {
		System.out.println(2 * x);
	}
}
		\end{java}

		\begin{java}
public static void for2(){
	int end;
	for (int i = 0; i < 3; i++){
		end = 6 * (i + 1) - 11;
		for (int j = 0; j < end; j = j + 3){
			System.out.println(10 * i + j);
		}
	}
}
		\end{java}
	

	\end{Exercice}

	\begin{Exercice}{Afficher des nombres}

		Reprenons un exercice déjà donné avec le \pc{\algorithmicwhile}.
		En utilisant un \pc{\algorithmicfor},
		écrire un algorithme qui reçoit un entier $n$ positif et affiche
		\begin{enumerate}[label=\alph*)]
			\item les nombres de 1 à $n$~;
			\item les nombres de 1 à $n$ en ordre décroissant~;
			\item les nombres de -n à n~;
			\item les multiples de 5 de 1 à n~;
			\item les multiples de n de 1 à 100.
		\end{enumerate}
	\end{Exercice}



\subsection{do while}
\bigskip

	\begin{Exercice}{Compréhension d’algorithmes}
	
		Quels sont les affichages réalisés lors de l’exécution
		de l'algorithme suivant~?

		\begin{java}
public static void dowhile1(){
	boolean isOdd, isBig;
	int p, x;
	x = 1;
	p = 1;
	do {
		p = 2 * p;
		x = x + p;
		isOdd = x % 2 == 0;
		isBig = x > 15;
	} while (!isOdd && !isBig);
		\end{java}

		\end{Exercice}

		\begin{Exercice}{Afficher des nombres}

			Reprenons un exercice déjà fait avec le \pc{\algorithmicwhile}
			et le \pc{\algorithmicfor}
			en utilisant cette fois un \pc{\algorithmicrepeat} \pc{\algorithmicwhile}.
			Écrire un algorithme qui reçoit un entier $n$ positif et affiche
			\begin{enumerate}[label=\alph*)]
				\item les nombres de 1 à $n$~;
				\item les nombres de 1 à $n$ en ordre décroissant~;
				\item les nombres de -n à n~;
				\item les multiples de 5 de 1 à n~;
				\item les multiples de n de 1 à 100.
			\end{enumerate}
		\end{Exercice}


		
\subsection{Exercices récapitulatifs sur les structures répétitives}
\bigskip

		\begin{Exercice}{Afficher les nombres impairs}
			Écrire un algorithme qui demande une série
			de valeurs entières à l’utilisateur
			et qui affiche celles qui sont impaires.
			L’algorithme commence par demander à l’utilisateur
			le nombre de valeurs qu’il désire donner.
		\end{Exercice}

		\begin{Exercice}{Compter les nombres impairs}
			Écrire un algorithme qui demande une série
			de valeurs entières à l’utilisateur
			et qui lui affiche le nombre de valeurs impaires
			qu’il a donné.
			Après chaque valeur entrée,
			l’algorithme demande à l’utilisateur s’il y en a encore d’autres.
		\end{Exercice}

		\begin{Exercice}{Choix de la valeur sentinelle}
			Quelle valeur sentinelle prendrait-on 
			pour additionner une série de cotes d’interrogations~? 
			Une série de températures~?
		\end{Exercice}

		\begin{Exercice}{Afficher les nombres impairs}

			Écrire un algorithme qui demande une série
			de valeurs entières non nulles à l’utilisateur
			et qui affiche celles qui sont impaires.
			La fin des données sera signalée 
			par la valeur sentinelle 0.
		\end{Exercice}

		\begin{Exercice}{Compter le nombre de réussites}

			Écrire un algorithme qui demande une série
			de cotes (entières, sur 20) à l’utilisateur
			et qui affiche le pourcentage de réussites.
			La fin des données sera signalée 
			par une valeur sentinelle que vous pouvez choisir.
		\end{Exercice}

		\begin{Exercice}{Suites}

			Écrire les algorithmes qui affichent les $n$ premiers termes des
			suites suivantes.  À vous de voir quel est, parmi les 2 modèles
			décrits dans les exemples précédents, le modèle de solution le plus
			adpaté.

			\begin{enumerate}[label=\alph*)]
				\item -1, -2, -3, -4, -5, -6\dots
				\item 1, 3, 6, 10, 15, 21, 28\dots
				\item 1, 0, 1, 0, 1, 0, 1, 0\dots
				\item 1, 2, 0, 1, 2, 0, 1, 2\dots
				\item 1, 2, 3, 1, 2, 3, 1, 2\dots
				\item 1, 2, 3, 2, 1, 2, 3, 2\dots
			\end{enumerate}			
		\end{Exercice}

\bigskip
\begin{Emphase}

	\paragraph{Remarque}

	Pour les exercices qui suivent, nous vous donnons peu d’indications sur la
	solution à mettre en \oe{}uvre.  À vous de jouer\dots{}

\end{Emphase}
\bigskip

		\begin{Exercice}{Lire un nombre}

			Écrire un algorithme qui demande à l’utilisateur un nombre entre
			1 et $n$ et le retourne.  Si la valeur donnée n’est pas dans
			l’intervalle souhaité, l’utilisateur est invité à rentrer une
			nouvelle valeur jusqu’à ce qu’elle soit correcte.
			
	\end{Exercice}

	\begin{Exercice}{Lancé de dés}

		Écrire un algorithme qui simule un lancé de dé. Il lance $n$ fois un dé
		et compte le nombre de fois qu’une certaine valeur est obtenue.

		\begin{java}
/**
 * Simule le lancé d'un dé et compte
 * le nombre de fois ou value apparait.
 * 
 * @param n le nombre de lancés
 * @param value la valeur comptée
 */
public static int rollDice(int n, int value){
	// ...
}
		\end{java}
	
	\end{Exercice}

	\begin{Exercice}{Factorielle}

		Écrire un algorithme qui retourne la factorielle de $n$ (entier positif ou
		nul). 
		
		Rappel~: la factorielle de $n$, notée $n$!, est le produit des $n$
		premiers entiers strictement positifs. 

		Par convention, 0! = 1.
	\end{Exercice}

	\begin{Exercice}{Produit de 2 nombres}
		Écrire un algorithme qui retourne le produit de deux entiers quelconques
		sans utiliser l’opérateur de multiplication, mais en minimisant le
		nombre d’opérations.
	\end{Exercice}

	\begin{Exercice}{Table de multiplication}

		\begin{minipage}[t]{10cm}
			Écrire un algorithme qui affiche la table de multiplication
			des nombres de 1 à 10
			(cf. l’exemple ci-contre).

		\end{minipage}
		\qquad
		\begin{minipage}[t]{4cm}
			\begin{verbatim}
			1 x 1 = 1
			1 x 2 = 2
			...
			1 x 10 = 10
			2 x 1 = 2
			...
			10 x 9 = 90
			10 x 10 = 100
			\end{verbatim}
		\end{minipage}		
	\end{Exercice}

	\begin{Exercice}{Double 6}

		Écrire un algorithme qui lance de façon répétée deux dés.
		Il s’arrête lorsqu’il obtient un double 6
		et retourne le nombre de lancés effectués.
	\end{Exercice}

	\begin{Exercice}{Nombre premier}

		Écrire un algorithme qui vérifie si un entier positif est un
		\textbf{nombre premier}. 

		Rappel~:~un nombre est premier s’il n’est divisible que par 1 et par
		lui-même. Le premier nombre premier est 2.
	\end{Exercice}

	\begin{Exercice}{Nombres premiers}

		Écrire un algorithme qui affiche les nombres premiers inférieurs ou
		égaux à un entier positif donné. Le module de cet algorithme fera appel
		de manière répétée mais économique à celui de l’exercice précédent.
	\end{Exercice}

	\begin{Exercice}{Somme de chiffres}

		Écrire un algorithme qui calcule la somme des chiffres qui forment un
		nombre naturel $n$. Attention, on donne au départ \textbf{le} nombre et
		pas ses chiffres. Exemple~: 133045 doit donner comme résultat 16,
		car 1 + 3 + 3 + 0 + 4 + 5 = 16.
	\end{Exercice}

	\begin{Exercice}{Nombre parfait}

		Écrire un algorithme qui vérifie si un entier positif est un
		\textbf{nombre parfait}, c’est-à-dire un nombre égal à la somme de ses
		diviseurs (sauf lui-même). 

		Par exemple, 6 est parfait car 6 = 1 + 2 + 3. 
		De même, 28 est parfait car 28 = 1 + 2 + 4 + 7 + 14.
	\end{Exercice}

	\begin{Exercice}{Décomposition en facteurs premiers}
		Écrire un algorithme qui affiche la décomposition 
		d’un entier en facteurs premiers. 
		Par exemple, $1001880$ donnerait comme décomposition
		$2^3 * 3^2 * 5 * 11^2 * 23$.
	\end{Exercice}

	\begin{Exercice}{Nombre miroir}

		Le miroir d’un nombre est le nombre obtenu
		en lisant les chiffres de droite à gauche.
		Ainsi le nombre miroir de $4209$ est $9024$.
		Écrire un algorithme qui calcule le miroir
		d’un nombre entier positif donné.
	\end{Exercice}

	\begin{Exercice}{Palindrome}

		Écrire un algorithme qui vérifie si un entier donné 
		forme un palindrome ou non. 
		Un nombre palindrome est un nombre qui lu dans un sens 
		(de gauche à droite) est identique au nombre lu dans l’autre sens 
		(de droite à gauche). 
		
		Par exemple, $1047401$ est un nombre palindrome.
	\end{Exercice}

	\begin{Exercice}{Jeu de la fourchette}

		Écrire un algorithme qui simule le jeu de la
		fourchette. Ce jeu consiste à essayer de découvrir un nombre quelconque
		compris entre 1 et 100 inclus, tiré au sort par l’ordinateur. 
		L’utilisateur a droit à huit essais
		maximum. À chaque essai, l’algorithme devra afficher un message
		indicatif «~nombre donné trop petit~» ou «~nombre donné trop grand~».
		En conclusion, soit «~bravo, vous avez trouvé en [nombre] essai(s)~» soit
		«~désolé, le nombre était [valeur]~».
	\end{Exercice}






