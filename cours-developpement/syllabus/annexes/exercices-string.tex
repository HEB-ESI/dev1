\clearpage
\section{Exercices~: chaine et String}
\bigskip

	\begin{Exercice}{Calcul de fraction}
		
		Écrire un algorithme qui reçoit une fraction sous forme de chaine, et
		retourne la valeur numérique de celle-ci.  Par exemple, si la fraction
		donnée est "5/8", l’algorithme renverra 0,625.  On peut considérer que
		la fraction donnée est correcte, elle est composée de 2 entiers séparés
		par le caractère de division '/'.

\end{Exercice}

\begin{Exercice}{Conversion de nom}
	Écrire un algorithme 
	qui reçoit le nom complet d’une personne 
	dans une chaine sous la forme "nom, prénom" 
	et la renvoie au format "prénom nom" 
	(sans virgule séparatrice). 
	Exemple~: "De Groote, Jan" deviendra "Jan De Groote".	
\end{Exercice}

\begin{Exercice}{Gauche et droite}

	Écrire un algorithme \pc{gauche} et un \pc{droite} recevant un paramètre
	entier $n$ qui retourne la chaine formée respectivement des $n$ premiers et
	des $n$ derniers caractères d’une chaine donnée.	

\end{Exercice}

\begin{Exercice}{Grammaire}
	Écrire un algorithme 
	qui met un mot en « ou » au pluriel. 
	Pour rappel, 
	un mot en « ou » prend un « s » à l’exception des 7 mots 
	bijou, caillou, chou, genou, hibou, joujou et pou qui prennent 
	un « x » au pluriel. 
	Exemple~: un clou, des clous, un hibou, des hiboux. 
	Si le mot soumis à l’algorithme n’est pas un mot en « ou », 
	un message adéquat sera affiché.
\end{Exercice}

\begin{Exercice}{Normaliser une chaine}
	Écrire un module qui reçoit une chaine et retourne une autre chaine,
	version normalisée de la première.
	Par normalisée, on entend~:~enlever tout ce qui n’est pas une lettre 
	et tout mettre en majuscule.
	\\Exemple~:~"Le <COBOL>, c’est la santé~!" devient "LECOBOLCESTLASANTE".
\end{Exercice}

\begin{Exercice}{Les palindromes}

	Cet exercice a déjà été réalisé pour des entiers, 
	en voici 2 versions « chaine »~! 
	
	\begin{enumerate}[label=\alph*)]
	
		\item Écrire un algorithme qui vérifie si un mot donné sous forme de
			chaine constitue un palindrome (comme par exemple "kayak", "radar"
			ou "saippuakivikauppias" (marchand de pierre de savon en finnois)
		
		\item Écrire un algorithme qui vérifie si une phrase donnée sous forme
			de chaine constitue un palindrome (comme par exemple "Esope reste
			ici et se repose" ou "Tu l’as trop écrasé, César, ce
			Port-Salut~!").  Dans cette seconde version, on fait abstraction
			des majuscules/minuscules et on néglige les espaces et tout signe
			de ponctuation.

	\end{enumerate}

\end{Exercice}

\begin{Exercice}{Le chiffre de César}
	\label{ex:cesar}

	\begin{quote}
	
		Depuis l’antiquité, les hommes politiques, les militaires, les hommes
		d’affaires cherchent à garder secret les messages importants qu’ils
		doivent envoyer.  L’empereur César utilisait une technique (on dit un
		\emph{chiffrement}) qui porte à présent son nom~: remplacer chaque
		lettre du message par la lettre qui se situe $k$ positions plus loin
		dans l’alphabet (cycliquement).
	
	\end{quote}

	Exemple~:~si $k$ vaut $2$, 
	alors le texte clair "CESAR" devient "EGUCT" lorsqu’il est chiffré 
	et le texte "ZUT" devient "BWV".

	Bien sûr, il faut que l’expéditeur du message et le récepteur
	se soient mis d’accord sur la valeur de $k$.

	On vous demande d’écrire un algorithme qui reçoit une chaine ne contenant
	que des lettres majuscules ainsi que la valeur de $k$ et qui retourne
	la version chiffrée du message.

	On vous demande également d’écrire l’algorithme de déchiffrement.
	Cet algorithme reçoit un message chiffré et la valeur de $k$ qui a été
	utilisée pour le chiffrer et retourne le message en clair.
	
	\paragraph{Remarque} Ce second algorithme est \textbf{très simple} dès lors
	que l'algorithme de chiffrement est écrit.

\end{Exercice}

\begin{Exercice}{Remplacement de sous-chaines}
	Écrire un algorithme 
	qui remplace dans une chaine donnée 
	toutes les sous-chaines ch1 par la sous-chaine ch2. 
	Attention, 
	cet exercice est plus coriace qu’il n’y parait à première vue\dots~
	Assurez-vous que votre code n’engendre pas de boucle infinie. 
\end{Exercice}	



