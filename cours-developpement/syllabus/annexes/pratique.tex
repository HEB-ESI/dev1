\chapter{Bonnes pratiques}
\label{bonnespratiques}

	Ce chapitre recense quelques bonnes pratiques d'écriture d'algorithme et de
	code. Il montre également des manières d'écrire du code peu lisibles et donc
	à éviter, voire à proscrire. Rappelons que le principal critère qui nous
	guide dans ce chapitre est la lisibilité et donc la facilite de
	compréhension à la lecture. 
	
	Les bonnes pratiques d'écriture ont également un aspect culturel et
	dépendant d'un langage. En effet, certaines communautés de programmeurs
	habitués à un langage adoptent des styles de programmation qui ne sont pas
	partagés par d'autres programmeurs, habitués à un autre langage. Nous
	présentons ici des bonnes pratiques d'algorithmique générale et des bonnes
	pratiques liées au langage Java. 

\minitoc
	
%===================================================
\section{Tester un booléen}
%===================================================

	Lorsqu’il s’agit de comparer deux nombres dans la condition
	d’un \K{if}, nous écrivons assez naturellement~:
	\pc{\K{if} (nb1<nb2) \dots}
	\\Il ne vous viendrait pas à l’idée d’écrire~:
	\pc{\K{if} (nb1<nb2 == true) \dots}
	
	\bigskip
		
	Avec un booléen, les choses sont moins simples.  Par exemple, 
	pour exécuter un traitement si la variable \pc{adult} est vraie, certaines
	personnes programmeuses débutantes écrivent ceci~:
	
	\clearpage
	\begin{wrong}
	\begin{pseudocode} 
		\If{adult == true} 
			\Stmt \dots 
		\EndIf 
	\end{pseudocode} 
	\end{wrong}

	Cette écriture est correcte mais inutilement lourde.  La variable étant déjà
	un booléen qui vaut vrai ou faux, il suffit d’écrire~:

	\begin{pseudocode}
		\If{adult}
			\Stmt \dots
		\EndIf
	\end{pseudocode}
	
	De même, si le test doit être vrai quand la variable booléenne est fausse,
	il suffit d’écrire~: 
	
	\begin{pseudocode}
		\If{NON adult}
			\Stmt \dots
		\EndIf
	\end{pseudocode}

	C'est une bonne pratique d'utiliser cette deuxième écriture. 


%===================================================
\section{Assigner un booléen}\label{B-ass-bool}
%===================================================

	Pour assigner une valeur booléenne à une variable, il n'est pas utile
	d'utiliser un \pc{if-else}, une assignation directe fait l'affaire. 

	Par exemple, si la variable booléenne \pc{isNul} doit valoir \pc{true} si un
	nombre est égal à 0 et \pc{false} sinon, nous pourrions écrire le bout de
	code ci-dessous.	

	\begin{wrong}
	\begin{pseudocode}
		\If{nb == 0}
			\Let isNul \Gets true
		\Else
			\Let isNul \Gets false
		\EndIf
	\end{pseudocode}
	\end{wrong}


	Ce code est correct mais inutilement compliqué.  Puisque on assigne
	\pc{true} si la condition est vraie et \pc{false} si la condition est
	fausse, il suffit d’assigner la condition.  Ce qui donne~:

	\begin{pseudocode}
		\Let isNul \Gets nb == 0
	\end{pseudocode}

	Beaucoup plus court et plus lisible n'est-ce-pas ? 


%===================================================
\section{Assigner une valeur en fonction d’une condition}\label{B-ass-val}
%===================================================

	Examinons l’exemple ci-contre qui assigne un prix \pc{price} qui doit être
	différent en fonction du droit ou non à un tarif réduit.  
	
	\begin{pseudocode}
		\If{isDiscount}
			\Let price \Gets 8 
		\Else
			\Let price \Gets 12 
		\EndIf
	\end{pseudocode}

	Nous voyons souvent l'extrait de code plus court. Donc mieux

	Cette manière d'écrire comporte effectivement (un peu) moins de lignes mais
	ce critère n’a que très peu d’importance.  Cette version n’est pas plus
	rapide (elle est même plus lente en cas de tarif réduit) et elle est moins
	lisible car le lecteur croit d’abord que le tarif est toujours de 12 avant
	de constater qu’il peut être différent.

	\begin{wrong}
	\begin{pseudocode}
		\Let price \Gets 12 
		\If{isDiscount}
			\Let price \Gets 8 
		\EndIf
	\end{pseudocode}
	\end{wrong}

	Privilégiez la première version. 
	
%===================================================
	\section{Interrompre une boucle \texttt{for}}\label{B-for-break}
%===================================================

	Certains développeurs et développeuses débutantes\footnote{En essayant
	d'avoir une écriture plus inclusive, je m'autorise un accord de proximité.}
	interrompent anticipativement une boucle \pc{for} en modifiant l'indice.
	C'est une mauvaise idée. 

	Dans l'exemple ci-dessous, l'indice \pc{i} reçoit une «~grande valeur~» pour
	sortir de la boucle \pc{for}.
	
	\begin{wrong}
	\begin{pseudocode}
		\For{i}{1}{n}
			\Stmt \dots
			\If{aCondition}
				\Let i \Gets n+1 \Comment {Pour interrompre la boucle}
			\EndIf
			\Stmt \dots
		\EndFor
	\end{pseudocode}
	\end{wrong}

	Cette manière de faire est à proscrire absolument et ce, quelle qu’en soit
	la raison.  Ce n’est pas une bonne idée en terme de lisibilité et dans
	certains langages c’est interdit par le compilateur ou cela ne fait pas ce
	qui est attendu.
	
	Si vous devez interrompre un parcours répétitif, lorsqu'une condition est
	remplie, nous insistons pour que vous utilisiez une boucle
	\pc{\algorithmicwhile} et l’utilisation d’un booléen pour contrôler la fin.
	
	Par exemple~:
		
	\begin{pseudocode}
		\Let i \Gets 1
		\Let isFinished \Gets false
		\While{i$\leq$n AND NON isFinished}
			\Stmt \dots
			\If{aCondition}
				\Let isFinished \Gets false
			\EndIf
			\Stmt \dots
			\Let i \Gets i+1
		\EndWhile
	\end{pseudocode}

	\bigskip
	Si le test est la première instruction de la boucle
	et que la condition est courte,
	on peut aussi se passer de la variable booléenne.
		
	\begin{pseudocode}
		\Let i \Gets 1
		\While{i$\leq$n AND NON aCondition}
			\Stmt \dots
			\Let i \Gets i+1
		\EndWhile
	\end{pseudocode}

	\medskip
	Si vous tenez absolument à interrompre la boucle \pc{\algorithmicfor},
	vous pouvez utiliser un \pc{return} comme expliqué au point suivant.
	 
%===================================================
\section{Plusieurs \pc{return}}
%===================================================

	Plus tôt dans ce cours nous avons énoncé comme règle~:

	\begin{quote}
	\og{}Un seul \pc{return} par algorithme\fg{}.
	\end{quote}

	Cette règle n’est pas partagée par tous les développeurs et développeuses.
	Il existe un courant qui tolère plusieurs \pc{return}
	dans certains cas précis.

	En voici quelques uns.

	Un cas concret est l'arrêt anticipé d'une boucle.  Certains écrivent~:

	\begin{pseudocode}
			\For{i}{1}{n}
				\If{\dots}
					\Return something
				\EndIf
			\EndFor
			\Return something else
	\end{pseudocode}
	
	\bigskip
	
	Une autre situation est lorsque la valeur à retourner dépend d’un test.  On
	voit alors des bouts de code qui ressemblent à~:

	\begin{pseudocode}
			\If{\dots}
				\Return quelquechose
			\Else
				\Return autrechose
			\EndIf
	\end{pseudocode}
	
	\medskip
	
	Ces écritures seront tolérées dans ce cours. Elles sont à utiliser avec 
	parcimonie et ne se justifient que dans des algorithmes assez courts.
	
