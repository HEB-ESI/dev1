\chapter{Bonnes pratiques}
\label{bonnespratiques}

	Ce chapitre recense quelques bonnes pratiques d'écriture d'algorithme et de
	code. Il montre également des manières d'écrire du code peu lisibles et donc
	à éviter, voire à proscrire. 
	
	Les bonnes pratiques d'écriture ont également un aspect culturel et
	dépendant d'un langage. En effet, certaines communautés de programmeurs
	habitués à un langage adoptent des styles de programmation qui ne sont pas
	partagés par d'autres programmeurs, habitués à un autre langage. Nous
	présentons ici des bonnes pratiques d'algorithmique générale et des bonnes
	pratiques liées au langage Java. 

	\begin{quote}
		
		{\Large «}~Lorsque je code en langage Java, je respecte les conventions
		du langage Java. 
		Lorsque je code en langage C++, je respecte les
		conventions du langage C++. 

		Lorsque je contribue à un projet, je respecte les conventions du
		langage et celles proposées par l'équipe en charge du projet.~{\Large»}

	\end{quote}

\minitoc


%===================================================
\section{Bonnes pratiques générales}
%===================================================

	Certaines conventions d'écriture, sont communes à tous les langages. Citons
	par exemple~:

	\begin{itemize}

		\item préferer des lignes de longueur maximale de 80 caractères. Ceci
			facilite la lecture\,;
		
		\item couper les lignes qui seront trop longues en suivant ces règles\,;

			\begin{itemize}
				
				\item couper après une virqule\,;
				\item couper avant un opérateur\,;

				\item préférer les coupures de haut niveau de priorité aux
					coupures de priorité faible\,;

				\item aligner la nouvelle ligne avec le début de l'expression
					se trouvant à la ligne précédente.

			\end{itemize}
			
	\end{itemize}

	Nous vous invitons à consulter les conventions d'écriture du code en Java
	—~même si le document est vieux~— ceci afin d'écrire un code plus lisible.
	Voir \cite{java-codeconv}

%===================================================
\section{Tester un booléen}
%===================================================

	Lorsqu’il s’agit de comparer deux nombres dans la condition d’un \K{if},
	nous écrivons assez naturellement~: \pc{\K{if} (nb1<nb2)} et il ne vous
	viendrait pas à l’idée d’écrire~: \pc{\K{if} (nb1<nb2 == true)}.
	
	\bigskip
		
	Avec un booléen, nous allons également éviter d'écrire \pc{== true} ou
	\pc{== false}.  Par exemple, pour exécuter un traitement si la variable
	\pc{isAdult} est vraie, certaines personnes programmeuses débutantes écrivent
	ceci~:
	
	\marginicon{dont}
	\begin{java}
		if (isAdult == true){
			// ...
		}
	\end{java}

	Cette écriture est correcte mais inutilement lourde.  La variable étant déjà
	un booléen qui vaut vrai ou faux, il suffit d’écrire~:

	\begin{java}
if (isAdult){
	// ...
}
	\end{java}

	C'est une bonne pratique d'utiliser cette deuxième écriture. 
	
	De même, si le test doit être vrai quand la variable booléenne est fausse,
	il suffit d’écrire~: 
	
	\begin{java}
if (!isAdult){
	// ...
}
	\end{java}
	

%===================================================
\section{Assigner un booléen}\label{B-ass-bool}
%===================================================

	Pour assigner une valeur booléenne à une variable, il n'est pas utile
	d'utiliser un \pc{if-else}, une assignation directe fait l'affaire. 

	Par exemple, si la variable booléenne \pc{isNul} doit valoir \pc{true} si un
	nombre est égal à 0 et \pc{false} sinon, nous pourrions écrire le bout de
	code ci-dessous.	

	\marginicon{dont}
	\begin{java}
if (nb == 0){
	isNul = true;
} else {
	isNul = false;
}
	\end{java}

	Ce code est correct mais inutilement compliqué.  Puisque on assigne
	\pc{true} si la condition est vraie et \pc{false} si la condition est
	fausse, il suffit d’assigner la condition.  Ce qui donne~:

	\begin{java}
isNul = nb == 0;
	\end{java}



%===================================================
\section{Assigner une valeur en fonction d’une condition}\label{B-ass-val}
%===================================================

	Examinons l’exemple ci-contre qui assigne un prix \pc{price} qui doit être
	différent en fonction du droit ou non à un tarif réduit.  

	\begin{java}
if (isDiscount){
	price = 8;
} else {
	price = 12;
}
	\end{java}
	
	Nous rencontrons parfois le code suivant. Plus court. Est-il nécessairement
	mieux ? 

	\marhinicon{dont}
	\begin{java}
price = 12;
if (isDiscount){
	price = 8;
}
	\end{java}

	Cette manière d'écrire comporte effectivement (un peu) moins de lignes mais
	ce critère n’a que très peu d’importance.  Cette version n’est pas plus
	rapide (elle est même plus lente en cas de tarif réduit) et elle est moins
	lisible car le lecteur croit d’abord que le tarif est toujours de 12 avant
	de constater qu’il peut être différent.

	Nous privilégirons la première version. 
	
%===================================================
	\section{Interrompre une boucle \texttt{for}}
	\label{B-for-break}
%===================================================

	Certains développeurs et développeuses débutantes\footnote{En essayant
	d'avoir une écriture plus inclusive, je m'autorise un accord de proximité.}
	interrompent anticipativement une boucle \pc{for} en modifiant l'indice.
	C'est une mauvaise idée. 

	Dans l'exemple ci-dessous, l'indice \pc{i} reçoit une «~grande valeur~» pour
	sortir de la boucle \pc{for}.
	
	\marginicon{dont}
	\begin{java}
for (int i=0; i < n; i++){
	// ...
	if (aCondition) {
		i = n + 1;		// pour interrompre la boucle
	}
	// ...
}
	\end{java}

	Cette manière de faire est à proscrire absolument et ce, quelle qu’en soit
	la raison.  Ce n’est pas une bonne idée en terme de lisibilité et dans
	certains langages c’est interdit par le compilateur ou cela ne fait pas ce
	qui est attendu.
	
	Si vous devez interrompre un parcours répétitif, lorsqu'une condition est
	remplie, nous insistons pour l'utilisation d'une boucle
	\pc{\algorithmicwhile} et l’utilisation d’un booléen pour contrôler la fin.
	
	Par exemple~:

	\begin{java}
int i = 0;
isFinished = false;
while (i < n && !isFinished){
	// ...
	if (aCondition){
		isFinished = true;
	}
	// ...
	i = i + 1;
}
	\end{java}
		
	\bigskip
	Si le test est la première instruction de la boucle
	et que la condition est courte,
	on peut aussi se passer de la variable booléenne.

	\begin{java}
int i = 0;
while (i < n && !aCondition){
	// ...
	i = i + 1;
}
	\end{java}
		
	\medskip
	Si vous tenez absolument à interrompre la boucle \pc{\algorithmicfor},
	vous pouvez utiliser un \pc{return} comme expliqué au point suivant.
	 
%===================================================
\section{Plusieurs \pc{return}}
%===================================================

	Plus tôt dans ce cours nous avons énoncé comme règle~:

	\begin{quote}
	\og{}Un seul \pc{return} par algorithme\fg{}.
	\end{quote}

	Cette règle n’est pas partagée par tous les développeurs et développeuses.
	Il existe un courant qui tolère plusieurs \pc{return}
	dans certains cas précis.

	En voici quelques uns.

	Un cas concret est l'arrêt anticipé d'une boucle.  Certains écrivent~:

	\begin{java}
for (int i = 0; i < n; i++){
	if (aCondition){
		return something;
	}
	// ...
	return somethingElse,
}
	\end{java}

	\bigskip
	
	Une autre situation est lorsque la valeur à retourner dépend d’un test.  On
	voit alors des bouts de code qui ressemblent à~:

	\begin{java}
if (aCondition){
	return something;
} else {
	return somethingElse;
}
	\end{java}

	
