\clearpage
\section{Exercice résolu~: le blackjack}
\label{exercice-resolu-blackjack}

Le blackjack est un jeu de casino dans lequel le joueur ou la joueuse joue
contre la banque. Le but du jeu est que la somme de ses cartes soit le
plus proche de 21 sans pour autant dépasser cette valeur. Chaque carte a
pour valeur son rang, sauf le valet (11), la dame (12) et roi (13) qui
valent \textbf{dix} et l'as (1) qui vaut \textbf{1 ou 11}.

Écrivons un algorithme qui reçoit les cartes représentant la main du joueur ou de
la joueuse dans un tableau d'entiers (de 1 à 13) et qui retourne la somme la
plus avantageuse pour la personne. C'est à dire, celle qui est le plus proche
de 21 sans pour autant dépasser cette valeur. Il faut donc attribuer aux as la
valeur 1 ou 11 selon le jeu de la personne.

Exemples~:

\begin{itemize}

	\item pour le tableau de cartes {[}5,12,1,4{]}, l'algorithme renvoie 20
		(5 + 10\,(la dame vaut 10) + 1 + 4)\,;

	\item pour le tableau de cartes {[}1,1,7{]} l'algorithme renvoie 19
		(1 + 11 + 7)\,;

	\item pour le tableau de cartes {[}2,3,11,1,1{]} l'algorithme renvoie 17 (2
		+ 3 + 10 + 1 + 1)\,;

	\item pour le tableau de cartes {[}13,3,11,1,1{]} l'algorithme renvoie 25
		(10 + 3 + 10 + 1 + 1).

\end{itemize}

\paragraph{Stratégie de résolution}

Avant d'espérer une solution correcte à un problème, il faut s'assurer
de bien le comprendre, d'identifier tous les cas particuliers et de
savoir soi-même calculer la bonne réponse dans ces cas là.

Lorsque le problème semble complexe, une stratégie peut être d'attaquer
d'abord une version plus simple du problème pour incorporer par la suite
les éléments mis de coté. C'est ce que nous allons faire ici.


\subsection{Une version simplifiée~: tous les as valent 1}

Commençons donc par considérer une version plus simple où tous les as valent 1.
Si on arrive à résoudre ce problème on réintroduira la possibilité pour un as
de valoir 11.

Dans cette version simplifiée, il s'agit de calculer la somme des valeurs des
cartes. Ce qu'il ne faut pas confondre avec la carte elle-même. Car si un
5 vaut 5, un 12 vaut 10.

\paragraph{La somme}

Comme chaque carte compte, on doit effectuer un parcours complet du
tableau. On peut schématiser la solution ainsi :

\begin{langagenaturel}
Algorithme: somme simplifié d'un tableau de cartes.\\

Soit cartes le tableau des cartes\\
Initialiser une somme à 0\\
Pour toutes les cartes \\
	\tab  Calculer la valeur de la carte\\
	\tab Ajouter cette valeur à la somme\\
\end{langagenaturel}

Dans l'algorithme que l'on vient d'écrire, il existe un seul élément qui
n'est pas trivial : le \textbf{calcul de la valeur d'une carte}. Il s'agit d'un
problème en soi, qu'il est bon de regarder et de résoudre séparément
sans polluer l'algorithme que l'on vient d'écrire

\paragraph{La valeur d'une carte}

Dans la plupart des cas, la valeur de la carte est égale à elle-même.
Les seules exceptions sont les figures (valet, dame et roi) qui valent
10. On peut l'exprimer ainsi :

\begin{langagenaturel}
Algorithme: Calcul de la valeur d'une carte\\

Soit carte une carte\\
Si la carte est > 10 alors\\
    \tab sa valeur est 10\\
sinon\\
    \tab sa valeur est égale à elle-même\\
\end{langagenaturel}

\paragraph{Solution en Java}

Si on exprime le tout en Java, ça donne :

\begin{java}
public static int valeur(int carte) {
    int valeur;
    if(carte > 10) {
        valeur = 10;
    } else {
        valeur = carte;
    }
    return valeur;
}

public static int sommeSimplifiée(int[] cartes) {
    int somme = 0;
    for(int carte : cartes) {
        somme += valeur(carte);
    }
    return somme;
}
\end{java}

Si on veut pouvoir le tester, il faut écrire une méthode principale qui
crée le tableau et appelle la méthode calculant la somme. Pour le
premier exemple de l'énoncé, cela donne

\begin{java}
public static void main(String[] args) {
    int[] cartes = {5,12,1,4};
    int somme = sommeSimplifiée(cartes);
    System.out.println(somme);
}
\end{java}

\subsection{Version complète~: un as vaut 1 ou 11}

Nous avons une solution mais rappelons-nous que nous avons simplifié le
problème. Réintroduisons la règle qui dit qu'un as vaut 1 ou 11.

\paragraph{1 ou 11 ? Comment décider~?}

Il faut bien comprendre les points suivants : 

\begin{itemize}

	\item un as va compter pour 11 si et seulement si ça ne fait pas dépasser 
		la valeur 21\,;
		
	\item même si plusieurs as sont présents dans la main du joueur, il ne peut
		y en avoir qu'un seul maximum qui compte pour 11, sinon le total
		dépasserait 21\,;

	\item on ne peut pas décider qu'un as vaut pour 11 tant qu'on n'a pas
		calculé la somme complète. 
		
		Par exemple avec {[}8,1,7{]}, lorsqu'on rencontre le 1, on pourrait
		être tenté de le compter comme un 11 (8 + 11 = 19\textless{}21) mais ce
		sera un problème par la suite pour ajouter le 7 (19
		+ 7 = 26\textgreater{}21).

\end{itemize}

Des constatation ci-dessus, on peut tirer un algorithme assez simple

\begin{langagenaturel}
Algorithme: somme non simplifié d'un tableau de cartes.\\

Soit cartes le tableau des cartes\\
Calculer la somme simplifiée, c'est à priori la bonne réponse\\
Mais s'il existe (au moins) un as dans les cartes \\
		\tab\tab\tab et que la somme simplifiée est < 12 alors\\
    \tab Ajouter 10 à la somme simplifiée\\
\end{langagenaturel}

Dans cet algorithme le seul élément non trivial qu'il reste est de savoir
s'il y a (au moins) un as dans la main du joueur. Voyons comment le
résoudre.

\paragraph{Au moins un as}

Se demander s'il y a un as dans la main du joueur (le tableau de cartes)
est un problème très proche d'autres qu'on a déjà résolu (ex: ``est-ce
que le tableau contient un 0 ?''). On a vu que la solution mettait en
oeuvre un \textbf{parcours partiel} du tableau. On a vu aussi qu'il y a
deux solutions types : avec et sans variable booléenne. Il suffit donc
de reprendre et d'adapter une des 2 approches déjà écrite par ailleurs.

\paragraph{Solution en Java}

La solution en Java peut utiliser les 2 méthodes déja écrites.

\begin{java}
public static int somme(int[] cartes) {
    int somme = sommeSimplifiée(cartes);
    if (existeAs(cartes) && somme<12) {
        somme += 10;
    }
    return somme;
}

public static boolean existeAs(int[] cartes) {
    int i = 0;
    boolean asTrouvé = false;
    while (i<cartes.length && !asTrouvé) {
        asTrouvé = (cartes[i]==1);
        i++;
    }
    return asTrouvé;
}
\end{java}

Revoyez l'algorithme de parcours partiel si vous ne comprenez pas bien
la méthode \textbf{existeAs}. Notez qu'on aurait pu utiliser l'autre
approche pour \textbf{existeAs} ce qui aurait donné :

\begin{java}
public static boolean existeAs(int[] cartes) {
    int i = 0;
    while (i<cartes.length && cartes[i]!=1) {
        i++;
    }
    return i<cartes.length;
}
\end{java}

Vous vous rappelez pourquoi le test du return doit être celui-là ?

Il reste une dernière chose à faire : modifier la méthode principale
afin qu'elle appelle la nouvelle version de la somme :

\begin{java}
public static void main(String[] args) {
    int[] cartes = {5,12,1,4};
    int somme = somme(cartes);
    System.out.println(somme);
}
\end{java}

Ce n'est pas tout-à-fait fini. Comme tout programmeur est faillible, il
reste une chose importante à faire : tester. Au minimum, en testant que
le programme donne la bonne réponse dans tous les exemples donnés dans
l'énoncé.

\subsection{Conclusion}

Il y trois enseignements à tirer de cet exercice~:

\begin{enumerate}

	\item Ne pas hésiter à attaquer d'abord une version simplifiée du problème.
		Cela permet d'arriver plus facilement à la solution finale. 

	Ici, il n'y a même pas eu de travail inutile puisque nous avons pu
	réutiliser tel quel le code écrit pour la version simplifiée. 

	\item Décomposer la solution en plusieurs algorithme. Si on identifie un
		sous-problème, il mérite une méthode à part plutôt que d'incorporer
		toutes les lignes de code dans une même méthode. On obtiendrait une
		longue méthode qui a beaucoup de défauts (plus difficile
		à lire/comprendre/corriger) et aucun avantage décisif (peut-être un
		rien plus rapide mais c'est négligeable). 

	\item Ne pas réinventer la roue ! Toujours se demander si le problème que
		l'on doit résoudre ne fait pas partie d'une catégorie de problèmes déjà
		résolu. Si c'est la cas, reprendre et adapter la solution déjà trouvée
		plutôt que de tout recommencer du début. 

		Par exemple, ici, on a compris que rechercher la présence d'un as est
		un parcours partiel de tableau à la recherche d'un élément donné. Un
		problème déjà étudié et résolu.

\end{enumerate}
