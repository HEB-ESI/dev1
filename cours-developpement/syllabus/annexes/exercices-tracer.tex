\clearpage
\section{Exercices~: tracer un algorithme}

Nous vous proposons de tracer vos algorithmes comme présenté dans le chapitre 
\ref{premalgos} dans la section \ref{tracer}, page \pageref{tracer}.


\begin{Exercice}{Tracer des bouts de code}
	Suivez l’évolution des variables pour les bouts
	d’algorithmes donnés.

	\begin{minipage}{4cm}
		\begin{java}
int a, b, c;
a = 42;
b = 24;
c = a + b;
c = c - 1;
a = 2 * b;
c = c + 1;
		\end{java}
	\end{minipage}
	\quad%
	\begin{minipage}{7cm}
		\begin{tabular}{|>{\centering\arraybackslash}m{1cm}|*{3}{>{\centering\arraybackslash}m{2cm}}|}
			\hline
			\verb_#_ & {a} & {b} & {c}\\
			\hline
			1 & {} & {} & {} \\
			2 & {} & {} & {} \\
			3 & {} & {} & {} \\
			4 & {} & {} & {} \\
			5 & {} & {} & {} \\
			6 & {} & {} & {} \\
			7 & {} & {} & {} \\
			\hline
		\end{tabular}
	\end{minipage}

	\bigskip
	\begin{minipage}{4cm}
          \begin{java}
int a, b, c;
double d;
a = 2;
b = a*a*a;
c = b - a*a;
d = Math.sqrt(c);
d = d/d;
          \end{java}
	\end{minipage}
	\quad%
	\begin{minipage}{7cm}
		\begin{tabular}{|>{\centering\arraybackslash}m{1cm}|*{3}{>{\centering\arraybackslash}m{2cm}}|}
			\hline
			\verb_#_ & {a} & {b} & {c}\\
			\hline
			1 & {} & {} & {} \\
			2 & {} & {} & {} \\
			3 & {} & {} & {} \\
			4 & {} & {} & {} \\
			5 & {} & {} & {} \\
			6 & {} & {} & {} \\
			\hline
		\end{tabular}
	\end{minipage}	
\end{Exercice}

\begin{Exercice}{Calcul de vitesse}

	Soit le problème suivant :
	\og
	Calculer la vitesse (en km/h) d’un véhicule 
	dont on donne la durée du parcours (en secondes) 
	et la distance parcourue (en mètres).
	\fg.

	Voici \textit{une} solution : 
\begin{java}
public static double vitesseKMH(double distanceM, double duréeS){
	double distanceKM, duréeH;
	distanceKM = 1000 * distanceM;
	duréeH = 3600 * duréeS;
	return distanceKM / duréeH;
}
\end{java}

L’algorithme, s’il est correct, devrait donner une vitesse de 1~km/h pour une
distance de 1000~mètres et une durée de 3600~secondes.  Testez cet algorithme
avec cet exemple.

\begin{java}
vitesseKMH(1000, 3600);
\end{java}

\begin{center}
	\begin{tabular}{|>{\centering\arraybackslash}m{1cm}|*{5}{>{\centering\arraybackslash}m{2cm}}|}
		\hline
		\verb_#_  &  &  & & &  \\			
		\hline
		1 & & & & & \\
		2 & & & & & \\
		3 & & & & & \\
		4 & & & & & \\
		5 & & & & & \\
		\hline
	\end{tabular}
\end{center}

Si vous trouvez qu’il n’est pas correct, voyez ce qu’il faudrait changer pour le
corriger.

\end{Exercice}


\begin{Exercice}{Allure, vitesse ou cote moyenne}

	Reprenez un exercice de la section \vref{annexe-specifier} — par exemple,
	le problème de l'allure du coureur (voir exercice \vref{ex:allure}) ou du
	calcul d'une cote moyenne (voir exercice \vref{ex:cotemoyenne}) — et tracer
	l'algorithme que vous avez écrit. 

\begin{center}
	\begin{tabular}{|>{\centering\arraybackslash}m{1cm}|*{5}{>{\centering\arraybackslash}m{2cm}}|}
		\hline
		\verb_#_  &  &  & & &  \\			
		\hline
		1 & & & & & \\
		2 & & & & & \\
		3 & & & & & \\
		4 & & & & & \\
		5 & & & & & \\
		\hline
	\end{tabular}
\end{center}

\end{Exercice}

