\clearpage
\section{Exercices~: structures alternatives}			

\begin{Emphase}
	
	\paragraph{Remarque} Dans certains des exercices qui suivent vous aurez
	besoin d'obtenir une valeur «~au hasard~». Vous pouvez supposer qu'il existe
	une instruction \pc{random(n)} qui donne une valeur au hasard comprise entre
	0 et n (strictement).

	Cette notion est abordée dans la section \ref{hasard}, p.\pageref{hasard}.

\end{Emphase}

\begin{Exercice}{Compréhension}
		
	Tracez cet algorithme avec les valeurs fournies et donnez la valeur de
	retour.

	\begin{java}
public static int exerciceA(int a, int b){
	int c;
	c = 2 * a;
	if (c > b){
		c = c - b;
	}
	return c;
}
	\end{java}	
\begin{multicols}{2}
	\begin{itemize}
		\item \pc{exerciceA(2, 5)} = \_\_\_
		\item \pc{exerciceA(4, 1)} = \_\_\_
	\end{itemize}
\end{multicols}	
	\end{Exercice}


\begin{Exercice}{Compréhension}
	Tracez ces algorithmes avec les valeurs fournies et donnez la valeur de retour.

	\begin{java}
public static int exerciceB(int a, int b){
	int c;
	if (a > b){
		c = a/b;
	} else {
		c = a%b;
	}
	return c;
}
	\end{java}


		\begin{multicols}{2}
			\begin{itemize}
				\item \pc{exerciceB(2, 3)} = \_\_\_
				\item \pc{exerciceB(4, 1)} = \_\_\_
			\end{itemize}
		\end{multicols}

		\bigskip

		\begin{java}
public static int exerciceC(int x1, int x2){
	boolean isBigger;
	isBigger = x1 > x2;
	if (isBigger){
		isBigger = isBigger && x1 ==  4;
	} else {
		isBigger = isBigger || x2 ==  3;
	}
	if (isBigger){
		x1 = x1 * 1000;
	}
	return x1 + x2;
}
		\end{java}

		\medskip
		\begin{multicols}{2}
			\begin{itemize}
				\item \pc{exerciceC(2, 3)} = \_\_\_
				\item \pc{exerciceC(4, 1)} = \_\_\_	
			\end{itemize}
		\end{multicols}	


	\end{Exercice}	

	\begin{Exercice}{Simplification d’algorithmes}

		Voici deux extraits d’algorithmes que vous pouvez supposer corrects
		du point de vue de la syntaxe mais contenant des lignes inutiles ou des
		lourdeurs d’écriture.  Remplacer chacune de ces portions d’algorithme
		par un minimum d’instructions qui auront un effet équivalent.

		\begin{minipage}[t]{7cm}
			\begin{java}
if (condition){
	b = true;
} else {
	b = false;
}
			\end{java}
	\end{minipage}
	\quad
	\begin{minipage}[t]{7cm}		
		\begin{java}
if (n1>n2){
	b = false;
} else {
	if (n1<=n2){
		b = true;
	}
}
		\end{java}
	\end{minipage}
	\end{Exercice}

	\begin{Exercice}{Maximum de 2 nombres}

		Écrire un algorithme qui, étant donné deux nombres quelconques,
		recherche et retourne le plus grand des deux. Attention~! On ne veut
		pas savoir si c’est le premier ou le deuxième qui est
		le plus grand mais bien quelle est cette plus grande valeur. Le
		problème est donc bien défini même si les deux nombres sont
		identiques.

		\textbf{Solution.}
		Une solution complète est disponible dans la fiche \vref{fiche:max2nb}.
	\end{Exercice}

	\begin{Exercice}{Calcul de salaire}
		Dans une entreprise, 
		une retenue spéciale de 15\% est pratiquée 
		sur la partie du salaire mensuel qui dépasse 1200~\texteuro. 
		Écrire un algorithme qui calcule le salaire net à partir du salaire brut. 
		En quoi l’utilisation de constantes convient-elle pour améliorer cet algorithme~?
	\end{Exercice}

	\begin{Exercice}{Fonction de Syracuse}

		Écrire un algorithme qui, étant donné un entier $n$ quelconque,
		retourne le résultat de la fonction
		$f(n)=
		\left\{
			\begin{array}{rl}
				n/2 & si \ n \ est\ pair\\
				3n+1 & si \ n \ est \ impair
			\end{array}
			\right.$
		\end{Exercice}

		\begin{Exercice}{Tarif réduit ou pas}
			Dans une salle de cinéma,
			le tarif plein pour une place est de 8\texteuro{}.
			Les personnes ayant droit au tarif réduit payent 7\texteuro{}.
			Écrire un algorithme qui reçoit un booléen
			indiquant si la personne peut bénéficier du tarif réduit
			et qui retourne le prix à payer.
		\end{Exercice}

		\begin{Exercice}{Maximum de 3 nombres}

			Écrire un algorithme qui, étant donné trois nombres quelconques,
			recherche et retourne le plus grand des trois.
		\end{Exercice}

		\begin{Exercice}{Le signe}

			Écrire un algorithme qui \textbf{affiche} un message indiquant
			si un entier est strictement négatif, nul ou strictement
			positif.
		\end{Exercice}

		\begin{Exercice}{Le type de triangle}
			Écrire un algorithme qui indique si un triangle
			dont on donne les longueurs de ces 3 cotés est~:
			équilatéral (tous égaux), isocèle (2 égaux)
			ou quelconque.
		\end{Exercice}

		\begin{Exercice}{Dés identiques}
			Écrire un algorithme qui lance trois dés
			et indique si on a obtenu 3 dés de valeur identique,
			2 ou aucun.
		\end{Exercice}

		\begin{Exercice}{Grade}

			Écrire un algorithme qui retourne le grade d’un étudiant 
			suivant la moyenne qu’il a obtenue.

			Un étudiant ayant obtenu 
			\begin{itemize}
				\item moins de 50\% n’a pas réussi~;
				\item de 50\% inclus à 60\% exclu a réussi~;
				\item de 60\% inclus à 70\% exclu a une satisfaction~;
				\item de 70\% inclus à 80\% exclu a une distinction~;
				\item de 80\% inclus à 90\% exclu a une grande distinction~;
				\item de 90\% inclus à 100\% inclus a la plus grande distinction.
			\end{itemize}
		\end{Exercice}


		\begin{Exercice}{Numéro du jour}

			Écrire un algorithme qui retourne le numéro du jour de la semaine
			reçu en paramètre (1 pour "lundi", 2 pour "mardi"\dots).
		\end{Exercice}

		\begin{Exercice}{Tirer une carte}

			Écrire un algorithme qui affiche l’intitulé d’une carte
			tirée au hasard dans un paquet de 52 cartes.
			Par exemple, "As de cœur", "3 de pique", "Valet de carreau"
			ou encore "Roi de trèfle".

			\textbf{Remarque}. Il est plus facile de déterminer séparément
			chacune des deux caractéristiques de la carte : couleur et valeur.
		\end{Exercice}

		\begin{Exercice}{Nombre de jours dans un mois}
			Écrire un algorithme qui retourne le nombre de jours dans un mois. 
			Le mois est lu sous forme d’un entier (1 pour janvier\dots).
			On considère dans cet exercice que le mois de février
			comprend toujours 28 jours.
		\end{Exercice}

		\begin{Exercice}{Moyenne pondérée}

			Un examen se déroule en plusieurs parties;
			\begin{itemize}
				\item la première partie a une pondération de 5\%\,;
				\item la seconde, 25\% et\,;
				\item la dernière 70\%.
			\end{itemize}
			Écrire un algorithme qui reçoit les 3 cotes (/20) et qui indique
			si l'étudiant a réussi l'examen ou non. L'algorithme affiche également
			la cote de l'examen. 
		\end{Exercice}	


		\begin{Exercice}{La fourchette}

			Écrire un algorithme qui, étant donné trois nombres, 
			retourne vrai si le premier des trois 
			appartient à l’intervalle donné par le plus petit et le plus grand 
			des deux autres (bornes exclues) et faux sinon. 
			Qu’est-ce qui change si on inclut les bornes~?
		\end{Exercice}

		\begin{Exercice}{Le prix des photocopies}

			Un magasin de photocopies facture 0,10 \texteuro{} 
			les dix premières photocopies, 
			0,09 \texteuro{} les vingt suivantes 
			et 0,08 \texteuro{} au-delà. 
			Écrivez un algorithme 
			qui reçoit le nombre de photocopies effectuées 
			et qui affiche la facture correspondante.
		\end{Exercice}

		\begin{Exercice}{Le stationnement alternatif}

			Dans une rue où se pratique le stationnement alternatif, du 1 au 15
			du mois, on se gare du côté des maisons ayant un numéro impair, et
			le reste du mois, on se gare de l’autre côté.  Écrire un algorithme
			qui, sur base de la date du jour et du numéro de maison devant
			laquelle vous vous êtes arrêté, retourne vrai si vous êtes bien
			stationné et faux sinon.  \end{Exercice}


