\clearpage
\section{Exercices~: division entière, quotient et reste}

\begin{Exercice}{Calculs}

	Voici quelques petits calculs à compléter faisant intervenir la division entière
	et le reste.  Par exemple~: «~14 DIV 3 = 4 reste 2~» signifie que 14 DIV 3 = 4 et
	14 MOD 3 = 2.

	\begin{multicols}{2}
		\begin{itemize}
			\item 11 DIV 3 = \_\_\ reste\ \_\_
			\item 3 DIV 11 = \_\_\ reste\ \_\_
			\item 11 DIV \_\_ = 2\ reste\ 3
			\item \_\_ DIV 3 = 3\ reste\ 1
		\end{itemize}
	\end{multicols}
\end{Exercice}

\begin{Exercice}{Les prix ronds}
	Voici un algorithme qui reçoit une somme d’argent exprimée en centimes
	et qui calcule le nombre (entier) de centimes qu’il
	faudrait ajouter à la somme pour tomber sur un prix rond en euros.
	Testez-le avec des valeurs numériques. Est-il correct~?

\begin{java}
public static int versPrixRond(int prixCentimes){
	return 100 - (prixCentimes % 100);
}
\end{java}

\begin{center}
	\begin{tabular}{|c|c|c|c|c|}
		\hline
		test \no & prixCentimes & réponse correcte & valeur retournée & Correct~? \\\hline
		\hline 
		1 & 130 & 70 &  & \\\hline
		2 & 40  & 60 &  & \\\hline
		3 & 99  & 1  &  & \\\hline
		4 & 100 & 0  &  & \\\hline
	\end{tabular}
\end{center}

\end{Exercice}

\bigskip
\bigskip
\bigskip
\begin{Emphase}
	
	\paragraph{Remarque}
	Les exercices qui suivent n’ont pas tous été déjà analysés et ils demandent
	des calculs faisant intervenir des divisions entières, des restes et/ou des
	expressions booléennes.  Comme d’habitude, écrivez la spécification si ça
	n’a pas encore été fait, donnez des exemples, rédigez un algorithme et
	vérifiez-le.

\end{Emphase}

		\begin{Exercice}{Nombre multiple de 5}

			\label{algo:mult5}
			Calculer si un nombre entier donné est un multiple de 5.

			\clearpage
			\paragraph{Solution.}
			Dans ce problème,
			il y a une donnée, le nombre à tester.
			La réponse est un booléen
			qui est à vrai si le nombre donné est un multiple de 5.
			\begin{center}
				\flowalgod{nombre (integer)}{multiple5}{boolean}
			\end{center}
			\textbf{Exemples.}
			\begin{multicols}{4}
				\begin{itemize}
					\item \pc{multiple5(4)} donne faux
					\item \pc{multiple5(15)} donne vrai
					\item \pc{multiple5(0)} donne vrai
					\item \pc{multiple5(-10)} donne vrai
				\end{itemize}
			\end{multicols}
			La technique pour vérifier si un nombre est
			un multiple de 5 est de vérifier que le reste
			de la division par 5 donne 0.
			Ce qui donne~:
			\begin{langagenaturel}
				return nombre \% 5 = 0
			\end{langagenaturel}
		
			Vérifions sur nos exemples~:
		\begin{center}
			\begin{tabular}{|c|c|c|c|c|}
				\hline
				test \no & nombre & réponse correcte & valeur retournée & Correct~? \\\hline
				\hline 
				1 & 4   & faux & faux & {\color{ForestGreen}$\checkmark$} \\\hline
				2 & 15  & vrai & vrai & {\color{ForestGreen}$\checkmark$} \\\hline
				3 & 0   & vrai & vrai & {\color{ForestGreen}$\checkmark$} \\\hline
				4 & -10 & vrai & vrai & {\color{ForestGreen}$\checkmark$} \\\hline
			\end{tabular}
		\end{center}
	\end{Exercice}

	\begin{Exercice}{Nombre entier positif se terminant par un 0}
		Calculer si un nombre donné se termine par un 0.
	\end{Exercice}

	\begin{Exercice}{Les centaines}
		Calculer la partie \emph{centaine}
		d’un nombre entier positif quelconque.
	\end{Exercice}

	\begin{Exercice}{Somme des chiffres}
		Calculer la somme des chiffres
		d’un nombre entier positif inférieur à 1000.
	\end{Exercice}

	\begin{Exercice}{Conversion secondes en heures}
		Étant donné un temps écoulé depuis minuit en secondes.
		Si on devait exprimer ce temps sous la forme
		habituelle (heure, minute et seconde),
		que vaudrait la partie "heure".

		Ex~:~10000 secondes donnera 2 heures.

		Aide~: L’heure n’est qu’un nombre exprimé en base 60~!
	\end{Exercice}

	\begin{Exercice}{Conversion secondes en minutes}
		Étant donné un temps écoulé depuis minuit en secondes.
		Si on devait exprimer ce temps sous la forme
		habituelle (heure, minute et seconde),
		que vaudrait la partie "minute".

		Ex~:~10000 secondes donnera 46 minutes.
	\end{Exercice}

	\begin{Exercice}{Conversion secondes en secondes}
		Étant donné un temps écoulé depuis minuit en secondes.
		Si on devait exprimer ce temps sous la forme
		habituelle (heure, minute et seconde),
		que vaudrait la partie "seconde".

		Ex~:~10000 secondes donnera 40 secondes.
	\end{Exercice}	

	\begin{Exercice}{Un double aux dés}

		Écrire un algorithme qui simule le lancer de deux dés
		et indique s’il y a eu un double 
		(les deux dés montrant une face identique).
	\end{Exercice}

	\begin{Exercice}{Année bissextile}

		\label{ex:bissextile}
		Écrire un algorithme qui vérifie si une année est bissextile.

		Pour rappel, les années bissextiles sont les années multiples de 4. 
		Font exception, les multiples de 100 
		(sauf les multiples de 400 qui sont bien bissextiles). 
		Ainsi $2012$ et $2400$ sont bissextiles mais pas $2010$ ni $2100$.
	\end{Exercice}



	\begin{Exercice}{Conversion en heures-minutes-secondes}

		Écrire un algorithme qui permet à l’utilisateur
		de donner le nombre de secondes écoulées depuis minuit
		et qui affiche le moment de la journée correspondant
		en heures-minutes-secondes.
		Par exemple, si on est 3726 secondes après minuit
		alors il est 1h2'6''.
	\end{Exercice}


