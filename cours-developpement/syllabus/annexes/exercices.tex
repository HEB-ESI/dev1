\chapter{Exercices}

\minitoc

%----------------------------------------------
\section{Spécifier le problème, exercices}
%----------------------------------------------
	
		Pour les exercices suivants, 
		nous vous demandons d’imiter la démarche décrite dans ce chapitre, 
		à savoir~:
		\begin{itemize}
			\item Déterminer quelles sont les données~;
				leur donner un nom et un type.
			\item Déterminer quel est le type du résultat.
			\item Déterminer un nom pertinent pour l’algorithme.
			\item Fournir un résumé graphique.
			\item Donner des exemples.
		\end{itemize}
	
		\begin{Exercice}{Somme de 2 nombres}
			Calculer la somme de deux nombres donnés.
			\paragraph{Solution.}%
			\footnote{%
				Nous allons de temps en temps 
				fournir des solutions.
				En algorithmique,
				il y a souvent \textbf{plus qu’une} solution possible.
				Ce n’est donc pas parce que vous avez trouvé autre chose
				que c’est mauvais.
				Mais il peut y avoir des solutions \textbf{meilleures}
				que d’autres; 
				n’hésitez jamais à montrer la vôtre
				à votre professeur pour avoir son avis.
			}
			Il y a ici clairement 2 données.
			Comme elles n’ont pas de rôle précis,
			on peut les appeler simplement \pc{nombre1}
			et \pc{nombre2}
			(\pc{nb1} et \pc{nb2} sont aussi de bons choix).
			L’énoncé ne dit pas si les nombres sont entiers ou pas;
			restons le plus général possible en prenant des réels.
			Le résultat sera de même type que les données.
			Le nom de l’algorithme pourrait être simplement \pc{somme}.
			Ce qui donne~:
			\begin{center}
				\flowalgodd{nombre1 (réel)}{nombre2 (réel)}{somme}{réel}
			\end{center}			 
			Et voici quelques exemples numériques~:	
				\pc{somme(3, 2)} donne $5$      \quad
				\pc{somme(-3, 2)} donne $-1$    \quad
				\pc{somme(3, 2.5)} donne $5.5$  \quad
				\pc{somme(-2.5, 2.5)} donne $0$.
		\end{Exercice}
	
		\begin{Exercice}{Moyenne de 2 nombres}
			Calculer la moyenne de deux nombres donnés.
		\end{Exercice}
		
		\begin{Exercice}{Surface d’un triangle}
			Calculer la surface d’un triangle connaissant sa base et sa hauteur.
		\end{Exercice}
	
		\begin{Exercice}{Périmètre d’un cercle}
			Calculer le périmètre d’un cercle dont on donne le rayon. 
		\end{Exercice}
	
		\begin{Exercice}{Surface d’un cercle}
			Calculer la surface d’un cercle dont on donne le rayon. 
		\end{Exercice}
	
		\begin{Exercice}{TVA}
			Si on donne un prix hors TVA, il faut lui ajouter 21\% 
			pour obtenir le prix TTC. Écrire un algorithme qui permet 
			de passer du prix HTVA au prix TTC.
		\end{Exercice}
	
		\begin{Exercice}{Les intérêts}
			Calculer les intérêts reçus après 1 an pour un montant placé en 
			banque à du 2\% d’intérêt.
		\end{Exercice}
	
		\begin{Exercice}{Placement}
			Étant donné le montant d’un capital placé (en \texteuro) 
			et le taux d’intérêt annuel (en \%), 
			calculer la nouvelle valeur de ce capital après un an.
		\end{Exercice}
	
		\begin{Exercice}{Prix TTC}
			Étant donné le prix unitaire d’un produit
			(hors TVA), le taux de TVA (en \%) et la quantité de produit vendue à
			un client, calculer le prix total à payer par ce client.
		\end{Exercice}
	
		\begin{Exercice}{Durée de trajet}
			Étant donné la vitesse moyenne en \textbf{m/s}
			d’un véhicule et la distance parcourue en \textbf{km} par ce véhicule,
			calculer la durée en secondes du trajet de ce véhicule.
		\end{Exercice}
	
		\begin{Exercice}{Allure et vitesse}
			L’allure d’un coureur est le temps qu’il met pour parcourir 1~km
			(par exemple, $4'37''$).
			On voudrait calculer sa vitesse (en km/h) à partir de son allure.
			Par exemple, la vitesse d’un coureur ayant une allure de
			$4'37''$ est de $13$~km/h. 
		\end{Exercice}
	
		\begin{Exercice}{Somme des chiffres}
			Calculer la somme des chiffres
			d’un nombre entier de 3 chiffres.
		\end{Exercice}
	
		\begin{Exercice}{Conversion HMS en secondes}
			Étant donné un moment dans la journée donné
			par trois nombres, à savoir, heure, minute et seconde, calculer le
			nombre de secondes écoulées depuis minuit.
		\end{Exercice}
	
		\begin{Exercice}{Conversion secondes en heures}
			Étant donné un temps écoulé depuis minuit.
			Si on devait exprimer ce temps sous la forme
			habituelle (heure, minute et seconde),
			que vaudrait la partie "heure".
	
			Ex~:~10000 secondes donnera 2 heures.
		\end{Exercice}
	
		\begin{Exercice}{Conversion secondes en minutes}
			Étant donné un temps écoulé depuis minuit.
			Si on devait exprimer ce temps sous la forme
			habituelle (heure, minute et seconde),
			que vaudrait la partie "minute".
	
			Ex~:~10000 secondes donnera 46 minutes.
		\end{Exercice}
	
		\begin{Exercice}{Conversion secondes en secondes}
			Étant donné un temps écoulé depuis minuit.
			Si on devait exprimer ce temps sous la forme
			habituelle (heure, minute et seconde),
			que vaudrait la partie "seconde".
	
			Ex~:~10000 secondes donnera 40 secondes.
		\end{Exercice}	
	
		\begin{Exercice}{Cote moyenne}
			Étant donné les résultats (cote entière sur
			20) de trois examens passés par un étudiant (exprimés par six nombres,
			à savoir, la cote et la pondération de chaque examen), calculer 
			la moyenne globale exprimée en pourcentage.
		\end{Exercice}


%-------------------------------------------
\section{Premiers algorithmes et programmes, exercices}
\label{prem-ex-simple}
%-------------------------------------------

			\begin{Exercice}{Moyenne de 2 nombres}
				\marginicon{java}
				Calculer la moyenne de deux nombres donnés.
			\end{Exercice}
			
			\begin{Exercice}{Surface d’un triangle}
				\marginicon{java}
				Calculer la surface d’un triangle 
				connaissant sa base et sa hauteur.
			\end{Exercice}
		
			\begin{Exercice}{Périmètre d’un cercle}
				\marginicon{java}
				Calculer le périmètre d’un cercle dont on donne le rayon. 
			\end{Exercice}
		
			\begin{Exercice}{Surface d’un cercle}
				\marginicon{java}
				Calculer la surface d’un cercle dont on donne le rayon. 
			\end{Exercice}
		
			\begin{Exercice}{TVA}
				\marginicon{java}
				Si on donne un prix hors TVA, il faut lui ajouter 21\% 
				pour obtenir le prix TTC. Écrire un algorithme qui permet 
				de passer du prix HTVA au prix TTC.
			\end{Exercice}
		
			\begin{Exercice}{Les intérêts}
				\marginicon{java}
				Calculer les intérêts reçus après 1 an 
				pour un montant placé en banque à du 2\% d’intérêt.
			\end{Exercice}
		
			\begin{Exercice}{Placement}
				\marginicon{java}
				Étant donné le montant d’un capital placé (en \texteuro) 
				et le taux d’intérêt annuel (en \%), calculer la
				nouvelle valeur de ce capital après un an.
			\end{Exercice}
		
			\begin{Exercice}{Conversion HMS en secondes}
				\marginicon{java}
				Étant donné un moment dans la journée donné
				par trois nombres, à savoir, heure, minute et seconde, calculer le
				nombre de secondes écoulées depuis minuit.
			\end{Exercice}
	
			\begin{Exercice}{Prix TTC}
				\marginicon{java}
				Étant donné le prix unitaire d’un produit
				(hors TVA), le taux de TVA (en \%) 
				et la quantité de produit vendue à un client, 
				calculer le prix total à payer par ce client.
			\end{Exercice}


\section{Tracer un algorithme}


			\begin{Exercice}{Tracer des bouts de code}
				Suivez l’évolution des variables pour les bouts
				d’algorithmes donnés.
	
				\begin{minipage}{4cm}
				\begin{pseudocode}[1]
					\Decl{a, b, c}{entiers}
					\Let a \Gets 42
					\Let b \Gets 24
					\Let c \Gets a + b
					\Let c \Gets c - 1
					\Let a \Gets 2 * b
					\Let c \Gets c + 1
				\end{pseudocode}
				\end{minipage}
				\quad%
				\begin{minipage}{7cm}
					\begin{tabular}{|>{\centering\arraybackslash}m{1cm}|*{3}{>{\centering\arraybackslash}m{2cm}}|}
					\hline
					\verb_#_ & {a} & {b} & {c}\\
					\hline
					1 & {} & {} & {} \\
					2 & {} & {} & {} \\
					3 & {} & {} & {} \\
					4 & {} & {} & {} \\
					5 & {} & {} & {} \\
					6 & {} & {} & {} \\
					7 & {} & {} & {} \\
					\hline
					\end{tabular}
				\end{minipage}
				
				\bigskip
				\begin{minipage}{4cm}
				\begin{pseudocode}[1]
					\Decl{a, b, c}{entiers}
					\Let a \Gets 2
					\Let b \Gets a$^3$
					\Let c \Gets b - a$^2$
					\Let a \Gets $\sqrt{c}$
					\Let a \Gets a / a
				\end{pseudocode}
				\end{minipage}
				\quad%
				\begin{minipage}{7cm}
					\begin{tabular}{|>{\centering\arraybackslash}m{1cm}|*{3}{>{\centering\arraybackslash}m{2cm}}|}
					\hline
					\verb_#_ & {a} & {b} & {c}\\
					\hline
					1 & {} & {} & {} \\
					2 & {} & {} & {} \\
					3 & {} & {} & {} \\
					4 & {} & {} & {} \\
					5 & {} & {} & {} \\
					6 & {} & {} & {} \\
					\hline
					\end{tabular}
				\end{minipage}	
			\end{Exercice}
		
			\begin{Exercice}{Calcul de vitesse}
				\marginicon{java}
				Soit le problème suivant :
				\og
					Calculer la vitesse (en km/h) d’un véhicule 
					dont on donne la durée du parcours (en secondes) 
					et la distance parcourue (en mètres).
				\fg.
				
				Voici \textit{une} solution : 
				\begin{pseudocode}[1]
				\Algo{vitesseKMH}{\Par{distanceM, duréeS}{réels}}{réel}
					\Decl{distanceKM, duréeH}{réels}
					\Let distanceKM \Gets 1000 * distanceM
					\Let duréeH \Gets 3600 * duréeS
					\Return $\frac{\textrm{distanceKM}}{\textrm{duréeH}}$
				\EndAlgo
				\end{pseudocode}

				L’algorithme, s’il est correct, devrait donner
				une vitesse de 1~km/h pour une distance de 1000~mètres
				et une durée de 3600~secondes.
				Testez cet algorithme avec cet exemple.

				\begin{center}
				\begin{tabular}{|>{\centering\arraybackslash}m{1cm}|*{5}{>{\centering\arraybackslash}m{2cm}}|}
					\hline
						\verb_#_  &  &  & & &  \\			
					\hline
						1 & & & & & \\
						2 & & & & & \\
						3 & & & & & \\
						4 & & & & & \\
						5 & & & & & \\
					\hline
				\end{tabular}
				\end{center}
				
				Si vous trouvez qu’il n’est pas correct,
				voyez ce qu’il faudrait changer pour le corriger.
			\end{Exercice}
		
			\begin{Exercice}{Allure et vitesse}
				\marginicon{java}
				L’allure d’un coureur est le temps qu’il met pour parcourir 1~km
				(par exemple, $4'37''$).
				On voudrait calculer sa vitesse (en km/h) à partir de son allure.
				Par exemple, la vitesse d’un coureur ayant une allure de
				$4'37''$ est de $12,996389892$~km/h. 
			\end{Exercice}
		
			\begin{Exercice}{Cote moyenne}
				\marginicon{java}
				Étant donné les résultats (cote entière sur
				20) de trois examens passés par un étudiant (exprimés par six nombres,
				à savoir, la cote et la pondération de chaque examen), calculer 
				la moyenne globale exprimée en pourcentage.
			\end{Exercice}



			\section{Division entière}
			
			\begin{Exercice}{Calculs}
				Voici quelques petits calculs à compléter
				faisant intervenir la division entière et le reste.
				Par exemple~: "14 DIV 3 = 4 reste 2"
				signifie que 14 DIV 3 = 4 et 14 MOD 3 = 2.
				
				\begin{multicols}{2}
					\begin{itemize}
					\item 11 DIV 3 = \_\_\ reste\ \_\_
					\item 3 DIV 11 = \_\_\ reste\ \_\_
					\item 11 DIV \_\_ = 2\ reste\ 3
					\item \_\_ DIV 3 = 3\ reste\ 1
					\end{itemize}
				\end{multicols}
			\end{Exercice}

			\begin{Exercice}{Les prix ronds}
				Voici un algorithme qui reçoit une somme d’argent exprimée en centimes
				et qui calcule le nombre (entier) de centimes qu’il
				faudrait ajouter à la somme pour tomber sur un prix rond en euros.
				Testez-le avec des valeurs numériques. Est-il correct~?
				
				\begin{pseudocode}
				\Algo{versPrixRond}{\Par{prixCentimes}{entier}}{entier}
					\Return 100 - (prixCentimes MOD 100)
				\EndAlgo
				\end{pseudocode}
				
				\begin{center}
				\begin{tabular}{|c|c|c|c|c|}
				\hline
				test \no & prixCentimes & réponse correcte & valeur retournée & Correct~? \\\hline
				\hline 
				1 & 130 & 70 &  & \\\hline
				2 & 40  & 60 &  & \\\hline
				3 & 99  & 1  &  & \\\hline
				4 & 100 & 0  &  & \\\hline
				\end{tabular}
				\end{center}
				
			\end{Exercice}
			
		%---------------------
		\subsection{Exercices récapitulatifs sur les difficultés de calcul}
		%---------------------
		\label{prem-ex-cplx}
		
			Les exercices qui suivent n’ont pas tous été déjà analysés
			et ils demandent des calculs faisant intervenir
			des divisions entières, des restes et/ou des expressions booléennes.
			Comme d’habitude, écrivez la spécification
			si ça n’a pas encore été fait,
			donnez des exemples, rédigez un algorithme
			et vérifiez-le.
		
			\begin{Exercice}{Nombre multiple de 5}
				\marginicon{java}
				\label{algo:mult5}
				Calculer si un nombre entier donné est un multiple de 5.
				\paragraph{Solution.}
				Dans ce problème,
				il y a une donnée, le nombre à tester.
				La réponse est un booléen
				qui est à vrai si le nombre donné est un multiple de 5.
				\begin{center}
				\flowalgod{nombre (entier)}{multiple5}{booléen}
				\end{center}
				\textbf{Exemples.}
				\begin{multicols}{4}
					\begin{itemize}
					\item \pc{multiple5(4)} donne faux
					\item \pc{multiple5(15)} donne vrai
					\item \pc{multiple5(0)} donne vrai
					\item \pc{multiple5(-10)} donne vrai
					\end{itemize}
				\end{multicols}
				La technique pour vérifier si un nombre est
				un multiple de 5 est de vérifier que le reste
				de la division par 5 donne 0.
				Ce qui donne~:
				\begin{pseudocode}[1]
					\Algo{multiple5}{\Par{nombre}{entier}}{booléen}
						\Return nombre MOD 5 = 0
					\EndAlgo
				\end{pseudocode}
				Vérifions sur nos exemples~:
				\begin{center}
				\begin{tabular}{|c|c|c|c|c|}
				\hline
				test \no & nombre & réponse correcte & valeur retournée & Correct~? \\\hline
				\hline 
				1 & 4   & faux & faux & {\color{ForestGreen}$\checkmark$} \\\hline
				2 & 15  & vrai & vrai & {\color{ForestGreen}$\checkmark$} \\\hline
				3 & 0   & vrai & vrai & {\color{ForestGreen}$\checkmark$} \\\hline
				4 & -10 & vrai & vrai & {\color{ForestGreen}$\checkmark$} \\\hline
				\end{tabular}
				\end{center}
			\end{Exercice}

			\begin{Exercice}{Nombre entier positif se terminant par un 0}
				Calculer si un nombre donné se termine par un 0.
			\end{Exercice}
	
			\begin{Exercice}{Les centaines}
				Calculer la partie \emph{centaine}
				d’un nombre entier positif quelconque.
			\end{Exercice}
	
			\begin{Exercice}{Somme des chiffres}
				Calculer la somme des chiffres
				d’un nombre entier positif inférieur à 1000.
			\end{Exercice}
		
			\begin{Exercice}{Conversion secondes en heures}
				Étant donné un temps écoulé depuis minuit.
				Si on devait exprimer ce temps sous la forme
				habituelle (heure, minute et seconde),
				que vaudrait la partie "heure".
		
				Ex~:~10000 secondes donnera 2 heures.
				
				Aide~: L’heure n’est qu’un nombre exprimé en base 60~!
			\end{Exercice}
		
			\begin{Exercice}{Conversion secondes en minutes}
				Étant donné un temps écoulé depuis minuit.
				Si on devait exprimer ce temps sous la forme
				habituelle (heure, minute et seconde),
				que vaudrait la partie "minute".
		
				Ex~:~10000 secondes donnera 46 minutes.
			\end{Exercice}
		
			\begin{Exercice}{Conversion secondes en secondes}
				Étant donné un temps écoulé depuis minuit.
				Si on devait exprimer ce temps sous la forme
				habituelle (heure, minute et seconde),
				que vaudrait la partie "seconde".
		
				Ex~:~10000 secondes donnera 40 secondes.
			\end{Exercice}	

			\begin{Exercice}{Un double aux dés}
				\marginicon{java}
				Écrire un algorithme qui simule le lancer de deux dés
				et indique s’il y a eu un double 
				(les deux dés montrant une face identique).
			\end{Exercice}
		
			\begin{Exercice}{Année bissextile}
				\marginicon{java}
				\label{ex:bissextile}
				Écrire un algorithme qui vérifie si une année est bissextile. 
				Pour rappel, les années bissextiles sont les années multiples de 4. 
				Font exception, les multiples de 100 
				(sauf les multiples de 400 qui sont bien bissextiles). 
				Ainsi $2012$ et $2400$ sont bissextiles mais pas $2010$ ni $2100$.
			\end{Exercice}
			 		 


			\begin{Exercice}{Conversion en heures-minutes-secondes}
				\marginicon{java}
				Écrire un algorithme qui permet à l’utilisateur
				de donner le nombre de secondes écoulées depuis minuit
				et qui affiche le moment de la journée correspondant
				en heures-minutes-secondes.
				Par exemple, si on est 3726 secondes après minuit
				alors il est 1h2'6''.
			\end{Exercice}


\section{Choix}			


	\begin{Exercice}{Compréhension}
		Tracez cet algorithme avec les valeurs fournies et donnez la valeur de retour.
		\begin{pseudocode}
		\Algo{exerciceA}{a, b~: entiers}{entier}
			\Decl{c}{entier}
			\Let c \Gets 2 * a
			\If{c > b}
				\Let c \Gets c - b
			\EndIf
			\Return c
		\EndAlgo
		\end{pseudocode}		
		\begin{multicols}{2}
		\begin{itemize}
		\item \pc{exerciceA(2, 5)} = \_\_\_
		\item \pc{exerciceA(4, 1)} = \_\_\_
		\end{itemize}
		\end{multicols}	
	\end{Exercice}

	\begin{Exercice}{Simplification d’algorithmes}
		Voici quelques extraits d’algorithmes corrects 
		du point de vue de la syntaxe 
		mais contenant des lourdeurs d’écriture.
		Simplifiez-les.
		
		\begin{pseudocode}
		\If{ok = vrai}
			\Write nombre
		\EndIf
		\end{pseudocode}
	
		\begin{pseudocode}
		\If{ok = faux}
			\Write nombre
		\EndIf
		\end{pseudocode}
	
		\begin{pseudocode}
		\If{ok = vrai OU ok = faux}
			\Write nombre
		\EndIf
		\end{pseudocode}
	
		\begin{pseudocode}
		\If{ok = vrai ET ok = faux}
			\Write nombre
		\EndIf
		\end{pseudocode}
	\end{Exercice}

	\begin{Exercice}{Compréhension}
		Tracez ces algorithmes avec les valeurs fournies et donnez la valeur de retour.
	
		\begin{pseudocode}
		\Algo{exerciceB}{b, a : entiers}{entier}
			\Decl{c}{entier}
			\If{a > b}
				\Let c \Gets a DIV b
			\Else
				\Let c \Gets b MOD a	
			\EndIf
			\Return c
		\EndAlgo
		\end{pseudocode}
	
		\begin{multicols}{2}
		\begin{itemize}
		\item \pc{exerciceB(2, 3)} = \_\_\_
		\item \pc{exerciceB(4, 1)} = \_\_\_
		\end{itemize}
		\end{multicols}
	
		\begin{pseudocode}
		\Algo{exerciceC}{x1, x2~: entiers}{entier}
			\Decl{ok}{booléen}
			\Let ok \Gets x1 > x2
			\If{ok}
				\Let ok \Gets ok ET x1 = 4
			\Else
				\Let ok \Gets ok OU x2 = 3
			\EndIf
			\If{ok}
				\Let x1 \Gets x1 * 1000
			\EndIf
			\Return x1 + x2
		\EndAlgo
		\end{pseudocode}
		
		\medskip
		\begin{multicols}{2}
		\begin{itemize}
		\item \pc{exerciceC(2, 3)} = \_\_\_
		\item \pc{exerciceC(4, 1)} = \_\_\_	
		\end{itemize}
		\end{multicols}	
		
	\end{Exercice}	
	
	\begin{Exercice}{Simplification d’algorithmes}
		Voici quelques extraits d’algorithmes corrects du point de vue de la
		syntaxe mais contenant des lignes inutiles ou des lourdeurs d’écriture.
		Remplacer chacune de ces portions d’algorithme par un minimum
		d’instructions qui auront un effet équivalent.
	
		\begin{minipage}[t]{7cm}
			\begin{pseudocode}
			\If{condition}
				\Let ok \Gets vrai
			\Else
				\Let ok \Gets faux
			\EndIf
			\end{pseudocode}
		\end{minipage}
		\quad
		\begin{minipage}[t]{7cm}		
			\begin{pseudocode}
			\If{a $>$ b}
				\Let ok \Gets faux
			\Else
				\If{a $\leq$ b}
					\Let ok \Gets vrai
				\EndIf
			\EndIf
			\end{pseudocode}
		\end{minipage}
	\end{Exercice}

	\begin{Exercice}{Maximum de 2 nombres}
		\marginicon{java}
		Écrire un algorithme qui, étant donné deux nombres quelconques,
		recherche et retourne le plus grand des deux. Attention~! On ne veut
		pas savoir si c’est le premier ou le deuxième qui est
		le plus grand mais bien quelle est cette plus grande valeur. Le
		problème est donc bien défini même si les deux nombres sont
		identiques.
		
		\textbf{Solution.}
		Une solution complète est disponible dans la fiche \vref{fiche:max2nb}.
	\end{Exercice}
	
	\begin{Exercice}{Calcul de salaire}
		Dans une entreprise, 
		une retenue spéciale de 15\% est pratiquée 
		sur la partie du salaire mensuel qui dépasse 1200~\texteuro. 
		Écrire un algorithme qui calcule le salaire net à partir du salaire brut. 
		En quoi l’utilisation de constantes convient-elle pour améliorer cet algorithme~?
	\end{Exercice}

	\begin{Exercice}{Fonction de Syracuse}
		\marginicon{java}
		Écrire un algorithme qui, étant donné un entier $n$ quelconque,
		retourne le résultat de la fonction
		$f(n)=
			\left\{
			\begin{array}{rl}
				n/2 & si \ n \ est\ pair\\
				3n+1 & si \ n \ est \ impair
			\end{array}
			\right.$
	\end{Exercice}

	\begin{Exercice}{Tarif réduit ou pas}
		Dans une salle de cinéma,
		le tarif plein pour une place est de 8\texteuro{}.
		Les personnes ayant droit au tarif réduit payent 7\texteuro{}.
		Écrire un algorithme qui reçoit un booléen
		indiquant si la personne peut bénéficier du tarif réduit
		et qui retourne le prix à payer.
	\end{Exercice}
		
	\begin{Exercice}{Maximum de 3 nombres}
		\marginicon{java}
		Écrire un algorithme qui, étant donné trois nombres quelconques,
		recherche et retourne le plus grand des trois.
	\end{Exercice}
	
	\begin{Exercice}{Le signe}
		\marginicon{java}
		Écrire un algorithme qui \textbf{affiche} un message indiquant
		si un entier est strictement négatif, nul ou strictement
		positif.
	\end{Exercice}

	\begin{Exercice}{Le type de triangle}
		Écrire un algorithme qui indique si un triangle
		dont on donne les longueurs de ces 3 cotés est~:
		équilatéral (tous égaux), isocèle (2 égaux)
		ou quelconque.
	\end{Exercice}

	\begin{Exercice}{Dés identiques}
		Écrire un algorithme qui lance trois dés
		et indique si on a obtenu 3 dés de valeur identique,
		2 ou aucun.
	\end{Exercice}

	\begin{Exercice}{Grade}
		\marginicon{java}
		Écrire un algorithme qui retourne le grade d’un étudiant 
		suivant la moyenne qu’il a obtenue.
		
		Un étudiant ayant obtenu 
		\begin{itemize}
			\item moins de 50\% n’a pas réussi~;
			\item de 50\% inclus à 60\% exclu a réussi~;
			\item de 60\% inclus à 70\% exclu a une satisfaction~;
			\item de 70\% inclus à 80\% exclu a une distinction~;
			\item de 80\% inclus à 90\% exclu a une grande distinction~;
			\item de 90\% inclus à 100\% inclus a la plus grande distinction.
		\end{itemize}
	\end{Exercice}


	\begin{Exercice}{Numéro du jour}
		\marginicon{java}
		Écrire un algorithme qui retourne le numéro du jour de la semaine
		reçu en paramètre (1 pour "lundi", 2 pour "mardi"\dots).
	\end{Exercice}

	\begin{Exercice}{Tirer une carte}
		\marginicon{java}
		Écrire un algorithme qui affiche l’intitulé d’une carte
		tirée au hasard dans un paquet de 52 cartes.
		Par exemple, "As de cœur", "3 de pique", "Valet de carreau"
		ou encore "Roi de trèfle".
		
		\textbf{Remarque}. Il est plus facile de déterminer séparément
		chacune des deux caractéristiques de la carte : couleur et valeur.
	\end{Exercice}
	
	\begin{Exercice}{Nombre de jours dans un mois}
		Écrire un algorithme qui retourne le nombre de jours dans un mois. 
		Le mois est lu sous forme d’un entier (1 pour janvier\dots).
		On considère dans cet exercice que le mois de février
		comprend toujours 28 jours.
	\end{Exercice}
		
	\begin{Exercice}{Réussir DEV1}
		\marginicon{java}
		Pour réussir l’UE (unité d’enseignement) DEV1,
		il faut que la cote attribuée à cette UE 
		soit supérieure ou égale à 50\%.
		Cette cote tient compte de votre examen intégré
		et de vos interrogations.
		Écrire un algorithme 
		qui reçoit la cote finale (sur 100)
		d’un étudiant pour l’UE DEV1
		et qui indique si l’étudiant a réussi cette UE.
	\end{Exercice}	
		
\begin{comment}
	\begin{Exercice}{Réussir GEN1}
		\marginicon{java}
		\label{algo:réussirGEN1}
		l'UE (Unité d'enseignement) GEN1 est composée de trois AA (activité d’apprentissage) :
		Mathématique, Communication anglophone et Comptabilité%
		\footnote{%
			Sans parler de Méthodologie qui ne donne pas lieu à une évaluation.
		}.
		Pour réussir cette unité d’enseignement,
		il faut que la cote attribuée à chaque AA soit supérieure ou égale à 50\%.
		Si c'est le cas, la cote attribuée à l'UE est une moyenne \textbf{pondérée}
		des trois cotes d'AA 
		(avec la pondération 6 pour Mathématique et 2 pour les autres AA).
		
		Écrire un algorithme qui reçoit les 3 cotes (sur 20) d’AA d’un étudiant
		pour l’UE GEN1 et qui \textbf{affiche} un message
		indiquant si l’étudiant a réussi ou pas cette UE.
		S’il a réussi, l’algorithme affiche également la cote d'UE (sur 20).
	\end{Exercice}		
\end{comment}

	\begin{Exercice}{La fourchette}
		\marginicon{java}
		Écrire un algorithme qui, étant donné trois nombres, 
		retourne vrai si le premier des trois 
		appartient à l’intervalle donné par le plus petit et le plus grand 
		des deux autres (bornes exclues) et faux sinon. 
		Qu’est-ce qui change si on inclut les bornes~?
	\end{Exercice}

	\begin{Exercice}{Le prix des photocopies}
		\marginicon{java}
		Un magasin de photocopies facture 0,10 \texteuro{} 
		les dix premières photocopies, 
		0,09 \texteuro{} les vingt suivantes 
		et 0,08 \texteuro{} au-delà. 
		Écrivez un algorithme 
		qui reçoit le nombre de photocopies effectuées 
		et qui affiche la facture correspondante.
	\end{Exercice}

	\begin{Exercice}{Le stationnement alternatif}
		\marginicon{java}
		Dans une rue où se pratique le stationnement alternatif, 
		du 1 au 15 du mois, on se gare du côté des maisons ayant un numéro impair, 
		et le reste du mois, on se gare de l’autre côté. 
		Écrire un algorithme qui, sur base de la date du jour et du numéro de maison
		devant laquelle vous vous êtes arrêté, 
		retourne vrai si vous êtes bien stationné et faux sinon.
	\end{Exercice}

