%----------------------------------------------
\section{Spécifier le problème, exercices}
%----------------------------------------------
	
		Pour les exercices suivants, 
		nous vous demandons d’imiter la démarche décrite dans ce chapitre, 
		à savoir~:
		\begin{itemize}
			\item Déterminer quelles sont les données~;
				leur donner un nom et un type.
			\item Déterminer quel est le type du résultat.
			\item Déterminer un nom pertinent pour l’algorithme.
			\item Fournir un résumé graphique.
			\item Donner des exemples.
		\end{itemize}
	
		\begin{Exercice}{Somme de 2 nombres}
			Calculer la somme de deux nombres donnés.
			\paragraph{Solution.}%
			\footnote{%
				Nous allons de temps en temps 
				fournir des solutions.
				En algorithmique,
				il y a souvent \textbf{plus qu’une} solution possible.
				Ce n’est donc pas parce que vous avez trouvé autre chose
				que c’est mauvais.
				Mais il peut y avoir des solutions \textbf{meilleures}
				que d’autres; 
				n’hésitez jamais à montrer la vôtre
				à votre professeur pour avoir son avis.
			}
			Il y a ici clairement 2 données.
			Comme elles n’ont pas de rôle précis,
			on peut les appeler simplement \lda{nombre1}
			et \lda{nombre2}
			(\lda{nb1} et \lda{nb2} sont aussi de bons choix).
			L’énoncé ne dit pas si les nombres sont entiers ou pas;
			restons le plus général possible en prenant des réels.
			Le résultat sera de même type que les données.
			Le nom de l’algorithme pourrait être simplement \lda{somme}.
			Ce qui donne~:
			\begin{center}
				\flowalgodd{nombre1 (réel)}{nombre2 (réel)}{somme}{réel}
			\end{center}			 
			Et voici quelques exemples numériques~:	
				\lda{somme(3, 2)} donne $5$      \quad
				\lda{somme(-3, 2)} donne $-1$    \quad
				\lda{somme(3, 2.5)} donne $5.5$  \quad
				\lda{somme(-2.5, 2.5)} donne $0$.
		\end{Exercice}
	
		\begin{Exercice}{Moyenne de 2 nombres}
			Calculer la moyenne de deux nombres donnés.
		\end{Exercice}
		
		\begin{Exercice}{Surface d’un triangle}
			Calculer la surface d’un triangle connaissant sa base et sa hauteur.
		\end{Exercice}
	
		\begin{Exercice}{Périmètre d’un cercle}
			Calculer le périmètre d’un cercle dont on donne le rayon. 
		\end{Exercice}
	
		\begin{Exercice}{Surface d’un cercle}
			Calculer la surface d’un cercle dont on donne le rayon. 
		\end{Exercice}
	
		\begin{Exercice}{TVA}
			Si on donne un prix hors TVA, il faut lui ajouter 21\% 
			pour obtenir le prix TTC. Écrire un algorithme qui permet 
			de passer du prix HTVA au prix TTC.
		\end{Exercice}
	
		\begin{Exercice}{Les intérêts}
			Calculer les intérêts reçus après 1 an pour un montant placé en 
			banque à du 2\% d’intérêt.
		\end{Exercice}
	
		\begin{Exercice}{Placement}
			Étant donné le montant d’un capital placé (en \texteuro) 
			et le taux d’intérêt annuel (en \%), 
			calculer la nouvelle valeur de ce capital après un an.
		\end{Exercice}
	
		\begin{Exercice}{Prix TTC}
			Étant donné le prix unitaire d’un produit
			(hors TVA), le taux de TVA (en \%) et la quantité de produit vendue à
			un client, calculer le prix total à payer par ce client.
		\end{Exercice}
	
		\begin{Exercice}{Durée de trajet}
			Étant donné la vitesse moyenne en \textbf{m/s}
			d’un véhicule et la distance parcourue en \textbf{km} par ce véhicule,
			calculer la durée en secondes du trajet de ce véhicule.
		\end{Exercice}
	
		\begin{Exercice}{Allure et vitesse}
			L’allure d’un coureur est le temps qu’il met pour parcourir 1~km
			(par exemple, $4'37''$).
			On voudrait calculer sa vitesse (en km/h) à partir de son allure.
			Par exemple, la vitesse d’un coureur ayant une allure de
			$4'37''$ est de $13$~km/h. 
		\end{Exercice}
	
		\begin{Exercice}{Somme des chiffres}
			Calculer la somme des chiffres
			d’un nombre entier de 3 chiffres.
		\end{Exercice}
	
		\begin{Exercice}{Conversion HMS en secondes}
			Étant donné un moment dans la journée donné
			par trois nombres, à savoir, heure, minute et seconde, calculer le
			nombre de secondes écoulées depuis minuit.
		\end{Exercice}
	
		\begin{Exercice}{Conversion secondes en heures}
			Étant donné un temps écoulé depuis minuit.
			Si on devait exprimer ce temps sous la forme
			habituelle (heure, minute et seconde),
			que vaudrait la partie "heure".
	
			Ex~:~10000 secondes donnera 2 heures.
		\end{Exercice}
	
		\begin{Exercice}{Conversion secondes en minutes}
			Étant donné un temps écoulé depuis minuit.
			Si on devait exprimer ce temps sous la forme
			habituelle (heure, minute et seconde),
			que vaudrait la partie "minute".
	
			Ex~:~10000 secondes donnera 46 minutes.
		\end{Exercice}
	
		\begin{Exercice}{Conversion secondes en secondes}
			Étant donné un temps écoulé depuis minuit.
			Si on devait exprimer ce temps sous la forme
			habituelle (heure, minute et seconde),
			que vaudrait la partie "seconde".
	
			Ex~:~10000 secondes donnera 40 secondes.
		\end{Exercice}	
	
		\begin{Exercice}{Cote moyenne}
			Étant donné les résultats (cote entière sur
			20) de trois examens passés par un étudiant (exprimés par six nombres,
			à savoir, la cote et la pondération de chaque examen), calculer 
			la moyenne globale exprimée en pourcentage.
		\end{Exercice}
