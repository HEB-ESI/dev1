
\section{Exercices~: structures et classes}

Dans ces exercices nous supposons que vous avez au préalable écrit les
structures et classes \pc{Moment}, \pc{Point}… telles que décrites dans le
chapitre \ref{structures-classes} p~\pageref{structures-classes}.

\begin{Exercice}{Conversion moment-secondes}
	Écrire un module qui reçoit en paramètre un
	moment d’une journée et qui retourne le nombre de secondes écoulées
	depuis minuit jusqu’à ce moment.
\end{Exercice}

\begin{Exercice}{Conversion secondes-moment}
	Écrire un module qui reçoit en paramètre un
	nombre de secondes écoulées dans une journée à partir de minuit et qui
	retourne le moment correspondant de la journée.
\end{Exercice}

\begin{Exercice}{Temps écoulé entre 2 moments}
	Écrire un module qui reçoit en paramètres deux
	moments d’une journée et qui retourne le nombre de secondes séparant
	ces deux moments.
\end{Exercice}

\begin{Exercice}{Milieu de 2 points}
	Écrire un module recevant deux points du plan 
	et qui retourne le point situé au milieu des deux.
\end{Exercice}

\begin{Exercice}{Distance entre 2 points}
	Écrire un module recevant les coordonnées de
	deux points distincts du plan et qui retourne
	la distance entre ces deux points.
\end{Exercice}

\begin{Exercice}{Un cercle}
	Définir un type \textbf{Cercle} pouvant décrire de façon
	commode un cercle quelconque dans un espace à deux dimensions. 	
	Écrire ensuite

	\begin{enumerate}[label=\alph*)]
		\item {
			un module calculant la surface du cercle reçu en paramètre~;}
		\item {
				un module recevant 2 points en paramètre et retournant le cercle dont le
			diamètre est le segment reliant ces 2 points~;}
		\item {
			un module qui détermine si un point donné est dans un cercle~;}
		\item {
				un module qui indique si 2 cercles ont une intersection.
			}
	\end{enumerate}
\end{Exercice}


\begin{Exercice}{Un rectangle}

	Définir un type \textbf{Rectangle} pouvant décrire de façon
	commode un rectangle dans un espace à deux dimensions et dont les côtés
	sont parallèles aux axes des coordonnées. 	
	Écrire ensuite

	\begin{enumerate}[label=\alph*)]
		\item {
			un module calculant le périmètre d’un rectangle reçu en paramètre~;}
		\item {
			un module calculant la surface d’un rectangle reçu en paramètre~;}
		\item {
				un module recevant en paramètre un rectangle R et les coordonnées
				d’un point P, et renvoyant 
				\textbf{vrai} si et seulement si le point P est à
			l’intérieur du rectangle R~;}
			%\item {
				%un module recevant en paramètre un rectangle R et les coordonnées
				%d’un point P, et renvoyant 
				%\textbf{vrai} si et seulement si le point P est sur le bord du
				%rectangle R~;}
			%\item {
				%un module recevant en paramètre deux rectangles et renvoyant la valeur
				%booléenne \textbf{vrai} si et seulement si ces deux rectangles ont une
				%intersection.}
	\end{enumerate}
\end{Exercice}

\begin{Exercice}{Validation de date}
	Écrire un algorithme qui valide une date reçue en paramètre 
	sous forme d’une structure.
\end{Exercice}


