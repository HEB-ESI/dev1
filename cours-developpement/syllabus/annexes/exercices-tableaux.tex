\clearpage
\section{Exercices~: tableaux}
\bigskip 


	\begin{Exercice}{Déclarer et initialiser}
		Écrire un algorithme qui déclare un tableau de 100 chaines
		et met votre nom dans la 3\ieme{} case du tableau.
	\end{Exercice}

	\begin{Exercice}{Initialiser un tableau à son indice}
		Écrire un algorithme qui déclare un tableau de 100 entiers
		et initialise chaque élément à la valeur de son indice.
		Ainsi, la case numéro $i$ contiendra la valeur $i$.
	\end{Exercice}

	\begin{Exercice}{Initialiser un tableau aux valeurs de 1 à 100}
		Écrire un algorithme qui déclare un tableau de 100 entiers
		et y met les nombres de 1 à 100.
	\end{Exercice}

	\begin{Exercice}{Compréhension}
		Expliquez la différence entre \pc{tab[i] \Gets tab[i+1]}
		et \pc{tab[i] \Gets tab[i]+1}.
	\end{Exercice}



	\begin{Exercice}{Trouver les entêtes}
		Écrire les entêtes (et uniquement les entêtes)
		des algorithmes qui résolvent les problèmes suivants~:
		\begin{enumerate}[label=\alph*)]
			\item
				Écrire un algorithme qui 
				inverse le signe de tous les éléments négatifs dans un tableau d’entiers.
			\item
				Écrire un algorithme qui
				donne le nombre d’éléments négatifs dans un tableau d’entiers.
			\item
				Écrire un algorithme qui
				détermine si un tableau d’entiers contient au moins un nombre négatif.
			\item
				Écrire un algorithme qui
				détermine si un tableau de chaines contient
				une chaine donnée en paramètre.
			\item
				Écrire un algorithme qui
				détermine si un tableau de chaines contient
				au moins deux occurrences de la même chaine,
				quelle qu’elle soit.
			\item
				Écrire un algorithme qui 
				retourne un tableau donnant les $n$ premiers nombres premiers,
				où $n$ est un paramètre de l’algorithme.
			\item
				Écrire un algorithme qui 
				reçoit un tableau d’entiers
				et retourne un tableau de booléens de la même taille
				où la case $i$ indique si oui ou non
				le nombre reçu dans la case $i$ est strictement positif.
		\end{enumerate}
	\end{Exercice}


	\begin{Exercice}{Inverser le signe des éléments}
		Écrire un algorithme qui 
		inverse le signe de tous les éléments négatifs dans un tableau d’entiers.
	\end{Exercice}

	\begin{Exercice}{Somme}

		Écrire un algorithme qui reçoit en paramètre le tableau \pc{integers}
		de $n$ entiers et qui retourne la somme de ses éléments.
	
	\end{Exercice}

	\begin{Exercice}{Nombre d’éléments d’un tableau}
		
		Écrire un algorithme qui reçoit en paramètre le tableau \pc{doubles} de
		$n$ réels et qui retourne le nombre d’éléments du tableau.
	
	\end{Exercice}

	\begin{Exercice}{Compter les éléments négatifs}
		Écrire un algorithme qui
		donne le nombre d’éléments négatifs dans un tableau d’entiers.
	\end{Exercice}


	\begin{Exercice}{Y a-t-il une copie valant 20/20~?}
		\begin{enumerate}
			
			\item Écrire un algorithme qui reçoit en paramètre le tableau
				\pc{cotes} de $n$ entiers représentant les cotes des étudiants
				et qui retourne un booléen indiquant s’il contient \textbf{au
				moins} une fois la valeur 20. 

			\item Écrire un algorithme qui reçoit en paramètre le tableau
				\pc{cotes} de $n$ entiers représentant les cotes des étudiants
				et qui retourne un booléen indiquant s’il contient
				\textbf{exactement} une fois la valeur 20.  
		
		\end{enumerate}
	
	\end{Exercice}

	\begin{Exercice}{Lancers de deux dés}

		Écrire un algorithme qui lance $n$ fois deux dés et compte le nombre de
		fois que chaque somme apparait. Cet algorithme retourne un tableau
		d'entiers\,; un élément par somme. 

	\end{Exercice}

	\begin{Exercice}{Nombre de jours dans un mois}

		Écrire un algorithme qui reçoit un numéro de mois (de 1 à 12)
		ainsi qu’une année et donne le nombre de jours dans ce mois
		(en tenant compte des années bissextiles).
		N’hésitez pas à réutiliser des algorithmes déjà écrits.
	\end{Exercice}



	\begin{Exercice}{Inscription / désinscription}

		\begin{enumerate}

			\item Comment modifier l’algorithme d’inscription (cfr.
				\ref{subscribe} p\,\pageref{subscribe}) pour s’assurer qu’un
				étudiant ne s’inscrive pas deux fois~?

			\item Comment modifier l’algorithme d’inscription pour refuser une
				inscription si le nombre maximal de participants est atteint en
				supposant que ce maximum est égal à la taille physique du
				tableau~?

			\item Que se passerait-il avec l’algorithme de désinscription tel
				qu’il est si on demande à désinscrire un étudiant non inscrit~?
				Que suggérez-vous comme changement~?

		\end{enumerate}
	
	\end{Exercice}


	\begin{Exercice}{Calcul de complexités}
		Quelle est la complexité d’un algorithme qui~:		
		\begin{enumerate}[label=\alph*)]
			\item 
				recherche le maximum d’un tableau de $n$ éléments~?
			\item 
				remplace par 0 toutes les occurrences du maximum 
				d’un tableau de $n$ éléments~?
			\item 
				vérifie si un tableau contient deux éléments égaux~?
				%\item 
				%	trie par recherche des minima successifs~?
			\item 
				vérifie si les éléments d’un tableau forment un palindrome~?
			\item
				cherche un élément dans un tableau 
				en essayant des cases au hasard jusqu’à le trouver~?
		\end{enumerate}
	\end{Exercice}


	\begin{Exercice}{Réflexion}
		L’algorithme de recherche dichotomique 
		est-il toujours à préférer à
		l’algorithme de recherche linéaire~?
	\end{Exercice}

\bigskip
\bigskip
\begin{Emphase}
	\paragraph{Remarque}
	Voici quelques exercices
	qui reprennent les différentes notions vues sur les tableaux.
	Pour chacun d’entre-eux~:
	\begin{itemize}
	\item
		demandez-vous quel est le problème le plus proche déjà
		rencontré et voyez comment adapter la solution~;
	\item
		indiquez la complexité de votre solution~;
	\item
		décomposez votre solution en plusieurs algorithmes 
		si ça peut améliorer la lisibilité~;
	\item
		testez votre solution dans le cas général
		et dans les éventuels cas particuliers.
	\end{itemize}
\end{Emphase}
\bigskip
	
	\begin{Exercice}{Renverser un tableau}
		
		Écrire un algorithme qui reçoit en paramètre 
		le tableau \pc{tabCar} de $n$ caractères, 
		et qui «~renverse~» ce tableau, 
		c’est-à-dire qui permute le premier élément avec le dernier, 
		le deuxième élément avec l’avant-dernier et ainsi de suite.
	\end{Exercice}
	
	\begin{Exercice}{Tableau symétrique~?}
		Écrire un algorithme qui reçoit en paramètre 
		le tableau \pc{tabChaines} de $n$ chaines 
		et qui vérifie si ce tableau est symétrique, 
		c’est-à-dire si le premier élément est identique au dernier, 
		le deuxième à l’avant-dernier et ainsi de suite.
	\end{Exercice}
		
	\begin{Exercice}{Maximum}
		
		Écrire un algorithme qui reçoit en paramètre le tableau
		\pc{tabEnt} de $n$ entiers et qui
		retourne la plus \textbf{grande} valeur de ce tableau.
	\end{Exercice}
		
	\begin{Exercice}{Minimum}
		
		Écrire un algorithme qui reçoit en paramètre le tableau
		\pc{tabEnt} de $n$ entiers et qui
		retourne la plus \textbf{petite} valeur de ce tableau. Idem pour le minimum.
	\end{Exercice}
	
	\begin{Exercice}{Indice du maximum/minimum}
		\label{ex:indiceminmax}
		
		Écrire un algorithme qui reçoit en paramètre 
		le tableau \pc{tabEnt} de $n$ entiers 
		et qui retourne l’indice de l’élément contenant 
		la plus grande valeur de ce tableau. 
		En cas d’ex-æquo, c’est l’indice le plus petit qui sera renvoyé.
		
		Que faut-il changer pour renvoyer l’indice le plus grand~?
		Et pour retourner l’indice du minimum~? 
		Réécrire l’algorithme de l’exercice précédent en utilisant celui-ci.
	\end{Exercice}
		
	\begin{Exercice}{Tableau ordonné~?}
		
		Écrire un algorithme qui reçoit en paramètre 
		le tableau \pc{valeurs} de $n$ entiers 
		et qui vérifie si ce tableau est ordonné 
		(strictement) croissant sur les valeurs. 
		L’algorithme retournera \pc{vrai} si le tableau est ordonné,
		\pc{faux} sinon.
	\end{Exercice}
	
	\begin{Exercice}{Remplir un tableau}
		On souhaite remplir un tableau de 20 éléments avec
		les entiers de 1 à 5, chaque nombre étant répété quatre fois.
		On voudrait deux variantes :
		\begin{enumerate}[label=\alph*)]
		\item
			D'abord une version
			où les nombres identiques sont groupés :
			d'abord tous les 1 puis tous les 2, \dots
			
			\begin{tabular}{|c|c|c|c|c|c|c|c|c|c|c|c|c|c|}
			\hline
			1 & 1 & 1 & 1 & 2 & 2 & 2 & 2 & 3 & 3 & 3 & 3 & 4 & 4 \\
			\hline
			\end{tabular}
			\begin{flushright}
			\par\hspace{3cm}
			\begin{tabular}{|c|c|c|c|c|c|}
			\hline
			4 & 4 & 5 & 5 & 5 & 5\\
			\hline
			\end{tabular}
			\end{flushright}
		\item
			Ensuite une version où on trouve dans le tableau
			les valeurs 1, 2, 3, 4, 5 qui se suivent, quatre fois.
			
			\begin{tabular}{|c|c|c|c|c|c|c|c|c|c|c|c|c|c|}
			\hline
			1 & 2 & 3 & 4 & 5 & 1 & 2 & 3 & 4 & 5 & 1 & 2 & 3 & 4 \\ 
			\hline
			\end{tabular}
			\begin{flushright}
			\par\hspace{3cm}
			\begin{tabular}{|c|c|c|c|c|c|}
			\hline
			5 & 1 & 2 & 3 & 4 & 5 \\
			\hline
			\end{tabular}
			\end{flushright}
		\end{enumerate}
		\smallskip
		\textbf{Remarque} : 
		Il existe de nombreuses façons de résoudre ce problème.
		On peut par exemple utiliser deux boucles imbriquées.
		On peut aussi adapter un algorithme de génération de suites.
	\end{Exercice}
	
	\begin{Exercice}{Positions du minimum}
		
		Écrire un algorithme qui reçoit en paramètre le tableau
		\pc{cotes} de $n$ entiers et qui
		affiche le ou les indice(s) des éléments 
		contenant la valeur minimale du tableau.
	
		\begin{enumerate}[label=\alph*)]
		\item 
			Écrire une première version «~classique~» avec deux parcours de tableau
		\item
			Écrire une deuxième version qui ne parcourt qu’une seule fois 
			\pc{cotes} en
			stockant dans un deuxième tableau (de quelle taille~?)
			les indices du plus petit élément
			rencontrés (ce tableau étant à chaque fois réinitialisé lorsqu’un
			nouveau minimum est rencontré)
		\item
			Écrire une troisième version qui \textbf{retourne} 
			le tableau contenant les indices.
			Écrire également un algorithme 
			qui appelle cette version puis affiche les indices reçus. 
		\end{enumerate}
	\end{Exercice}
	
	\begin{Exercice}{Occurrence des chiffres}
		
		Écrire un algorithme qui reçoit un nombre entier positif ou nul en
		paramètre et qui retourne un tableau de 10 entiers indiquant, pour
		chacun de ses chiffres, le nombre de fois qu’il apparait dans ce
		nombre.  
		
		Ainsi, pour le nombre 10502851125, on retournera le tableau
		[2,3,2,0,0,3,0,0,1,0].
	
	\end{Exercice}
	
	\begin{Exercice}{Les doublons}
		
		Écrire un algorithme qui vérifie 
		si un tableau de chaines
		contient au moins 2 éléments égaux.
	\end{Exercice}	
	\bigskip
	
	\begin{Exercice}{Le crible d’Ératosthène}

		\begin{quote}
			\og{} Le crible d’Ératosthène est un procédé 
			qui permet de trouver tous les nombres premiers inférieurs 
			à un certain entier naturel donné N.
			L’algorithme procède par élimination~: 
			il s’agit de supprimer d’une table des entiers de 2 à N 
			tous les multiples d’un entier. 
			En supprimant tous les multiples, 
			à la fin il ne restera que les entiers qui ne sont multiples d’aucun entier, 
			et qui sont donc les nombres premiers.
			On commence par rayer les multiples de 2, 
			puis à chaque fois on raye les multiples du plus petit entier restant.
			On peut s’arrêter lorsque le carré du plus petit entier restant 
			est supérieur au plus grand entier restant, car dans ce cas, 
			tous les non-premiers ont déjà été rayés précédemment.\fg{}
			(source~: Wikipédia)
		\end{quote}
		
		Le tableau dont il est question peut être un simple tableau
		de booléens; le booléen en position "i" indiquant 
		si le nombre "i" est premier ou pas.

		Écrire un algorithme qui reçoit un entier $n$ et affiche tous les
		entiers premiers de $1$ à $n$\footnote{Tester votre programme avec de
		«~grandes~» valeurs de $n$.}.

	\end{Exercice}
	
	\begin{Exercice}{Mastermind}
		
		\begin{quote}
		
			Dans le jeu du Mastermind, un joueur A doit trouver une combinaison
			de $n$ pions de couleur, choisie et tenue secrète par un autre
			joueur B.  Cette combinaison peut contenir éventuellement des pions
			de même couleur.  À chaque proposition du joueur A, le joueur
			B indique le nombre de pions de la proposition qui sont corrects et
			bien placés et le nombre de pions corrects mais mal placés. 
		
			Les propositions du joueur A, ainsi que la combinaison secrète du
			joueur B sont contenues dans des tableaux de $n$ composantes de
			type chaine.
		
		\end{quote}
		
		Écrire deux algorithmes~:

		\begin{itemize}
			\item l'un qui renvoie le nombre de pions de la bonne couleur et 
				bien placés et\,;
			\item l'autre qui renvoie le nombre de pions de la bonne couleur 
				et mal placés. 
		\end{itemize}

	\end{Exercice}
	
