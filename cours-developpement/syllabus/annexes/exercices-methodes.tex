\clearpage
\section{Exercices~: modules et méthodes}

		\begin{Exercice}{Tracer des algorithmes}
			Indiquer quels nombres sont successivement affichés 
			lors de l’exécution des algorithmes ex1, ex2, ex3 et ex4.

			\begin{java}
public static void ex1(){
	int x, y;
	x = add(3, 4);
	System.out.println(x);
	x = 3;
	y = 5;
	y = add(x, y);
	System.out.println(y);
}

public static int add(int a, int b){
	int sum;
	sum = a + b;
	return sum;
}
			\end{java}

			\begin{java}
public static void ex2(){
	int a, b;
	a = add(3, 4);		// cfr. ci-dessus
	System.out.println(a);
	a = 3;
	b = 5;
	b = sub(b, a);
	System.out.println(b);
}

public static int sub(int a, int b){
	return a - b;
}
			\end{java}

	\end{Exercice}

	\begin{Exercice}{Appels de module}
		Parmi les instructions suivantes (où les variables
		\pc{a}, \pc{b} et \pc{c}
		sont des entiers), lesquelles font correctement appel 
		à l’algorithme d’en-tête suivant~?

		\begin{java}
public static int pgcd(int a, int b){
	// ...
}
		\end{java}

		\begin{itemize}
			\item[$\Box$] \pc{a = pgcd(24, 32)} 
			\item[$\Box$] \pc{a = pgcd(a, 24)}
			\item[$\Box$] \pc{b = 3 * pgcd(a + b, 2 * c) + 120}
			\item[$\Box$] \pc{pgcd(20, 30)}
			\item[$\Box$] \pc{a = pgcd(a, b, c)}
			\item[$\Box$] \pc{a = pgcd(a, b) + pgcd(a, c)}
			\item[$\Box$] \pc{a = pgcd(a, pgcd(a, b))}
			\item[$\Box$] \pc{System.out.println(pgcd(a, b))}
			\item[$\Box$] \pc{pgcd(a, b) = c}
		\end{itemize}

	\end{Exercice}

	\begin{Exercice}{Maximum de 4 nombres}
		Écrivez un algorithme qui calcule le maximum de 4 nombres.
	\end{Exercice}

	\begin{Exercice}{Écart entre 2 durées}
		Étant donné deux durées données chacune par trois
		nombres (heure, minute, seconde),
		écrire un algorithme qui calcule
		le délai écoulé entre ces deux durées en heure(s), minute(s),
		seconde(s) sachant que la deuxième durée donnée 
		est plus petite que la première.
	\end{Exercice}

	\begin{comment}
	\begin{Exercice}{Réussir GEN1}
		Reprenons l’exercice \vref{algo:réussirGEN1}.
		Cette fois-ci on ne veut rien afficher
		mais fournir deux résultats~:
		un booléen indiquant si l’étudiant a réussi ou pas
		et un entier indiquant sa cote (qui n’a de sens que s’il a réussi). 
	\end{Exercice}
	\end{comment}

	\begin{Exercice}{Tirer une carte}

		L’exercice suivant a déjà été résolu. 
		Refaites une solution modulaire.

		Écrire un algorithme qui affiche l’intitulé d’une carte
		tirée au hasard dans un paquet de 52 cartes.
		Par exemple, "As de cœur", "3 de pique", "Valet de carreau"
		ou encore "Roi de trèfle".
	\end{Exercice}

	\begin{Exercice}{Nombre de jours dans un mois}
		Écrire un algorithme qui donne le nombre de jours dans un mois.
		Il reçoit en paramètre le numéro du mois (1 pour janvier\dots)
		ainsi que l’année.
		Pour le mois de février, 
		il faudra répondre 28 ou 29 selon que l’année fournie
		est bissextile ou pas.
		Vous devez réutiliser au maximum ce que vous avez déjà fait
		lors d’exercices précédents (cf. exercice \vref{ex:bissextile}).
	\end{Exercice}

	\begin{Exercice}{Valider une date}
		Écrire un algorithme qui valide 
		une date donnée par trois entiers~:~ l’année, le mois et le jour.
		Vous devez réutiliser au maximum ce que vous avez déjà fait
		lors d’exercices précédents.
	\end{Exercice}

	\begin{Exercice}{Généraliser un algorithme}
		
		Dans l’exercice \vref{algo:mult5}, nous avons écrit un algorithme pour
		tester si un nombre est divisible par 5.  Si on vous demande à présent
		un algorithme pour tester si un nombre est divisible par 3, vous le
		feriez sans peine.  Idem pour tester la divisibilité par 2, 4, 6, 7, 8,
		9\dots{} mais vous vous lasseriez bien vite.

		Écrivez un seul algorithme, plus général, qui résoud tous ces problèmes
		en une seule fois. 
	
	\end{Exercice}


