\clearpage
\section{Exercices~: premiers algorithmes et programmes}
\label{prem-ex-simple}

Dans ces exercices, nous vous proposons d'écrire un algorithme et de le traduire
en un programme Java. Certains exercices ont déjà été travaillés dans la section 
précédente (voir annexe \ref{annexe-specifier}, p.\pageref{annexe-specifier}).

\begin{Exercice}{Moyenne de 2 nombres}	
	Calculer la moyenne de deux nombres donnés.
\end{Exercice}

\begin{Exercice}{Surface d’un triangle}
	Calculer la surface d’un triangle 
	connaissant sa base et sa hauteur.
\end{Exercice}

\begin{Exercice}{Périmètre d’un cercle}
	Calculer le périmètre d’un cercle dont on donne le rayon. 
\end{Exercice}

\begin{Exercice}{Surface d’un cercle}
	Calculer la surface d’un cercle dont on donne le rayon. 
\end{Exercice}

\begin{Exercice}{TVA}
	Si on donne un prix hors TVA, il faut lui ajouter 21\% 
	pour obtenir le prix TTC. Écrire un algorithme qui permet 
	de passer du prix HTVA au prix TTC.
\end{Exercice}

\begin{Exercice}{Les intérêts}	
	Calculer les intérêts reçus après 1 an 
	pour un montant placé en banque à du 2\% d’intérêt.
\end{Exercice}

\begin{Exercice}{Placement}
	Étant donné le montant d’un capital placé (en \texteuro) 
	et le taux d’intérêt annuel (en \%), calculer la
	nouvelle valeur de ce capital après un an.
\end{Exercice}

\begin{Exercice}{Conversion HMS en secondes}
	Étant donné un moment dans la journée donné
	par trois nombres, à savoir, heure, minute et seconde, calculer le
	nombre de secondes écoulées depuis minuit.
\end{Exercice}

\begin{Exercice}{Prix total}
	Étant donné le prix unitaire d’un produit
	(hors TVA), le taux de TVA (en \%) 
	et la quantité de produit vendue à un client, 
	calculer le prix total à payer par ce client.
\end{Exercice}

