%==========================================
%\chapter{Gérer les données dans un tableau}
\chapter{Algorithmes d'insertions et de recherches}
\label{chap:insertion-recherche}
%==========================================

	Souvent lorsque les applications stockent des données. Ces données changent. 
	Il est nécessaire d'en ajouter, de modifier certaines, d'en rechercher, d'en
	supprimer. 

	\minitoc
	\clearpage

	\textbf{Exemple}	Imaginons qu'une école organise un événement libre et
	gratuit pour les étudiants et les étudiantes.  Mais pour y assister, il est
	nécessaire de s’inscrire. Nous voulons fournir ce qu’il faut pour~:
	
	\begin{itemize}
	\item inscrire un ou une étudiant·e~;
	\item vérifier l'inscription~;
	\item désinscrire~;
	\item afficher la liste des personnes inscrites.
	\end{itemize}

	Nous allons, pour l'exemple, stocker les numéros d'étudiants et d'étudiantes
	dans un tableau%
	\footnote{%
		Ce n'est bien sûr pas la bonne manière de faire pour une application 
		réelle pour laquelle nous utiliserions probablement une base de données…
		mais c'est une autre histoire… et un autre cours. 
	}.
	Comme le tableau ne sera généralement pas rempli, nous allons créer un
	tableau «~trop grand~» et gérer sa taille logique.  Nous avons deux
	variables~:
	
	\begin{itemize}
	\item \pc{registered}~: le tableau des numéros d’étudiants
	\item \pc{nRegistered}~: le nombre d’étudiant·es actuellement inscrit·es 
		(la taille logique du tableau)
	\end{itemize}
	
	Nous allons envisager deux cas~:
	\begin{enumerate}
	\item les numéros seront placés dans le tableau sans ordre imposé;
	\item les numéros seront placés dans l’ordre.		
	\end{enumerate}
	
	% ==============================================
	\section{Insertion dans un tableau non trié} 
	% ==============================================
		
		Les valeurs sont stockées dans le tableau 
		dans un ordre quelconque.
		Par exemple~:
		\begin{center}
			\pc{registered} = 
			\smallskip
			\begin{tabular}{*{10}{|>{\centering\arraybackslash}m{8mm}}|}
				\hline
				42010 & 41390 & 43342 & 42424 & ? & ? & ? & ? & \dots \\
				\hline
			\end{tabular}
			\smallskip
			\pc{nRegistered = 4}
		\end{center}

		Il y a pour le moment quatre personnes inscrites dont les numéros sont
		ceux repris dans les quatre premières cases du tableau.
		
		\subsection{Inscription}
		%---------------------------------------------
		
			Pour inscrire un étudiant ou une étudiante, il suffit de l’ajouter
			à la suite dans le tableau.  Par exemple, en partant de la situation
			décrite ci-dessus, l’inscription de l’étudiant numéro 42123 aboutit
			à la situation suivante~:

			\begin{center}
				\pc{registered} = 
				\smallskip
				\begin{tabular}{*{10}{|>{\centering\arraybackslash}m{8mm}}|}
					\hline
					42010 & 41390 & 43342 & 42424 & 42123 & ? & ? & ? & \dots \\
					\hline
				\end{tabular}
				\smallskip
				\pc{nRegistered = 5}
			\end{center}
			
			L’algorithme est assez simple~:
			
			\begin{pseudocode}
				\LComment{Inscrit un étudiant de numéro donné.}
				\Algo{register}{\Par{registered\In\Out}{\Array{}{integers}}, 
					\Par{nRegistered\In\Out}{integer},
					\\\hfill \Par{number\In}{integer}}{}
					\LComment{On peut imaginer vérifier que l’étudiant n’est 
					pas déjà inscrit}
					\LComment{On peut imaginer vérifier qu’il reste 
					de la place}
					\LComment{(c-à-d que le tableau n’est pas plein)}
					\Let registered[nRegistered] \Gets number
					\Let nRegistered \Gets nRegistered + 1
				\EndAlgo
			\end{pseudocode}

			En langage Java, il n'est pas possible de passer \pc{nRegistered} en 
			entrée sortie puisque le passage de paramètre se fait par valeur et 
			que \pc{int} est un type primitif. Il faudra que ce soit en valeur de 
			retour. Nous pourrions écrire~:

			\begin{java}
	public static int register(int[] registered, int nRegistered,
			int number){
		// idem
		registered[nRegistered] = number;
		return nRegistered + 1;
	}
			\end{java}

			Dans ce cas, c'est bien le code appelant qui devra maintenir la 
			valeur de la taille logique à jour. 
		
		\subsection{Vérifier une inscription}
		%---------------------------------------------
	
			Pour vérifier si un étudiant est déjà inscrit,
			il faut le rechercher dans la partie utilisée du tableau.
			Cette recherche se fait via un parcours avec sortie
			prématurée. 
			
			L'algorithme ne va pas se contenter de retourner un booléen
			précisant si le numéro est trouvé mais va retourner l'indice dans le
			tableau de ce numéro, -1 sinon. 
			
			\begin{pseudocode}
				\LComment{Vérifie si un étudiant est inscrit 
				et donne sa position (-1 si non inscrit)}
				\Algo{check}{\Par{registered\In}{\Array{n}{integers}}, 
					\Par{nRegistered\In}{integer}, 
					\\\hfill\Par{number\In}{integer}}{integer}
					\Decl{i}{integer}
					\Let i \Gets 0
					\While{i < nRegistered ET registered[i] $\ne$ number}
						\Let i \Gets i + 1
					\EndWhile
					\If{i < nRegistered}
						\Return i
					\Else
						\Return -1
					\EndIf
				\EndAlgo
			\end{pseudocode}

			La traduction en langage Java ne pose aucune difficulté et est
			laissée en exercice. 
			
			La fiche \vref{fiche:tab-recherche-non-triee} reprend cet algorithme
			dans un cadre plus général.

		\subsection{Désinscription}
		%---------------------------------------------
		
			Pour désinscrire un étudiant, il faut le trouver dans le tableau et
			l’enlever.  Attention, il ne peut pas y avoir de trou dans un
			tableau (il n’y a pas de case vide).  Le plus simple est d’y
			déplacer le dernier élément.  Par exemple, reprenons la situation
			après inscription de l’étudiant numéro 42123.

			\begin{center}
				\pc{registered} = 
				\smallskip
				\begin{tabular}{*{10}{|>{\centering\arraybackslash}m{8mm}}|}
					\hline
					42010 & 41390 & 43342 & 42424 & 42123 & ? & ? & ? & \dots \\
					\hline
				\end{tabular}
				\smallskip
				\pc{nRegistered = 5}
			\end{center}

			La désinscription de l’étudiant numéro 41390 donnerait ce qui suit.
			
			Nous avons volontairement indiqué le 42123 en 5\ieme{} position.  Il
			est toujours là mais ne sera pas considéré par les algorithmes
			puisque cette case est au-delà de la taille logique (donnée par la
			variable \pc{nRegistered}).  

			\begin{center}
				\pc{registered} = 
				\smallskip
				\begin{tabular}{*{10}{|>{\centering\arraybackslash}m{8mm}}|}
					\hline
					42010 & 42123 & 43342 & 42424 & 42123 & ? & ? & ? & \dots \\
					\hline
				\end{tabular}
				\smallskip
				\pc{nRegistered = 4}
			\end{center}
			
			L’algorithme est assez simple à écrire en utilisant la recherche
			écrite juste avant~:
			
			\begin{pseudocode}
				\LComment{Désinscrit l’étudiant donné}
				\Algo{unsubscribe}{
						\Par{registered\In\Out}{\Array{}{integers}}, 
						\\\hfill\Par{nRegistered\In\Out}{integer}, 
						\Par{number\In}{integer}
						}{}
					\Decl{pos}{integer}
					\Let pos \Gets check(registered, nRegistered, number)
					\LComment{On pourrait vérifier et générer une erreur 
					si l’étudiant n’est pas inscrit}
					\Let registered[pos] \Gets registered[nRegistered-1]
					\Let nRegistered \Gets nRegistered - 1					
				\EndAlgo
			\end{pseudocode}
		
			Comme précédement, en langage Java, il n'est pas possible de passer
			\pc{nRegistered} en entrée sortie.  Il faudra que ce soit en valeur
			de retour. Nous pourrions écrire~:

			\begin{java}
	public static int unsubscribe(int[] registered, 
			int nRegistered, int number){
		int pos = check(registered, nRegistered, number);
		registered[pos] = registered[nRegistered - 1];
		return nRegistered - 1;
	}
			\end{java}

		\subsection{Exercices}
		%---------------------------------------------
		
		\begin{enumerate}
			
			\item Éviter la double inscription.

				Comment modifier l’algorithme d’inscription
				pour s’assurer qu’un étudiant ne s’inscrive pas deux fois~?

			\item Limiter le nombre d'inscriptions.

				Comment modifier l’algorithme d’inscription pour refuser une
				inscription si le nombre maximal de participant est atteint en
				supposant que ce maximum est égal à la taille physique du
				tableau~?

			\item Vérifier la désinscription. 

				Que se passerait-il avec l’algorithme
				de désinscription tel qu’il est
				si on demande à désinscrire un étudiant non inscrit~?
				Que suggérez-vous comme changement~?

			\item Optimiser la désinscription.

				Dans l’algorithme de désinscription tel qu’il est,
				voyez-vous un cas où on effectue une opération inutile~?
				Avez-vous une proposition pour l’éviter~?
		
		\end{enumerate}

	% ==============================================
	\section{Insertion dans un tableau trié} 
	% ==============================================

		Imaginons à présent que nous maintenons un ordre
		dans les données du tableau.
		
		En reprenant l’exemple utilisé pour les données non triées,
		nous avons~:
		\begin{center}
			\pc{registered} = 
			\smallskip
			\begin{tabular}{*{10}{|>{\centering\arraybackslash}m{8mm}}|}
				\hline
				41390 & 42010 & 42424 & 43342 & ? & ? & ? & ? & \dots \\
				\hline
			\end{tabular}
			\smallskip
			\pc{nRegistered = 4}
		\end{center}
	
		— Qu’est-ce que ça change au niveau des algorithmes~?  \\
		— Beaucoup~!

		Par exemple, plus question de placer un nouvel inscrit à la suite des
		autres; il faut trouver sa place.
		
		\subsection{Rechercher la position d’une inscription}
		%----------------------------------------------------

			Tous les algorithmes que nous allons voir dans le cadre de données
			triées ont une partie en commun~: la recherche de la position du
			numéro, s’il est présent ou de la position où il aurait dû être s’il
			est absent.  Commençons par cela.
			
			L’algorithme est assez proche de celui de la vérification d’une
			inscription vu précédemment si ce n’est que l'algorithme peut
			s’arrêter dès qu'un numéro plus grand est trouvé. Nous allons
			décomposer l'algorithme en deux parties~: la première pour la
			recherche de la position à laquelle devrait se trouver le numéro et
			la seconde pour vérifier que c'est bien le bon numéro qui se trouve
			à cette position. En effet, lorsque le numéro n'est pas présent dans
			le tableau, c'est un numéro plus grand qui s'y trouve.   

			\begin{pseudocode}
				\LComment{Recherche un étudiant.}
				\LComment{return: la position où a été trouvé l’étudiant}
				\LComment{        ou -1 s'il est pas inscrit}
				\Algo{find}{
						\Par{registered\In}{\Array{}{integers}}, 
						\Par{nRegistered\In}{integer}, 
						\\\hfill\Par{number\In}{integer}
						}{integer}
					\Decl{pos}{integer} \Gets 
						findPosition(registered, nRegistered, number)
					\If{registered[pos] == number}
						\Return pos
					\Else
						\Return -1
					\EndIf
				\EndAlgo
				
				\Empty
				\LComment{Recherche une position}
				\LComment{return: la position a laquelle est }
				\LComment{ou devrait être le numéro}
				\Algo{findPosition}{
						\Par{registered\In}{\Array{}{integers}}, 
						\Par{nRegistered\In}{integer}, 
						\\\hfill\Par{number\In}{integer}
						}{integer}
					\Decl{pos}{integer} \Gets 0
					\While{pos < nRegistered AND registered[pos] < number}
						\Let pos \Gets pos + 1
					\EndWhile
					\Return pos
				\EndAlgo
			\end{pseudocode}
			
			Comprenez-vous pourquoi~:
			\begin{itemize}
			\item
				\pc{registered[nRegistered] < number} contien un '<' dans la 
				condition du tant que et pas un '$\neq$'~?
			\item
				la condition du \pc{if} est composée de deux parties
				et utilise un =~?
			\end{itemize}

			La liste étant triée il faut s'arrêter de chercher dès que la valeur
			n'est plus inférieure à la valeur cherchée. Dès qu'elle est plus
			grande c'est que la valeur \textbf{n'est pas} dans la liste. 

			De même pour la condition du \pc{if}, il faut d'abord tester si la
			recherche n'est pas arrivée à la fin du nombre d'inscrits et
			d'inscrites. Ensuite, il faut voir si le numéro à la position
			trouvée est bien le bon. 

			La traduction en Java est immédiate. Le \pc{if} tel que présenté
			dans l'algorithm peut se réécrire grâce à l'opérateur
			ternaire\index{ternaire} de Java. Profitons en pour le présenter. 

			\begin{grammaire}
				\grammarrule{ConditionalExpression:}
				    \grammarrule{Condition} ? \grammarrule{Expression} : \grammarrule{Expression}
			\end{grammaire}

			Si la condition est vraie, l'expression prend la première valeur et
			si la condition est fausse, ce sera la seconde valeur. Par exemple~:
			\pc{heure < 12 ? ``matin'' : ``soir''} vaut ``soir'' si \pc{heure}
			vaut 15. 
			
			\begin{java}
	/** 
	 * Recherche une inscription dans le tableau.
	 * 
	 * @param registered le tableau d'inscrit·es
	 * @param nRegistered le nombre d'inscrit·es
	 * @param number le numéro à chercher
	 * @return la position dans le tableau, -1 si non présent
	 */
	public static int find(int[] registered, int nRegistered, 
			int number){
		int pos = findPosition(registered, nRegistered, number);
		return registered[pos] == number ? pos : -1;
	}

	/**
	 * Recherche la position que le nombre soit présent ou pas. 
	 * @param registered le tableau d'inscrit·es
	 * @param nRegistered le nombre d'inscrit·es
	 * @param number le numéro à chercher
	 * @return la position à laquelle est ou devrait être le numéro. 
	 */
	public static int findPosition{int[] registered, int nRegistered,
			int number){
		int pos = 0;
		while (pos < nRegistered && registered[pos] < number){
			pos = pos + 1;
		}
		return pos;
	}
			\end{java}

		\subsection{Inscrire un étudiant}
		%---------------------------------------------
				
			L’inscription d’un étudiant se fait en trois étapes~:
			\begin{enumerate}
			\item
				recherche de l’étudiant via l’algorithme vu à l'instant,
				ce qui nous donne la place où le placer;

				Nous ne vérifions pas si l'étudiant ou l'étudiante est déjà 
				présent·e dans le tableau. 

			\item
				libération de la place pour l’y mettre;
				
				Cette opération n’est pas triviale.  Si vous tenez des cartes en
				main, il est facile d’insérer une nouvelle carte à n’importe
				quelle position.  Avec un tableau, il en va tout autrement~;
				pour insérer un élément à un endroit donné, il faut décaler tous
				les suivants d’une position sur la droite.
			
			\item
				placer l’élément à la position libérée.
			\end{enumerate} 
			
			Ce qui nous donne~:
			
			\begin{pseudocode}
				\LComment{Inscrit un étudiant de numéro donné.}
				\Algo{subscribe}{
						\Par{registered\In\Out}{\Array{}{integers}}, 
						\Par{nRegistered\In\Out}{integer}, 
						\\\hfill\Par{number\In}{integer}
						}{}
					\Decl{pos}{integer}
					\Stmt pos \Gets findPosition(registered, nRegistered, number)
					\Stmt shiftRight(registered, pos, nRegistered-1)
					\Let registered[pos] \Gets number
					\Let nRegistered \Gets nRegistered + 1
				\EndAlgo
				\Empty
				\LComment{Décale d’une position à droite les éléments 
					entre la position début et fin}
				\Algo{shiftRight}{
						\Par{tab\In\Out}{\Array{}{integers}}, 
						\Par{begin\In}{integer}, 
						\Par{end\In}{integer}
						}{}
					\For[-1]{i}{end}{begin}
						\Let tab[i+1] \Gets tab[i]
					\EndFor
				\EndAlgo
			\end{pseudocode}
	
			En langage Java, la traduction est immédiate~:

			\begin{java}
/**
 * Inscrit un ou une étudiant·e.
 * 
 * @param registered le tableau d'inscrit·es
 * @param nRegistered le nombre d'inscrit·es
 * @param number le numéro à chercher
 */
public static void subscribe(int[] registred, int nRegistered,
				int number){
	int pos = findPosition(registered, nRegistered, number);
	shiftRight(registred,pos, nRegistered);
	registered[pos] = number;
	nRegistered = nRegistered + 1;
}

/**
 * Décale les éléments du tableau de begin à end un à droite.
 * @param tab le tableau
 * @param begin l'indice de départ
 * @param end l'indice de fin
 */
public shiftRight(int[] tab, int begin, int end){
	for (int i = end, i >= begin; i = i - 1){
		tab[i + 1] = tab[i];
	}
}
			\end{java}

			
		\subsection{Désinscription}
		%---------------------------------------------
		
			Ici, il va falloir décaler vers la gauche
			les éléments qui suivent celui à supprimer.

			\begin{pseudocode}
				\LComment{désinscrit l’étudiant donné}
				\Algo{unsubscribe}{
						\Par{registered\In\Out}{\Array{n}{integers}}, 
						\Par{nRegistered\In\Out}{integer}, 
						\\\hfill\Par{number\In}{integer}
						}{}
					\Decl{pos}{integer}
					\Stmt pos \Gets findPosition(registered, nRegistered, number)
					\Stmt shiftLeft(registered, pos, nRegistered-1)
					\Let nRegistered \Gets nRegistered - 1					
				\EndAlgo

				\Empty
				\LComment{Décale d’une position à gauche les éléments entre 
					la position début et fin}
				\Algo{shiftLeft}{
						\Par{tab\In\Out}{\Array{n}{integers}}, 
						\\\hfill\Par{begin\In}{integer}, 
						\Par{end\In}{integer}
						}{}
					\For{i}{begin}{end}
						\Let tab[i-1] \Gets tab[i]
					\EndFor
				\EndAlgo
			\end{pseudocode}
			
			La fiche \vref{fiche:tab-recherche-triee} reprend cet algorithme
			dans un cadre plus général.

			En langage Java, la traduction est immédiate~:

			\begin{java}
/**
 * Désinscrit un ou une étudiant·e.
 * 
 * @param registered le tableau d'inscrit·es
 * @param nRegistered le nombre d'inscrit·es
 * @param number le numéro à chercher
 */
public static void subscribe(int[] registred, int nRegistered,
				int number){
	int pos = findPosition(registered, nRegistered, number);
	shiftLeft(registred,pos, nRegistered);
	nRegistered = nRegistered - 1;
}

/**
 * Décale les éléments du tableau de begin à end un à gauche
 * (et écrase donc le premier).
 * @param tab le tableau
 * @param begin l'indice de départ
 * @param end l'indice de fin
 */
public shiftLeft(int[] tab, int begin, int end){
	for (int i = begin, i <= end; i = i + 1){
		tab[i - 1] = tab[i];
	}
}
			\end{java}

		\subsection{Intérêt de trier les valeurs~?}
		%---------------------------------------------

			Est-ce que trier les valeurs est vraiment intéressant~?
			
			La recherche est un peu plus rapide. En moyenne deux fois plus
			rapide si l'élément ne s'y trouve pas. Par contre, l'ajout et la
			suppression d'un élément sont plus lents.  Les algorithmes sont plus
			complexes à écrire et à comprendre.  Le gain ne parait pas évident
			et en effet, en l’état, il ne l’est pas.
			
			Mais c’est sans compter un autre algorithme de recherche, beaucoup
			plus rapide, la recherche dichotomique, que nous allons voir
			maintenant.
			
	% ==============================================
	\section{La recherche dichotomique\index{recherche dichotomique}} 
	% ==============================================

		La recherche dichotomique est un algorithme de recherche d’une valeur
		dans un tableau trié.  Il a pour essence de réduire à chaque étape la
		taille de l’ensemble de recherche de moitié, jusqu’à ce qu’il ne reste
		qu’un seul élément dont la valeur devrait être celle recherchée, sauf
		bien entendu en cas d’inexistence de cette valeur dans l’ensemble de
		départ. 
	
		Pour l’expliquer, nous allons considérer un tableau d’entiers
		complètement rempli.  Il sera facile d’adapter la solution à un tableau
		partiellement rempli (avec une taille logique) ou un tableau contenant
		tout autre type de valeurs qui peut s’ordonner.
		
		{\sffamily\bfseries\upshape
		Description de l’algorithme}
	
		Soit \pc{value} la valeur recherchée dans une zone délimitée 
		par les indices \pc{leftIndex} et \pc{rightIndex}. 
		Il faut commencer par déterminer l’élément médian, 
		c’est-à-dire celui qui se trouve «~au milieu~» 
		de la zone de recherche%
		\footnote{%
			Cet élément médian n’est pas tout à fait au milieu 
			dans le cas d’une zone contenant un nombre pair d’éléments.
		}~; 
		son indice sera déterminé par la formule
	
		{\centering
		\pc{medianIndex \Gets (leftIndex + rightIndex) DIV 2}\par{}
		}

		\marginicon{definition}
		L'algorithme compare alors \pc{value} avec la valeur de cet élément
		médian~; il est possible que ce soit la valeur cherchée; sinon, il
		partage la zone de recherche en deux parties~: une qui \textbf{ne
		contient certainement pas} la valeur cherchée et une qui
		\textbf{pourrait la contenir}.  C’est cette deuxième partie qui devient
		la nouvelle zone de recherche.  Il adapte \pc{leftIndex} ou
		\pc{rightIndex} en conséquence.  L'algorithme réitère le processus
		jusqu’à ce que la valeur cherchée soit trouvée ou que la zone de
		recherche soit réduite à sa plus simple expression, c’est-à-dire un seul
		élément.
			
		{\sffamily\bfseries\upshape
		Exemple de recherche fructueuse}

		Supposons que l’on recherche la valeur \textbf{23} dans un tableau de 20
		entiers.  La zone ombrée représente à chaque fois la partie de
		recherche, qui est au départ le tableau trié dans son entièreté.  Au
		départ, \pc{leftIndex} vaut 0 et \pc{rightIndex} vaut 19.

		\begin{center}
		\scriptsize
		\begin{tabular}{*{20}
			{>{\centering\itshape\arraybackslash}m{1pt}}}
		 \textcolor{gray}{0} &
		 \textcolor{gray}{1} &
		 \textcolor{gray}{2} &
		 \textcolor{gray}{3} &
		 \textcolor{gray}{4} &
		 \textcolor{gray}{5} &
		 \textcolor{gray}{6} &
		 \textcolor{gray}{7} &
		 \textcolor{gray}{8} &
		 \textcolor{gray}{9} &
		 \textcolor{gray}{10} &
		 \textcolor{gray}{11} &
		 \textcolor{gray}{12} &
		 \textcolor{gray}{13} &
		 \textcolor{gray}{14} &
		 \textcolor{gray}{15} &
		 \textcolor{gray}{16} &
		 \textcolor{gray}{17} &
		 \textcolor{gray}{18} &
		 \textcolor{gray}{19}
			 \\
		\end{tabular}
		\begin{tabular}{|*{20}{>{\centering\arraybackslash}m{1pt}|}}
			\hline
			{\cellcolor{gray!25}  1} &
			{\cellcolor{gray!25}  3} &
			{\cellcolor{gray!25}  5} &
			{\cellcolor{gray!25}  7} &
			{\cellcolor{gray!25}  7} &
			{\cellcolor{gray!25}  9} &
			{\cellcolor{gray!25}  9} &
			{\cellcolor{gray!25} 10} &
			{\cellcolor{gray!25} 10} &
			{\cellcolor{gray!25} 15} &
			{\cellcolor{gray!25} 16} &
			{\cellcolor{gray!25} 20} &
			{\cellcolor{gray!25} 23} &
			{\cellcolor{gray!25} 23} &
			{\cellcolor{gray!25} 25} &
			{\cellcolor{gray!25} 28} &
			{\cellcolor{gray!25} 28} &
			{\cellcolor{gray!25} 28} &
			{\cellcolor{gray!25} 29} &
			{\cellcolor{gray!25} 29}\\\hline
		\end{tabular}
		\end{center}
		
		\bigskip

		\textit{Première étape}~:
		\pc{medianIndex \Gets (0 + 19) DIV 2}, c’est-à-dire 9. 
		
		Comme la valeur en position 9 est 15, 
		la valeur cherchée doit se trouver à sa droite, et
		on prend comme nouvelle zone de recherche celle délimitée par
		\pc{leftIndex} qui vaut 10 et \pc{rightIndex} qui vaut 19.
		
		\medskip
		
		\begin{center}
		\scriptsize
		\begin{tabular}{*{20}{>{\centering\itshape\arraybackslash}m{1pt}}}
		 \textcolor{gray}{0} &
		 \textcolor{gray}{1} &
		 \textcolor{gray}{2} &
		 \textcolor{gray}{3} &
		 \textcolor{gray}{4} &
		 \textcolor{gray}{5} &
		 \textcolor{gray}{6} &
		 \textcolor{gray}{7} &
		 \textcolor{gray}{8} &
		 \textcolor{gray}{9} &
		 \textcolor{gray}{10} &
		 \textcolor{gray}{11} &
		 \textcolor{gray}{12} &
		 \textcolor{gray}{13} &
		 \textcolor{gray}{14} &
		 \textcolor{gray}{15} &
		 \textcolor{gray}{16} &
		 \textcolor{gray}{17} &
		 \textcolor{gray}{18} &
		 \textcolor{gray}{19}
			 \\
		\end{tabular}
		\begin{tabular}{|*{20}{>{\centering\arraybackslash}m{1pt}|}}
			\hline
			{ 1} &
			{  3} &
			{  5} &
			{  7} &
			{  7} &
			{  9} &
			{  9} &
			{ 10} &
			{ 10} &
			{ 15} &
			{\cellcolor{gray!25} 16} &
			{\cellcolor{gray!25} 20} &
			{\cellcolor{gray!25} 23} &
			{\cellcolor{gray!25} 23} &
			{\cellcolor{gray!25} 25} &
			{\cellcolor{gray!25} 28} &
			{\cellcolor{gray!25} 28} &
			{\cellcolor{gray!25} 28} &
			{\cellcolor{gray!25} 29} &
			{\cellcolor{gray!25} 29}\\\hline
		\end{tabular}
		\end{center}

		\bigskip

		\textit{Deuxième étape}~:
		\pc{medianIndex \Gets (10 + 19) DIV 2}, c’est-à-dire 14. 
		
		Comme on y trouve la valeur 25, 
		on garde les éléments situés à la gauche de celui-ci~; 
		la nouvelle zone de recherche est [10, 13].
				
		\begin{center}
		\scriptsize
		\begin{tabular}{*{20}{>{\centering\itshape\arraybackslash}m{1pt}}}
		 \textcolor{gray}{0} &
		 \textcolor{gray}{1} &
		 \textcolor{gray}{2} &
		 \textcolor{gray}{3} &
		 \textcolor{gray}{4} &
		 \textcolor{gray}{5} &
		 \textcolor{gray}{6} &
		 \textcolor{gray}{7} &
		 \textcolor{gray}{8} &
		 \textcolor{gray}{9} &
		 \textcolor{gray}{10} &
		 \textcolor{gray}{11} &
		 \textcolor{gray}{12} &
		 \textcolor{gray}{13} &
		 \textcolor{gray}{14} &
		 \textcolor{gray}{15} &
		 \textcolor{gray}{16} &
		 \textcolor{gray}{17} &
		 \textcolor{gray}{18} &
		 \textcolor{gray}{19}
			 \\
		\end{tabular}
		\begin{tabular}{|*{20}{>{\centering\arraybackslash}m{1pt}|}}
			\hline
			{ 1} &
			{  3} &
			{  5} &
			{  7} &
			{  7} &
			{  9} &
			{  9} &
			{ 10} &
			{ 10} &
			{ 15} &
			{\cellcolor{gray!25} 16} &
			{\cellcolor{gray!25} 20} &
			{\cellcolor{gray!25} 23} &
			{\cellcolor{gray!25} 23} &
			{ 25} &
			{ 28} &
			{ 28} &
			{ 28} &
			{ 29} &
			{ 29}\\\hline
		\end{tabular}
		\end{center}

		\bigskip

		\textit{Troisième étape}~:
		\pc{medianIndex \Gets (10 + 13) DIV 2}, c’est-à-dire 11 
		où se trouve l’élément 20. 
		La zone de recherche devient
		\pc{leftIndex} vaut 12 et \pc{rightIndex} vaut 13.

		
		\begin{center}
		\scriptsize
		\begin{tabular}{*{20}{>{\centering\sffamily\itshape\arraybackslash}m{1pt}}}
		 \textcolor{gray}{0} &
		 \textcolor{gray}{1} &
		 \textcolor{gray}{2} &
		 \textcolor{gray}{3} &
		 \textcolor{gray}{4} &
		 \textcolor{gray}{5} &
		 \textcolor{gray}{6} &
		 \textcolor{gray}{7} &
		 \textcolor{gray}{8} &
		 \textcolor{gray}{9} &
		 \textcolor{gray}{10} &
		 \textcolor{gray}{11} &
		 \textcolor{gray}{12} &
		 \textcolor{gray}{13} &
		 \textcolor{gray}{14} &
		 \textcolor{gray}{15} &
		 \textcolor{gray}{16} &
		 \textcolor{gray}{17} &
		 \textcolor{gray}{18} &
		 \textcolor{gray}{19}
			 \\
		\end{tabular}
		\begin{tabular}{|*{20}{>{\centering\arraybackslash}m{1pt}|}}
			\hline
			{ 1} &
			{  3} &
			{  5} &
			{  7} &
			{  7} &
			{  9} &
			{  9} &
			{ 10} &
			{ 10} &
			{ 15} &
			{ 16} &
			{ 20} &
			{\cellcolor{gray!25} 23} &
			{\cellcolor{gray!25} 23} &
			{ 25} &
			{ 28} &
			{ 28} &
			{ 28} &
			{ 29} &
			{ 29}\\\hline
		\end{tabular}
		\end{center}

		\bigskip

		\textit{Quatrième étape~:}
		\pc{medianIndex \Gets (12 + 13) DIV 2}, 
		c’est-à-dire 12 où se trouve la valeur cherchée~; 
		la recherche est terminée.

	{\sffamily\bfseries\upshape
	Exemple de recherche infructueuse}

		Recherchons à présent la valeur \textbf{8}. 
		Les étapes de la recherche vont donner successivement
		
		\begin{center}
		\scriptsize
		\begin{tabular}{*{20}{>{\centering\sffamily\itshape\arraybackslash}m{1pt}}}
		 \textcolor{gray}{0} &
		 \textcolor{gray}{1} &
		 \textcolor{gray}{2} &
		 \textcolor{gray}{3} &
		 \textcolor{gray}{4} &
		 \textcolor{gray}{5} &
		 \textcolor{gray}{6} &
		 \textcolor{gray}{7} &
		 \textcolor{gray}{8} &
		 \textcolor{gray}{9} &
		 \textcolor{gray}{10} &
		 \textcolor{gray}{11} &
		 \textcolor{gray}{12} &
		 \textcolor{gray}{13} &
		 \textcolor{gray}{14} &
		 \textcolor{gray}{15} &
		 \textcolor{gray}{16} &
		 \textcolor{gray}{17} &
		 \textcolor{gray}{18} &
		 \textcolor{gray}{19}
			 \\
		\end{tabular}
		\begin{tabular}{|*{20}{>{\centering\arraybackslash}m{1pt}|}}
			\hline
			{\cellcolor{gray!25}  1} &
			{\cellcolor{gray!25}  3} &
			{\cellcolor{gray!25}  5} &
			{\cellcolor{gray!25}  7} &
			{\cellcolor{gray!25}  7} &
			{\cellcolor{gray!25}  9} &
			{\cellcolor{gray!25}  9} &
			{\cellcolor{gray!25} 10} &
			{\cellcolor{gray!25} 10} &
			{\cellcolor{gray!25} 15} &
			{\cellcolor{gray!25} 16} &
			{\cellcolor{gray!25} 20} &
			{\cellcolor{gray!25} 23} &
			{\cellcolor{gray!25} 23} &
			{\cellcolor{gray!25} 25} &
			{\cellcolor{gray!25} 28} &
			{\cellcolor{gray!25} 28} &
			{\cellcolor{gray!25} 28} &
			{\cellcolor{gray!25} 29} &
			{\cellcolor{gray!25} 29}\\\hline
		\end{tabular}
		\end{center}

		\begin{center}
		\scriptsize
		\begin{tabular}{|*{20}{>{\centering\arraybackslash}m{1pt}|}}
			\hline
			{\cellcolor{gray!25}  1} &
			{\cellcolor{gray!25}  3} &
			{\cellcolor{gray!25}  5} &
			{\cellcolor{gray!25}  7} &
			{\cellcolor{gray!25}  7} &
			{\cellcolor{gray!25}  9} &
			{\cellcolor{gray!25}  9} &
			{\cellcolor{gray!25} 10} &
			{\cellcolor{gray!25} 10} &
			{ 15} &
			{ 16} &
			{ 20} &
			{ 23} &
			{ 23} &
			{ 25} &
			{ 28} &
			{ 28} &
			{ 28} &
			{ 29} &
			{ 29}\\\hline
		\end{tabular}
		\end{center}

		\begin{center}
		\scriptsize
		\begin{tabular}{|*{20}{>{\centering\arraybackslash}m{1pt}|}}
			\hline
			{ 1} &
			{  3} &
			{  5} &
			{  7} &
			{  7} &
			{\cellcolor{gray!25}  9} &
			{\cellcolor{gray!25}  9} &
			{\cellcolor{gray!25} 10} &
			{\cellcolor{gray!25} 10} &
			{ 15} &
			{ 16} &
			{ 20} &
			{ 23} &
			{ 23} &
			{ 25} &
			{ 28} &
			{ 28} &
			{ 28} &
			{ 29} &
			{ 29}\\\hline
		\end{tabular}
		\end{center}

		\begin{center}
		\scriptsize
		\begin{tabular}{|*{20}{>{\centering\arraybackslash}m{1pt}|}}
			\hline
			{ 1} &
			{  3} &
			{  5} &
			{  7} &
			{  7} &
			{\cellcolor{gray!25}  9} &
			{  9} &
			{ 10} &
			{ 10} &
			{ 15} &
			{ 16} &
			{ 20} &
			{ 23} &
			{ 23} &
			{ 25} &
			{ 28} &
			{ 28} &
			{ 28} &
			{ 29} &
			{ 29}\\\hline
		\end{tabular}
		\end{center}

		\smallskip

		Arrivé à ce stade, la zone de recherche s’est réduite à un seul élément.
		Ce n’est pas celui qu’on recherche mais c’est à cet endroit qu’il se
		serait trouvé~; c’est donc là qu’on pourra éventuellement l’insérer.  Le
		paramètre de sortie prend la valeur 5.

		{\sffamily\bfseries
		Algorithme}
		
		\begin{pseudocode}
			\Algo{dichotomousResearch}{
					\\\hfill
					\Par{myArray\In}{\Array{}{integers}}, 
					\Par{value\In}{integer}, 
					\Par{pos\InOut}{integer}
					}{boolean}
				\Decl{rightIndex, leftIndex, medianIndex}{integer}
				\Decl{candidate}{integer}
				\Decl{isFind}{boolean}
				\Empty
				\Let leftIndex \Gets 0
				\Let rightIndex \Gets myArray.length - 1
				\Let isFind \Gets false
				\Empty
				\While{NON isFind AND leftIndex {${\leq}$} rightIndex}
					\Let medianIndex \Gets (leftIndex + rightIndex) DIV 2
					\Let candidate \Gets myArray[medianIndex]
					\If{candidate == value} 
						\Let isFind \Gets vrai
					\ElsIf{candidate < value}
						\Let leftIndex \Gets medianIndex + 1
						\RComment on garde la partie droite
					\Else
						\Let rightIndex \Gets medianIndex – 1
						\RComment on garde la partie gauche
					\EndIf
				\EndWhile
				\Empty
				\If{isFind}
					\Let pos \Gets medianIndex
				\Else
					\Let pos \Gets leftIndex
					\RComment dans le cas où la valeur n’est pas trouvée,
					\Empty 
					\RComment on vérifiera que leftIndex donne la valeur 
					où elle pourrait être insérée.
				\EndIf
				\Empty
				\Return isFind
			\EndAlgo
		\end{pseudocode}

		La traduction de l'algorithme en Java est laissé à titre d'exercice. La
		difficulté réside dans le choix de la valeur de retour et du paramètre
		en entrée/sortie \pc{pos}. En Java, il faudra choisir de ne pas obtenir
		un booléen mais bien l'indice de la valeur si elle est trouvée et -1
		sinon. Nous ne pourrons pas avoir recevoir l'indice «~à insérer~» dans
		le cas ou la valeur ne se trouve pas dans le tableau.

		\clearpage
	% ==============================================
	\section{Introduction à la complexité} 
	% ==============================================

		\begin{quote}
		
			L’algorithme de recherche dichotomique est beaucoup plus rapide que
			l’algorithme de recherche linéaire. 
	
		\end{quote}
		
		Mais qu’est-ce que cela veut dire exactement~? \\
		En est-on sûr~? \\
		Comment le mesure-t-on~? \\
		Quels critères utilise-t-on~? \\

		De façon générale, comment comparer la vitesse de deux algorithmes
		différents qui résolvent le même problème.
		
		\subsection{Une approche pratique~: la simulation numérique}
		%--------------------------------------------------
	
			Une manière naïve de comparer la rapidité des algorithmes serait de
			les traduire dans un langage de programmation, de les exécuter et de
			comparer les temps d'exécution.

			Cette technique pose toutefois quelques problèmes~:			
			\begin{itemize}
				
				\item il faut que les programmes soient exécutés 
					dans des environnements strictement identiques 
					ce qui n’est pas toujours le cas ou facile à vérifier;
				
				\item certains environnements peuvent être favorables à un
					algorithme par rapport à l’autre ce qui ne serait pas le cas
					d’un autre environnement;
					
					Par exemple, certaines architectures
					sont plus rapides dans les calculs entiers
					mais moins dans les calculs flottants.
				
				\item le test ne porte que sur un (voir quelques uns) jeu(x) de
					tests.  
					
					Comment en tirer un enseignement général~?\\
					Que se passerait-il avec des données plus importantes~?\\
					Avec des données différentes~?
		
			\end{itemize}	

			En fait, un chiffre précis ne nous intéresse pas.  Ce qui est
			vraiment intéressant, c’est de savoir comment l’algorithme va se
			comporter avec de grandes données.  C’est ce qu’apporte l’approche
			suivante.
		
		\subsection{Une approche théorique~: la complexité}
		%--------------------------------------------------
		
			Le principe de cette approche est de déduire de l'algorithme le
			nombre d’opérations de base à effectuer en fonction de la
			\textbf{taille} du problème à résoudre et en déduire comment il va
			se comporter sur de «~gros~» problèmes~?
		
			Dans le cas de la recherche dans un tableau, la taille du problème
			est la taille du tableau dans lequel l'élément est cherché.  Nous
			pouvons considérer que l’opération de base est la comparaison avec
			un élément du tableau. 
			
			La question est alors la suivante~: pour un tableau de taille $n$,
			à combien de comparaisons faut-il procéder pour trouver l’élément
			(ou se rendre compte de son absence)~?
		
			{\bfseries
			Pour la recherche linéaire}
		
				Cela dépend évidemment de la position de la valeur à trouver. 
				Dans le meilleur des cas c’est 1, 
				dans le pire c’est $n$ mais on peut dire, 
				qu’en moyenne, 
				cela entraine «~$n/2$~» comparaisons 
				(que ce soit pour la trouver ou se rendre compte de son absence).
		
			{\bfseries
			Pour la recherche dichotomique}
		
				Ici, la zone de recherche est divisée par 2, à chaque étape. 
				Imaginons une liste de 64 éléments~: 
				après 6 étapes, on arrive à un élément isolé. 
				Pour une liste de taille $n$, 
				on peut en déduire que le nombre de comparaisons 
				est au maximum l’exposant qu’il faut donner à 2 pour obtenir $n$, 
				soit «~$\log_2(n)$~».
		
			{\sffamily\bfseries\upshape
			Comparaisons}
		
			Voici un tableau comparatif du nombre de comparaisons en fonction de
			la taille $n$
		
			\begin{center}
				\small
				\begin{tabular}{|c|c|c|c|c|c|c|}
			\hline
			\rowcolor{black!40}
			$n$ & 10 & 100 & 1000 & 10.000 & 100.000 & 1 million \\
			\hline
			recherche linéaire & 5 & 50 & 500 & 5.000 & 50.000 & 500.000 \\
			\hline
			recherche dichotomique & 4 & 7 & 10 & 14 & 17 & 20 \\
			\hline
			\end{tabular}
			\end{center}
			
			Nous pouvons voir que c’est surtout pour des grands problèmes que la
			recherche dichotomique montre toute son efficacité.  Et nous voyons
			aussi que l'important pour mesurer la complexité est l’ordre de
			grandeur du nombre de comparaisons. 
		
			Nous dirons que la recherche simple est un algorithme de complexité
			linéaire (c’est-à-dire que le nombre d’opérations est de l’ordre de
			$n$ ou proportionnel à $n$ ou encore que doubler la taille du
			problème fait doubler aussi le temps de calcul). Ceci se note en
			langage plus mathématique~: \textbf{$O(n)$} (prononcé «~grand $O$ de
			$n$~»). 
			
			Pour la recherche dichotomique, la complexité est logarithmique.
			Elle se note \\$O(\log_2(n))$ ce qui veut dire que doubler la taille
			du problème augmente d'une seule itération la boucle. C'est un
			fameux gain. 
			
			Comparons les complexités les plus courantes.
			\begin{center}
			\small
			\begin{tabular}{|c|l|l|l|l|l|l|l|}
			\hline
			\rowcolor{black!40}
			$n$ & 10 & 100 & 1000 & $10^4$ & $10^5$ & $10^6$ & $10^9$ \\
			\hline
			$O(1)$ & 1 & 1 & 1 & 1 & 1 & 1 & 1\\
			\hline
			$O(\log_2(n))$ & 4 & 7 & 10 & 14 & 17 & 20 & 30\\
			\hline
			$O(n)$ & 10 & 100 & 1000 & $10^4$ & $10^5$ & $10^6$ & $10^9$ \\
			\hline
			$O(n^2)$ & 100 & $10^4$ & $10^6$ & $10^8$ & $10^10$ & $10^{12}$ & $10^{18}$ \\
			\hline
			$O(n^3)$ & 1000 & $10^6$ & $10^9$ & $10^{12}$ & $10^{15}$ & $10^{18}$ & $10^{27}$ \\
			\hline
			$O(2^n)$ & 1024 & $10^{30}$ & $10^{301}$ & $10^{3010}$ & $10^{30102}$ 
			& $10^{301029}$ & {\tiny $10^{301029995}$} \\
			\hline
			\end{tabular}
			\end{center}
		
			Nous pouvons en conclure que si trouver un algorithme qui résout un
			algorithme est sans doute bien, trouver un algorithme qui a une
			complexité exponentielle ne sert à rien en pratique. 

			Par exemple un algorithme de recherche de complexité exponentielle,
			exécuté sur une machine pouvant effectuer un milliard de
			comparaisons par secondes, prendrait plus de dix mille milliards
			d’années pour trouver une valeur dans un tableau de 100 éléments.
