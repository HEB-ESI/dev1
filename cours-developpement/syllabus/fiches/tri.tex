%================================
\begin{Fiche}{Tri d’un tableau}
%================================
\label{fiche:tri}

\noindent Trier un tableau d’entiers par ordre croissant. Les algorithmes de tri sont présentés aux sections \vrefrange{chap:tri-insertion}{chap:tri-bubble}.

\Section{Spécification}
	
	\textbf{Données}~: le tableau non trié, à trier.
		
	\textbf{Résultat}~: ce même tableau trié.

\Section{Solution}

\subsection*{Le tri par insertion}
		
	\begin{pseudocode}
		\LComment Trie le tableau reçu en paramètre (via un tri par insertion).
		\Algo{insertionSort}{\Par{myArray\InOut}{\Array{}{integers}}}{}
			\Decl{i, j, value}{integers}
			\For{i}{1}{myArray.length - 1}
				\Let value \Gets myArray[i]
				\LComment recherche de l’endroit où insérer value dans le 
				\LComment sous-tableau trié et décalage simultané des éléments
				\Let j \Gets i - 1
				\While{j ${\geq}$ 0 AND value < myArray[j]}
					\Let myArray[j+1] \Gets myArray[j]
					\Let j \Gets j – 1
				\EndWhile
				\Let myArray[j+1] \Gets value
			\EndFor
		\EndAlgo
	\end{pseudocode}

        \inputjava{algo-standards/insertionSort.java}


	
\subsection*{Le tri par sélection des minima successifs}
		
	\begin{pseudocode}
		\LComment Trie le tableau reçu en paramètre 
		(via un tri par sélection des minima successifs).
		\Algo{selectionSort}{\Par{myArray\InOut}{\Array{}{integers}}}{}
			\Decl{i, minIndex}{integer}
			\For{i}{0}{myArray.length – 2}
			\RComment i correspond à l’étape 
			\RComment de l’algorithme
				\Let minIndex \Gets searchMinIndex( myArray, i, 
					myArray.length - 1 )
				\Stmt swap( myArray[i], myArray[minIndex] )
			\EndFor
		\EndAlgo

		\Empty
		\LComment Retourne l’indice du minimum entre les indices début 
		et fin du tableau reçu.
		\Algo{searchMinIndex}{
			\Par{myArray\In}{\Array{}{integers}}, 
			\\\hfill\Par{début\In, fin\In}{integers}}{integer}
			\Decl{minIndex, i}{integers}
			\Let minIndex \Gets début
			\For{i}{début+1}{fin}
				\If{myArray[i] < myArray[minIndex]}
					\Let minIndex \Gets i
				\EndIf
			\EndFor
			\Return minIndex
		\EndAlgo

		\Empty
		\LComment {Échange le contenu de 2 variables.}
		\Algo{swap}{\Par{a\InOut, b\InOut}{integers}}{}
			\Decl{aux}{integer}
			\Let aux \Gets a
			\Let a \Gets b
			\Let b \Gets aux
		\EndAlgo
	\end{pseudocode}

	\inputjava{algo-standards/selectionSort.java}


	

\subsection*{Le tri bulle}

	\begin{pseudocode}
	\LComment Trie le tableau reçu en paramètre (via un tri bulle).
		\Algo{bubbleSort}{\Par{myArray\InOut}{\Array{}{integers}}}{}
			\Decl{burstIndex, bubbleIndex}{integers}
			\For{burstIndex}{0}{myArray.length - 2}
				\For[-1]{bubbleIndex}{myArray.length – 2}{burstIndex}
					\If{myArray[bubbleIndex] > myArray[bubbleIndex + 1]}
                                        	\LComment voir algorithme précédent
						\Stmt swap( myArray[bubbleIndex], myArray[bubbleIndex + 1] )
					\EndIf
				\EndFor
			\EndFor
		\EndAlgo
	\end{pseudocode}

        \inputjava{algo-standards/bubbleSort.java}

\end{Fiche}
