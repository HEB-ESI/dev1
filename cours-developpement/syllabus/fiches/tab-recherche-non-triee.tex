%================================
\begin{Fiche}{Tableau non trié}
%================================
\label{fiche:tab-recherche-non-triee}

\Section{Problème}
	
	Rechercher, ajouter, supprimer des données dans un tableau non trié. Par
	exemple dans un tableau d'entiers

\Section{Rechercher}

	\SubSection{Spécification}

	Retourner l'indice d'une donnée trouvée dans un tableau non trié ou -1 si
	elle n'est pas trouvée. 

	Nous supposons que le tableau n'est pas rempli, l'algorithme recevra donc 
	une valeur représentant le nombre d'éléments du tableau. Cette valeur est
	inférieure à la taille du tableau bien sûr. 
	
	\textbf{Données}~: le tableau à analyser, le nombre d'éléments dans ce
	tableau (taille logique), la valeur à rechercher
		
	\textbf{Résultat}~: la position de l'élément si il est dans le tableau et -1
	sinon.
	
	\SubSection{Solution}
	
		Il suffit de parcourir le tableau jusqu'à ce que l'on trouve la valeur
		ou bien que l'on arrive à la fin. 

		\begin{langagenaturel}
			tant que l'on n'est pas à la fin et la valeur n'est pas trouvée\\
			\tab incrémenter l'indice\\
			fin tant que\\
			si on n'est pas à la fin alors\\
			\tab on a trouvé\\
			sinon\\
			\tab on n'a pas trouvé et on retourne -1\\
			fin si\\
		\end{langagenaturel}

		\begin{pseudocode}
			\LComment{Vérifie si un nombre est dans un tableau 
			et donne sa position (-1 sinon)}
			\Algo{indexValue}{\Par{is\In}{\Array{}{integers}}, 
					\\\hfill\Par{nValues\In}{integer}, 
					\Par{value\In}{integer}}{integer}
				\Decl{i}{integer}
				\Let i \Gets 0
				\While{i < nValues AND is[i] $\ne$ value}
					\Let i \Gets i + 1
				\EndWhile
				\If{i < nValues}
					\Return i
				\Else
					\Return -1
				\EndIf
			\EndAlgo
		\end{pseudocode}

		\begin{java}
public static int indexValue(int[] is, int nValues, int value){
	int i = 0;
	while (i < nValues && is[i] != value){
		i++;
	}
	if (i < nValues){
		return i;
	} else {
		return -1;
	}
}
		\end{java}


\Section{Ajouter}

	\SubSection{Spécification}

	Ajouter une donnée non encore présente dans le tableau de données non
	triées.
	
	\textbf{Données}~: le tableau à modifier, le nombre d'éléments dans ce
	tableau, la valeur à ajouter
		
	\textbf{Résultat}~: le tableau reçu est modifié en lui ajoutant la valeur si
	elle n'y était pas déjà
	
	\SubSection{Solution}

	Si l'algorithme reçoit le tableau, le nombre d'éléments et la valeur
	à ajouter, il suffit d'insérer la valeur dans le tableau… après avoir
	vérifier qu'elle n'est pas présente. Cette vérification peut se faire
	simplement en utilisant l'algorithme \pc{indexValue}. Si celui-ci retourne
	-1, c'est que la valeur n'est pas présente. 

	\begin{langagenaturel}
		if nValues < taille du tableau AND indexValue = -1\\
		\tab ajouter la valeur à la fin du tableau
	\end{langagenaturel}

		\begin{pseudocode}
			\LComment{Ajoute un nombre non encore présent dans le tableau.}
			\Algo{add}{\Par{is\In\Out}{\Array{}{integer}}, 
				\\\hfill\Par{nValues\In\Out}{integer}, \Par{value\In}{integer}}{}
				\If{nValues $<$ is.length 
						AND indexValue(is, nValues, value) $=$ -1}
					\Let is[nValues] \Gets value
					\Let nValues \Gets nValues + 1
				\EndIf
			\EndAlgo
		\end{pseudocode}

		\paragraph{Remarque} 
		
		En langage Java, le passage de paramètre se faisant par valeur (cfr.
		Chapitre \vref{chap:insertion-recherche}), la taille logique du tableau
		doit être retournée et maintenue à jour par le code appelant. En
		langage Java, cet algorithme aurait l'allure suivante~:

		\begin{java}
public static int add(int[] is, int nValues, int value){
	if (nValues < is.length() 
			&& indexValue(is, nValues, value) == -1){
		is[nValues] = value;
		nValues = nValues + 1;
	}
	return nValues;
}
		\end{java}


\Section{Supprimer}

	\SubSection{Spécification}

	Supprimer une donnée d'un tableau de données non triées.

	\textbf{Données}~: le tableau à modifier, le nombre d'éléments dans ce
	tableau, la valeur à supprimer
		
	\textbf{Résultat}~: le tableau reçu est modifié en lui supprimant la valeur
	
	\SubSection{Solution}

	Comme le tableau n'est pas trié, pour supprimer une valeur, il suffit de
	trouver l'index de cette valeur et de placer la dernière valeur à sa place. 

	\begin{langagenaturel}
		index = l'index de la valeur à supprimer\\
		if index != 1\\
		\tab is[index] = is[nValues-1]\\
		\tab nValues--\\
		end if\\
	\end{langagenaturel}

		\begin{pseudocode}
			\LComment{Supprime un nombre donné dans le tableau.}
			\Algo{delete}{
				\Par{is\In\Out}{\Array{n}{integer}}, 
				\\\hfill\Par{nValues\In\Out}{integer}, 
				\Par{value\In}{integer}
			}{}
				\Decl{index}{integer}
				\Let index \Gets indexValue(is, nValues, value)
				\If{index $\ne$ -1}
					\Let is[index] \Gets is[nValues-1]
					\Let nValues \Gets nValues - 1
				\EndIf			
			\EndAlgo
		\end{pseudocode}

		\begin{java}
public static int delete(int[] is, int nValues, int value){
	int index = indexValue(is, nValues, value);
	if (index != -1){
		is[index] = is[nValues];
		nValues = nValues - 1;
	}
	return nValues;
}
		\end{java}

%\Section{Variante}


	
\end{Fiche}
