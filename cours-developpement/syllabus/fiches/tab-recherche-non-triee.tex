%================================
\begin{Fiche}{Tableau non trié}
%================================
\label{fiche:tab-recherche-non-triee}

\Section{Problème}
	
	Rechercher, ajouter, supprimer des données non triées dans un tableau
	d’entiers non triés.

\Section{Rechercher}

	Retourner l'indice d'une donnée trouvée dans un tableau non trié ou -1 si
	elle n'est pas trouvée. 

	Nous supposons que le tableau n'est pas rempli, l'algorithme recevra donc 
	une valeur représentant le nombre d'éléments du tableau. Cette valeur est
	inférieure à la taille du tableau bien sûr. 
	
	\textbf{Données}~: le tableau à analyser, le nombre d'éléments dans ce
	tableau (taille logique), la valeur à rechercher
		
	\textbf{Résultat}~: la position de l'élément si il est dans le tableau et -1
	sinon
	
		\begin{pseudocode}
			\LComment{Vérifie si un nombre est dans un tableau 
			et donne sa position (-1 sinon)}
			\Algo{indexValue}{\Par{is\In}{\Array{}{integers}}, 
					\\\hfill\Par{nbValues\In}{integer}, 
					\Par{value\In}{integer}}{integer}
				\Decl{i}{integer}
				\Let i \Gets 0
				\While{i < nbValues AND is[i] $\ne$ value}
					\Let i \Gets i + 1
				\EndWhile
				\If{i < nbValues}
					\Return i
				\Else
					\Return -1
				\EndIf
			\EndAlgo
		\end{pseudocode}

\Section{Ajouter}
	Ajouter une donnée non encore présente dans le tableau de données non triées
	
	\textbf{Données}~: le tableau à modifier, le nombre d'éléments dans ce
	tableau, la valeur à ajouter
		
	\textbf{Résultat}~: le tableau reçu est modifié en lui ajoutant la valeur si
	elle n'y était pas déjà
	
		\begin{pseudocode}
			\LComment{Ajoute un nombre non encore présente dans le isleau.}
			\Algo{add}{\Par{is\In\Out}{\Array{}{integer}}, 
				\\\hfill\Par{nValues\In\Out}{integer}, \Par{value\In}{integer}}{}
				\If{nValues $<$ is.length 
						AND indexValue(is, nValues, value) $\ne$ -1}
					\Let is[nValues] \Gets value
					\Let nValues \Gets nValues + 1
				\EndIf
			\EndAlgo
		\end{pseudocode}

		\paragraph{Remarque} Si l'algorithme précédent se traduit immédiatement
		en langage Java, ce n'est pas le cas de celui-ci. Pourquoi ? 
	
		Le passage de paramètre se faisant par valeur (cfr. chapitre
		\ref{chap:insertion-recherche} p.~\pageref{chap:insertion-recherche}),
		la taille logique du tableau doit être retournée et maintenue à jour par
		le code appelant. En langage Java, cet algorithme aurait l'allure
		suivante~:

		\begin{java}
public static int add(int[] is, int nValues, int value){
	if (nValues < is.length() 
			&& indexValue(is, nValues, value) != -1){
		is[nValues] = value;
		nValues = nValues + 1;
	}
	return nValues;
}
		\end{java}


\Section{Supprimer}
	Supprimer une donnée d'un tableau de données non triées
	
	\textbf{Données}~: le tableau à modifier, le nombre d'éléments dans ce
	tableau, la valeur à supprimer
		
	\textbf{Résultat}~: le tableau reçu est modifié en lui supprimant la valeur
		
		\begin{pseudocode}
			\LComment{Supprime un nombre donné dans le tableau.}
			\Algo{delete}{
				\Par{is\In\Out}{\Array{n}{integer}}, 
				\\\hfill\Par{nValues\In\Out}{integer}, 
				\Par{value\In}{integer}
			}{}
				\Decl{index}{integer}
				\Let index \Gets indexValue(is, nValues, value)
				\If{index $\ne$ -1}
					\Let is[index] \Gets is[nValues-1]
					\Let nValues \Gets nValues - 1
				\EndIf			
			\EndAlgo
		\end{pseudocode}

%\Section{Variante}


	
\end{Fiche}
