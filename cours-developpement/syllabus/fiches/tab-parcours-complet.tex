%================================
\begin{Fiche}{Parcours complet d’un tableau}
%================================
\label{fiche:tab-parcours-complet}

\Section{Problème}
	Afficher tous les éléments d’un tableau d’entiers.

\Section{Spécification}
	
	\begin{itemize}
	\item \textbf{Données}~: le tableau à afficher
	\item \textbf{Résultat}~: aucun.
	\item \textbf{Affiche} : les éléments du tableau, dans l'ordre.
	\end{itemize}

\Section{Solution}

	Puisqu’on parcourt tout le tableau, on peut utiliser une boucle \emph{pour}
	(\pc{for}).
	
	\begin{pseudocode}
		\Algo{display}{\Par{is}{\Array{}{integer}}}{}
			\For{i}{0}{is.length - 1}
				\Write is[i]
			\EndFor
		\EndAlgo
	\end{pseudocode}

\Section{Quand l’utiliser~?}

	Ce type de solution peut être utilisé à chaque fois qu’il est nécessaire
	d'examiner \textbf{tous} les éléments d’un tableau, quel que soit le
	traitement voulu~: les afficher, les sommer, les comparer\dots
	
	
\end{Fiche}
