%================================
\begin{Fiche}{Parcours complet d’un tableau}
%================================
\label{fiche:tab-parcours-complet}

\Section{Problème} 

Parcours complet d'un tableau pour afficher tous les éléments. Par exemple un
tableau de \pc{double}.

\Section{Spécification}
	
	\begin{itemize}
	\item \textbf{Données}~: le tableau à afficher
	\item \textbf{Résultat}~: aucun.
	\item \textbf{Affiche} : les éléments du tableau, dans l'ordre.
	\end{itemize}

\Section{Solution}

	Puisqu’on parcourt tout le tableau, on peut utiliser une boucle \emph{pour}
	(\pc{for}).
	
	\begin{pseudocode}
		\Algo{display}{\Par{ds}{\Array{}{real}}}{}
			\For{i}{0}{ds.length - 1}
				\Write ds[i]
			\EndFor
		\EndAlgo
	\end{pseudocode}

	\begin{java}
// Utilisation d'un for
public static void display (double[] ds){
	for (int i=0; i < ds.length; i++){
		System.out.println(ds[i]);
	}
}
	\end{java}

	Si l'on n'a pas besoin de modifier la valeur dans le tableau et que l'on
	veut simplement utiliser la valeur — dans ce cas, pour l'afficher — on peut
	utiliser une boucle \textit{enhanced for}\index{foreach} comme suit~:

	\begin{java}
// Utilisation d'un enhanced for (foreach)
public static void display (double[] ds){
	for (double d: ds){
		System.out.println(d);
	}
}
	\end{java}

\Section{Quand l’utiliser~?}

	Ce type de solution peut être utilisé à chaque fois qu’il est nécessaire
	d'examiner \textbf{tous} les éléments d’un tableau, quel que soit le
	traitement voulu~: les afficher, les sommer, les comparer\dots

	L'\textit{enhanced for} ne permet pas de modifier les éléments du tableau. 
	
	
\end{Fiche}
