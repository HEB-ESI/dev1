%================================
\begin{Fiche}{Maximum dans un tableau}
%================================
\label{fiche:tab-max}

\Section{Problème}
	Trouver la valeur maximale dans un tableau d’entiers.

\Section{Spécification}
	
	\textbf{Données}~: le tableau à analyser

	\textbf{Résultat}~: la valeur du maximum

\Section{Solution}

	Il faut veiller à initialiser le maximum avec la première valeur du tableau
	pour parcourir le reste du tableau à la recherche d’une valeur plus grande.
	
	\begin{pseudocode}
		\Algo{max}{\Par{is}{\Array{}{integer}}}{integer}
			\Decl{max}{integer}
			\Let max \Gets is[0]
			\For{i}{1}{is.length - 1}
				\If{is[i] > max}
					\Let max \Gets is[i]
				\EndIf
			\EndFor
			\Return max
		\EndAlgo
	\end{pseudocode}

\Section{Variante}

	Si c'est la position et non la valeur qui nous intéresse, une solution 
	serait~:

	\begin{pseudocode}
		\Algo{max}{\Par{is}{\Array{}{integer}}}{integer}
			\Decl{maxIndex}{integer}
			\Let maxIndex \Gets 0
			\For{i}{1}{is.length - 1}
				\If{is[i]>is[maxIndex]}
					\Let maxIndex \Gets i
				\EndIf
			\EndFor
			\Return maxIndex
		\EndAlgo
	\end{pseudocode}
	
	
\end{Fiche}
