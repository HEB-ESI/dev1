%==============================
\begin{Fiche}{Un calcul simple}
%==============================
\label{fiche:calcul-simple}

\Section{Le problème}
Calculer la surface d’un rectangle à partir de sa longueur et sa largeur.
	
\Section{Spécification}

	\textbf{Données} (toutes réelles et non négatives)~:
		\begin{itemize}
			\item la longueur du rectangle~;
			\item la largeur.
		\end{itemize}
		
	\textbf{Résultat}~: un réel représentant la surface du rectangle.

	\begin{center}	
		\flowalgodd{length (real)}{width (real)}{rectangleArea}{real}
	\end{center}

\Section{Exemples}

	\begin{itemize}
	\item \pc{rectangleArea(3, 2)} donne $6$
	\item \pc{rectangleArea(3.5, 1)} donne $3.5$
	\item \pc{rectangleArea(4, 0)} donne $0$
	\end{itemize}

\Section{Solution}

	La surface d’un rectangle est obtenue en multipliant
	la largeur par la longueur.
	\[
		\textrm{surface} = \textrm{longueur} * \textrm{largeur}
	\]

	\begin{langagenaturel}
		surface = longueur * largeur
	\end{langagenaturel}

	\begin{pseudocode}
		\Algo{rectangleArea}{\Par{length, width}{real}}{real}
			\Return length * width
		\EndAlgo
	\end{pseudocode}

	\begin{java}
public static double rectangleArea(double length, double width){
	return length * width;
	\end{java}

\Section{Vérification / tests}

	\begin{center}
		\begin{tabular}{|c|cccc|c|}
		\hline
		\rowcolor{black!40}
		test \no & longueur & largeur & réponse attendue & réponse fournie & {} \\
		\hline 
		1 & 3   & 2 & 6   & 6   & {\color{ForestGreen}$\checkmark$} \\\hline
		2 & 3.5 & 1 & 3.5 & 3.5 & {\color{ForestGreen}$\checkmark$} \\\hline
		3 & 4 & 0 & 0 & 0 & {\color{ForestGreen}$\checkmark$} \\\hline
		\end{tabular}
	\end{center}				

\Section{Quand l’utiliser~?}

	Ce type de solution peut être utilisé à chaque fois
	que la réponse s’obtient par un calcul simple sur les données.
	Si le calcul est plus complexe, 
	il peut être utile de le décomposer pour accroitre la lisibilité
	(cf. fiche \vref{fiche:calcul-complexe}) 
	
\end{Fiche}
