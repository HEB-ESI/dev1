%================================
\begin{Fiche}{Parcours partiel d’un tableau}
%================================
\label{fiche:tab-parcours-partiel}

\Section{Problème}

	Parcours partiel d'un tableau pour rechercher un zéro. Par exemple avec un
	tableau de \pc{double}.

\Section{Spécification}
	
	\textbf{Données}~: le tableau à tester
		
	\textbf{Résultat}~: 
		un booléen à vrai si il existe une valeur nulle dans le tableau.
	
	\begin{center}	
		\flowalgod{ds (tableau de pseudo-réels)}{containsNulValue}{booléen}
	\end{center}

\Section{Solution}

	Contrairement au parcours complet (cf. fiche
	\vref{fiche:tab-parcours-complet}) nous allons utiliser un \emph{tant que}
	(\pc{while}) car nous voulons arrêter l'algorithme dès que la valeur
	cherchée est trouvée.
	
	Il existe essentiellement deux solutions, avec ou sans variable booléenne.
	En général, la première solution [A] sera plus claire si le test est court.

	\subsubsection*{[A] Sans variable booléenne}
	
		\begin{pseudocode}
			\Algo{containsNulValue}{\Par{ds}{\Array{}{double}}}{boolean}
				\Decl{i}{integer}
				\Let i \Gets 0
				\While{i < ds.length AND ds[i] $\neq$ 0}
					\Let i \Gets i + 1
				\EndWhile
				\Return i < n \RComment Si i<n -> arrêt prématuré 
					\Let\RComment c-à-d que l'on a trouvé un 0.
			\EndAlgo
		\end{pseudocode}

		\begin{java}
public static boolean containsNulValue(double[] ds){
	int i = 0;
	while (i < ds.length && ds[i] != 0){
		i++;
	}
	/* Si i < n c'est un arrêt prémature et c'est que l'on a trouvé */
	return i < n
}
		\end{java}
	
		Il faut être attentif à \textbf{ne pas inverser} les deux parties du
		test sous peine de générer une erreur parce que l'on essaie d'accéder
		à un élément hors du tableau.  Il faut absolument vérifier que l’indice
		est bon avant de tester la valeur à cet indice. Se ce n'est pas clair,
		revoir la notion d'évaluation paresseuse (court-circuit). Voir
		\vref{court-circuit}.
		

	\subsubsection*{[B] Avec variable booléenne}

		\begin{pseudocode}
			\Algo{containsNulValue}{\Par{is}{\Array{}{integer}}}{boolean}
				\Decl{i}{integer}
				\Decl{isZero}{boolean}
				\Let isZero \Gets false
				\Let i \Gets 0
				\While{i < is.length AND NON isZero}
					\Let isZero \Gets is[i] ==  0
					\Let i \Gets i + 1
				\EndWhile
				\Return isZero
			\EndAlgo
		\end{pseudocode}

		\begin{java}
public static boolean containsNulValue(double[] ds){
	int i = 0;
	boolean isZero = false;
	while (i < ds.length && !isZero){
		isZero = ds[i] == 0;
		i++;
	}
	return isZero;
}
		\end{java}

		Au sortir de la boucle, l’indice \pc{i} ne désigne pas l’élément qui
		nous a permis d’arrêter mais le suivant.  Si nécessaire, on peut
		remplacer l’intérieur de la boucle par~:

		\begin{java}
if (ds[i] == 0){
	isZero = true;
} else {
	i++;
}
		\end{java}
		
		Dans notre exemple, nous cherchons un élément particulier (un 0).  Dans
		le cas où l'on vérifie si tous les éléments possèdent une certaine
		propriété (être positifs par exemple), nous veillerons à adapter le nom
		du booléen et son utilisation (par exemple un booléen appelé
		\pc{areAllPositives}, initialisé à \pc{true} avec un \pc{\dots AND
		areAllPositives} dans le test.

		Dans tous les cas, faites attention à ne pas acdéder à l'élément
		—~utiliser \pc{ds[i]}~— si vous n’êtes pas sûr de l'indice \pc{i}.
		C’est particulièrement vrai après la boucle.

		\paragraph{Remarque} C'est mieux de respecter les bonnes pratiques
		d'écriture et d'écrire directement un \pc{return} sans \pc{if} dans ce
		cas.  Voir annexe \vref{B-for-break}.
		
					
\Section{Quand l’utiliser~?}

	Ce type de solution peut être utilisé pour tout parcours d'un tableau où un
	arrêt prématuré est possible en se posant les questions suivantes~:

	\begin{itemize}
	\item Est-ce que tous les éléments sont positifs~?
	\item Est-ce que les élément sont triés~?
	\item Est-ce qu’un élément précis est présent~?
	\item \dots
	\end{itemize}
	
\end{Fiche}
