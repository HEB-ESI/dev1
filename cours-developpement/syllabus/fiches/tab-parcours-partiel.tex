%================================
\begin{Fiche}{Parcours partiel d’un tableau}
%================================
\label{fiche:tab-parcours-partiel}

\Section{Problème}
	Recherche d’un zéro dans un tableau.

\Section{Spécification}
	
	\textbf{Données}~: le tableau à tester
		
	\textbf{Résultat}~: 
		un booléen à vrai si il existe une valeur nulle dans le tableau.
	
	\begin{center}	
		\flowalgod{is (tableau d’entiers)}{containsNulValue}{booléen}
	\end{center}

\Section{Solution}

	Contrairement au parcours complet (cf. fiche
	\vref{fiche:tab-parcours-complet}) nous allons utiliser un \emph{tant que} 
	(\pc{while}) car nous voulons arrêter l'algorithme dès que la valeur cherchée
	est trouvée.
	
	Il existe essentiellement deux solutions, avec ou sans variable booléenne.
	En général, la solution [A] sera plus claire si le test est court.

	\subsubsection*{[A] Sans variable booléenne}
	
		\begin{pseudocode}
			\Algo{containsNulValue}{\Par{is}{\Array{}{integer}}}{boolean}
				\Decl{i}{integer}
				\Let i \Gets 0
				\While{i < is.length AND is[i] $\neq$ 0}
					\Let i \Gets i + 1
				\EndWhile
				\Return i < n \RComment Si i<n -> arrêt prématuré 
					\Let\RComment c-à-d que l'on a trouvé un 0.
			\EndAlgo
		\end{pseudocode}
	
		Il faut être attentif à \textbf{ne pas inverser} les deux parties du
		test.  Il faut absolument vérifier que l’indice est bon avant de tester
		la valeur à cet indice.  Revoyez la notion de court-circuit
		(\vref{court-circuit} ).
		
	\subsubsection*{[B] Avec variable booléenne}

		\begin{pseudocode}
			\Algo{containsNulValue}{\Par{is}{\Array{}{integer}}}{boolean}
				\Decl{i}{integer}
				\Decl{isZero}{boolean}
				\Let isZero \Gets false
				\Let i \Gets 0
				\While{i < is.length AND NON isZero}
					\Let isZero \Gets is[i] ==  0
					\Let i \Gets i + 1
				\EndWhile
				\Return isZero
			\EndAlgo
		\end{pseudocode}

		Au sortir de la boucle, l’indice \pc{i} ne désigne pas l’élément qui
		nous a permis d’arrêter mais le suivant.  Si nécessaire, on peut
		remplacer l’intérieur de la boucle par~:

		\begin{pseudocode}
			\If{is[i] ==  0}
				\Let isZero \Gets true
			\Else
				\Let i \Gets i + 1
			\EndIf
		\end{pseudocode}
		
		Dans notre exemple, nous cherchons un élément particulier (un 0).  Dans
		le cas où l'on vérifie si tous les éléments possèdent une certaine
		propriété (être positifs par exemple), nous veillerons à adapter le nom
		du booléen et son utilisation (par exemple un booléen appelé
		\pc{areAllPositives}, initialisé à \pc{true} avec un \pc{\dots AND
		areAllPositives} dans le test.

		Dans tous les cas, faites attention à ne pas utiliser \pc{is[i]} si
		vous n’êtes pas sûr du \pc{i}.  C’est particulièrement vrai après la
		boucle.

		Attention aussi à éviter les mauvaises pratiques 
		expliquées à l’annexe \vref{B-for-break}.
		
					
\Section{Quand l’utiliser~?}

	Ce type de solution peut être utilisé pour tout parcours d'un tableau où un
	arrêt prématuré est possible. 

	\begin{itemize}
	\item Est-ce que tous les éléments sont positifs~?
	\item Est-ce que les élément sont triés~?
	\item Est-ce qu’un élément précis est présent~?
	\item \dots
	\end{itemize}
	
\end{Fiche}
