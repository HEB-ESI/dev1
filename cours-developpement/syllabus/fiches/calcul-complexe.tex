%================================
\begin{Fiche}{Un calcul complexe}
%================================
\label{fiche:calcul-complexe}

\Section{Problème}

	Calculer la vitesse moyenne (en km/h) d’un véhicule dont on donne la
	distance parcourue (en mètres) et la durée du parcours (en secondes). 

\Section{Spécification}
	
	\textbf{Données} (toutes réelles et non négatives)~:
		\begin{itemize}
		\item la distance parcourue par le véhicule (en m)~;
		\item la durée du parcours (en s).
		\end{itemize}
		
	\textbf{Résultat}~: la vitesse moyenne du véhicule (en km/h).

	\begin{center}
	\flowalgodd{meters (real)}{seconds (real)}{speed}{real}
	\end{center}

\Section{Exemples}

	\begin{itemize}
	\item \pc{speed(100,10)} donne $36$
	\item \pc{speed(10000,3600)} donne $10$
	\end{itemize}

\Section{Solution}

	La vitesse moyenne est liée à la distance et à la durée par la formule~:
	\[
		\textrm{vitesse} = \frac{\textrm{distance}}{\textrm{durée}}
	\]

	\begin{flushright}
	pour autant que les unités soient cohérentes.
	\end{flushright}

	Ainsi pour obtenir une vitesse en km/h, il faut convertir la distance en
	kilomètres et la durée en heures.  Ce qui donne~:
		
	\begin{pseudocode}
		\Algo{speed}{\Par{meters, seconds}{real}}{real}
			\Decl{km, hours}{real}
			\Let km \Gets meters / 1000
			\Let hours \Gets seconds / 3600
			\Return km / hours
		\EndAlgo
	\end{pseudocode}

\Section{Vérification}

	\begin{center}
		\begin{tabular}{|c|cccc|c|}
		\hline
			\rowcolor{black!40}
		test \no & dist. (m) & durée (s) & réponse attendue 
			& réponse fournie & {} \\
		\hline 
		1 & 100   & 10   & 36 & 36 & {\color{ForestGreen}$\checkmark$} \\\hline
		2 & 10000 & 3600 & 10 & 10 & {\color{ForestGreen}$\checkmark$} \\\hline
		\end{tabular}
	\end{center}								

\Section{Solution Java}

\begin{java}
public static double speed(double meters, double seconds){
	double km;
	double hours;
	km = meters / 1000;
	hours = seconds / 3600;
	return km / hours;
}
\end{java}

\Section{Quand l’utiliser~?}

	Ce type de solution peut être utilisé à chaque fois que la réponse s’obtient
	par un calcul complexe sur les données qu’il est bon de décomposer pour
	aider à sa lecture.  Si le calcul est plutôt simple, on peut le garder en
	une seule expression (cf. fiche \vref{fiche:calcul-simple}).
	
\end{Fiche}
