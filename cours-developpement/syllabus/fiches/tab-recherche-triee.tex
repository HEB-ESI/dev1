%================================
\begin{Fiche}{Tableau trié}
%================================
\label{fiche:tab-recherche-triee}

\Section{Problème}
	
	Rechercher, ajouter, supprimer des données triées dans un tableau d’entiers
	triés.

\Section{Rechercher}

	Rechercher la position où a été trouvé l’élément ou la position où il aurait
	dû être
	
	\textbf{Données}~: le tableau à analyser, le nombre d'éléments dans ce
	tableau, la valeur à rechercher
		
	\textbf{Résultat}~: la position où a été trouvée la valeur ou la position où
	elle aurait dû être

		\begin{pseudocode}
				\LComment{Recherche un élément.}
				\LComment{- index~: indique la position où a été trouvée la valeur}
				\LComment{ou la position où elle aurait dû être}
				\Algo{findIndex}{
						\Par{is\In}{\Array{}{integer}}, 
						\Par{nValues\In}{integer}, 
						\\\hfill\Par{value\In}{integer},
						}{index integer}
					\Decl{index}{integer} \Gets 0
					\While{index < nValues AND is[index] < value}
						\Let index \Gets index + 1
					\EndWhile
					\Return index
				\EndAlgo
		\end{pseudocode}

\Section{Vérifier} 

	Rechercher l'indice d'une donnée trouvée dans un tableau trié ou -1 si elle
	n'est pas trouvée
	
	\textbf{Données}~: le tableau à analyser, le nombre d'éléments dans ce
	tableau, la valeur à rechercher
		
	\textbf{Résultat}~: la position de l'élément si il est dans le tableau et -1
	sinon
	
	Cette opération est triviale.
			
		\begin{pseudocode}
			\LComment{Vérifie si un nombre est dans un tableau d'entiers trié}
			\LComment{et donne sa position (-1 si non présent)}
			\Algo{verify}{
				\Par{is\In}{\Array{}{integers}}, 
				\Par{nValues\In}{integer}, 
				\\\hfill
				\Par{value\In}{integer}
			}{integer}
				\Decl{index}{integer}
				\Decl{isFound}{boolean}
				\Stmt index \Gets findIndex( is, nValues, value)
				\If{is[index] == value}
					\Return index
				\Else
					\Return -1
				\EndIf
			\EndAlgo
		\end{pseudocode}

\Section{Ajouter}

	Ajouter une donnée non encore présente dans le tableau de données triées.
	
	\textbf{Données}~: le tableau à modifier, le nombre d'éléments dans ce
	tableau, la valeur à ajouter
		
	\textbf{Résultat}~: le tableau reçu est modifié en lui ajoutant la valeur si
	elle n'y était pas déjà
	
		\begin{pseudocode}
				\LComment{Ajouter un nombre donné.}
				\Algo{add}{
						\Par{is\In\Out}{\Array{}{integer}}, 
						\Par{nValues\In}{integer}, 
						\\\hfill\Par{value\In}{integer}
						}{integer}
					\Decl{index}{integer}
					\Decl{isFound}{boolean}
					\If{verify(is, nValues, value) == -1}
						\Stmt index \Gets findIndex(is, nValues, value)
						\Stmt shiftRight(is, index, nValues)
						\Let is[index] \Gets value
						\Let nValues \Gets nValues + 1
					\EndIf
					\Return nValues
				\EndAlgo
				\Empty
				\LComment{Décale d’une position à droite les éléments}
				\LComment{entre la position début et fin}
				\Algo{shiftRight}{
						\Par{is\In\Out}{\Array{}{integer}}, 
						\Par{begin\In}{integer}, 
						\Par{end\In}{integer}
						}{}
					\For[-1]{i}{end}{begin}
						\Let is[i+1] \Gets is[i]
					\EndFor
				\EndAlgo
			\end{pseudocode}


\Section{Supprimer}
	Supprimer une donnée d'un tableau de données triées
	
	\textbf{Données}~: le tableau à modifier, le nombre d'éléments dans ce
	tableau, la valeur à supprimer
		
	\textbf{Résultat}~: le tableau reçu est modifié en lui supprimant la valeur
		
		\begin{pseudocode}
			\LComment{supprimer l'élément donné}
			\Algo{delete}{
				\Par{is\In\Out}{\Array{}{integer}}, 
				\Par{nValues\In\Out}{integer}, 
				\\\hfill\Par{value\In}{integer}
			}{integer}
				\Decl{index}{integer}
				\Decl{isFound}{boolean}
				\If{verify(is, nValues, value) $\neq$ -1}
					\Stmt index \Gets findIndex(is, nValues, value)
					\Stmt shiftLeft(is, index + 1, nValues)
					\Let nValues \Gets nValues - 1	
				\EndIf
				\Return nValues
			\EndAlgo
			\Empty
			\LComment{Décale d’une position à gauche les éléments }
			\LComment{entre la position début et fin}
			\Algo{shiftLeft}{
				\Par{tab\In\Out}{\Array{}{integers}}, 
				\Par{begin\In}{integer}, 
				\Par{end\In}{integer}
			}{}
				\For{i}{begin}{end}
					\Let is[i-1] \Gets is[i]
				\EndFor
			\EndAlgo
			\end{pseudocode}


\end{Fiche}
