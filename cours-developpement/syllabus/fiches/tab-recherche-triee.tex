%================================
\begin{Fiche}{Tableau trié}
%================================
\label{fiche:tab-recherche-triee}

\Section{Problème}
	
	Rechercher, ajouter, supprimer des données triées dans un tableau trié. Par
	exemple dans un tableau d'entiers. 

\Section{Rechercher}

\SubSection{Spécification}

	Rechercher la position où a été trouvé l’élément ou la position où il aurait
	dû être.
	
	\textbf{Données}~: le tableau à analyser, le nombre d'éléments dans ce
	tableau, la valeur à rechercher
		
	\textbf{Résultat}~: la position où a été trouvée la valeur ou la position où
	elle aurait dû être

	\SubSection{Solution}

	L'algorithme retourne la position où a été trouvée la valeur \textbf{ou}
	celle où elle devrait être. En parcourant le tableau, il suffit de
	s'arrêter dès que l'élément du tableau est plus grand ou égal à la valeur
	recherchée. 

		\begin{pseudocode}
				\LComment{Recherche un élément.}
				\LComment{- index~: indique la position où a été trouvée la valeur}
				\LComment{ou la position où elle aurait dû être}
				\Algo{findIndex}{
						\Par{is\In}{\Array{}{integer}}, 
						\Par{nValues\In}{integer}, 
						\\\hfill\Par{value\In}{integer},
						}{index integer}
					\Decl{index}{integer} \Gets 0
					\While{index < nValues AND is[index] < value}
						\Let index \Gets index + 1
					\EndWhile
					\Return index
				\EndAlgo
		\end{pseudocode}

		\begin{java}
public static int findIndex(int[] is, int nValues, int value){
	int index = 0;
	while (index < nValues && is[index] < value){
		index++;
	}
	return index;
}
		\end{java}


\Section{Ajouter}

	\SubSection{Spécification}
	
	Ajouter une donnée non encore présente dans le tableau de données triées.
	
	\textbf{Données}~: le tableau à modifier, le nombre d'éléments dans ce
	tableau, la valeur à ajouter
		
	\textbf{Résultat}~: le tableau reçu est modifié en lui ajoutant la valeur si
	elle n'y était pas déjà

	\SubSection{Solution}

	Pour faire un ajout sans doublon dans un tableau triée, il faut~:

	\begin{enumerate}
		\item vérifier que la valeur n'est pas déjà présente\,;
		\item rechercher l'endroit ou elle devrait être placée\,;
		\item déplace les valeurs plus grande d'une position vers la droite\,;
		\item insère la valeur à sa place. 
	\end{enumerate}

	En décomposant le problème, voici une solution~:

	\begin{pseudocode}
		\LComment{Ajouter un nombre donné.}
		\Algo{add}{
				\Par{is\In\Out}{\Array{}{integer}}, 
				\Par{nValues\In}{integer}, 
				\\\hfill\Par{value\In}{integer}
				}{integer}
			\Decl{index}{integer}
			\Decl{isFound}{boolean}
			\If{verify(is, nValues, value) == -1}
				\Stmt index \Gets findIndex(is, nValues, value)
				\Stmt shiftRight(is, index, nValues)
				\Let is[index] \Gets value
				\Let nValues \Gets nValues + 1
			\EndIf
			\Return nValues
		\EndAlgo
		\LComment{Vérifie si un nombre est dans un tableau d'entiers trié}
		\LComment{et donne sa position (-1 si non présent)}
		\Algo{verify}{
			\Par{is\In}{\Array{}{integers}}, 
			\Par{nValues\In}{integer}, 
			\\\hfill
			\Par{value\In}{integer}
		}{integer}
			\Decl{index}{integer}
			\Decl{isFound}{boolean}
			\Stmt index \Gets findIndex( is, nValues, value)
			\If{is[index] == value}
				\Return index
			\Else
				\Return -1
			\EndIf
		\EndAlgo
		\LComment{Décale d’une position à droite les éléments}
		\LComment{entre la position début et fin}
		\Algo{shiftRight}{
				\Par{is\In\Out}{\Array{}{integer}}, 
				\Par{begin\In}{integer}, 
				\Par{end\In}{integer}
				}{}
			\For[-1]{i}{end}{begin}
				\Let is[i+1] \Gets is[i]
			\EndFor
		\EndAlgo
	\end{pseudocode}

	\begin{java}
/**
 * Ajoute un nombre s'il n'est pas présent.
 * 
 * @param is le tableau d'éléments
 * @param nValues le nombre d'éléments du tableau (taille logique)
 * @param value la valeur à insérer
 */
public static int add(int[] is, int nValues, int value){
	int index;
	if (verify(is, nValues, value) == -1){
		index = findIndex(is, nValues, value);
		shiftRight(is, index, nValues);
		is[index] = value;
		nValues++;
	}
	return nValues;
}

/**
 * Vérifie si la valeur est dans le tableau. 
 *
 * @param is le tableau d'éléments
 * @param nValues le nombre d'éléments du tableau (taille logique)
 * @param value la valeur à insérer
 */
public static int verify(int[] is, int nValues, int value){
	int index = findIndex(is, nValues, value);
	if (is[index] == value) {
		return index;
	} else {
		return -1;
	}
}

/**
 * Décale d'une position vers la droite les éléments 
 * entre position début et fin. 
 * 
 * @param is le tableau d'éléments
 * @param begin la position de début
 * @pasam end la position de fin
 */
public static void shiftRight(int[] is, int begin, int end){
	for (int i = end; i >= begin; i--){
		is[i+1] = is[i];
	}
}
	\end{java}

\clearpage
\Section{Supprimer}

	\SubSection{Spécification}

	Supprimer une donnée d'un tableau de données triées.
	
	\textbf{Données}~: le tableau à modifier, le nombre d'éléments dans ce
	tableau, la valeur à supprimer
		
	\textbf{Résultat}~: le tableau reçu est modifié en lui supprimant la valeur
		
	\SubSection{Solution}

	Pour supprimer un élément, dès dès lors qu'il est trouvé, il suffit de décaler tous les élémens à sa droite vers la gauche et adapter la taille logique. 

	\begin{pseudocode}
		\LComment{supprimer l'élément donné}
		\Algo{delete}{
			\Par{is\In\Out}{\Array{}{integer}}, 
			\Par{nValues\In\Out}{integer}, 
			\\\hfill\Par{value\In}{integer}
		}{integer}
			\Decl{index}{integer}
			\Decl{isFound}{boolean}
			\If{verify(is, nValues, value) $\neq$ -1}
				\Stmt index \Gets findIndex(is, nValues, value)
				\Stmt shiftLeft(is, index + 1, nValues)
				\Let nValues \Gets nValues - 1	
			\EndIf
			\Return nValues
		\EndAlgo
		\Empty
		\LComment{Décale d’une position à gauche les éléments }
		\LComment{entre la position début et fin}
		\Algo{shiftLeft}{
			\Par{tab\In\Out}{\Array{}{integers}}, 
			\Par{begin\In}{integer}, 
			\Par{end\In}{integer}
		}{}
			\For{i}{begin}{end}
				\Let is[i-1] \Gets is[i]
			\EndFor
		\EndAlgo
	\end{pseudocode}

	\begin{java}
public static int delete(int[] is, int nValues, int value){
	int index;
	if (verify(is, nValues, value) != -1){
		index = findIndex(is, nValues, value);
		shiftLeft(is, index+1, nValues);
		nValues = nValues - 1;
	}
	return nValues;
}

public static void shiftLeft(int[] is, int begin, int end){
	for (int i=begin; i <= end; i++){
		is[i-1] = is[i];
	}
}
	\end{java}

\end{Fiche}
