%================================
\begin{Fiche}{Passage d’un tableau en paramètre}
%================================
\label{fiche:tab-passage-param}

\Section{Problème}

	Le type \emph{tableau} étant un type à part entière, il est tout-à-fait
	éligible comme type pour les paramètres et la valeur de retour d’un
	algorithme.

\Section{Passer un tableau en paramètre}
		\begin{itemize}
			\item \In : 
				indique que l’algorithme va consulter les valeurs du tableau
				reçu en paramètre.  Les éléments doivent donc avoir été
				initialisés avant d’appeler l’algorithme. Exemple~:
			
				\begin{pseudocode}
					\LComment{Affiche les éléments d’un tableau de n entiers}
					\Algo{display}{\Par{is\In}{\Array{}{integer}}}{} 
						\For{i}{0}{is.length - 1}
							\Write is[i]
						\EndFor
					\EndAlgo 
					
					\LComment{Utilisation possible}
					\Decl{myIntegers}{\Array{10}{entiers}}
					\Let myIntegers \Gets \{2,3,5,7,11,13,17,19,23,29\}
					\Stmt afficher(myIntegers)
				\end{pseudocode}
				
				\textbf{Rappel} C'est  le passage de paramètre par défaut si
				aucune flèche n’est indiquée.
				
			\item \In\Out :
				indique que l’algorithme va consulter/modifier les valeurs 
				du tableau reçu en paramètre. Exemple~:
			
				\begin{pseudocode}
					\LComment{Inverse le signe des éléments d'un tableau de 
						n entiers}
					\Algo{oppositeValues}{\Par{is\In\Out}{\Array{}{integer}}}{} 
						\For{i}{0}{is.length - 1}
							\Let is[i] \Gets -is[i]
						\EndFor
					\EndAlgo 
		
					\LComment{Utilisation possible}
					\Decl{myIntegers}{\Array{5}{entiers}}
					\Let myIntegers \Gets \{2,-3,5,-7,11\}
					\Stmt oppositeValues(myIntegers)
				\end{pseudocode}

			\end{itemize}

			\paragraph{Rappel Java~:} Un tableau Java étant un type référence,
			c'est toujours la valeur de la référence qui est passée en
			paramètre.  En ce sens, un passage en entrée ou en entrée / sortie
			est identique en Java et le tableau doit toujours exister
			c'est-à-dire avoir été créé au préalable. 

		\Section{Retourner un tableau}
		%---------------------------------
			
			Comme pour n’importe quel autre type, un algorithme peut retourner
			un tableau.  Ce sera à lui de le déclarer et de lui donner des
			valeurs.

			\textbf{Exemple~:}
			\begin{pseudocode}
				\LComment Crée un tableau d’entiers de taille n, 
					l’initialise à 0 et le retourne.
				\Algo{create}{n : integer}{\Array{n}{integer}}
					\Decl{is}{\Array{n}{integer}}
					\For{i}{0}{n-1}
						\Let is[i] \Gets 0
					\EndFor
					\Return is
				\EndAlgo
				\Empty
				\LComment{Utilisation possible}
				\Algo{test}{}{}
					\Decl{myIntegers}{\Array{}{integer}}
					\Let myIntegers \Gets create(20)
					\Stmt display(myIntegers)
				\EndAlgo
			\end{pseudocode}
	
\end{Fiche}
