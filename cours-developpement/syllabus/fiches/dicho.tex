%================================
\begin{Fiche}{Recherche dichotomique}
%================================
\label{fiche:dicho}

\Section{Problème}

	Trouver rapidement la position d’une valeur donnée dans un tableau
	\textbf{trié} d’entiers.  Si la valeur n’est pas présente, on donnera la
	position où elle aurait du se trouver.

\Section{Spécification}
	
	\textbf{Données}~: 
		le tableau à analyser
		et la valeur recherchée
		
	\textbf{Résultat}~:
		un booléen indiquant si la valeur a été trouvée
		ec un entier indiquant
			soit la position où la valeur a été trouvée
			soit la position où elle aurait du être.

\Section{Solution}

		L’algorithme rapide que nous avons vu est la recherche
		dichotomique.
		
		\begin{pseudocode}
			\Algo{dichotomousResearch}{
					\\\hfill
					\Par{myArray\In}{\Array{}{integers}}, 
					\Par{value\In}{integer}, 
					\Par{pos\InOut}{integer}
					}{boolean}
				\Decl{rightIndex, leftIndex, medianIndex}{integer}
				\Decl{candidate}{integer}
				\Decl{isFind}{boolean}
				\Empty
				\Let leftIndex \Gets 0
				\Let rightIndex \Gets myArray.length - 1
				\Let isFind \Gets false
				\Empty
				\While{NON isFind AND leftIndex {${\leq}$} rightIndex}
					\Let medianIndex \Gets (leftIndex + rightIndex) DIV 2
					\Let candidate \Gets myArray[medianIndex]
					\If{candidate == value} 
						\Let isFind \Gets vrai
					\ElsIf{candidate < value}
						\Let leftIndex \Gets medianIndex + 1
						\RComment on garde la partie droite
					\Else
						\Let rightIndex \Gets medianIndex – 1
						\RComment on garde la partie gauche
					\EndIf
				\EndWhile
				\Empty
				\If{isFind}
					\Let pos \Gets medianIndex
				\Else
					\Let pos \Gets leftIndex
					\RComment dans le cas où la valeur n’est pas trouvée,
					\Empty 
					\RComment on vérifiera que leftIndex donne la valeur 
					où elle pourrait être insérée.
				\EndIf
				\Empty
				\Return isFind
			\EndAlgo
		\end{pseudocode}
		
	
\end{Fiche}
