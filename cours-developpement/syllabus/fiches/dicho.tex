%================================
\begin{Fiche}{Recherche dichotomique}
%================================
\label{fiche:dicho}

\Section{Problème}

	Trouver rapidement la position d’une valeur donnée dans un tableau
	\textbf{trié} d’entiers.  Si la valeur n’est pas présente, on donnera la
	position où elle aurait du se trouver.

	Voir Section~\vref{chap:recherche-dichotomique}.

\Section{Spécification}
	
	\textbf{Données}~: le tableau à analyser et la valeur recherchée
		
	\textbf{Résultat}~: un booléen indiquant si la valeur a été trouvée et un
	entier indiquant soit la position où la valeur a été trouvée soit la
	position où elle aurait du être.

\Section{Solution}

		L’algorithme rapide que nous avons vu est la recherche
		dichotomique.
		
		\begin{java}
public static int dichotomousSearch(int[] myArray, int value){
	int leftIndex = 0;
	int medianIndex;
	int rightIndex = myArray.length;
	int candidate;
	boolean isFound = false;

	while (!isFound && leftIndex < rightIndex){
		medianIndex = (leftIndex + rightIndex) / 2;
		candidate = myArray[medianIndex];
		if (candidate == value){
			isFound = true;
		} else if (candidate < value){
			leftIndex = medianIndex + 1;
		} else {
			rightIndex = medianIndex - 1;
		}
	}

	if (isFound){
		return medianIndex;
	} else {
		return -1,
	}
}
		\end{java}
		
		Dans le cas d'un langage permettant d'autres passages de paramètres que
		le passage par valeur (cfr.~Section~\vref{paramètres}), nous pourrions
		écrire un algorithme~:

		\begin{itemize}
			
			\item qui retourne un booléen présisant si la valeur est trouvée
				et\,;

			\item qui donne la position à laquelle se trouve la valeur ou la
				position à laquelle elle devrait se trouver ce qui serait
				pratique en cas d'insertion. 

		\end{itemize}

		Un tel algorithme aurait cette allure~:
	
		\begin{pseudocode}
			\Algo{dichotomousSearch}{
					\\\hfill
					\Par{myArray\In}{\Array{}{integers}}, 
					\Par{value\In}{integer}, 
					\Par{pos\InOut}{integer}
					}{boolean}
				\Decl{rightIndex, leftIndex, medianIndex}{integer}
				\Decl{candidate}{integer}
				\Decl{isFound}{boolean}
				\Empty
				\Let leftIndex \Gets 0
				\Let rightIndex \Gets myArray.length - 1
				\Let isFound \Gets false
				\Empty
				\While{NON isFound AND leftIndex {${\leq}$} rightIndex}
					\Let medianIndex \Gets (leftIndex + rightIndex) DIV 2
					\Let candidate \Gets myArray[medianIndex]
					\If{candidate == value} 
						\Let isFound \Gets vrai
					\ElsIf{candidate < value}
						\Let leftIndex \Gets medianIndex + 1
						\RComment on garde la partie droite
					\Else
						\Let rightIndex \Gets medianIndex – 1
						\RComment on garde la partie gauche
					\EndIf
				\EndWhile
				\Empty
				\If{isFound}
					\Let pos \Gets medianIndex
				\Else
					\Let pos \Gets leftIndex
					\RComment dans le cas où la valeur n’est pas trouvée,
					\Empty 
					\RComment on vérifiera que leftIndex donne la valeur 
					où elle pourrait être insérée.
				\EndIf
				\Empty
				\Return isFound
			\EndAlgo
		\end{pseudocode}
\end{Fiche}
