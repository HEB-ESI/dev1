%===================================
\begin{Fiche}{Un nombre pair}
%===================================
\label{fiche:calcul-pair}

\Section{Problème}
Un nombre reçu en paramètre est-il pair\footnote{\textit{Even} en anglais et \textit{odd} pour impair.}~?

\Section{Spécification}

	\textbf{Données}~: le nombre entier dont on veut savoir si il est pair.
		
	\textbf{Résultat}~: un booléen à \textit{vrai} si le \textit{nombre} est pair et \textit{faux} sinon.

	\begin{center}	
		\flowalgod{number (integer)}{isEven}{boolean}
	\end{center}

\Section{Exemples}

	\begin{itemize}
	\item \pc{isEven(2016)} donne $vrai$
	\item \pc{isEven(2015)} donne $faux$
	\end{itemize}
	
\Section{Solution}

	Un nombre est pair si il est multiple de 2. 
	C’est-à-dire si le reste de sa division par 2 vaut 0.
	%\[
		%estPair\ est\ vrai\ \equiv nombre\ MOD\ 2 = 0
	%\]

	\begin{pseudocode}
		\Algo{isEven}{\Par{number}{integer}}{boolean}
			\Return number MOD 2 == 0
		\EndAlgo
	\end{pseudocode}

	Attention à éviter les mauvaises écritures expliquées à l’annexe
	\vref{B-ass-bool}~: pas de \pc{\K{si-sinon}}.
	
\Section{Quand l’utiliser~?}

	À chaque fois qu’un résultat booléen dépend d’un calcul simple.
	Si le calcul est plus compliqué, on peut le décomposer comme
	indiqué dans la fiche \vref{fiche:calcul-complexe}.
	
	On peut également s’inspirer de cette solution
	quand il faut donner sa valeur à une variable booléenne.
		
\end{Fiche}
