%======================================
\begin{Fiche}{Maximum de deux nombres}
%======================================
\label{fiche:max2nb}

\Section{Problème}
	Quel est le maximum de deux nombres~?

\Section{Analyse}

	Voilà un classique de l’algorithmique.  Attention~! On ne veut pas savoir
	\emph{lequel} est le plus grand mais juste la valeur.  Il n’y a donc pas
	d’ambigüité si les deux nombres sont égaux.

	\textbf{Données}~: deux nombres réels.
		
	\textbf{Résultat}~: un réel contenant la plus grande des deux valeurs données.

	\begin{center}	
		\flowalgodd{nb1 (real)}{nb2 (real)}{max2}{real}
	\end{center}

\Section{Exemples}

	\vspace*{-3mm}
	\begin{multicols}{3}
		\begin{itemize}
		\item \pc{max2(-3, 4)} donne $4$
		\item \pc{max2(7, 4)} donne $7$
		\item \pc{max2(4, 4)} donne $4$
		\end{itemize}
	\end{multicols}
	\vspace*{-6mm}
	
\Section{Solution}

	\begin{pseudocode}
	\Algo{max2}{\Par{nb1}{real}, \Par{nb2}{real}}{real}
		\Decl{max}{real}
		\If{nb1 > nb2}
			\Let max \Gets nb1
		\Else
			\Let max \Gets nb2
		\EndIf
		\Return max
	\EndAlgo
	\end{pseudocode}

	Attention à éviter les mauvaises écritures 
	expliquées à l’annexe \vref{B-ass-val}.

	
\Section{Quand l’utiliser~?}

	Cet algorithme peut bien sûr être facilement adapté à la recherche du
	minimum.
		
\end{Fiche}
