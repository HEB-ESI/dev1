% ====================================================================
\chapter{Les variables structurées et les classes}
% ====================================================================

Dans la résolution de problème, l'écriture d'algorithmes et de programmes nous
ne pourrons pas nous contenter des types de données simples. Nous allons devoir
manipuler des donneés plus complexes. C'est là qu'interviennent les structures 
en algorithmique et les classes en langage orienté objet.

Comme un tableau, une structure permet de stocker plusieurs valeurs.  Mais,
à l’inverse des tableaux, chaque valeur peut être de type différent et est
désignée par un nom et pas par un numéro.
		
Cela permet de créer un nouveau type permettant de représenter des données qui
ne peuvent s’inscrire dans les types simples déjà vus.  Par exemple~:

\begin{itemize}

	\item Une \textbf{date} est composée de trois éléments (le jour, le mois,
		l’année). Le mois peut s’exprimer par un entier (15/10/2015) ou par une
		chaine (15~octobre~2015).
	
	\item Un \textbf{moment} de la journée est un triple d’entiers (heures,
		minutes, secondes).
		
	\item La localisation d’un \textbf{point} dans un plan nécessite la
		connaissance de deux coordonnées cartésiennes (l’abscisse \textit{x} et
		l’ordonnée \textit{y}) ou polaires (le rayon \textit{r} et l’angle
		\textit{$\theta$}).
	
	\item Une \textbf{adresse} est composée de plusieurs données~:~ un nom de
		rue, un numéro de maison, parfois un numéro de boite postale, un code
		postal, le nom de la localité, un nom ou code de pays pour une adresse
		à l’étranger\dots
	
\end{itemize}
	
\minitoc

\section{Structure et classe}
\index{structure}
\index{classe}

Pour stocker et manipuler de tels ensembles de données, nous utiliserons des
\textbf{types structurés} ou \textbf{structures}.  Une \textbf{structure} est
donc un ensemble fini d’éléments pouvant être de types distincts.  Chaque
élément de cet ensemble, appelé \textbf{champ} (\textit{field}) ou
\textbf{attribut} (\textit{attribute}) de la structure, possède un nom unique.
		
Notez qu’un attribut d’une structure peut lui-même être une structure.  Par
exemple, une \textbf{carte d’identité} inclut parmi ses informations une
\textbf{date} de naissance, l’\textbf{adresse} de son propriétaire\dots
	
Les langages orientés objet permettent de déclarer leurs propres types. Cette
déclaration se fera au sein d'une \textbf{classe}. 

Comme nous l'avons déjà vu lorsque nous écrivions notre premier programme (cfr.
section \ref{helloworld} p~\pageref{helloworld}) chaque programme Java se trouve
dans une classe. Nous allons faire cohabiter des fichiers Java contenant une
méthode principale (\pc{main}) et des méthodes statiques et d'autres fichiers
Java définissant une classe qui représente un objet, une structure. Cette section
ne fera qu'effleurer la notion d'objet que nous détaillerons dans le cours DEV2 
«~Développement II~». 

	% ===================================
	\subsection{Définition d’une structure}
	% ===================================
	
		\marginicon{definition}
		La définition d’un type structuré adoptera le modèle suivant~:
	
		\begin{pseudocode}
		\Struct{StructureName}
			\Decl{fieldName1}{type1}
			\Decl{fieldName2}{type2}
			\Stmt \dots
			\Decl{fieldNameN}{typeN}
		\EndStruct
		\end{pseudocode}
	
		\pc{fieldName1}, \dots, \pc{fieldNameN} sont les noms des
		différents attributs (ou champs) de la structure, et \pc{type1}, \dots,
		\pc{typeN} les types de ces attributs.  Tous les types peuvent être
		envisagés~: les types «~simples~» (integer, real, boolean, string), les
		tableaux et même d’autres types structurés dont la structure aura été
		préalablement définie.
	
		Pour exemple, nous définissons ci-dessous quatre
		types structurés que nous utiliserons souvent par la suite~:
	
		\begin{minipage}{.5\linewidth}
		\begin{pseudocode}
		\Struct{Date}
			\Decl{day}{integer}
			\Decl{month}{integer}
			\Decl{year}{integer}
		\EndStruct
		\end{pseudocode}
		\end{minipage}
	\ 
		\begin{minipage}{.5\linewidth}
		\begin{pseudocode}
		\Struct{Moment}
			\Decl{hour}{integer}
			\Decl{minute}{integer}
			\Decl{second}{integer}
		\EndStruct
		\end{pseudocode}
		\end{minipage}


		\begin{minipage}{.5\linewidth}
		\begin{pseudocode}
		\Struct{Point}
			\Decl{x}{real}
			\Decl{y}{real}
		\EndStruct{}
		\Empty
		\end{pseudocode}
		\end{minipage}
	\ 
		\begin{minipage}{.5\linewidth}
		\begin{pseudocode}
		\Struct{Student}
			\Decl{name}{string}
			\Decl{matricule}{integer}
			\Decl{section}{string}
			\EndStruct{}
		\end{pseudocode}
	\end{minipage}


	\subsection{Définition d'une classe}

	Sans surprise, le mot clé utilisé sera \pc{class} et une classe se définit
	comme suit (dans une version très simplifiée de la grammaire)~:

	\begin{grammaire}
		\grammarrule{ClassDeclaration:}
		    [public] class \grammarrule{Identifier}
    		    \grammarrule{ClassBody}

		\grammarrule{ClassBody:}
		    \grammarrule{FieldDeclaration}
		    \grammarrule{…}
	\end{grammaire}

	\textit{FieldDeclaration} est une série de déclarations habituelles de
	variables de la forme \pc{Type Identifier;}.

	Définir une classe \pc{Date} équivalente à la structure ci-dessus, aura
	l'allure suivante~:

	\begin{java}
public class Date{
	int day;
	int month;
	int year;
}
	\end{java}


	
	% =====================================================
	\section{Déclaration d’une variable}
	% =====================================================
	
		Une fois un type structuré ou une classe défini, 
		la déclaration de variables de ce type 
		est identique à celle des variables simples. 
		Par exemple~:
	
		\begin{pseudocode}
		\Decl{birthday, dDay}{Date}
		\Decl{departure, arrival, aBeautifullMoment}{Moment}
		\Decl{a, b, gravity}{Point}
		\end{pseudocode}

		\begin{java}
Date birthday;
Date Dday;
Moment departure, arrival, aBeautifullMoment;
Point a, b;
Point gravity;
		\end{java}
	
	% ====================================================
	\section{Utilisation des variables de type structuré}
	% ====================================================
	
		En général, pour manipuler une variable structurée ou en modifier le
		contenu, il faut agir au niveau de ses champs en utilisant les
		opérations permises selon leur type.  Pour accéder à l’un des champs
		d’une variable structurée, il faut mentionner le nom de ce champ ainsi
		que celui de la variable dont il fait partie.  Nous utiliserons la
		notation «~pointée~»~:
	
		\begin{pseudocode}
			\Stmt variableName.fieldName
		\end{pseudocode}
	
		Exemples d’instructions utilisant les variables
		déclarées au paragraphe précédent~:
	
		\begin{pseudocode}
		\Let birthday.day \Gets 15
		\Let birthday.month \Gets 10
		\Let birthday.year \Gets 2014
		\Let arrival.hour \Gets departure.hour + 2
		\Let gravity.x \Gets (a.x + b.x) / 2
		\end{pseudocode}

		Il est également possible d'utiliser une variable structurée
		de manière globale (c’est-à-dire d’une façon qui agit simultanément sur
		chacun de ses champs).  Le cas le plus courant est l’affectation interne
		entre deux variables structurées de même type, par exemple~:
	
		\begin{pseudocode}
		\Let arrival \Gets departure
		\end{pseudocode}
	
		qui résume les trois instructions suivantes~:
	
		\begin{pseudocode}
		\Let arrival.heure \Gets departure.heure
		\Let arrival.minute \Gets departure.minute
		\Let arrival.seconde \Gets departure.seconde
		\end{pseudocode}
		
		
		Par facilité d’écriture, l'assignation peut se faire en une seule fois
		via des «~\{\}~». 
		
		Exemple~:
	
		\begin{pseudocode}
		\Let birthday \Gets \{19, 5, 1955\}
		\end{pseudocode}
	
		Une variable structurée peut aussi être le paramètre ou la valeur de
		retour d’un algorithme.  Par exemple, l’entête d’un algorithme renvoyant
		le nombre de secondes écoulées entre une heure de départ et d’arrivée
		serait~:
	
		\begin{pseudocode}
		\Entete{secondsElapsed}{\Par{departure\In, arrival\In}{Moment}}{integer}
		\end{pseudocode}
	
		On pourra aussi lire ou afficher une variable structurée%
		\footnote{%
			Bien que, dans certains langages, 
			ces opérations devront être décomposées en une lecture ou
			écriture de chaque champ de la structure.
		}.
	
		\begin{pseudocode}
		\Read aMoment
		\Write aMoment
		\end{pseudocode}
	
		Par contre, il n’est pas autorisé d’utiliser les opérateurs de
		comparaison ({\textless}, {\textgreater}) pour comparer des variables
		structurées (même de même type), car une relation d’ordre n’accompagne
		pas toujours les structures utilisées.  En effet, s’il est naturel de
		vouloir comparer des dates ou des moments, comment définir une relation
		d’ordre avec les points du plan ou avec des cartes d’identités~?
	
		Si le besoin de comparer des variables structurées se fait sentir, il
		faudra dans ce cas écrire des modules de comparaison adaptés aux
		structures utilisées.

		\textbf{Et en langage Java ?}

		Pour une classe définie comme dans la section précédente, Java permet 
		de déclarer une variable de ce type et d'accéder aux attributs avec une
		notation pointée. 

		\begin{java}
Date birthday = new Date();
birthday.day = 15;
// ... 
		\end{java}

		Notez la présence de \textbf{\pc{new Date()}}. Instruction obligatoire,
		le \textbf{constructeur} réserve un emplacement mémoire destiné
		à recevoir les valeurs de chaque attribut. En langage orienté objet,
		nous disons que le constructeur, crée — ou instancie — l'objet. 

		Nous verrons en DEV2 «~Développement II~» que de donner l'accès publique
		— c'est-à-dire à tous — aux attributs via \pc{birthday.day} par exemple)
		n'est pas la bonne pratique. Il vaut mieux mettre en œuvre
		\textbf{l'encapsulation} qui consiste à rendre les attributs privés
		(inaccessibles en dehors de la classe) et de fournir des méthodes
		— appelées accesseurs — donnant ces accès. \index{encapsulation}

		Java permet d'utiliser une variable de manière globale puisque c'est une 
		variable de type référence. Nous pourrons bien sûr écrire~:

		\begin{java}
arrival = departure;			
		\end{java}

		\textbf{Notez bien} que cette assignation est une copie de référence et
		que les deux variables références le même objet. Modifier l'un, modifie
		l'autre. En ce sens, c'est différent de ce que nous faisons dans nos
		algorithmes. 

		L'assignation globale telle que présentée précédemment dans les
		algorithmes ne pourra se faire qu'en écrivant explicitement un
		constructeur\footnote{Spoiler. Votre IDE peut vous aider dans cette
		tâche.} recevant les valeurs utiles en paramètres. Un
		\textbf{constructeur}\index{constructeur}~:

		\begin{itemize}
			\item a le même nom que la classe;
			\item n'a pas de type de retour;
			\item construit l'objet~; réserve l'emplacement mémoire nécessaire;
			\item est appelé grâce au mot clé \textbf{\pc{new}}
		\end{itemize}

		La classe \pc{Date}, précédemment écrite, pourrait s'écrire~:

		\begin{java}
public class Date{
	int day;
	int month;
	int year;

	public Date(){
		// permet d'instancier une date sans valeurs de départ
	}

	public Date(int aDay, int aMonth, int aYear){
		day = aDay;
		month = aMonth;
		year = aYear;
	}
}
		\end{java}


		… et l'assignation globable~:

		\begin{java}
Date birthday = new Date{19, 5, 1955};		
		\end{java}

		Une classe définie par l'utilisateur — comme la classe \pc{Date} — peut
		être un paramètre d'une méthode ou un type de retour. 

		Pour la lecture, il faudra lire chaque attribut séparément et créer
		l'objet — grâce au constructeur — ensuite. 

		Pour afficher la valeur d'un objet, il est nécessaire~:

		\begin{description}

			\item[là où l'on veut l'affichage] de faire une simple demande
				d'affichage d'une variable 

\begin{java}
System.out.println(birthday);
\end{java}

			\item[dans la classe] d'ajouter une méthode \pc{toString} (et
				il faut impérativement respecter le nom)\footnote{Comme pour le 
				constructeur, votre IDE peut vous aider.}

				\begin{java}
// ...
public String toString(){
	return day + "/" + month + "/" + year;
}
				\end{java}

		\end{description}

		Nous avons montré ici, ce qu'il faut faire en langage Java pour pouvoir
		utiliser l'équivalent des structures en algorithmique. La notion de
		classe dans les langages orientés objets est beaucoup plus large et
		s'accompagne de concepts essentiels de programmation tels que
		l'encapsulation, le plymorphisme et l'héritage. Ces concepts seront
		abordés en DEV2 - «~Développement II~».
	
	% ==========================================================
	\section{Remarque}
	% ==========================================================
	
		Tous les exemples ci-dessus présentent des champs simples (entier ou
		chaine) mais il n'y a en fait aucune contrainte : un champ peut-être un
		tableau ou même une autre structure.
		
		Par exemple, la structure suivante définit une personne qui possède, en
		plus de son nom, une date de naissance (la structure \pc{Date}) et
		3 numéros de téléphone (un tableau de 3 chaines).
		
		\begin{pseudocode}
			\Struct{Person}
			\Decl{name}{string}
			\Decl{birthday}{Date}
			\Decl{phoneNumbers}{\Array{3}{string}}
			\EndStruct
		\end{pseudocode}

		Et voici un exemple d'utilisation.

		\begin{pseudocode}
			\Decl{me}{Person}
			\Let me.birthday.day \Gets 23
			\Let me.phoneNumbers[0] \Gets "0444/42.42.42"
		\end{pseudocode}
		
		En langage Java, cet exemple peut être immédiatement traduit par~:

		\begin{java}
public class Person{
	String name;
	Date birthday;
	String[] phoneNumbers;

	public Person(){
		birthday = new Date();
		phoneNumbers = new String(3);
	}
}
		\end{java}

		\paragraph{Notez bien} la nécessité de créer le tableau et l'objet date
		dans le constructeur. 

		L'exemple d'utilisation devient~:

		\begin{java}
Person me = new Person();
me.birthday.day = 23;
me.phoneNumbers[0] = "0444/42.42.42";
		\end{java}

		À nouveau, nous verrons les bonnes pratiques dans le prochain cours de 
		développement. 
	
