% ====================
\chapter{Les chaines}\index{chaine}
% ====================

	Dans ce chapitre, nous allons apprendre à manipuler du texte en étudiant les
	différentes fonctions associées aux variables de type chaine.  Ce sera aussi
	l’occasion de revoir quelques techniques algorithmiques déjà acquises pour
	les nombres (alternatives, boucles) afin de les consolider dans un contexte
	plus \og littéraire \fg.

	
	Très tôt dans ce cours, nous avons introduit le type \pc{chaine} mais nous
	n’en avons encore fait que des usages basiques, essentiellement des
	affichages.  Il est temps d’aller plus loin et de les \emph{manipuler},
	c’est-à-dire d’examiner et de modifier, déplacer… le contenu de chaines.
	Procédons par étapes.
	
	\minitoc

	\clearpage
\section{Longueur}
%=================
	
	\marginicon{definition}\index{chaine (longueur d’une)} La \textbf{longueur}
	d’une chaine est le nombre de caractères qu’elle contient.  Pour la
	connaitre on utilisera la notation pointée \pc{unechaine.length}.
	
	Par exemple~:

	\begin{itemize}
	\item \pc{"Bonjour".length} vaut 7.
	\item \pc{"Une chaine".length} vaut 10 (l’espace compte pour 1).
	\item \pc{"A".length} vaut 1.
	\item \pc{"".length} est égal à 0.
	\end{itemize}

	En langage Java également, une chaine connait sa taille~:

	\begin{java}
"Bonjour".length() // vaut 7		
	\end{java}

	\paragraph{Remarque}
	\begin{itemize}
		\item Notez la présence des parenthèses en Java. 
		
		\item Nous n'écrirons pas de parenthèse dans nos algorithmes mais ce
			n'est pas une faute d'en mettre. 
	
	\end{itemize}

	
\section{Chaine et caractère}
%============================

	Certains langages, comme Java, distinguent clairement le type chaine
	(\pc{String}) et le type caractère\index{caractère} (\pc{char}).  Ainsi, en
	Java, le littéral \pc{'A'} représente un caractère qui est différent du
	littéral \pc{"A"} qui représente une chaine (composée d’un seul caractère).
	Le type \pc{String} est un type référence tandis que le type \pc{char} est
	un type primitif. 

	Cette distinction est essentiellement dictée par des détails
	d’implémentation; un caractère pouvant se représenter de façon plus économe
	qu’une chaine. En langage Java toujours essayer de comparer — par un \pc{"A"
	== 'A'} — un chaine et un caractère génère une erreur de compilation. 

	D’autres langages, comme Python, n’ont pas de type caractère spécifique.
	D’autres encore, comme C, ont un type caractère mais pas de type chaine
	à proprement parler; ils sont vus comme des tableaux de caractères. 

	Dans nos algorithmes nous introduirons un type \pc{character} (caractère)
	mais sans le distinguer clairement d’une chaine.  Ainsi, un caractère pourra
	être utilisé comme une chaine et une chaine d’un seul caractère pourra être
	considéré comme un caractère.  En somme, \pc{character} n’est qu’un
	raccourci pour dire~: \emph{une chaine de longueur 1}. 
	
\section{Le contenu d’une chaine}
%=================================

	Pour pouvoir manipuler une chaine, il faut pouvoir accéder aux caractères
	qui la composent.  Comme il s’agit d’une opération de base souvent utilisée,
	nous allons introduire une écriture compacte, empruntée aux tableaux~:
	\pc{nomChaine[i]} désigne le i\ieme{} caractère de la chaine nomChaine (en
	commençant à 1).
	
	Par exemple~:
	\begin{pseudocode}
		\Decl{text}{string} \Gets "Hello"
		\Write text[5]	\RComment Affiche "o"
		\Let text[2] \Gets "a" \RComment {texte vaut "Hallo"; }
			\Let\RComment un caractère doit être remplacé par un seul caractère.
	\end{pseudocode}

	\paragraph{Exemple.}
	Écrivons un algorithme qui vérifie si un mot donné
	contient une lettre donnée.
	
	\begin{pseudocode}
		\Algo{contains}{\Par{word}{string}, \Par{letter}{character}}{boolean}
			\Decl{i}{entier}
			\Let i \Gets 1
			\While{i $\le$ word.length AND word[i] $\neq$ letter}
				\Let i \Gets i + 1
			\EndWhile
			\Return i $\le$ word.length
		\EndAlgo
	\end{pseudocode}

	En langage Java, l'API propose une méthode — \pc{char charAt(int)}
	—  permettant d'accéder à une lettre d'un mot et la retournant. Ce qui se
	traduit en langage Java

	\begin{java}
public static boolean contains(String word, char letter){
	int i = 1;
	while (i <= word.length() && word.charAt(i) != letter){
		i = i + 1;
	}
	return i <= word.length();
}	
	\end{java}

	\clearpage
\section{La concaténation}\index{concaténation}
%=================================

	Il est fréquent de devoir rassembler plusieurs chaines 
	pour former une seule chaine plus grande, 
	il s’agit de l’opération de \textbf{concaténation}.
	Introduisons également une notation compacte, le \pc{+}. 
	
	
	Par exemple~: 
	\begin{pseudocode}
		\Decl{texte}{string}
		\Let texte \Gets "al" + "go" + "rithmique"
		\RComment{assigne la chaine "algortihmique"}
		\Let\RComment{à la variable texte.}
	\end{pseudocode}	


	\paragraph{Exemple.}
	Écrivons un algorithme qui inverse toutes les lettres d’un mot.
	Ainsi, "algo" deviendra "ogla".
	
	\begin{pseudocode}
		\Algo{mirror}{\Par{\InOut word}{string}}{string}
			\Decl{mirror}{string}
			\Let mirror \Gets ""
			\For{i}{1}{word.length}
				\Let mirror \Gets word[i] + mirror
			\EndFor
			\Return mirror
		\EndAlgo
	\end{pseudocode}

	En langage Java, ce n'est pas immédiat de remplacer un caractère par un
	autre dans une chaine. Ceci parce que \pc{character} et \pc{String} sont
	deux types différents. De plus, l'un est de type primitif comme nous l'avons
	déjà dit et l'autre de type référence. 

	\index{api}

	Java — comme tous les autres langages d'ailleurs — propose en plus des
	instructions élémentaires telles que \pc{if} et \pc{while} par exemple, une
	API (\textit{application programming interface}). Une API est un ensemble de
	classes et de méthodes, de manière générale~; de services, qu'offre le
	langage pour aider le développeur. Nous reviendrons abondamment sur ces
	concepts dans les autres cours de développement. 

	Dans le cas qui nous occupe, une manière élégante de résoudre le problème,
	est l'utilisation de \pc{StringBuilder} qui, comme son nom l'indique, permet
	de manipuler plus finement une chaine de caractères. 
	Nous pouvons écrire~:

	\begin{java}
public static String mirror(String word){
	StringBuilder sb = new StringBuilder(word);
	for (int i = 0; i < word.length(); i = i + 1){
		sb.setCharAt(word.length() - i - 1, word.charAt(i));
	}
	return sb.toString();
}
	\end{java}
	

\section{Manipuler les caractères}
%================================

	Voici quelques algorithmes et méthodes de l'API Java \index{api} pratiques 
	pour la manipulation de chaines. 

	Nous pourrions écrire ces algorithmes et ces méthodes de l'API mais ce serait
	long et fastidieux. Dans nos exercices, nous supposerons que les algorithmes
	existent et pour chacun d'eux, nous montrerons comment les écrire en langage
	Java en donnant un exemple simple. 
	
	\begin{description}
	
		\item[\pc{isLetter(c: character)\Gives~boolean}]
			Cette fonction indique si un caractère \textbf{est une lettre}. 
		Par exemple elle retourne vrai pour "a", "e", "G", "K", 
		mais faux pour "4", "\$", "@"\dots %$ 

		\pc{boolean Character.isLetter(char c)}

		\begin{java}
Character. isLetter('#');  // false
char c = 'A';
Character.isLetter(c);   // true			
		\end{java}
	
	
	\item[\pc{isLowerCase(c: character)\Gives~boolean}]	
		Permet de savoir si le caractère \textbf{est une lettre minuscule}.

		\pc{boolean Character.isLowerCase(char c)}

		\begin{java}
Character. isLowerCase('A');  // false
char c = 'a';
Character.isLowerCase(c);   // true			
		\end{java}

	\item[\pc{isUpperCase(c: character)\Gives~boolean}]	
		Permet de savoir si le caractère \textbf{est une lettre majuscule}.
		
		\pc{boolean Character.isUpperCase(char c)}

		\begin{java}
Character. isUpperCase('a');  // false
char c = 'A';
Character.isUpperCase(c);   // true			
		\end{java}
	

	\item[\pc{isDigit(c: character)\Gives~boolean}]	
		Permet de savoir si un caractère \textbf{est un chiffre}. 
		Elle retourne vrai pour les dix caractères 
		'0', '1', '2', '3', '4', '5', '6', '7', '8' et '9'%
		\footnote{%
			Et aussi pour les chiffres dans d'autres langues, cfr. javadoc, 
			\url{https://docs.oracle.com/javase/9/docs/api/java/lang/Character.html#isDigit-char-}
		}

		\begin{java}
Character. isDigit('a');  // false
char c = '1';
Character.isDigit(c);   // true			
		\end{java}
	
	\item[\pc{toUpperCase(s: string)\Gives~string}]
		Retourne une chaine où toutes les lettres du texte
		ont été converties en majuscules.

		En langage Java, c'est la chaine qui est convertie. \\
		\pc{String.toUpperCase()}

		\begin{java}
String s = "hello";
s.toUpperCase();
System.out.println(s);		// affiche HELLO
		\end{java}

	\item[\pc{toLowerCase(s: string)\Gives~string}]
		Retourne une chaine où toutes les lettres du texte
		ont été converties en minuscules.
		
		En langage Java, c'est la chaine qui est convertie. \\
		\pc{String.toLowerCase()}

		\begin{java}
String s = "WORLD";
s.toLowerCase();
System.out.println(s);		// affiche world
		\end{java}
	
	\end{description}
	

\section{L’alphabet}
%===================

	Il peut aussi être pratique de connaitre la position d’une lettre dans
	l’alphabet pour résoudre des exercices.  Ceci se fera à l’aide de la
	fonction~:

	\begin{description}
	
		\item[\pc{letterIndex(c: character)\Gives~integer}]
		Retourne toujours un entier entre 1 et 26. 

		Par exemple \pc{letterIndex("E")} donnera 5, 
		ainsi que \pc{letterIndex("e")}. 
		
		Cette fonction traite donc de la même manière les majuscules et les
		minuscules.  \pc{letterIndex} retournera aussi 5 pour les caractères "é",
		"è", "ê", "ë"\dots).  

		En langage Java, il est assez simple de connaitre la position d'une
		lettre dans l'alphabet dès lors que l'on sait qu'à chaque lettre est
		associée un code (Unicode) et que ces codes se suivent. La
		\href{https://www.unicode.org/charts/PDF/U0000.pdf}{table unicode
		contenant les lettres de l'alphabet} nous montre que le code associé
		à la lettre \pc{'A'} est \texttt{0x0041} ou 65 en base 10, \pc{'B'}
		\texttt{0x0042} et ainsi de suite. 
		
		Cette table nous montre également que les lettres minuscules arrivent
		ensuite~; \pc{'a'} a comme code \texttt{0x0061} ou 97 en base 10,
		\pc{'b'} \texttt{0x0062}…

		Retourner la position d'une lettre devient aussi simple qu'une
		soustraction. En supposant qu'il ne faille pas vérifier que le caractère
		reçu est bien une lettre, nous pourrions écrire~:

		\begin{java}
public static int letterIndex(char c){
	return Character.toLowerCase(c) - 0x61 + 1;
}
		\end{java}

		Pour traiter les caractères accentués, c'est plus complexe et ça sort un
		peu du cadre d'un premier cours de développement. Bien sûr, nous pouvons
		nous en sortir avec une série de \pc{if} de la forme \pc{if(c == 'é' ||
		c== 'è'…)} mais ce serait fastidieux. 

		L'API Java peut faire ce travail pour nous si c'est demandé gentiment.
		Il existe une classe de l'API pour «~normaliser~» les caractères d'une
		chaine. Cette normalisation consiste à vérifier que les caractères
		accentués le sont sous la forme «~lettre-accent~». C'est-à-dire qu'un
		\pc{à} est bien codé \pc{a`}. Dans ce cas, \pc{replaceAll} peut
		supprimer les accents et la méthode devient\footnote{Un \texttt{import
		java.text.Normalizer} est nécessaire.}~:

		\begin{java}
public static int letterIndex(char c){
	c = Normalizer
		.normalize(""+c, Normalizer.Form.NFD)
		.replaceAll("[\u0300-\u036F]", "")
		.charAt(0);
	return Character.toLowerCase(c) - 0x61 + 1;
}
		\end{java}
		
	\end{description}
	
	Il peut être utile d’avoir un outil qui fait l’opération inverse, 
	à savoir associer la lettre de l’alphabet correspondant à une position donnée. 
	Pour cela, nous aurons~: 

	\begin{description}

		\item[\pc{indexToUpperChar(n: integer)\Gives~character}]
		Retourne la forme majuscule de la n\ieme{} lettre de l’alphabet 
		(où \textit{n} sera obligatoirement compris entre 1 et 26). 
		Par exemple, \pc{lettreMaj(13)} retourne "M".

		En langage Java, c'est immédiat par \pc{return 0x41 + n}.

	\item[\pc{indexToLowerChar(n: integer)\Gives~character}]
		Idem pour les minuscules.
		
		En langage Java, c'est immédiat par \pc{return 0x61 + n}.

	\end{description}

	
\section{Chaine et nombre}
%=========================

	Si une chaine contient un nombre,
	on doit pouvoir le convertir, et inversement.

	\begin{description}
	
		\item[\pc{toString(n: real)\Gives~string}]
		Transforme un nombre en chaine.
		Ex: \pc{chaine(42)} retourne la chaine "42"
		et \pc{chaine (3,14)} donnera "3,14". 

		En langage Java, cette conversion est immédiate et se fait de 
		manière transparente.

		\item[\pc{toReal(s : string) \Gives~real}]
		Transforme une chaine contenant des caractères numériques 
		en nombre.
		Ainsi, \pc{nombre("3,14")} retournera 3,14. 
		C’est une erreur de l’utiliser avec une chaine
		qui ne représente pas un nombre.
	
		En langage Java, pour chaque type, il existe une méthode qui fait cette
		transformation. Pour le type \pc{double}, il suffit d'écrire~:

		\begin{java}
String s = "3.14";		
double d = Double.parseDouble(s);			
		\end{java}<++>

	\end{description}
	
\section{Extraction de sous-chaines}
% ==================================

	\begin{description}
	\item[\pc{substring(s: string, beginIndex: integer, endIndex: integer) 
		\Gives~string}]
		
		Permet d’extraire une sous-chaine d'une chaine à partir d'une position
		donnée jusqu'à une autre. \pc{beginIndex} est la position de départ et
		\pc{endIndex - 1} la position de fin… ce qui implique que \pc{endIndex
		- beginIndex} est la longueur de la sous chaine. 

		En langage Java, il existe également une méthode \pc{substring}~:

		\begin{java}
String s = "Karine, Jenny et Vicky sont belles";			
System.out.println(s.substring(8,34)); 
	// Jenny et Vicky sont belles
		\end{java}
	
	\end{description}

	Il faut bien sûr être vigilant pour ne pas sortir des bornes de la chaine
	sinon, une erreur est générée. 

	Cette fonction est très utile pour sélectionner des portions d’une chaine
	contenant des informations codées sous un certain format.  Prenons par
	exemple une date stockée dans une chaine \textit{stringDate} de format 
	"JJ/MM/AAAA" (de longueur 10).  Pour extraire~:
	
	\begin{itemize}
		\item le jour, \pc{substring(stringDate, 0, 2)};
		\item le mois, \pc{substring(stringDate, 3, 5)};
		\item le jour, \pc{substring(stringDate, 6, 10)};
	\end{itemize}

	
\clearpage
\section{Recherche de sous-chaine}
% ==================================
	
	\begin{description}
	\item[\pc{indexOf(s: string, subs: string)\Gives~integer}]	
	
		Permet de savoir si une sous-chaine donnée est présente dans une chaine
		donnée.  Elle permet d’éviter d’écrire le code correspondant à une
		recherche.  La valeur de l’entier renvoyé est la position où commence la
		sous-chaine recherchée.  Si la sous-chaine ne s’y trouve pas, la
		fonction retourne -1.

		En langage Java, il existe également une méthode \pc{indexOf}~:

		\begin{java}
String s = "Des 3 nombrils, " 
	+ "c'est Karine la plus intelligente.";
s.indexOf("Karine"); // 22
		\end{java}

	\end{description}

	\paragraph{Remarque} Pour simplement savoir si la sous-chaine est présente 
	sans demander sa position, la fonction \pc{contains} retourne un booléen et 
	prend les mêmes paramètres. 
	


