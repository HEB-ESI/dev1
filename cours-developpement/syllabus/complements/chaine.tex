% ====================
\chapter{Les chaines et les \texttt{String}}\index{chaine}
% ====================

	Dans ce chapitre, nous allons apprendre à manipuler du texte en étudiant les
	différentes fonctions associées aux variables de type chaine.  Ce sera aussi
	l’occasion de revoir quelques techniques algorithmiques déjà acquises pour
	les nombres (alternatives, boucles) afin de les consolider dans un contexte
	plus \og littéraire \fg.

	
	Très tôt dans ce cours, nous avons introduit le type \pc{chaine} mais nous
	n’en avons encore fait que des usages basiques, essentiellement des
	affichages.  Il est temps d’aller plus loin et de les \emph{manipuler},
	c’est-à-dire d’examiner et de modifier, déplacer… le contenu de chaines.
	
	Procédons par étapes.
	
	\minitoc

	\clearpage


\section{Interface de programmation applicative}
\label{chap:api}
\index{API}\index{application programming interface}

\marginicon{definition}
Une interface de programmation applicative, \textit{application programming
interface} en anglais ou encore simplement \textbf{API} est un ensemble de
méthodes (de classes et de constantes) fournies — dans ce cas par le langage
Java — au développeur.

L'API Java complète peut-être découverte à la lecture de la documentation
fournie avec le langage à cette adresse par exemple~:
\url{https://docs.oracle.com/en/java/javase/11/docs/api/index.html}~\cite{javadoc}

Un des éléments mesurant la force d'un langage est la taille de son API
à laquelle s'ajoute la qualité de sa documentation. Dans cette API se
retrouvent les «~actions ordinaires et habituelles~» que demandent les
développeurs et développeuses. Intéressons-nous à la partie concernant les
chaines de caractères. 

\section{Chaine et caractère}
%============================

	\index{littéral}
	Certains langages, comme Java, distinguent clairement le type chaine
	(\pc{String}) et le type caractère\index{caractère} (\pc{char}).  Ainsi, en
	Java, le littéral \pc{'A'} représente un caractère qui est différent du
	littéral \pc{"A"} qui représente une chaine (composée d’un seul caractère).
	Le type \pc{String} est un type référence tandis que le type \pc{char} est
	un type primitif. 

	Cette distinction est essentiellement dictée par des détails
	d’implémentation; un caractère pouvant se représenter de façon plus économe
	qu’une chaine. En langage Java toujours essayer de comparer — par un
	\pc{"A" == 'A'} — un chaine et un caractère génère une erreur de
	compilation. 

	D’autres langages, comme Python, n’ont pas de type caractère spécifique.
	D’autres encore, comme C, ont un type caractère mais pas de type chaine
	à proprement parler; ils sont vus comme des tableaux de caractères. 


\section{Longueur}
%=================
	
	\index{chaine (longueur d’une)} La \textbf{longueur}
	d’une chaine est le nombre de caractères qu’elle contient.  Pour la
	connaitre on utilisera la notation pointée \pc{uneChaine.length()}.
	
	Par exemple~:

	\begin{itemize}
	\item \pc{"Bonjour".length()} vaut 7.
	\item \pc{"Une chaine".length()} vaut 10 (l’espace compte pour 1).
	\item \pc{"A".length()} vaut 1.
	\item \pc{"".length()} est égal à 0.
	\end{itemize}

	\begin{java}
"Bonjour".length() // vaut 7		
	\end{java}

	\paragraph{Remarque}
	Notez la présence des parenthèses.

	
	
\section{Le contenu d’une chaine}
%=================================

	Pour pouvoir manipuler une chaine, il faut pouvoir accéder aux caractères
	qui la composent.  
	Le langage Java, l'API, propose une méthode — \pc{char charAt(int)}
	—  permettant d'accéder à une lettre d'un mot et la retournant. 
	
	\begin{java}
char c = "Hello".charAt(2);		// le caractère 'l'
	\end{java}

	\paragraph{Exemple.}
	Écrivons un algorithme qui vérifie si un mot donné
	contient une lettre donnée.
	
	\begin{java}
public static boolean contains(String word, char letter){
	int i = 1;
	while (i <= word.length() && word.charAt(i) != letter){
		i = i + 1;
	}
	return i <= word.length();
}	
	\end{java}

	\paragraph{Remarque}

	L'exemple est rhétorique car l'API Java propose une méthode \pc{contains}
	répondant à cette question.  



\section{La concaténation}\index{concaténation}
%=================================

	Il est fréquent de devoir rassembler plusieurs chaines pour former une
	seule chaine plus grande, il s’agit de l’opération de
	\textbf{concaténation}. En Java, c'est simplement l'opérateur \pc{+} qui
	est utilisé. Dès lors que les opérandes sont des chaines, l'opérateur
	concatène au lieu d'additionner. 
	
\begin{java}
String text;
text = "al" + "go" + "rithmique";
\end{java}<++>
	
	\paragraph{Exemple.}
	Écrivons un algorithme qui inverse toutes les lettres d’un mot.
	Ainsi, "algo" deviendra "ogla".
	
	\begin{java}
public static String mirror(String word){
	String mirror = "";
	for (int i = 0; i < word.length(); i = i + 1){
		mirror = word.charAt(i) + mirror;
	}
	return mirror;
}
	\end{java}

	\paragraph{Remarque}

	L'instruction de la ligne 4 concatène un caractère (\pc{word.charAt(i)})
	avec une chaine (\pc{mirror}) ce qui n'est, à priori, pas possible. Dans ce
	cas, Java convertit le caractère en une chaine avant d'effectuer la
	concaténation de manière tout à fait transparente. 

	Dans cette instruction à la ligne 4 toujours, il y a création d'une nouvelle
	chaine de caractère à chaque itération de la boucle. Si l'on veut gagner un
	peu de place en mémoire, il est possible de créer une chaine de la bonne
	taille dès le début et d'y placer les caractères dans l'ordre qui nous
	convient au fur et à mesure. Ce serait mieux n'est-ce pas ? 
	
	En langage Java, ce n'est pas immédiat de remplacer un caractère par un
	autre dans une chaine. Ceci parce que \pc{character} et \pc{String} sont
	deux types différents. De plus, l'un est de type primitif comme nous l'avons
	déjà dit et l'autre de type référence. 

	Une manière élégante de résoudre le problème, est l'utilisation de
	\pc{StringBuilder} qui, comme son nom l'indique, permet de manipuler plus
	finement une chaine de caractères.  Nous pouvons écrire~:

	\begin{java}
public static String mirror(String word){
	StringBuilder sb = new StringBuilder(word);
	for (int i = 0; i < word.length(); i = i + 1){
		sb.setCharAt(word.length() - i - 1, word.charAt(i));
	}
	return sb.toString();
}
	\end{java}

	\begin{itemize}
		\item instruction 2~: création d'un espace de la mème longueur que la 
			chaine \pc{word};

		\item instruction 4~: placement du caractère à la position d'indice $i$
			dans \pc{word} à la position d'indice «~taille du mot - $i$ -1~»
			dans le \pc{StringBuilder}.

	\end{itemize}
	

\section{Manipuler les caractères}
%================================

	Voici quelques méthodes de l'API Java \index{api} pratiques 
	pour la manipulation de chaines. 

	Nous pourrions écrire ces algorithmes et ces méthodes de l'API mais ce serait
	long et fastidieux. Vous pouvez y réfléchir à titre d'exercice. 

	Certaines méthodes sont des méthodes de la classe \pc{String}, d'autres de
	la classe \pc{Character}. Nous vous encourageons à consulter la Javadoc
	\cite{javadoc}. 
	
	
	\begin{description}
	
		\item[\pc{isLetter}]
			Cette fonction indique si un caractère \textbf{est une lettre}. 
		Par exemple elle retourne vrai pour "a", "e", "G", "K", 
		mais faux pour "4", "\$", "@"\dots %$ 

		\pc{boolean Character.isLetter(char c)}

		\begin{java}
Character. isLetter('#');  // false
char c = 'A';
Character.isLetter(c);   // true			
		\end{java}
	
	
	\item[\pc{isLowerCase}]	
		Permet de savoir si le caractère \textbf{est une lettre minuscule}.

		\pc{boolean Character.isLowerCase(char c)}

		\begin{java}
Character. isLowerCase('A');  // false
char c = 'a';
Character.isLowerCase(c);   // true			
		\end{java}

	\item[\pc{isUpperCase}]	
		Permet de savoir si le caractère \textbf{est une lettre majuscule}.
		
		\pc{boolean Character.isUpperCase(char c)}

		\begin{java}
Character. isUpperCase('a');  // false
char c = 'A';
Character.isUpperCase(c);   // true			
		\end{java}
	

	\item[\pc{isDigit}]	
		Permet de savoir si un caractère \textbf{est un chiffre}. 
		Elle retourne vrai pour les dix caractères 
		'0', '1', '2', '3', '4', '5', '6', '7', '8' et '9'
		(et aussi pour les chiffres dans d'autres langues, cfr. 
		\href{https://docs.oracle.com/javase/9/docs/api/java/lang/Character.html\#isDigit-char-}{javadoc}).

		\begin{java}
Character. isDigit('a');  // false
char c = '1';
Character.isDigit(c);   // true			
		\end{java}
	
	\item[\pc{toUpperCase}]
		Convertit la chaine en passant tous les caractères en majuscules.
		
		\pc{String.toUpperCase()}

		\begin{java}
String s = "hello";
s.toUpperCase();
System.out.println(s);		// affiche HELLO
		\end{java}

	\item[\pc{toLowerCase}]
		Convertit la chaine en passant tous les caractères en minuscules.
		
		\pc{String.toLowerCase()}

		\begin{java}
String s = "WORLD";
s.toLowerCase();
System.out.println(s);		// affiche world
		\end{java}
	
	\end{description}
	

\section{L’alphabet}
%===================

	Il peut aussi être pratique de connaitre la position d’une lettre dans
	l’alphabet pour résoudre des exercices. Cette méthode retournerait toujours
	un entier entre 1 et 26 (ou entre 0 et 25). Par exemple,
	\pc{letterIndex('E')} donnerait 5. Cette fonction traiterait de la même
	manière les lettres majuscules et les lettres minuscules. Ainsi,
	\pc{letterIndex('e')} retournerait également 5. 

	Il n'existe pas de telle méthode dans l'API Java. 
	
	Pour les caractères non accentués, il est assez simple de connaitre la
	position d'une lettre dans l'alphabet dès lors que l'on sait qu'à chaque
	lettre est associée un code (Unicode) et que ces codes se suivent. La
	\href{https://www.unicode.org/charts/PDF/U0000.pdf}{table unicode contenant
	les lettres de l'alphabet} nous montre que le code associé à la lettre
	\pc{'A'} est \texttt{0x0041} ou 65 en base 10, \pc{'B'} \texttt{0x0042} et
	ainsi de suite \cite{pbt-unicode}. 
		
	Cette table nous montre également que les lettres minuscules arrivent
	ensuite~; \pc{'a'} a comme code \texttt{0x0061} ou 97 en base 10, \pc{'b'}
	\texttt{0x0062}…

	Retourner la position d'une lettre devient aussi simple qu'une
	soustraction. En supposant qu'il ne faille pas vérifier que le caractère
	reçu est bien une lettre, nous pourrions écrire~:

	\begin{java}
public static int letterIndex(char c){
	return Character.toLowerCase(c) - 0x61 + 1;
}
	\end{java}

	Pour traiter les caractères accentués, c'est plus complexe et ça sort un
	peu du cadre d'un premier cours de développement. Bien sûr, nous pouvons
	nous en sortir avec une série de \pc{if} de la forme \pc{if(c == 'é' || c==
	'è'…)} mais ce serait fastidieux. 

	L'API Java peut faire ce travail pour nous si c'est demandé gentiment.  Il
	existe une classe de l'API pour «~normaliser~» les caractères d'une chaine.
	Cette normalisation consiste à vérifier que les caractères accentués le
	sont sous la forme «~lettre-accent~». C'est-à-dire qu'un \pc{à} est bien
	codé \pc{a`}. Dans ce cas, \pc{replaceAll} peut supprimer les accents et
	l'on pourrait écrire une méthode \pc{letterIndex} qui fonctionne avec des
	lettres accentuées comme suit\footnote{Un \texttt{import
	java.text.Normalizer} est nécessaire.}~:

	\begin{java}
public static int letterIndex(char c){
	c = Normalizer
		.normalize(""+c, Normalizer.Form.NFD)
		.replaceAll("[\u0300-\u036F]", "")
		.charAt(0);
	return Character.toLowerCase(c) - 0x61 + 1;
}
	\end{java}

	… mais ça sort un peu du cadre de ce premier cours de développement. 
		
	
	Il peut être utile d’avoir un outil qui fait l’opération inverse, à savoir
	associer la lettre de l’alphabet correspondant à une position donnée.
	C'est immédiat avec une simple soustraction~: 

	\begin{description}

		\item[\pc{indexToUpperChar}]
		Retourne la forme majuscule de la n\ieme{} lettre de l’alphabet 
		(où \textit{n} sera obligatoirement compris entre 1 et 26). 
		Par exemple, \pc{lettreMaj(13)} retourne "M".

		C'est immédiat par \pc{return 0x41 + n}.

		\item[\pc{indexToLowerChar}]
		Idem pour les minuscules.
		
		C'est immédiat par \pc{return 0x61 + n}.

	\end{description}

	
\section{Chaine et nombre}
%=========================

	Si une chaine contient un nombre, il doit être possible de convertir la
	chaine en nombre et inversement.

	\begin{description}
	
		\item[Nombre vers chaine]			
			La conversion d'un nombre en chaine est immédiate lors l'une
			affectation dans une variable de type \pc{String}

			\begin{java}
				String s = 42;
			\end{java}

			Lors d'une concaténation, les conversions d'un nombre vers une
			chaine sont immédiates à ceci près qu'il faut parfois être attentif
			à l'ordre des opérandes.

			\begin{java}
				String s1 = 1 + "2";		// "12"
				String s2 = 1 + 2;			// "3"
				String s3 = 1 + "2" + "3";	// "123"
				String s4 = 1 + 2 + "3";	// "33"
			\end{java}

		\item[Chaine vers nombre]
		Transforme une chaine contenant des caractères numériques 
		en nombre. Pour chaque type, il existe une méthode qui fait cette
		transformation. 

		\begin{java}
String s = "3.14";		
double d = Double.parseDouble(s);			
s = "5";
int i = Integer.parseInt(s);
		\end{java}

	\end{description}
	
\section{Extraction de sous-chaines}
% ==================================

	\begin{description}
	\item[\pc{substring}]
		
		Permet d’extraire une sous-chaine d'une chaine à partir d'une position
		donnée jusqu'à une autre. 

		\begin{java}
String s = "Karine, Jenny et Vicky paraissent belles";			
System.out.println(s.substring(8,40)); 
	// Jenny et Vicky paraissent belles
		\end{java}
	
	\end{description}

	Il faut bien sûr être vigilant pour ne pas sortir des bornes de la chaine
	sinon, une erreur est générée. 

	Cette fonction est très utile pour sélectionner des portions d’une chaine
	contenant des informations codées sous un certain format.  Prenons par
	exemple une date stockée dans une chaine \textit{stringDate} de format 
	"JJ/MM/AAAA" (de longueur 10).  Pour extraire~:
	
	\begin{itemize}
		\item le jour, \pc{substring(stringDate, 0, 2)};
		\item le mois, \pc{substring(stringDate, 3, 5)};
		\item le jour, \pc{substring(stringDate, 6, 10)};
	\end{itemize}

	
\section{Recherche de sous-chaine}
% ==================================
	
	\begin{description}
	\item[\pc{indexOf}]	
	
		Permet de savoir si une sous-chaine donnée est présente dans une chaine
		donnée.  Elle permet d’éviter d’écrire le code correspondant à une
		recherche.  La valeur de l’entier renvoyé est la position où commence la
		sous-chaine recherchée.  Si la sous-chaine ne s’y trouve pas, la
		fonction retourne -1.

		\begin{java}
String s = "Des 3 nombrils, " 
	+ "c'est Karine la plus intelligente.";
s.indexOf("Karine"); // 22
		\end{java}

	\end{description}

	\paragraph{Remarque} Pour simplement savoir si la sous-chaine est présente 
	sans demander sa position, la fonction \pc{contains} retourne un booléen et 
	prend les mêmes paramètres. 
	


